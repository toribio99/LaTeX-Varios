\documentclass[14pt,letterpaper]{extarticle} % 14pt reales (extsizes)

% ====== Paquetes recomendados ======
\usepackage[utf8]{inputenc}   % Codificación UTF-8
\usepackage[T1]{fontenc}      % Salida de caracteres correcta
\usepackage[spanish]{babel}   % Idioma español
\usepackage{lmodern}          % Fuente moderna
\usepackage{geometry}         % Márgenes
\geometry{left=1cm,right=1cm,top=2cm,bottom=2cm}

\usepackage{amsmath,amssymb}  % Matemáticas
\usepackage{graphicx}         % Imágenes
\usepackage{hyperref}         % Hipervínculos
\usepackage{fancyhdr}         % Encabezados/pies
\usepackage{enumitem}         % Listas cómodas

% Carpeta donde estarán las imágenes
\graphicspath{{imagenes/}}

% Hipervínculos
\hypersetup{
  colorlinks=true,
  linkcolor=blue,
  urlcolor=blue,
  citecolor=blue,
  pdfauthor={Toribio De J Arrieta F},
  pdftitle={Problemas de Física: MRU y MRUA con ilustraciones (extarticle 14pt)}
}

% Encabezado / Pie
\pagestyle{fancy}
\fancyhf{}
\fancyhead[L]{Problemas de Física — MRU/MRUA}
\fancyhead[R]{\thepage}
\fancyfoot[C]{\textit{Elaborado por Toribio Arrieta para Sheyra Celedón Villa}}

% Título
\title{\Huge Problemas de Física con Ilustraciones\\[2mm]
\Large MRU y MRUA aplicados a situaciones cotidianas}
\author{\Large Profesor Tutor: Toribio De J Arrieta F}
\date{\today}

\begin{document}
\maketitle

\tableofcontents
\listoffigures
\bigskip

% =======================================================
\section{David y Juliana: ¿llega a tiempo? (MRUA)}

\subsection{Planteamiento}
David debe recoger a Juliana en 2 minutos. Juliana vive a 20 km de su casa y David acelera en su moto uniformemente a razón de \(4\,\text{m/s}^2\). ¿Alcanzará a recogerla a tiempo?

\begin{figure}[h!]
  \centering
  \includegraphics[width=0.95\textwidth]{situacion_inicial_david.png}
  \caption{Situación inicial: David en su moto, casa de Juliana a 20 km.}
\end{figure}
\noindent\textbf{Enlace de descarga:} \href{run:imagenes/situacion_inicial_david.png}{situacion\_inicial\_david.png}

\subsection{Análisis y resolución}
\begin{itemize}[leftmargin=1.3cm]
  \item Distancia: \(d=20\,\text{km}=20\,000\,\text{m}\)
  \item Tiempo disponible: \(t=2\,\text{min}=120\,\text{s}\)
  \item Aceleración: \(a=4\,\text{m/s}^2\)
  \item Velocidad inicial: \(v_0=0\) (desde el reposo)
\end{itemize}

\subsubsection*{Distancia que recorre en 120 s}
\[
d = v_0 t + \tfrac{1}{2} a t^2
\quad\Rightarrow\quad
d = \tfrac{1}{2}(4)(120)^2 = 28\,800\,\text{m}.
\]

\subsubsection*{¿Llega a tiempo?}
Debe recorrer \(20\,000\,\text{m}\) y puede recorrer \(28\,800\,\text{m}\) en \(120\,\text{s}\).
\textbf{Sí, llega a tiempo.}

\subsubsection*{Tiempo para recorrer 20 km}
\[
d=\tfrac{1}{2} a t^2 \Rightarrow t=\sqrt{\tfrac{2d}{a}}
=\sqrt{\tfrac{2\cdot 20\,000}{4}}=\sqrt{10\,000}=100\,\text{s}.
\]
\textbf{Le sobran 20 s} (dispone de 120 s).

\subsection{Situación solución}
\begin{figure}[h!]
  \centering
  \includegraphics[width=0.95\textwidth]{situacion_solucion_david.png}
  \caption{Solución: llega en \(100\,\text{s}\); sobran \(20\,\text{s}\).}
\end{figure}
\noindent\textbf{Enlace de descarga:} \href{run:imagenes/situacion_solucion_david.png}{situacion\_solucion\_david.png}

\bigskip
\hrule
\bigskip

% =======================================================
\section{Peatón vs. vehículo (MRU): ¿será atropellado?}

\subsection{Planteamiento}
Un peatón se encuentra 120 m por delante de un vehículo y pretende cruzar una calle de 20 m de ancho con velocidad constante \(4\,\text{m/s}\). ¿Será atropellado?

\begin{figure}[h!]
  \centering
  \includegraphics[width=0.95\textwidth]{situacion_inicial.png}
  \caption{Situación inicial: peatón listo para cruzar; vehículo a 120 m del cruce.}
\end{figure}
\noindent\textbf{Enlace de descarga:} \href{run:imagenes/situacion_inicial.png}{situacion\_inicial.png}

\subsection{Análisis y resolución}
\textbf{Tiempo del peatón para cruzar:}
\[
t_p=\frac{20}{4}=5\,\text{s}.
\]
\textbf{Tiempo del vehículo para llegar al cruce:}
\[
t_v=\frac{120}{v}.
\]
Para que ocurra atropello, \(t_v\le 5\Rightarrow \frac{120}{v}\le 5 \Rightarrow v\ge 24\,\text{m/s}\) (\(\approx 86.4\,\text{km/h}\)).

\subsection{Conclusión}
El peatón \textbf{sólo sería atropellado} si \(v\ge 24\,\text{m/s}\).
Si \(v<24\,\text{m/s}\), cruza antes de que llegue el vehículo.

\subsection{Situación solución}
\begin{figure}[h!]
  \centering
  \includegraphics[width=0.95\textwidth]{situacion_solucion.png}
  \caption{Solución ilustrada: comparación de tiempos de llegada al cruce.}
\end{figure}
\noindent\textbf{Enlace de descarga:} \href{run:imagenes/situacion_solucion.png}{situacion\_solucion.png}

\bigskip
\hrule
\bigskip

% =======================================================
\section{Mazda: verificación de la aceleración anunciada (MRUA)}

\subsection{Planteamiento}
Mazda anuncia aceleración \(6\,\text{m/s}^2\).
Un comprador parte del reposo y reporta \(v=280\,\text{m/s}\) en \(t=50\,\text{s}\).
¿Coincide con la publicidad?

\begin{figure}[h!]
  \centering
  \includegraphics[width=0.95\textwidth]{situacion_inicial_mazda.png}
  \caption{Situación inicial: auto en reposo, listo para acelerar.}
\end{figure}
\noindent\textbf{Enlace de descarga:} \href{run:imagenes/situacion_inicial_mazda.png}{situacion\_inicial\_mazda.png}

\subsection{Cálculo de la aceleración real}
\[
v = v_0 + a t \;\Rightarrow\; a=\frac{v-v_0}{t}
=\frac{280-0}{50}=5.6\,\text{m/s}^2.
\]

\subsection{Conclusión}
La aceleración medida (\(5.6\,\text{m/s}^2\)) es \textbf{menor} que la anunciada (\(6\,\text{m/s}^2\)).
\textbf{No coincide exactamente} con la propaganda.

\subsection{Situación solución}
\begin{figure}[h!]
  \centering
  \includegraphics[width=0.95\textwidth]{situacion_solucion_mazda.png}
  \caption{Solución: aceleración real \(5.6\,\text{m/s}^2\) frente a \(6\,\text{m/s}^2\) anunciada.}
\end{figure}
\noindent\textbf{Enlace de descarga:} \href{run:imagenes/situacion_solucion_mazda.png}{situacion\_solucion\_mazda.png}

\bigskip
\hrule
\bigskip

% =======================================================
\section*{Comentario final del profesor}
Estos ejercicios muestran cómo las ecuaciones de MRU y MRUA permiten evaluar con precisión situaciones reales:  
llegadas a tiempo, seguridad vial y verificación de datos de fabricantes.

\end{document}
