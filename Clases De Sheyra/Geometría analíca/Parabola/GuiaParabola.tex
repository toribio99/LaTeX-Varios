% !TEX TS-program = lualatex
% !TEX encoding = UTF-8

\documentclass[12pt,a4paper]{article}

\usepackage{fontspec}
\usepackage[spanish,shorthands=off]{babel}
\usepackage{amsmath,amssymb}
\usepackage{geometry}
\geometry{margin=2.5cm}
\usepackage{tikz}
\usetikzlibrary{calc,arrows.meta}
\usepackage{xcolor}

\title{\Large Guía de Geometría Analítica: La Parábola

\small{Elaborado para : \textsc{\bf{Sheyra Celedón}}}}

\author{\bf{\textsc{Toribio de J Arrieta F}}}
\date{\today}

\begin{document}
	\maketitle

	\section{¿Qué es una parábola?}

	Una \textbf{parábola} es una curva que se forma en el plano cartesiano y tiene una forma muy especial: \emph{es como una U} (puede estar abierta hacia arriba, hacia abajo, hacia la derecha o hacia la izquierda).

	\bigskip

	\textbf{Definición geométrica.} Una parábola es el conjunto de todos los puntos del plano que están a la \textcolor{red}{\textbf{misma distancia}} de un punto fijo llamado \textbf{foco} y de una recta fija llamada \textbf{directriz}.

	\bigskip

	\textbf{¿Dónde vemos parábolas en la vida real?}
	\begin{itemize}
		\item En el chorro de agua de una fuente.
		\item En la trayectoria de una pelota cuando la lanzamos.
		\item En las antenas parabólicas.
		\item En los faros de los carros (los reflectores tienen forma parabólica).
	\end{itemize}

	\section{Elementos de la parábola}

	Toda parábola tiene varios elementos importantes que debemos conocer:

	\begin{itemize}
		\item \textbf{Vértice (V):} Es el punto donde la parábola cambia de dirección. Es el punto más bajo si abre hacia arriba, o el más alto si abre hacia abajo. \emph{(o sea, es como la punta de la U)}.

		\item \textbf{Foco (F):} Es un punto fijo dentro de la parábola. Todos los puntos de la parábola están a la misma distancia del foco que de la directriz.

		\item \textbf{Directriz:} Es una recta fija que está fuera de la parábola.

		\item \textbf{Eje de simetría:} Es una recta que pasa por el vértice y el foco, y divide la parábola en dos partes iguales (como un espejo).

		\item \textbf{Parámetro (p):} Es la distancia del vértice al foco (y también del vértice a la directriz). Este valor nos dice qué tan "abierta" está la parábola.
	\end{itemize}

	\section{Parábola vertical con vértice en el origen}

	Empecemos con el caso más sencillo: cuando el vértice de la parábola está en el origen \((0,0)\) y la parábola abre hacia arriba o hacia abajo.

	\subsection*{Ecuación}

	La ecuación de una parábola vertical con vértice en \((0,0)\) es:
	\[
	\boxed{x^2=4py}
	\]

	\textbf{¿Qué significa cada cosa?}
	\begin{itemize}
		\item \(x^2\): La variable \(x\) está elevada al cuadrado.
		\item \(p\): Es el parámetro (distancia del vértice al foco).
		\item Si \(\mathbf{p>0}\): la parábola abre \textcolor{red}{\textbf{hacia arriba}}.
		\item Si \(\mathbf{p<0}\): la parábola abre \textcolor{blue}{\textbf{hacia abajo}}.
	\end{itemize}

	\subsection*{Elementos importantes}
	\begin{itemize}
		\item Vértice: \(V(0,0)\)
		\item Foco: \(F(0,p)\)
		\item Directriz: \(y=-p\)
		\item Eje de simetría: el eje \(y\) (la recta \(x=0\))
	\end{itemize}

	\subsection*{Ejemplo 1: Parábola que abre hacia arriba}

	\textbf{Ejercicio.} Encuentra el foco y la directriz de la parábola \(x^2=8y\).

	\bigskip

	\textbf{Solución.} Comparamos con la forma \(x^2=4py\):
	\[
	x^2=8y \quad\Rightarrow\quad 4p=8 \quad\Rightarrow\quad p=2.
	\]
	Como \(p=2>0\), la parábola abre hacia arriba.

	\bigskip

	\textbf{Elementos:}
	\begin{itemize}
		\item Vértice: \(V(0,0)\)
		\item Foco: \(F(0,2)\) (porque el foco está a distancia \(p=2\) del vértice, hacia arriba)
		\item Directriz: \(y=-2\) (una recta horizontal a distancia \(p=2\) del vértice, hacia abajo)
	\end{itemize}

	\subsection*{Gráfica de la parábola \(x^2=8y\)}

	\begin{center}
		\begin{tikzpicture}[scale=0.6]
			% límites
			\def\xmin{-8}\def\xmax{8}
			\def\ymin{-4}\def\ymax{10}

			% cuadrícula
			\draw[very thin,gray!50] (\xmin,\ymin) grid (\xmax,\ymax);

			% ejes
			\draw[-{Latex},thick] (\xmin,0)--(\xmax,0) node[right] {$x$};
			\draw[-{Latex},thick] (0,\ymin)--(0,\ymax) node[above] {$y$};

			% marcas numéricas
			\foreach \x in {-8,-7,...,8}
			\draw (\x,0)--(\x,0.12) node[below=6pt,scale=0.7]{\x};
			\foreach \y in {-4,-3,...,10}
			\draw (0,\y)--(0.12,\y) node[left=6pt,scale=0.7]{\y};

			% parábola x^2 = 8y => y = x^2/8
			\draw[thick,red,domain=-8:8,samples=100] plot (\x,{\x*\x/8});

			% vértice
			\filldraw[black] (0,0) circle (3pt) node[below right] {$V(0,0)$};

			% foco
			\filldraw[blue] (0,2) circle (3pt) node[right] {$F(0,2)$};

			% directriz
			\draw[dashed,blue,thick] (-8,-2)--(8,-2) node[right] {$y=-2$};

			% eje de simetría
			\draw[dotted,thick] (0,-4)--(0,10);
			\node[green!30!black,scale=0.9,rotate=90] at (.6,6) {Eje de simetría};

		\end{tikzpicture}
	\end{center}

	\bigskip

	\subsection*{Ejemplo 2: Parábola que abre hacia abajo}

	\textbf{Ejercicio.} Encuentra el foco y la directriz de la parábola \(x^2=-12y\).

	\bigskip

	\textbf{Solución.} Comparamos con la forma \(x^2=4py\):
	\[
	x^2=-12y \quad\Rightarrow\quad 4p=-12 \quad\Rightarrow\quad p=-3.
	\]
	Como \(p=-3<0\), la parábola abre hacia abajo.

	\bigskip

	\textbf{Elementos:}
	\begin{itemize}
		\item Vértice: \(V(0,0)\)
		\item Foco: \(F(0,-3)\) (porque el foco está a distancia \(|p|=3\) del vértice, hacia abajo)
		\item Directriz: \(y=3\) (una recta horizontal a distancia \(|p|=3\) del vértice, hacia arriba)
	\end{itemize}

	\subsection*{Gráfica de la parábola \(x^2=-12y\)}

	\begin{center}
		\begin{tikzpicture}[scale=0.6]
			% límites
			\def\xmin{-8}\def\xmax{8}
			\def\ymin{-10}\def\ymax{5}

			% cuadrícula
			\draw[very thin,gray!50] (\xmin,\ymin) grid (\xmax,\ymax);

			% ejes
			\draw[-{Latex},thick] (\xmin,0)--(\xmax,0) node[right] {$x$};
			\draw[-{Latex},thick] (0,\ymin)--(0,\ymax) node[above] {$y$};

			% marcas numéricas
			\foreach \x in {-8,-7,...,8}
			\draw (\x,0)--(\x,0.12) node[below=6pt,scale=0.7]{\x};
			\foreach \y in {-10,-9,...,5}
			\draw (0,\y)--(0.12,\y) node[left=6pt,scale=0.7]{\y};

			% parábola x^2 = -12y => y = -x^2/12
			\draw[thick,red,domain=-8:8,samples=100] plot (\x,{-\x*\x/12});

			% vértice
			\filldraw[black] (0,0) circle (3pt) node[above right] {$V(0,0)$};

			% foco
			\filldraw[blue] (0,-3) circle (3pt) node[right] {$F(0,-3)$};

			% directriz
			\draw[dashed,blue,thick] (-8,3)--(8,3) node[right] {$y=3$};

			% eje de simetría
			\draw[dotted,thick] (0,-10)--(0,5);
			\node[green!30!black,scale=0.9,rotate=90] at (0.6,-7) {Eje de simetría};

		\end{tikzpicture}
	\end{center}

	\section{Parábola horizontal con vértice en el origen}

	Ahora veamos el caso cuando la parábola abre hacia la derecha o hacia la izquierda (en lugar de hacia arriba o abajo).

	\subsection*{Ecuación}

	La ecuación de una parábola horizontal con vértice en \((0,0)\) es:
	\[
	\boxed{y^2=4px}
	\]

	\textbf{¿Qué significa cada cosa?}
	\begin{itemize}
		\item \(y^2\): Ahora la variable \(y\) está elevada al cuadrado (en lugar de \(x\)).
		\item \(p\): Es el parámetro (distancia del vértice al foco).
		\item Si \(\mathbf{p>0}\): la parábola abre \textcolor{red}{\textbf{hacia la derecha}}.
		\item Si \(\mathbf{p<0}\): la parábola abre \textcolor{blue}{\textbf{hacia la izquierda}}.
	\end{itemize}

	\subsection*{Elementos importantes}
	\begin{itemize}
		\item Vértice: \(V(0,0)\)
		\item Foco: \(F(p,0)\)
		\item Directriz: \(x=-p\)
		\item Eje de simetría: el eje \(x\) (la recta \(y=0\))
	\end{itemize}

	\subsection*{Ejemplo 3: Parábola que abre hacia la derecha}

	\textbf{Ejercicio.} Encuentra el foco y la directriz de la parábola \(y^2=16x\).

	\bigskip

	\textbf{Solución.} Comparamos con la forma \(y^2=4px\):
	\[
	y^2=16x \quad\Rightarrow\quad 4p=16 \quad\Rightarrow\quad p=4.
	\]
	Como \(p=4>0\), la parábola abre hacia la derecha.

	\bigskip

	\textbf{Elementos:}
	\begin{itemize}
		\item Vértice: \(V(0,0)\)
		\item Foco: \(F(4,0)\) (porque el foco está a distancia \(p=4\) del vértice, hacia la derecha)
		\item Directriz: \(x=-4\) (una recta vertical a distancia \(p=4\) del vértice, hacia la izquierda)
	\end{itemize}

	\subsection*{Gráfica de la parábola \(y^2=16x\)}

	\begin{center}
		\begin{tikzpicture}[scale=0.6]
			% límites
			\def\xmin{-6}\def\xmax{12}
			\def\ymin{-10}\def\ymax{10}

			% cuadrícula
			\draw[very thin,gray!50] (\xmin,\ymin) grid (\xmax,\ymax);

			% ejes
			\draw[-{Latex},thick] (\xmin,0)--(\xmax,0) node[right] {$x$};
			\draw[-{Latex},thick] (0,\ymin)--(0,\ymax) node[above] {$y$};

			% marcas numéricas
			\foreach \x in {-6,-5,...,12}
			\draw (\x,0)--(\x,0.12) node[below=6pt,scale=0.7]{\x};
			\foreach \y in {-10,-9,...,10}
			\draw (0,\y)--(0.12,\y) node[left=6pt,scale=0.7]{\y};

			% parábola y^2 = 16x => x = y^2/16
			\draw[thick,red,domain=-10:10,samples=100] plot ({\x*\x/16},\x);

			% vértice
			\filldraw[black] (0,0) circle (3pt) node[below left] {$V(0,0)$};

			% foco
			\filldraw[blue] (4,0) circle (3pt) node[below] {$F(4,0)$};

			% directriz
			\draw[dashed,blue,thick] (-4,-10)--(-4,10) node[above] {$x=-4$};

			% eje de simetría
			\draw[dotted,thick] (-6,0)--(12,0);
			\node[above, green!30!black,scale=0.9] at (8,0.3) {Eje de simetría};

		\end{tikzpicture}
	\end{center}

	\section{Parábola con vértice fuera del origen}

	Hasta ahora hemos visto parábolas con vértice en \((0,0)\). Pero, ¿qué pasa si el vértice está en otro punto \((h,k)\)?

	\subsection*{Parábola vertical con vértice en \((h,k)\)}

	Si la parábola abre hacia arriba o hacia abajo y tiene vértice en \((h,k)\), la ecuación es:
	\[
	\boxed{(x-h)^2=4p(y-k)}
	\]

	\textbf{Elementos:}
	\begin{itemize}
		\item Vértice: \(V(h,k)\)
		\item Foco: \(F(h,k+p)\)
		\item Directriz: \(y=k-p\)
		\item Eje de simetría: \(x=h\)
	\end{itemize}

	\subsection*{Parábola horizontal con vértice en \((h,k)\)}

	Si la parábola abre hacia la derecha o hacia la izquierda y tiene vértice en \((h,k)\), la ecuación es:
	\[
	\boxed{(y-k)^2=4p(x-h)}
	\]

	\textbf{Elementos:}
	\begin{itemize}
		\item Vértice: \(V(h,k)\)
		\item Foco: \(F(h+p,k)\)
		\item Directriz: \(x=h-p\)
		\item Eje de simetría: \(y=k\)
	\end{itemize}

	\subsection*{Ejemplo 4: Parábola con vértice fuera del origen}

	\textbf{Ejercicio.} Encuentra el vértice, foco y directriz de la parábola \((x-2)^2=8(y+1)\).

	\bigskip

	\textbf{Solución.} Comparamos con la forma \((x-h)^2=4p(y-k)\):
	\[
	(x-2)^2=8(y-(-1)) \quad\Rightarrow\quad h=2,\quad k=-1,\quad 4p=8 \quad\Rightarrow\quad p=2.
	\]
	Como \(p=2>0\), la parábola abre hacia arriba.

	\bigskip

	\textbf{Elementos:}
	\begin{itemize}
		\item Vértice: \(V(2,-1)\)
		\item Foco: \(F(2,-1+2)=F(2,1)\)
		\item Directriz: \(y=-1-2=-3\)
		\item Eje de simetría: \(x=2\)
	\end{itemize}

	\subsection*{Gráfica de la parábola \((x-2)^2=8(y+1)\)}

	\begin{center}
		\begin{tikzpicture}[scale=0.6]
			% límites
			\def\xmin{-4}\def\xmax{8}
			\def\ymin{-5}\def\ymax{8}

			% cuadrícula
			\draw[very thin,gray!50] (\xmin,\ymin) grid (\xmax,\ymax);

			% ejes
			\draw[-{Latex},thick] (\xmin,0)--(\xmax,0) node[right] {$x$};
			\draw[-{Latex},thick] (0,\ymin)--(0,\ymax) node[above] {$y$};

			% marcas numéricas
			\foreach \x in {-4,-3,...,8}
			\draw (\x,0)--(\x,0.12) node[below=6pt,scale=0.7]{\x};
			\foreach \y in {-5,-4,...,8}
			\draw (0,\y)--(0.12,\y) node[left=6pt,scale=0.7]{\y};

			% parábola (x-2)^2 = 8(y+1) => y = (x-2)^2/8 - 1
			\draw[thick,red,domain=-2:6,samples=100] plot (\x,{(\x-2)*(\x-2)/8-1});

			% vértice
			\filldraw[black] (2,-1) circle (3pt) node[below right] {$V(2,-1)$};

			% foco
			\filldraw[blue] (2,1) circle (3pt) node[right] {$F(2,1)$};

			% directriz
			\draw[dashed,blue,thick] (-4,-3)--(8,-3) node[right] {$y=-3$};

			% eje de simetría
			\draw[dotted,thick] (2,-5)--(2,8);
			\node[green!30!black,scale=0.9,rotate=9] at (2.2,7.5) {$x=2$};
			\node[green!30!black,scale=0.9,rotate=90] at (2.6,4.3) {Eje de simetría};

		\end{tikzpicture}
	\end{center}

	\section{Forma desarrollada de la parábola}

	A veces la ecuación de la parábola no está en la forma \((x-h)^2=4p(y-k)\) que hemos visto, sino que está \textbf{desarrollada} (o sea, con todos los términos multiplicados). En ese caso necesitamos \textbf{completar el cuadrado} para encontrar el vértice y el foco.

	\subsection*{Ejemplo 5: Completar el cuadrado}

	\textbf{Ejercicio.} Encuentra el vértice y el foco de la parábola \(x^2-4x-8y+12=0\).

	\bigskip

	\textbf{Solución.} Vamos a reorganizar la ecuación para dejarla en la forma \((x-h)^2=4p(y-k)\).

	\bigskip

	\textbf{Paso 1:} Agrupamos los términos con \(x\) y dejamos los demás del otro lado:
	\[
	x^2-4x=8y-12.
	\]

	\textbf{Paso 2:} Completamos el cuadrado en el lado izquierdo. Para eso, tomamos el coeficiente de \(x\) (que es \(-4\)), lo dividimos entre 2 y lo elevamos al cuadrado:
	\[
	\left(\frac{-4}{2}\right)^2=(-2)^2=4.
	\]
	Sumamos 4 a ambos lados:
	\[
	x^2-4x+4=8y-12+4 \quad\Rightarrow\quad (x-2)^2=8y-8.
	\]

	\textbf{Paso 3:} Factorizamos el lado derecho:
	\[
	(x-2)^2=8(y-1).
	\]

	\textbf{Paso 4:} Ahora comparamos con la forma \((x-h)^2=4p(y-k)\):
	\[
	h=2,\quad k=1,\quad 4p=8 \quad\Rightarrow\quad p=2.
	\]

	\bigskip

	\textbf{Elementos:}
	\begin{itemize}
		\item Vértice: \(V(2,1)\)
		\item Foco: \(F(2,1+2)=F(2,3)\)
		\item Directriz: \(y=1-2=-1\)
	\end{itemize}

	\section{Resumen: ¿Cómo identificar una parábola?}

	\begin{center}
		\begin{tabular}{|c|c|c|}
			\hline
			\textbf{Ecuación} & \textbf{Orientación} & \textbf{Abre hacia} \\
			\hline
			\(x^2=4py\) con \(p>0\) & Vertical & Arriba \\
			\hline
			\(x^2=4py\) con \(p<0\) & Vertical & Abajo \\
			\hline
			\(y^2=4px\) con \(p>0\) & Horizontal & Derecha \\
			\hline
			\(y^2=4px\) con \(p<0\) & Horizontal & Izquierda \\
			\hline
		\end{tabular}
	\end{center}

	\bigskip

	\textbf{Regla práctica:}
	\begin{itemize}
		\item Si \textcolor{red}{\textbf{la variable \(x\) está al cuadrado}}, la parábola es \textbf{vertical} (abre hacia arriba o abajo).
		\item Si \textcolor{blue}{\textbf{la variable \(y\) está al cuadrado}}, la parábola es \textbf{horizontal} (abre hacia la derecha o izquierda).
	\end{itemize}

	\section{Ejercicios propuestos}

	\textbf{1.} Encuentra el foco y la directriz de la parábola \(x^2=12y\).

	\bigskip

	\textbf{2.} Encuentra el foco y la directriz de la parábola \(y^2=-8x\).

	\bigskip

	\textbf{3.} Encuentra el vértice y el foco de la parábola \((x+3)^2=16(y-2)\).

	\bigskip

	\textbf{4.} Encuentra el vértice y el foco de la parábola \(y^2+6y-4x+1=0\) (pista: completa el cuadrado).

	\bigskip

	\textbf{5.} Dibuja la gráfica de la parábola \(x^2=-4y\) e indica el vértice, foco y directriz.

	\bigskip
	\bigskip

	\section{Soluciones detalladas de los ejercicios}

	\subsection*{Solución del Ejercicio 1}

	\textbf{Enunciado:} Encuentra el foco y la directriz de la parábola \(x^2=12y\).

	\bigskip

	\textbf{Solución paso a paso:}

	\bigskip

	\textbf{Paso 1: Identificar el tipo de parábola.}

	La ecuación es \(x^2=12y\). Como la variable \(x\) está elevada al cuadrado, se trata de una \textcolor{red}{\textbf{parábola vertical}} (abre hacia arriba o hacia abajo).

	\bigskip

	\textbf{Paso 2: Comparar con la forma estándar.}

	La forma estándar de una parábola vertical con vértice en el origen es:
	\[
	x^2=4py
	\]

	Comparamos nuestra ecuación con esta forma:
	\[
	x^2=12y \quad\Leftrightarrow\quad x^2=4py
	\]

	Por lo tanto:
	\[
	4p=12
	\]

	\textbf{Paso 3: Calcular el valor de \(p\).}

	Despejamos \(p\):
	\[
	4p=12 \quad\Rightarrow\quad p=\frac{12}{4}=3
	\]

	Como \(\mathbf{p=3>0}\), la parábola \textcolor{red}{\textbf{abre hacia arriba}}.

	\bigskip

	\textbf{Paso 4: Determinar el vértice.}

	Como la ecuación es de la forma \(x^2=4py\) (sin desplazamiento), el vértice está en el origen:
	\[
	\boxed{V(0,0)}
	\]

	\textbf{Paso 5: Calcular el foco.}

	Para una parábola vertical con vértice en \((0,0)\), el foco está en \((0,p)\):
	\[
	\boxed{F(0,3)}
	\]

	\emph{Interpretación:} El foco está 3 unidades arriba del vértice, sobre el eje \(y\).

	\bigskip

	\textbf{Paso 6: Determinar la directriz.}

	Para una parábola vertical con vértice en \((0,0)\), la directriz es la recta \(y=-p\):
	\[
	\boxed{y=-3}
	\]

	\emph{Interpretación:} La directriz es una recta horizontal que está 3 unidades debajo del vértice.

	\bigskip

	\textbf{Paso 7: Identificar el eje de simetría.}

	El eje de simetría es la recta vertical que pasa por el vértice:
	\[
	x=0 \quad\text{(el eje \(y\))}
	\]

	\bigskip

	\textbf{Resumen de elementos:}
	\begin{itemize}
		\item Vértice: \(V(0,0)\)
		\item Foco: \(F(0,3)\)
		\item Directriz: \(y=-3\)
		\item Eje de simetría: \(x=0\)
		\item Orientación: Abre hacia arriba
		\item Parámetro: \(p=3\)
	\end{itemize}

	\subsection*{Gráfica del Ejercicio 1: Parábola \(x^2=12y\)}

	\begin{center}
		\begin{tikzpicture}[scale=0.55]
			% límites
			\def\xmin{-10}\def\xmax{10}
			\def\ymin{-5}\def\ymax{12}

			% cuadrícula
			\draw[very thin,gray!50] (\xmin,\ymin) grid (\xmax,\ymax);

			% ejes
			\draw[-{Latex},thick] (\xmin,0)--(\xmax,0) node[right] {$x$};
			\draw[-{Latex},thick] (0,\ymin)--(0,\ymax) node[above] {$y$};

			% marcas numéricas en x
			\foreach \x in {-10,-9,...,10}
			\draw (\x,0)--(\x,0.12) node[below=6pt,scale=0.65]{\x};

			% marcas numéricas en y
			\foreach \y in {-5,-4,...,12}
			\draw (0,\y)--(0.12,\y) node[left=6pt,scale=0.65]{\y};

			% parábola x^2 = 12y => y = x^2/12
			\draw[thick,red,domain=-10:10,samples=100] plot (\x,{\x*\x/12});

			% vértice
			\filldraw[black] (0,0) circle (3.5pt) node[below right=2pt,scale=1] {$V(0,0)$};

			% foco
			\filldraw[blue] (0,3) circle (3.5pt) node[right=3pt,scale=1] {$F(0,3)$};

			% directriz
			\draw[dashed,blue,very thick] (\xmin,-3)--(\xmax,-3);
			\node[blue,scale=1] at (-8,-3.6) {Directriz: $y=-3$};

			% eje de simetría (línea punteada vertical)
			\draw[dotted,thick,green!60!black] (0,\ymin)--(0,\ymax);
			\node[green!60!black,scale=0.9,rotate=90] at (0.7,9) {Eje de simetría};

			% etiqueta de la parábola
			\node[red,scale=1.1] at (7,5) {$x^2=12y$};

			% algunos puntos de la parábola para ilustrar
			\filldraw[red] (6,3) circle (2pt);
			\filldraw[red] (-6,3) circle (2pt);
			\node[red,scale=0.8] at (6,3.7) {$(6,3)$};
			\node[red,scale=0.8] at (-6,3.7) {$(-6,3)$};

			% distancia del vértice al foco
			\draw[<->,orange,very thick] (0.5,0)--(0.5,3);
			\node[orange,scale=0.9] at (1.8,1.5) {$p=3$};

		\end{tikzpicture}
	\end{center}

	\bigskip

	\textbf{Verificación:} Podemos verificar que cualquier punto de la parábola está a la misma distancia del foco que de la directriz. Por ejemplo, el punto \((6,3)\):
	\begin{itemize}
		\item Distancia al foco \(F(0,3)\): \(\sqrt{(6-0)^2+(3-3)^2}=\sqrt{36}=6\)
		\item Distancia a la directriz \(y=-3\): \(|3-(-3)|=6\)
	\end{itemize}
	¡Las distancias son iguales!

	\newpage

	\subsection*{Solución del Ejercicio 2}

	\textbf{Enunciado:} Encuentra el foco y la directriz de la parábola \(y^2=-8x\).

	\bigskip

	\textbf{Solución paso a paso:}

	\bigskip

	\textbf{Paso 1: Identificar el tipo de parábola.}

	La ecuación es \(y^2=-8x\). Como la variable \(y\) está elevada al cuadrado, se trata de una \textcolor{blue}{\textbf{parábola horizontal}} (abre hacia la derecha o hacia la izquierda).

	\bigskip

	\textbf{Paso 2: Comparar con la forma estándar.}

	La forma estándar de una parábola horizontal con vértice en el origen es:
	\[
	y^2=4px
	\]

	Comparamos nuestra ecuación con esta forma:
	\[
	y^2=-8x \quad\Leftrightarrow\quad y^2=4px
	\]

	Por lo tanto:
	\[
	4p=-8
	\]

	\textbf{Paso 3: Calcular el valor de \(p\).}

	Despejamos \(p\):
	\[
	4p=-8 \quad\Rightarrow\quad p=\frac{-8}{4}=-2
	\]

	Como \(\mathbf{p=-2<0}\), la parábola \textcolor{blue}{\textbf{abre hacia la izquierda}}.

	\bigskip

	\textbf{Paso 4: Determinar el vértice.}

	Como la ecuación es de la forma \(y^2=4px\) (sin desplazamiento), el vértice está en el origen:
	\[
	\boxed{V(0,0)}
	\]

	\textbf{Paso 5: Calcular el foco.}

	Para una parábola horizontal con vértice en \((0,0)\), el foco está en \((p,0)\):
	\[
	\boxed{F(-2,0)}
	\]

	\emph{Interpretación:} El foco está 2 unidades a la izquierda del vértice, sobre el eje \(x\).

	\bigskip

	\textbf{Paso 6: Determinar la directriz.}

	Para una parábola horizontal con vértice en \((0,0)\), la directriz es la recta \(x=-p\):
	\[
	x=-(-2)=2 \quad\Rightarrow\quad \boxed{x=2}
	\]

	\emph{Interpretación:} La directriz es una recta vertical que está 2 unidades a la derecha del vértice.

	\bigskip

	\textbf{Paso 7: Identificar el eje de simetría.}

	El eje de simetría es la recta horizontal que pasa por el vértice:
	\[
	y=0 \quad\text{(el eje \(x\))}
	\]

	\bigskip

	\textbf{Resumen de elementos:}
	\begin{itemize}
		\item Vértice: \(V(0,0)\)
		\item Foco: \(F(-2,0)\)
		\item Directriz: \(x=2\)
		\item Eje de simetría: \(y=0\)
		\item Orientación: Abre hacia la izquierda
		\item Parámetro: \(p=-2\)
	\end{itemize}

	\subsection*{Gráfica del Ejercicio 2: Parábola \(y^2=-8x\)}

	\begin{center}
		\begin{tikzpicture}[scale=0.55]
			% límites
			\def\xmin{-12}\def\xmax{5}
			\def\ymin{-10}\def\ymax{10}

			% cuadrícula
			\draw[very thin,gray!50] (\xmin,\ymin) grid (\xmax,\ymax);

			% ejes
			\draw[-{Latex},thick] (\xmin,0)--(\xmax,0) node[right] {$x$};
			\draw[-{Latex},thick] (0,\ymin)--(0,\ymax) node[above] {$y$};

			% marcas numéricas en x
			\foreach \x in {-12,-11,...,5}
			\draw (\x,0)--(\x,0.12) node[below=6pt,scale=0.65]{\x};

			% marcas numéricas en y
			\foreach \y in {-10,-9,...,10}
			\draw (0,\y)--(0.12,\y) node[left=6pt,scale=0.65]{\y};

			% parábola y^2 = -8x => x = -y^2/8
			\draw[thick,red,domain=-10:10,samples=100] plot ({-\x*\x/8},\x);

			% vértice
			\filldraw[black] (0,0) circle (3.5pt) node[above right=2pt,scale=1] {$V(0,0)$};

			% foco
			\filldraw[blue] (-2,0) circle (3.5pt) node[above=3pt,scale=1] {$F(-2,0)$};

			% directriz
			\draw[dashed,blue,very thick] (2,\ymin)--(2,\ymax);
			\node[blue,scale=1] at (2,8.5) {Directriz: $x=2$};

			% eje de simetría (línea punteada horizontal)
			\draw[dotted,thick,green!60!black] (\xmin,0)--(\xmax,0);
			\node[green!60!black,scale=0.9] at (-9,0.7) {Eje de simetría};

			% etiqueta de la parábola
			\node[red,scale=1.1] at (-8,5) {$y^2=-8x$};

			% algunos puntos de la parábola para ilustrar
			\filldraw[red] (-2,4) circle (2pt);
			\filldraw[red] (-2,-4) circle (2pt);
			\node[red,scale=0.8] at (-2.8,4.5) {$(-2,4)$};
			\node[red,scale=0.8] at (-2.8,-4.5) {$(-2,-4)$};

			% distancia del vértice al foco
			\draw[<->,orange,very thick] (0,-0.6)--(-2,-0.6);
			\node[orange,scale=0.9] at (-1,-1.4) {$|p|=2$};

		\end{tikzpicture}
	\end{center}

	\bigskip

	\textbf{Verificación:} Podemos verificar que el punto \((-2,4)\) de la parábola está a la misma distancia del foco que de la directriz:
	\begin{itemize}
		\item Distancia al foco \(F(-2,0)\): \(\sqrt{(-2-(-2))^2+(4-0)^2}=\sqrt{16}=4\)
		\item Distancia a la directriz \(x=2\): \(|(-2)-2|=|-4|=4\)
	\end{itemize}
	¡Las distancias son iguales!

	\newpage

	\subsection*{Solución del Ejercicio 3}

	\textbf{Enunciado:} Encuentra el vértice y el foco de la parábola \((x+3)^2=16(y-2)\).

	\bigskip

	\textbf{Solución paso a paso:}

	\bigskip

	\textbf{Paso 1: Identificar el tipo de parábola.}

	La ecuación es \((x+3)^2=16(y-2)\). Como la variable \(x\) (dentro del paréntesis) está elevada al cuadrado, se trata de una \textcolor{red}{\textbf{parábola vertical}}.

	\bigskip

	\textbf{Paso 2: Comparar con la forma estándar.}

	La forma estándar de una parábola vertical con vértice en \((h,k)\) es:
	\[
	(x-h)^2=4p(y-k)
	\]

	Reescribimos nuestra ecuación para compararla:
	\[
	(x+3)^2=16(y-2) \quad\Leftrightarrow\quad (x-(-3))^2=16(y-2)
	\]

	\textbf{Paso 3: Identificar los valores de \(h\), \(k\) y \(p\).}

	Comparando con \((x-h)^2=4p(y-k)\):
	\[
	\begin{aligned}
		h &= -3 \\
		k &= 2 \\
		4p &= 16 \quad\Rightarrow\quad p=\frac{16}{4}=4
	\end{aligned}
	\]

	Como \(\mathbf{p=4>0}\), la parábola \textcolor{red}{\textbf{abre hacia arriba}}.

	\bigskip

	\textbf{Paso 4: Determinar el vértice.}

	El vértice está en el punto \((h,k)\):
	\[
	\boxed{V(-3,2)}
	\]

	\textbf{Paso 5: Calcular el foco.}

	Para una parábola vertical con vértice en \((h,k)\), el foco está en \((h,k+p)\):
	\[
	F(h,k+p)=F(-3,2+4)=F(-3,6)
	\]
	\[
	\boxed{F(-3,6)}
	\]

	\emph{Interpretación:} El foco está 4 unidades arriba del vértice.

	\bigskip

	\textbf{Paso 6: Determinar la directriz.}

	Para una parábola vertical con vértice en \((h,k)\), la directriz es la recta \(y=k-p\):
	\[
	y=k-p=2-4=-2
	\]
	\[
	\boxed{y=-2}
	\]

	\emph{Interpretación:} La directriz es una recta horizontal que está 4 unidades debajo del vértice.

	\bigskip

	\textbf{Paso 7: Identificar el eje de simetría.}

	El eje de simetría es la recta vertical que pasa por el vértice:
	\[
	x=-3
	\]

	\bigskip

	\textbf{Resumen de elementos:}
	\begin{itemize}
		\item Vértice: \(V(-3,2)\)
		\item Foco: \(F(-3,6)\)
		\item Directriz: \(y=-2\)
		\item Eje de simetría: \(x=-3\)
		\item Orientación: Abre hacia arriba
		\item Parámetro: \(p=4\)
	\end{itemize}

	\subsection*{Gráfica del Ejercicio 3: Parábola \((x+3)^2=16(y-2)\)}

	\begin{center}
		\begin{tikzpicture}[scale=0.55]
			% límites
			\def\xmin{-12}\def\xmax{6}
			\def\ymin{-4}\def\ymax{14}

			% cuadrícula
			\draw[very thin,gray!50] (\xmin,\ymin) grid (\xmax,\ymax);

			% ejes
			\draw[-{Latex},thick] (\xmin,0)--(\xmax,0) node[right] {$x$};
			\draw[-{Latex},thick] (0,\ymin)--(0,\ymax) node[above] {$y$};

			% marcas numéricas en x
			\foreach \x in {-12,-11,...,6}
			\draw (\x,0)--(\x,0.12) node[below=6pt,scale=0.65]{\x};

			% marcas numéricas en y
			\foreach \y in {-4,-3,...,14}
			\draw (0,\y)--(0.12,\y) node[left=6pt,scale=0.65]{\y};

			% parábola (x+3)^2 = 16(y-2) => y = (x+3)^2/16 + 2
			\draw[thick,red,domain=-11:5,samples=100] plot (\x,{(\x+3)*(\x+3)/16+2});

			% vértice
			\filldraw[black] (-3,2) circle (3.5pt) node[blue,scale=1] at (-1.8,1.3) {$V(-3,2)$};
			%\node[blue,scale=1] at (-2,1.2) {$V(-3,2)$};

			% foco
			\filldraw[blue] (-3,6) circle (3.5pt) node[blue,scale=1] at (-4.7,6) {$F(-3,6)$};
			%node[right=3pt,scale=1] {$F(-3,6)$};

			% directriz
			\draw[dashed,blue,very thick] (\xmin,-2)--(\xmax,-2);
			\node[blue,scale=1] at (-9,-2.7) {Directriz: $y=-2$};

			% eje de simetría (línea punteada vertical)
			\draw[dotted,thick,green!40!black] (-3,\ymin)--(-3,\ymax);
			\node[green!40!black,scale=0.9,rotate=90] at (-2.2,11) {Eje: $x=-3$};

			% etiqueta de la parábola
			\node[red,scale=1] at (-5,8) {$(x+3)^2=16(y-2)$};

			% algunos puntos de la parábola para ilustrar
			\filldraw[red] (1,3) circle (2pt);
			\filldraw[red] (-7,3) circle (2pt);
			\node[red,scale=0.8] at (1.8,2.5) {$(1,3)$};
			\node[red,scale=0.8] at (-7.8,2.5) {$(-7,3)$};

			% distancia del vértice al foco
			\draw[<->,orange,very thick] (-2.4,2)--(-2.4,6);
			\node[orange,scale=0.9] at (-4,4) {$p=4$};

			% distancia del vértice a la directriz
			\draw[<->,purple,very thick] (-3.6,2)--(-3.6,-2);
			\node[purple,scale=0.9] at (-4.5,.5) {$p=4$};

		\end{tikzpicture}
	\end{center}

	\bigskip

	\textbf{Observación:} Nota que el vértice está desplazado del origen: 3 unidades a la izquierda (porque \(h=-3\)) y 2 unidades hacia arriba (porque \(k=2\)). La distancia del vértice al foco es igual a la distancia del vértice a la directriz, ambas iguales a \(p=4\).
	
	\vspace{3mm}

	%\newpage

	\subsection*{Solución del Ejercicio 4}

	\textbf{Enunciado:} Encuentra el vértice y el foco de la parábola \(y^2+6y-4x+1=0\) (pista: completa el cuadrado).

	\bigskip

	\textbf{Solución paso a paso:}

	\bigskip

	\textbf{Paso 1: Identificar el tipo de parábola.}

	La ecuación tiene el término \(y^2\) (la variable \(y\) está al cuadrado), por lo que es una \textcolor{blue}{\textbf{parábola horizontal}}.

	\bigskip

	\textbf{Paso 2: Reorganizar la ecuación.}

	Agrupamos los términos que contienen \(y\) en un lado y los demás en el otro:
	\[
	y^2+6y-4x+1=0
	\]
	\[
	y^2+6y=4x-1
	\]

	\textbf{Paso 3: Completar el cuadrado en \(y\).}

	Para completar el cuadrado, tomamos el coeficiente de \(y\) (que es 6), lo dividimos entre 2 y lo elevamos al cuadrado:
	\[
	\left(\frac{6}{2}\right)^2=3^2=9
	\]

	Sumamos 9 a ambos lados de la ecuación:
	\[
	y^2+6y+9=4x-1+9
	\]
	\[
	y^2+6y+9=4x+8
	\]

	\textbf{Paso 4: Factorizar el lado izquierdo.}

	El lado izquierdo es un trinomio cuadrado perfecto:
	\[
	(y+3)^2=4x+8
	\]

	\textbf{Paso 5: Factorizar el lado derecho.}

	Factorizamos el 4 en el lado derecho:
	\[
	(y+3)^2=4(x+2)
	\]

	Reescribimos para que coincida con la forma estándar:
	\[
	(y-(-3))^2=4(x-(-2))
	\]

	\textbf{Paso 6: Comparar con la forma estándar.}

	La forma estándar de una parábola horizontal con vértice en \((h,k)\) es:
	\[
	(y-k)^2=4p(x-h)
	\]

	Comparando con nuestra ecuación \((y-(-3))^2=4(x-(-2))\):
	\[
	\begin{aligned}
		h &= -2 \\
		k &= -3 \\
		4p &= 4 \quad\Rightarrow\quad p=1
	\end{aligned}
	\]

	Como \(\mathbf{p=1>0}\), la parábola \textcolor{blue}{\textbf{abre hacia la derecha}}.

	\bigskip

	\textbf{Paso 7: Determinar el vértice.}

	El vértice está en el punto \((h,k)\):
	\[
	\boxed{V(-2,-3)}
	\]

	\textbf{Paso 8: Calcular el foco.}

	Para una parábola horizontal con vértice en \((h,k)\), el foco está en \((h+p,k)\):
	\[
	F(h+p,k)=F(-2+1,-3)=F(-1,-3)
	\]
	\[
	\boxed{F(-1,-3)}
	\]

	\emph{Interpretación:} El foco está 1 unidad a la derecha del vértice.

	\bigskip

	\textbf{Paso 9: Determinar la directriz.}

	Para una parábola horizontal con vértice en \((h,k)\), la directriz es la recta \(x=h-p\):
	\[
	x=h-p=-2-1=-3
	\]
	\[
	\boxed{x=-3}
	\]

	\emph{Interpretación:} La directriz es una recta vertical que está 1 unidad a la izquierda del vértice.

	\bigskip

	\textbf{Paso 10: Identificar el eje de simetría.}

	El eje de simetría es la recta horizontal que pasa por el vértice:
	\[
	y=-3
	\]

	\bigskip

	\textbf{Resumen de elementos:}
	\begin{itemize}
		\item Ecuación en forma estándar: \((y+3)^2=4(x+2)\)
		\item Vértice: \(V(-2,-3)\)
		\item Foco: \(F(-1,-3)\)
		\item Directriz: \(x=-3\)
		\item Eje de simetría: \(y=-3\)
		\item Orientación: Abre hacia la derecha
		\item Parámetro: \(p=1\)
	\end{itemize}

	\subsection*{Gráfica del Ejercicio 4: Parábola \(y^2+6y-4x+1=0\)}

	\begin{center}
		\begin{tikzpicture}[scale=0.6]
			% límites
			\def\xmin{-5}\def\xmax{8}
			\def\ymin{-9}\def\ymax{3}

			% cuadrícula
			\draw[very thin,gray!50] (\xmin,\ymin) grid (\xmax,\ymax);

			% ejes
			\draw[-{Latex},thick] (\xmin,0)--(\xmax,0) node[right] {$x$};
			\draw[-{Latex},thick] (0,\ymin)--(0,\ymax) node[above] {$y$};

			% marcas numéricas en x
			\foreach \x in {-5,-4,...,8}
			\draw (\x,0)--(\x,0.12) node[below=6pt,scale=0.7]{\x};

			% marcas numéricas en y
			\foreach \y in {-9,-8,...,3}
			\draw (0,\y)--(0.12,\y) node[left=6pt,scale=0.7]{\y};

			% parábola (y+3)^2 = 4(x+2) => x = (y+3)^2/4 - 2
			\draw[thick,red,domain=-9:3,samples=100] plot ({(\x+3)*(\x+3)/4-2},\x);

			% vértice
			\filldraw[black] (-2,-3) circle (3.5pt) node[below left=2pt,scale=1] {$V(-2,-3)$};

			% foco
			\filldraw[blue] (-1,-3) circle (3.5pt) node[above=3pt,scale=1] {$F(-1,-3)$};

			% directriz
			\draw[dashed,blue,very thick] (-3,\ymin)--(-3,\ymax);
			\node[blue,scale=1] at (-3,2) {$x=-3$};

			% eje de simetría (línea punteada horizontal)
			\draw[dotted,thick,green!60!black] (\xmin,-3)--(\xmax,-3);
			\node[green!60!black,scale=0.9] at (4.5,-3.6) {Eje de Simetría: $y=-3$};

			% etiqueta de la parábola
			\node[red,scale=1] at (4,-1) {$y^2+6y-4x+1=0$};
			\node[red,scale=0.9] at (4,-1.7) {o bien $(y+3)^2=4(x+2)$};

			% algunos puntos de la parábola para ilustrar
			\filldraw[red] (-1,-1) circle (2pt);
			\filldraw[red] (-1,-5) circle (2pt);
			\node[red,scale=0.75] at (-0.3,-0.7) {$(-1,-1)$};
			\node[red,scale=0.75] at (-0.3,-5.3) {$(-1,-5)$};

			% distancia del vértice al foco
			\draw[<->,orange,very thick] (-2,-2.4)--(-1,-2.4);
			\node[orange,scale=0.9] at (-1.5,-1.7) {$p=1$};

			% distancia del vértice a la directriz
			\draw[<->,purple,very thick] (-2,-3.6)--(-3,-3.6);
			\node[purple,scale=0.9] at (-2.5,-4.3) {$p=1$};

		\end{tikzpicture}
	\end{center}

	\bigskip

	\textbf{Proceso de completar el cuadrado resumido:}
	\[
	\begin{aligned}
		y^2+6y-4x+1&=0 \\
		y^2+6y&=4x-1 \\
		y^2+6y+9&=4x-1+9 \quad\text{(sumamos 9)} \\
		(y+3)^2&=4x+8 \\
		(y+3)^2&=4(x+2) \quad\text{(forma estándar)}
	\end{aligned}
	\]

	\newpage

	\subsection*{Solución del Ejercicio 5}

	\textbf{Enunciado:} Dibuja la gráfica de la parábola \(x^2=-4y\) e indica el vértice, foco y directriz.

	\bigskip

	\textbf{Solución paso a paso:}

	\bigskip

	\textbf{Paso 1: Identificar el tipo de parábola.}

	La ecuación es \(x^2=-4y\). Como la variable \(x\) está elevada al cuadrado, se trata de una \textcolor{red}{\textbf{parábola vertical}}.

	\bigskip

	\textbf{Paso 2: Comparar con la forma estándar.}

	La forma estándar de una parábola vertical con vértice en el origen es:
	\[
	x^2=4py
	\]

	Comparamos nuestra ecuación:
	\[
	x^2=-4y \quad\Leftrightarrow\quad x^2=4py
	\]

	Por lo tanto:
	\[
	4p=-4
	\]

	\textbf{Paso 3: Calcular el valor de \(p\).}

	Despejamos \(p\):
	\[
	4p=-4 \quad\Rightarrow\quad p=\frac{-4}{4}=-1
	\]

	Como \(\mathbf{p=-1<0}\), la parábola \textcolor{red}{\textbf{abre hacia abajo}}.

	\bigskip

	\textbf{Paso 4: Determinar el vértice.}

	Como la ecuación es de la forma \(x^2=4py\) (sin desplazamiento), el vértice está en el origen:
	\[
	\boxed{V(0,0)}
	\]

	\textbf{Paso 5: Calcular el foco.}

	Para una parábola vertical con vértice en \((0,0)\), el foco está en \((0,p)\):
	\[
	F(0,p)=F(0,-1)
	\]
	\[
	\boxed{F(0,-1)}
	\]

	\emph{Interpretación:} El foco está 1 unidad debajo del vértice, sobre el eje \(y\).

	\bigskip

	\textbf{Paso 6: Determinar la directriz.}

	Para una parábola vertical con vértice en \((0,0)\), la directriz es la recta \(y=-p\):
	\[
	y=-p=-(-1)=1
	\]
	\[
	\boxed{y=1}
	\]

	\emph{Interpretación:} La directriz es una recta horizontal que está 1 unidad arriba del vértice.

	\bigskip

	\textbf{Paso 7: Identificar el eje de simetría.}

	El eje de simetría es la recta vertical que pasa por el vértice:
	\[
	x=0 \quad\text{(el eje \(y\))}
	\]

	\bigskip

	\textbf{Paso 8: Calcular algunos puntos de la parábola.}

	Para dibujar la parábola, calculamos algunos puntos. De la ecuación \(x^2=-4y\), despejamos \(y\):
	\[
	y=-\frac{x^2}{4}
	\]

	Algunos puntos:
	\begin{itemize}
		\item Si \(x=0\): \(y=-\frac{0^2}{4}=0\) \(\Rightarrow\) Punto \((0,0)\)
		\item Si \(x=2\): \(y=-\frac{2^2}{4}=-1\) \(\Rightarrow\) Punto \((2,-1)\)
		\item Si \(x=-2\): \(y=-\frac{(-2)^2}{4}=-1\) \(\Rightarrow\) Punto \((-2,-1)\)
		\item Si \(x=4\): \(y=-\frac{4^2}{4}=-4\) \(\Rightarrow\) Punto \((4,-4)\)
		\item Si \(x=-4\): \(y=-\frac{(-4)^2}{4}=-4\) \(\Rightarrow\) Punto \((-4,-4)\)
		\item Si \(x=6\): \(y=-\frac{6^2}{4}=-9\) \(\Rightarrow\) Punto \((6,-9)\)
		\item Si \(x=-6\): \(y=-\frac{(-6)^2}{4}=-9\) \(\Rightarrow\) Punto \((-6,-9)\)
	\end{itemize}

	\bigskip

	\textbf{Resumen de elementos:}
	\begin{itemize}
		\item Vértice: \(V(0,0)\)
		\item Foco: \(F(0,-1)\)
		\item Directriz: \(y=1\)
		\item Eje de simetría: \(x=0\)
		\item Orientación: Abre hacia abajo
		\item Parámetro: \(p=-1\)
	\end{itemize}

	\subsection*{Gráfica del Ejercicio 5: Parábola \(x^2=-4y\)}

	\begin{center}
		\begin{tikzpicture}[scale=0.6]
			% límites
			\def\xmin{-8}\def\xmax{8}
			\def\ymin{-10}\def\ymax{3}

			% cuadrícula
			\draw[very thin,gray!50] (\xmin,\ymin) grid (\xmax,\ymax);

			% ejes
			\draw[-{Latex},thick] (\xmin,0)--(\xmax,0) node[right] {$x$};
			\draw[-{Latex},thick] (0,\ymin)--(0,\ymax) node[above] {$y$};

			% marcas numéricas en x
			\foreach \x in {-8,-7,...,8}
			\draw (\x,0)--(\x,0.12) node[below=6pt,scale=0.7]{\x};

			% marcas numéricas en y
			\foreach \y in {-10,-9,...,3}
			\draw (0,\y)--(0.12,\y) node[left=6pt,scale=0.7]{\y};

			% parábola x^2 = -4y => y = -x^2/4
			\draw[thick,red,domain=-8:8,samples=100] plot (\x,{-\x*\x/4});

			% vértice
			\filldraw[black] (0,0) circle (3.5pt) node[above right=2pt,scale=1] {$V(0,0)$};

			% foco
			\filldraw[blue] (0,-1) circle (3.5pt) node[right=3pt,scale=1] {$F(0,-1)$};

			% directriz
			\draw[dashed,blue,very thick] (\xmin,1)--(\xmax,1);
			\node[blue,scale=1] at (-6,1.6) {Directriz: $y=1$};

			% eje de simetría (línea punteada vertical)
			\draw[dotted,thick,green!60!black] (0,\ymin)--(0,\ymax);
			\node[green!60!black,scale=0.9,rotate=90] at (0.7,-7) {Eje de simetría};

			% etiqueta de la parábola
			\node[red,scale=1.1] at (5,-2) {$x^2=-4y$};

			% puntos calculados
			\filldraw[red] (2,-1) circle (2.5pt);
			\filldraw[red] (-2,-1) circle (2.5pt);
			\filldraw[red] (4,-4) circle (2.5pt);
			\filldraw[red] (-4,-4) circle (2.5pt);
			\filldraw[red] (6,-9) circle (2.5pt);
			\filldraw[red] (-6,-9) circle (2.5pt);

			% etiquetas de puntos
			\node[red,scale=0.75] at (2.7,-0.5) {$(2,-1)$};
			\node[red,scale=0.75] at (-2.7,-0.5) {$(-2,-1)$};
			\node[red,scale=0.75] at (4.7,-3.5) {$(4,-4)$};
			\node[red,scale=0.75] at (-4.7,-3.5) {$(-4,-4)$};
			\node[red,scale=0.75] at (6.7,-8.5) {$(6,-9)$};
			\node[red,scale=0.75] at (-6.7,-8.5) {$(-6,-9)$};

			% distancia del vértice al foco
			\draw[<->,orange,very thick] (0.6,0)--(0.6,-1);
			\node[orange,scale=0.9] at (1.8,-0.5) {$|p|=1$};

			% distancia del vértice a la directriz
			\draw[<->,purple,very thick] (-0.6,0)--(-0.6,1);
			\node[purple,scale=0.9] at (-1.8,0.5) {$|p|=1$};

		\end{tikzpicture}
	\end{center}

	\bigskip

	\textbf{Tabla de puntos de la parábola:}

	\begin{center}
		\begin{tabular}{|c|c|}
			\hline
			\(x\) & \(y=-\dfrac{x^2}{4}\) \\
			\hline
			\(-6\) & \(-9\) \\
			\hline
			\(-4\) & \(-4\) \\
			\hline
			\(-2\) & \(-1\) \\
			\hline
			\(0\) & \(0\) \\
			\hline
			\(2\) & \(-1\) \\
			\hline
			\(4\) & \(-4\) \\
			\hline
			\(6\) & \(-9\) \\
			\hline
		\end{tabular}
	\end{center}

	\bigskip

	\textbf{Observación importante:} Como la parábola abre hacia abajo, el vértice \(V(0,0)\) es el \textbf{punto más alto} de la curva. El foco \(F(0,-1)\) está dentro de la parábola (debajo del vértice), y la directriz \(y=1\) está fuera de la parábola (arriba del vértice). Todos los puntos de la parábola están a la misma distancia del foco que de la directriz.

	\newpage

	\section{Ejercicios: De elementos a ecuación}

	En estos ejercicios te doy algunos elementos de la parábola y tú debes \textbf{encontrar los elementos faltantes} y luego \textbf{escribir la ecuación de la parábola}.

	\bigskip

	\textbf{Ejercicio 6.} Se tiene una parábola con vértice en \(V(0,0)\) y foco en \(F(0,5)\).
	\begin{itemize}
		\item[(a)] ¿La parábola es vertical u horizontal?
		\item[(b)] ¿Hacia dónde abre la parábola?
		\item[(c)] Encuentra el valor de \(p\).
		\item[(d)] Encuentra la directriz.
		\item[(e)] Escribe la ecuación de la parábola.
	\end{itemize}

	\bigskip

	\textbf{Ejercicio 7.} Se tiene una parábola con vértice en \(V(0,0)\) y directriz \(x=3\).
	\begin{itemize}
		\item[(a)] ¿La parábola es vertical u horizontal?
		\item[(b)] ¿Hacia dónde abre la parábola?
		\item[(c)] Encuentra el valor de \(p\).
		\item[(d)] Encuentra el foco.
		\item[(e)] Escribe la ecuación de la parábola.
	\end{itemize}

	\bigskip

	\textbf{Ejercicio 8.} Se tiene una parábola con vértice en \(V(2,-3)\), foco en \(F(2,1)\).
	\begin{itemize}
		\item[(a)] ¿La parábola es vertical u horizontal?
		\item[(b)] ¿Hacia dónde abre la parábola?
		\item[(c)] Encuentra el valor de \(p\).
		\item[(d)] Encuentra la directriz.
		\item[(e)] Escribe la ecuación de la parábola.
	\end{itemize}

	\bigskip

	\textbf{Ejercicio 9.} Se tiene una parábola con vértice en \(V(-1,4)\) y directriz \(y=6\).
	\begin{itemize}
		\item[(a)] ¿La parábola es vertical u horizontal?
		\item[(b)] ¿Hacia dónde abre la parábola?
		\item[(c)] Encuentra el valor de \(p\).
		\item[(d)] Encuentra el foco.
		\item[(e)] Escribe la ecuación de la parábola.
	\end{itemize}

	\bigskip

	\textbf{Ejercicio 10.} Se tiene una parábola con vértice en \(V(3,2)\), que pasa por el punto \(P(7,4)\) y abre hacia la derecha.
	\begin{itemize}
		\item[(a)] ¿La parábola es vertical u horizontal?
		\item[(b)] Usa el punto \(P\) para encontrar el valor de \(p\).
		\item[(c)] Encuentra el foco.
		\item[(d)] Encuentra la directriz.
		\item[(e)] Escribe la ecuación de la parábola.
	\end{itemize}

	\newpage

	\section{Soluciones: De elementos a ecuación}

	\subsection*{Solución del Ejercicio 6}

	\textbf{Datos:} Vértice \(V(0,0)\) y foco \(F(0,5)\).

	\bigskip

	\textbf{(a) ¿La parábola es vertical u horizontal?}

	Observamos que el vértice y el foco tienen la misma coordenada \(x\) (ambos tienen \(x=0\)), pero diferente coordenada \(y\) (\(y=0\) para el vértice y \(y=5\) para el foco). Esto significa que el foco está sobre el eje \(y\), por lo tanto:

	\[
	\boxed{\text{La parábola es \textbf{vertical}}}
	\]

	\textbf{(b) ¿Hacia dónde abre la parábola?}

	Como el foco \(F(0,5)\) está \emph{arriba} del vértice \(V(0,0)\) (porque \(5>0\)):

	\[
	\boxed{\text{La parábola abre \textbf{hacia arriba}}}
	\]

	\textbf{(c) Encuentra el valor de \(p\).}

	El parámetro \(p\) es la distancia del vértice al foco:
	\[
	p = y_{\text{foco}} - y_{\text{vértice}} = 5-0=5
	\]
	\[
	\boxed{p=5}
	\]

	\textbf{(d) Encuentra la directriz.}

	Para una parábola vertical con vértice en el origen, la directriz es \(y=-p\):
	\[
	y=-p=-5
	\]
	\[
	\boxed{\text{Directriz: } y=-5}
	\]

	\textbf{(e) Escribe la ecuación de la parábola.}

	Para una parábola vertical con vértice en \((0,0)\), la ecuación es \(x^2=4py\):
	\[
	x^2=4py=4(5)y=20y
	\]
	\[
	\boxed{x^2=20y}
	\]

	\subsection*{Gráfica del Ejercicio 6}

	\begin{center}
		\begin{tikzpicture}[scale=0.5]
			\def\xmin{-12}\def\xmax{12}
			\def\ymin{-7}\def\ymax{15}

			\draw[very thin,gray!50] (\xmin,\ymin) grid (\xmax,\ymax);
			\draw[-{Latex},thick] (\xmin,0)--(\xmax,0) node[right] {$x$};
			\draw[-{Latex},thick] (0,\ymin)--(0,\ymax) node[above] {$y$};

			\foreach \x in {-12,-10,...,12}
			\draw (\x,0)--(\x,0.12) node[below=6pt,scale=0.65]{\x};
			\foreach \y in {-7,-5,...,15}
			\draw (0,\y)--(0.12,\y) node[left=6pt,scale=0.65]{\y};

			\draw[thick,red,domain=-12:12,samples=100] plot (\x,{\x*\x/20});

			\filldraw[black] (0,0) circle (3.5pt) node[below right=2pt,scale=1] {$V(0,0)$};
			\filldraw[blue] (0,5) circle (3.5pt) node[right=3pt,scale=1] {$F(0,5)$};
			\draw[dashed,blue,very thick] (\xmin,-5)--(\xmax,-5);
			\node[blue,scale=1] at (-9,-5.7) {Directriz: $y=-5$};

			\draw[dotted,thick,green!60!black] (0,\ymin)--(0,\ymax);
			\node[green!60!black,scale=0.9,rotate=90] at (0.8,12) {Eje de simetría};

			\node[red,scale=1.1] at (8,8) {$x^2=20y$};

			\draw[<->,orange,very thick] (0.7,0)--(0.7,5);
			\node[orange,scale=0.9] at (2.5,2.5) {$p=5$};
		\end{tikzpicture}
	\end{center}

	\newpage

	\subsection*{Solución del Ejercicio 7}

	\textbf{Datos:} Vértice \(V(0,0)\) y directriz \(x=3\).

	\bigskip

	\textbf{(a) ¿La parábola es vertical u horizontal?}

	La directriz es una recta vertical (\(x=3\)). Para una parábola, la directriz es perpendicular al eje de simetría. Si la directriz es vertical, el eje de simetría es horizontal, por lo tanto:

	\[
	\boxed{\text{La parábola es \textbf{horizontal}}}
	\]

	\textbf{(b) ¿Hacia dónde abre la parábola?}

	La directriz está a la \emph{derecha} del vértice (porque \(x=3>0\)). Como la parábola abre en dirección \emph{opuesta} a la directriz:

	\[
	\boxed{\text{La parábola abre \textbf{hacia la izquierda}}}
	\]

	\textbf{(c) Encuentra el valor de \(p\).}

	Para una parábola horizontal con vértice en el origen, la directriz es \(x=-p\). En nuestro caso:
	\[
	x=-p=3 \quad\Rightarrow\quad -p=3 \quad\Rightarrow\quad p=-3
	\]
	\[
	\boxed{p=-3}
	\]

	El signo negativo confirma que la parábola abre hacia la izquierda.

	\textbf{(d) Encuentra el foco.}

	Para una parábola horizontal con vértice en \((0,0)\), el foco está en \((p,0)\):
	\[
	F(p,0)=F(-3,0)
	\]
	\[
	\boxed{\text{Foco: } F(-3,0)}
	\]

	\textbf{(e) Escribe la ecuación de la parábola.}

	Para una parábola horizontal con vértice en \((0,0)\), la ecuación es \(y^2=4px\):
	\[
	y^2=4px=4(-3)x=-12x
	\]
	\[
	\boxed{y^2=-12x}
	\]

	\subsection*{Gráfica del Ejercicio 7}

	\begin{center}
		\begin{tikzpicture}[scale=0.5]
			\def\xmin{-14}\def\xmax{6}
			\def\ymin{-12}\def\ymax{12}

			\draw[very thin,gray!50] (\xmin,\ymin) grid (\xmax,\ymax);
			\draw[-{Latex},thick] (\xmin,0)--(\xmax,0) node[right] {$x$};
			\draw[-{Latex},thick] (0,\ymin)--(0,\ymax) node[above] {$y$};

			\foreach \x in {-14,-12,...,6}
			\draw (\x,0)--(\x,0.12) node[below=6pt,scale=0.65]{\x};
			\foreach \y in {-12,-10,...,12}
			\draw (0,\y)--(0.12,\y) node[left=6pt,scale=0.65]{\y};

			\draw[thick,red,domain=-12:12,samples=100] plot ({-\x*\x/12},\x);

			\filldraw[black] (0,0) circle (3.5pt) node[above right=2pt,scale=1] {$V(0,0)$};
			\filldraw[blue] (-3,0) circle (3.5pt) node[above=3pt,scale=1] {$F(-3,0)$};
			\draw[dashed,blue,very thick] (3,\ymin)--(3,\ymax);
			\node[blue,scale=1] at (3,10) {Directriz: $x=3$};

			\draw[dotted,thick,green!60!black] (\xmin,0)--(\xmax,0);
			\node[green!60!black,scale=0.9] at (-10,0.8) {Eje de simetría};

			\node[red,scale=1.1] at (-9,6) {$y^2=-12x$};

			\draw[<->,orange,very thick] (0,-0.8)--(-3,-0.8);
			\node[orange,scale=0.9] at (-1.5,-2) {$|p|=3$};
		\end{tikzpicture}
	\end{center}

	\newpage

	\subsection*{Solución del Ejercicio 8}

	\textbf{Datos:} Vértice \(V(2,-3)\) y foco \(F(2,1)\).

	\bigskip

	\textbf{(a) ¿La parábola es vertical u horizontal?}

	El vértice y el foco tienen la misma coordenada \(x\) (ambos tienen \(x=2\)), pero diferente coordenada \(y\). El foco está sobre una recta vertical que pasa por el vértice, por lo tanto:

	\[
	\boxed{\text{La parábola es \textbf{vertical}}}
	\]

	\textbf{(b) ¿Hacia dónde abre la parábola?}

	Como el foco \(F(2,1)\) está \emph{arriba} del vértice \(V(2,-3)\) (porque \(1>-3\)):

	\[
	\boxed{\text{La parábola abre \textbf{hacia arriba}}}
	\]

	\textbf{(c) Encuentra el valor de \(p\).}

	El parámetro \(p\) es la distancia del vértice al foco:
	\[
	p = y_{\text{foco}} - y_{\text{vértice}} = 1-(-3)=1+3=4
	\]
	\[
	\boxed{p=4}
	\]

	\textbf{(d) Encuentra la directriz.}

	Para una parábola vertical con vértice en \((h,k)\), la directriz es \(y=k-p\):
	\[
	y=k-p=-3-4=-7
	\]
	\[
	\boxed{\text{Directriz: } y=-7}
	\]

	\textbf{(e) Escribe la ecuación de la parábola.}

	Para una parábola vertical con vértice en \((h,k)=(2,-3)\), la ecuación es \((x-h)^2=4p(y-k)\):
	\[
	(x-2)^2=4p(y-k)=4(4)(y-(-3))=16(y+3)
	\]
	\[
	\boxed{(x-2)^2=16(y+3)}
	\]

	\subsection*{Gráfica del Ejercicio 8}

	\begin{center}
		\begin{tikzpicture}[scale=0.5]
			\def\xmin{-6}\def\xmax{10}
			\def\ymin{-9}\def\ymax{8}

			\draw[very thin,gray!50] (\xmin,\ymin) grid (\xmax,\ymax);
			\draw[-{Latex},thick] (\xmin,0)--(\xmax,0) node[right] {$x$};
			\draw[-{Latex},thick] (0,\ymin)--(0,\ymax) node[above] {$y$};

			\foreach \x in {-6,-5,...,10}
			\draw (\x,0)--(\x,0.12) node[below=6pt,scale=0.65]{\x};
			\foreach \y in {-9,-8,...,8}
			\draw (0,\y)--(0.12,\y) node[left=6pt,scale=0.65]{\y};

			\draw[thick,red,domain=-4:8,samples=100] plot (\x,{(\x-2)*(\x-2)/16-3});

			\filldraw[black] (2,-3) circle (3.5pt) node[right=3pt,scale=1] {$V(2,-3)$};
			\filldraw[blue] (2,1) circle (3.5pt) node[right=3pt,scale=1] {$F(2,1)$};
			\draw[dashed,blue,very thick] (\xmin,-7)--(\xmax,-7);
			\node[blue,scale=1] at (-4,-7.7) {Directriz: $y=-7$};

			\draw[dotted,thick,green!60!black] (2,\ymin)--(2,\ymax);
			\node[green!60!black,scale=0.9,rotate=90] at (2.8,6) {Eje: $x=2$};

			\node[red,scale=1] at (6,3) {$(x-2)^2=16(y+3)$};

			\draw[<->,orange,very thick] (1.3,-3)--(1.3,1);
			\node[orange,scale=0.9] at (-0.5,-1) {$p=4$};
		\end{tikzpicture}
	\end{center}

	\newpage

	\subsection*{Solución del Ejercicio 9}

	\textbf{Datos:} Vértice \(V(-1,4)\) y directriz \(y=6\).

	\bigskip

	\textbf{(a) ¿La parábola es vertical u horizontal?}

	La directriz es una recta horizontal (\(y=6\)). Si la directriz es horizontal, el eje de simetría es vertical, por lo tanto:

	\[
	\boxed{\text{La parábola es \textbf{vertical}}}
	\]

	\textbf{(b) ¿Hacia dónde abre la parábola?}

	El vértice está en \(V(-1,4)\), con \(y=4\). La directriz está en \(y=6\), es decir, \emph{arriba} del vértice (porque \(6>4\)). Como la parábola abre en dirección \emph{opuesta} a la directriz:

	\[
	\boxed{\text{La parábola abre \textbf{hacia abajo}}}
	\]

	\textbf{(c) Encuentra el valor de \(p\).}

	Para una parábola vertical con vértice en \((h,k)\), la directriz es \(y=k-p\). En nuestro caso:
	\[
	y=k-p \quad\Rightarrow\quad 6=4-p \quad\Rightarrow\quad -p=6-4=2 \quad\Rightarrow\quad p=-2
	\]
	\[
	\boxed{p=-2}
	\]

	El signo negativo confirma que la parábola abre hacia abajo.

	\textbf{(d) Encuentra el foco.}

	Para una parábola vertical con vértice en \((h,k)\), el foco está en \((h,k+p)\):
	\[
	F(h,k+p)=F(-1,4+(-2))=F(-1,2)
	\]
	\[
	\boxed{\text{Foco: } F(-1,2)}
	\]

	\textbf{(e) Escribe la ecuación de la parábola.}

	Para una parábola vertical con vértice en \((h,k)=(-1,4)\), la ecuación es \((x-h)^2=4p(y-k)\):
	\[
	(x-(-1))^2=4p(y-4)=4(-2)(y-4)=-8(y-4)
	\]
	\[
	\boxed{(x+1)^2=-8(y-4)}
	\]

	\subsection*{Gráfica del Ejercicio 9}

	\begin{center}
		\begin{tikzpicture}[scale=0.55]
			\def\xmin{-8}\def\xmax{6}
			\def\ymin{-4}\def\ymax{8}

			\draw[very thin,gray!50] (\xmin,\ymin) grid (\xmax,\ymax);
			\draw[-{Latex},thick] (\xmin,0)--(\xmax,0) node[right] {$x$};
			\draw[-{Latex},thick] (0,\ymin)--(0,\ymax) node[above] {$y$};

			\foreach \x in {-8,-7,...,6}
			\draw (\x,0)--(\x,0.12) node[below=6pt,scale=0.65]{\x};
			\foreach \y in {-4,-3,...,8}
			\draw (0,\y)--(0.12,\y) node[left=6pt,scale=0.65]{\y};

			\draw[thick,red,domain=-7:5,samples=100] plot (\x,{-(\x+1)*(\x+1)/8+4});

			\filldraw[black] (-1,4) circle (3.5pt) node[right=3pt,scale=1] {$V(-1,4)$};
			\filldraw[blue] (-1,2) circle (3.5pt) node[right=3pt,scale=1] {$F(-1,2)$};
			\draw[dashed,blue,very thick] (\xmin,6)--(\xmax,6);
			\node[blue,scale=1] at (-6,6.7) {Directriz: $y=6$};

			\draw[dotted,thick,green!60!black] (-1,\ymin)--(-1,\ymax);
			\node[green!60!black,scale=0.9,rotate=90] at (-0.2,6.5) {Eje: $x=-1$};

			\node[red,scale=1] at (3,1) {$(x+1)^2=-8(y-4)$};

			\draw[<->,orange,very thick] (-0.3,4)--(-0.3,2);
			\node[orange,scale=0.9] at (1.2,3) {$|p|=2$};

			\draw[<->,purple,very thick] (-1.7,4)--(-1.7,6);
			\node[purple,scale=0.9] at (-3.2,5) {$|p|=2$};
		\end{tikzpicture}
	\end{center}

	\newpage

	\subsection*{Solución del Ejercicio 10}

	\textbf{Datos:} Vértice \(V(3,2)\), pasa por el punto \(P(7,4)\), y abre hacia la derecha.

	\bigskip

	\textbf{(a) ¿La parábola es vertical u horizontal?}

	Nos dicen que la parábola abre \emph{hacia la derecha}, lo cual es una orientación horizontal:

	\[
	\boxed{\text{La parábola es \textbf{horizontal}}}
	\]

	\textbf{(b) Usa el punto \(P\) para encontrar el valor de \(p\).}

	Para una parábola horizontal con vértice en \((h,k)=(3,2)\), la ecuación es:
	\[
	(y-k)^2=4p(x-h) \quad\Rightarrow\quad (y-2)^2=4p(x-3)
	\]

	Como el punto \(P(7,4)\) está en la parábola, debe satisfacer la ecuación. Sustituimos \(x=7\) y \(y=4\):
	\[
	(4-2)^2=4p(7-3)
	\]
	\[
	2^2=4p(4)
	\]
	\[
	4=16p
	\]
	\[
	p=\frac{4}{16}=\frac{1}{4}
	\]
	\[
	\boxed{p=\frac{1}{4}}
	\]

	Como \(p>0\), esto confirma que la parábola abre hacia la derecha.

	\textbf{(c) Encuentra el foco.}

	Para una parábola horizontal con vértice en \((h,k)\), el foco está en \((h+p,k)\):
	\[
	F(h+p,k)=F\left(3+\frac{1}{4},2\right)=F\left(\frac{13}{4},2\right)=F(3.25,2)
	\]
	\[
	\boxed{\text{Foco: } F\left(\frac{13}{4},2\right) \text{ o } F(3.25,2)}
	\]

	\textbf{(d) Encuentra la directriz.}

	Para una parábola horizontal con vértice en \((h,k)\), la directriz es \(x=h-p\):
	\[
	x=h-p=3-\frac{1}{4}=\frac{12}{4}-\frac{1}{4}=\frac{11}{4}=2.75
	\]
	\[
	\boxed{\text{Directriz: } x=\frac{11}{4} \text{ o } x=2.75}
	\]

	\textbf{(e) Escribe la ecuación de la parábola.}

	Para una parábola horizontal con vértice en \((h,k)=(3,2)\), la ecuación es \((y-k)^2=4p(x-h)\):
	\[
	(y-2)^2=4p(x-3)=4\left(\frac{1}{4}\right)(x-3)=(x-3)
	\]
	\[
	\boxed{(y-2)^2=x-3}
	\]

	También podemos escribirla como: \((y-2)^2=1(x-3)\)

	\subsection*{Gráfica del Ejercicio 10}

	\begin{center}
		\begin{tikzpicture}[scale=0.7]
			\def\xmin{1}\def\xmax{9}
			\def\ymin{-2}\def\ymax{6}

			\draw[very thin,gray!50] (\xmin,\ymin) grid (\xmax,\ymax);
			\draw[-{Latex},thick] (\xmin,0)--(\xmax,0) node[right] {$x$};
			\draw[-{Latex},thick] (0,\ymin)--(0,\ymax) node[above] {$y$};

			\foreach \x in {1,2,...,9}
			\draw (\x,0)--(\x,0.12) node[below=6pt,scale=0.7]{\x};
			\foreach \y in {-2,-1,...,6}
			\draw (0,\y)--(0.12,\y) node[left=6pt,scale=0.7]{\y};

			\draw[thick,red,domain=-2:6,samples=100] plot ({(\x-2)*(\x-2)+3},\x);

			\filldraw[black] (3,2) circle (3.5pt) node[below left=2pt,scale=1] {$V(3,2)$};
			\filldraw[blue] (3.25,2) circle (3.5pt) node[above=3pt,scale=0.9] {$F(3.25,2)$};
			\filldraw[red] (7,4) circle (3pt) node[above right=2pt,scale=1] {$P(7,4)$};

			\draw[dashed,blue,very thick] (2.75,\ymin)--(2.75,\ymax);
			\node[blue,scale=0.9,rotate=90] at (2.75,4.5) {Directriz: $x=2.75$};

			\draw[dotted,thick,green!60!black] (\xmin,2)--(\xmax,2);
			\node[green!60!black,scale=0.9] at (7,2.6) {Eje: $y=2$};

			\node[red,scale=1] at (5.5,0) {$(y-2)^2=x-3$};

			\draw[<->,orange,very thick] (3,1.4)--(3.25,1.4);
			\node[orange,scale=0.8] at (3.1,0.8) {$p=\frac{1}{4}$};
		\end{tikzpicture}
	\end{center}

	\bigskip

	\textbf{Observación:} Este ejercicio es especial porque usamos un punto de la parábola para calcular \(p\). Sustituimos las coordenadas del punto en la ecuación general y resolvemos para \(p\).

	\bigskip
	\bigskip

	\noindent
	\textbf{Nota final:} Esta guía cubre los conceptos fundamentales de la parábola en geometría analítica. Recuerda que la clave para entender la parábola es identificar cuál variable está al cuadrado y el signo del parámetro \(p\). ¡Con práctica dominarás este tema, Sheyra!

\end{document}
