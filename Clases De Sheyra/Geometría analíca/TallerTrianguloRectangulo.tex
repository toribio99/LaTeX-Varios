% !TEX TS-program = pdflatex
% !TEX encoding = UTF-8

\documentclass[12pt,a4paper]{article}

\usepackage[utf8]{inputenc}
\usepackage[T1]{fontenc}
\usepackage[spanish,shorthands=off]{babel} % <-- evita conflicto con < y >
\usepackage{amsmath,amssymb}
\usepackage{geometry}
\geometry{margin=2.5cm}
\usepackage{tikz}
\usetikzlibrary{calc,arrows.meta}

\usetikzlibrary{calc,arrows.meta} % <-- para -{Latex}

\title{\Large Taller de Geometría Analítica
	
\small{Elaborado para : \textsc{\bf{Sheyra Celedón}}}}
	
\author{\bf{\textsc{Toribio de J Arrieta F}}}
\date{\today}

\begin{document}
	\maketitle
	
\section{Comprobar que el triángulo es rectángulo}
	
	\textbf{Objetivo.} Comprobar que el triángulo planteado en el plano cartesiano con vértices
	\[
	A(3,-6),\quad B(8,-2),\quad C(-1,-1)
	\]
	es un triángulo rectángulo. 
	
	\vspace{5mm}
	
	\textbf{Solución} Para iniciar la solución de este ejercicio usaremos al teorema de Pitágoras
	
	
	\bigskip
	
	\textbf{Teorema de Pitágoras.}
	En \emph{todo triángulo rectángulo}, \textcolor{red}{\textbf{la suma de los cuadrados de las longitudes de los catetos}} {\textbf{(\underline{o sea los dos lados más cortos del triángulo})} \textcolor{red}{\textbf{  es igual al cuadrado de la longitud de la hipotenusa}}:
	\[
	a^2+b^2=c^2.
	\]
	\emph{¿Cuándo se aplica?}
	\begin{itemize}
		\item Cuando sabemos o sospechamos que hay un ángulo de \(90^\circ\).
		\item Para calcular un lado conociendo los otros dos, \textcolor{red}{ \underline{\textbf{o para verificar si un triángulo es}}  \underline{\textbf{rectángulo}}}: si para tres lados \(a,b,c\) (con \(c\) el mayor) se cumple \(a^2+b^2=c^2\), entonces el triángulo es rectángulo y \(c\) es la hipotenusa.
	\end{itemize}
	
	Entonces el plan es: calcular las distancias de cada lado del triángulo y comprobar si cuando elevo al cuadrado la distancia de dos de sus lados y al sumar los dos resultados me da un valor igual al cuadrado del tercer lado elevado al cuadrado
	
	\bigskip
	
	Recordemos que la \textbf{Fórmula de distancia en el plano.}
	Para dos puntos dice que si tengo dos puntos  \(P(x_1,y_1)\) y \(Q(x_2,y_2)\), la distancia entre ellos se calcula mediante :
	\[
	d(P,Q)=\sqrt{(x_2-x_1)^2+(y_2-y_1)^2}.
	\]
	
	\bigskip
	
	\textbf{Calculemos las longitudes de los diferentes lados.}
	
	\vspace{3mm}
	
	\underline{Lado \(AB\)} (entre \(A(3,-6)\) y \(B(8,-2)\)):
	\[
	\begin{aligned}
		AB &= \sqrt{(8-3)^2+(-2-(-6))^2}
		= \sqrt{5^2+4^2}
		= \sqrt{25+16}
		= \sqrt{41}\approx 6.40.
	\end{aligned}
	\]
	
	\underline{Lado \(AC\)} (entre \(A(3,-6)\) y \(C(-1,-1)\)):
	\[
	\begin{aligned}
		AC &= \sqrt{(-1-3)^2+(-1-(-6))^2}
		= \sqrt{(-4)^2+5^2}
		= \sqrt{16+25}
		= \sqrt{41}\approx 6.40.
	\end{aligned}
	\]
	
	\underline{Lado \(BC\)} (entre \(B(8,-2)\) y \(C(-1,-1)\)):
	\[
	\begin{aligned}
		BC &= \sqrt{(8-(-1))^2+(-2-(-1))^2}
		= \sqrt{9^2+(-1)^2}
		= \sqrt{81+1}
		= \sqrt{82}\approx 9.06.
	\end{aligned}
	\]
	
	\bigskip
	
	\textbf{Ahora apliquemos el Teorema de Pitágoras con los datos obtenidos.}
	
	Como \(BC\) es el lado más largo, lo comparamos con \(AB\) y \(AC\):
	\[
	AB^2+AC^2=(\sqrt{41})^2+(\sqrt{41})^2=41+41=82=(\sqrt{82})^2=BC^2.
	\]
	Entoces, como se cumple que: \(AB^2+AC^2=BC^2\). Por lo tanto, el triángulo es \textbf{rectángulo} y además el ángulo recto está en el vértice \(\mathbf{A}\); la hipotenusa es \(\overline{BC}\).
	
	\bigskip
	
	\textbf{Gráfica en el plano cartesiano.}
	
	\begin{center}
		\begin{tikzpicture}[scale=0.65]
			% límites
			\def\xmin{-3}\def\xmax{10}
			\def\ymin{-8}\def\ymax{4}
			
			% cuadrícula
			\draw[very thin,gray!50] (\xmin,\ymin) grid (\xmax,\ymax);
			
			% ejes (sin usar '->', usamos -{Latex})
			\draw[-{Latex},thick] (\xmin,0)--(\xmax,0) node[right] {$x$};
			\draw[-{Latex},thick] (0,\ymin)--(0,\ymax) node[above] {$y$};
			
			% marcas numéricas en x
			\foreach \x in {-3,-2,...,10}
			\draw (\x,0) -- (\x,0.12) node[below=6pt,scale=0.8] {\x};
			
			% marcas numéricas en y
			\foreach \y in {-8,-7,...,4}
			\draw (0,\y) -- (0.12,\y) node[left=6pt,scale=0.8] {\y};
			
			% puntos
			\coordinate (A) at (3,-6);
			\coordinate (B) at (8,-2);
			\coordinate (C) at (-1,-1);
			
			% triángulo
			\draw[thick] (A)--(B)--(C)--cycle;
			\filldraw[black] (A) circle (2pt) node[below right] {$A(3,-6)$};
			\filldraw[black] (B) circle (2pt) node[above right] {$B(8,-2)$};
			\filldraw[black] (C) circle (2pt) node[below left] {$C(-1,-1)$};
			
			% cuadradito del ángulo recto en A
			\path (A) coordinate (Aa);
			\path ($(A)+(0.47,0.375)$) coordinate (Pab);
			\path ($(A)+(-0.375,0.47)$) coordinate (Pac);
			\draw[thick] (Pab) -- ($(Pab)+(-0.375,0.47)$) -- (Pac);
			
			% etiquetas de lados
			\node at ($(A)!0.55!(B)$) [below right,scale=0.9] {$AB=\sqrt{41}$};
			\node at ($(A)!0.55!(C)$) [below left,scale=0.9] {$AC=\sqrt{41}$};
			\node at ($(B)!0.55!(C)$) [above,scale=0.9] {$BC=\sqrt{82}$};
		\end{tikzpicture}
	\end{center}
	

		
		\section{¿Qué es el punto medio y cómo se calcula?}
		Dado un segmento que une dos puntos \(P(x_1,y_1)\) y \(Q(x_2,y_2)\), el
		\textbf{punto medio} \(M\) es el punto exactamente a la misma distancia de \(P\) y de \(Q\).
		Su fórmula es:
		\[
		\boxed{\,M\left(\dfrac{x_1+x_2}{2},\ \dfrac{y_1+y_2}{2}\right)\,}
		\]
		
		\textbf{¿Cuándo se aplica?}
		\begin{itemize}
			\item \textbf{Para encontrar el centro de un segmento} cuando se conocen sus extremos.
			\item \textbf{Para verificar} si dos segmentos se \emph{bisen} (por ejemplo, diagonales de un paralelogramo).
			\item \textbf{Para construir figuras simétricas} o ubicar centros en problemas de geometría analítica.
		\end{itemize}
		
		\section*{Datos que obtenemos del problema}
		Cuadrilátero con vértices:
		\[
		A(0,0),\quad B(0,4),\quad C(3,5),\quad D(3,-1).
		\]
		Buscamos los puntos medios de las diagonales \(\overline{AC}\) y \(\overline{BD}\).
		
		\section*{Calculemos el punto medio de \(\overline{AC}\) mediante la fórmula que haz recibido del profesor de matemáticas}
		Aplicamos la fórmula del punto medio a \(A(0,0)\) y \(C(3,5)\):
		\[
		\begin{aligned}
			M_{AC} \;=\;
			\left(\frac{x_A+x_C}{2},\ \frac{y_A+y_C}{2}\right)
			&=\left(\frac{0+3}{2},\ \frac{0+5}{2}\right) \\
			&=\left(\frac{3}{2},\ \frac{5}{2}\right)
			=(1.5,\ 2.5).
		\end{aligned}
		\]
		
		\section*{Calculemos ahora el punto medio de \(\overline{BD}\) usando nuevamente la fórmula del punto medio}
		Aplicamos la fórmula del punto medio a \(B(0,4)\) y \(D(3,-1)\):
		\[
		\begin{aligned}
			M_{BD} \;=\;
			\left(\frac{x_B+x_D}{2},\ \frac{y_B+y_D}{2}\right)
			&=\left(\frac{0+3}{2},\ \frac{4+(-1)}{2}\right) \\
			&=\left(\frac{3}{2},\ \frac{3}{2}\right)
			=(1.5,\ 1.5).
		\end{aligned}
		\]
		\emph{Nota.} Sumar un número negativo: \(4+(-1)=3\).
		
		\section*{Ahora te presento en el plano cartesiano la figura}
		\begin{center}
			\begin{tikzpicture}[scale=0.8]
				% límites del dibujo
				\def\xmin{-1}\def\xmax{5}
				\def\ymin{-2}\def\ymax{6}
				
				% cuadrícula
				\draw[very thin,gray!50] (\xmin,\ymin) grid (\xmax,\ymax);
				
				% ejes con puntas tipo LaTeX (evita usar -> para no chocar con babel)
				\draw[-{Latex},thick] (\xmin,0)--(\xmax,0) node[right] {$x$};
				\draw[-{Latex},thick] (0,\ymin)--(0,\ymax) node[above] {$y$};
				
				% marcas numéricas en x
				\foreach \x in {-1,0,1,2,3,4,5}
				\draw (\x,0) -- (\x,0.12) node[below=6pt,scale=0.8] {\x};
				
				% marcas numéricas en y
				\foreach \y in {-2,-1,0,1,2,3,4,5,6}
				\draw (0,\y) -- (0.12,\y) node[left=6pt,scale=0.8] {\y};
				
				% puntos del cuadrilátero
				\coordinate (A) at (0,0);
				\coordinate (B) at (0,4);
				\coordinate (C) at (3,5);
				\coordinate (D) at (3,-1);
				
				% diagonales y contorno
				\draw[thick] (A)--(B)--(C)--(D)--cycle;
				\draw[dashed,thick] (A)--(C);
				\draw[dashed,thick] (B)--(D);
				
				% puntos medios
				\coordinate (MAC) at (1.5,2.5);
				\coordinate (MBD) at (1.5,1.5);
				
				% dibujar puntos y etiquetas
				\filldraw (A) circle (2pt) node[below left] {$A(0,0)$};
				\filldraw (B) circle (2pt) node[above left] {$B(0,4)$};
				\filldraw (C) circle (2pt) node[above right] {$C(3,5)$};
				\filldraw (D) circle (2pt) node[below right] {$D(3,-1)$};
				
				\filldraw[black] (MAC) circle (2.2pt) node[above right] {$M_{AC}(1.5,\,2.5)$};
				\filldraw[black] (MBD) circle (2.2pt) node[below right] {$M_{BD}(1.5,\,1.5)$};
				
			\end{tikzpicture}
		\end{center}
		
		\bigskip
		\noindent
		\textbf{Conclusión.} Los puntos medios de las diagonales son
		\[
		\boxed{M_{AC}=(1.5,\,2.5)} \qquad \text{y} \qquad \boxed{M_{BD}=(1.5,\,1.5)}.
		\]
		En la gráfica se observa que \textbf{ambos quedan centrados sobre las diagonales} correspondientes.
		
			
			\section{Definición de triángulo isósceles}
			Un \textbf{triángulo isósceles} es un triángulo que tiene \textbf{dos lados de igual longitud}.
			Los ángulos opuestos a esos lados también son iguales. El lado que \emph{no} es igual se
			llama \emph{base} y el vértice común a los dos lados iguales se llama \emph{vértice del isósceles}.
			
			\subsection*{Ejemplos visuales}
			\begin{center}
				\begin{tikzpicture}[scale=0.9]
					% Ejemplo 1
					\begin{scope}[shift={(0,0)}]
						\coordinate (A) at (0,0);
						\coordinate (B) at (4,0);
						\coordinate (C) at (2,3);
						\draw[thick] (A)--(B)--(C)--cycle;
						% marquitas de igualdad en AC y BC
						\draw ($(A)!0.5!(C)$) +(-0.12,0.12) -- +(0.12,-0.12);
						\draw ($(B)!0.5!(C)$) +(-0.12,0.12) -- +(0.12,-0.12);
						\node at (2,-0.65) {Isósceles con base $\overline{AB}$};
					\end{scope}
					% Ejemplo 2
					\begin{scope}[shift={(7,0)}]
						\coordinate (D) at (0,0);
						\coordinate (E) at (3,2.5);
						\coordinate (F) at (6,0);
						\draw[thick] (D)--(E)--(F)--cycle;
						% marquitas de igualdad en DE y EF
						\draw ($(D)!0.5!(E)$) +(-0.12,-0.12) -- +(0.12,0.12);
						\draw ($(E)!0.5!(F)$) +(-0.12,-0.12) -- +(0.12,0.12);
						\node at (3,-0.65) {Isósceles con base $\overline{DF}$};
					\end{scope}
				\end{tikzpicture}
			\end{center}
			
			\section*{Datos que obtenemos del problema}
			Puntos: \(X(-7,2),\ Y(3,-4),\ Z(1,4)\).\\
			\textbf{Objetivo}: Se quiere determinar si el triángulo \(XYZ\) es isósceles comprobando las longitudes de sus lados.
			
			\section*{Fórmula de la distancia (recordatorio)}
			Para dos puntos \(P(x_1,y_1)\) y \(Q(x_2,y_2)\),
			\[
			\boxed{\,d(P,Q)=\sqrt{(x_2-x_1)^2+(y_2-y_1)^2}\,}
			\]
			
			\section*{Cálculo de cada lado}
			\subsection*{Lado \(\mathbf{XY}\)}
			\textbf{Fórmula:}
			\(d(X,Y)=\sqrt{(x_Y-x_X)^2+(y_Y-y_X)^2}\).
			\[
			\begin{aligned}
				XY &= \sqrt{(3-(-7))^2+(-4-2)^2}
				= \sqrt{10^2+(-6)^2}
				= \sqrt{100+36}
				= \sqrt{136}\approx 11.66.
			\end{aligned}
			\]
			
			\subsection*{Lado \(\mathbf{XZ}\)}
			\textbf{Fórmula:}
			\(d(X,Z)=\sqrt{(x_Z-x_X)^2+(y_Z-y_X)^2}\).
			\[
			\begin{aligned}
				XZ &= \sqrt{(1-(-7))^2+(4-2)^2}
				= \sqrt{8^2+2^2}
				= \sqrt{64+4}
				= \sqrt{68}\approx 8.25.
			\end{aligned}
			\]
			
			\subsection*{Lado \(\mathbf{YZ}\)}
			\textbf{Fórmula:}
			\(d(Y,Z)=\sqrt{(x_Z-x_Y)^2+(y_Z-y_Y)^2}\).
			\[
			\begin{aligned}
				YZ &= \sqrt{(1-3)^2+(4-(-4))^2}
				= \sqrt{(-2)^2+8^2}
				= \sqrt{4+64}
				= \sqrt{68}\approx 8.25.
			\end{aligned}
			\]
			
			\section*{Conclusión}
			Observamos que \(XZ = YZ = \sqrt{68}\) y \(XY=\sqrt{136}\).
			Como \(\,XZ=YZ\), el triángulo \(XYZ\) es \textbf{isósceles} con vértice en \(\mathbf{Z}\) y base \(\overline{XY}\).
			
			\section*{Gráfica en el plano cartesiano}
			\begin{center}
				\begin{tikzpicture}[scale=0.7]
					% límites
					\def\xmin{-8}\def\xmax{5}
					\def\ymin{-5}\def\ymax{5}
					
					% cuadrícula
					\draw[very thin,gray!50] (\xmin,\ymin) grid (\xmax,\ymax);
					
					% ejes
					\draw[-{Latex},thick] (\xmin,0)--(\xmax,0) node[right] {$x$};
					\draw[-{Latex},thick] (0,\ymin)--(0,\ymax) node[above] {$y$};
					
					% marcas numéricas
					\foreach \x in {-8,-7,...,5}
					\draw (\x,0)--(\x,0.12) node[below=6pt,scale=0.8]{\x};
					\foreach \y in {-5,-4,...,5}
					\draw (0,\y)--(0.12,\y) node[left=6pt,scale=0.8]{\y};
					
					% puntos
					\coordinate (X) at (-7,2);
					\coordinate (Y) at (3,-4);
					\coordinate (Z) at (1,4);
					
					% triángulo y resaltado de lados iguales
					\draw[thick] (X)--(Y)--(Z)--cycle;
					\draw[very thick] (X)--(Z);
					\draw[very thick] (Y)--(Z);
					
					% puntos visibles
					\filldraw (X) circle (2pt) node[above left] {$X(-7,2)$};
					\filldraw (Y) circle (2pt) node[below right] {$Y(3,-4)$};
					\filldraw (Z) circle (2pt) node[above right] {$Z(1,4)$};
					
					% etiquetas de longitudes (opcional)
					\node[scale=0.9] at ($(X)!0.5!(Z)$) [above] {$\sqrt{68}$};
					\node[scale=0.9] at ($(Y)!0.5!(Z)$) [right] {$\sqrt{68}$};
					\node[scale=0.9] at ($(X)!0.5!(Y)$) [below] {$\sqrt{136}$};
				\end{tikzpicture}
			\end{center}
			
			
			\section{Cálculo de la distancia de un punto a una recta}
			 \textbf{La distancia del punto} \(P(x_0,y_0)\)  \textbf{a una recta} dada en forma general
			\[
			Ax+By+C=0
			\]
			se calcula con
			\[
			\boxed{\,d(P,\ell)=\dfrac{|Ax_0+By_0+C|}{\sqrt{A^2+B^2}}\,}
			\]
			El valor absoluto garantiza que la distancia sea \emph{positiva}.\\[2mm]
			\textbf{¿Cuándo se aplica?} Siempre que conozcamos un punto y una recta (en forma \(Ax+By+C=0\)) y queramos la \textbf{\emph{distancia perpendicular desde el punto hasta la recta}}.
			
			\section*{Datos que obtenemos del ejercicio}
			Punto \(P(-6,5)\) y recta \(\ell:\ 2x+y-1=0\).\\
			Aquí \(A=2,\ B=1,\ C=-1,\ x_0=-6,\ y_0=5\).
			
			\section*{Cálculo de la distancia}
			\textbf{Fórmula a usar:}
			\[
			d(P,\ell)=\dfrac{|Ax_0+By_0+C|}{\sqrt{A^2+B^2}}.
			\]
			
			\textbf{Reemplazo de datos:}  \textcolor{red}{Noté que en tus apuntes tienes un error de signo y es por que al final no colocas el valor absoluto, por eso a tí te da valor negativo y acá te lo presento con valor positivo.}
			\[
			\begin{aligned}
				d(P,\ell)
				&=\dfrac{|\,2(-6)+1(5)-1\,|}{\sqrt{2^2+1^2}}
				=\dfrac{|\,{-12}+5-1\,|}{\sqrt{4+1}}
				=\dfrac{|\,{-8}\,|}{\sqrt{5}}
				=\dfrac{8}{\sqrt{5}}.
			\end{aligned}
			\]
			
			\textbf{(Opcional) Racionalización del denominador:}
			\[
			\dfrac{8}{\sqrt{5}}=\dfrac{8\sqrt{5}}{5}.
			\]
			
			\section*{Interpretación geométrica o representación gráfica}
			\subsection*{Esta parte es para que te hagas una idea o entiendas con un dibujo lo que se está calculando o buscando con tanto vericueto}
			La distancia que acabamos de encontrar corresponde al segmento perpendicular desde \(P\) hasta la recta.
			Si llamamos \(Q\) al pie de la perpendicular, o sea el punto de la recta desde donde se mide la distancia al punto, entonces \(PQ\perp \ell\) y \(PQ=d(P,\ell)\). \textbf{Te sugiero que con la ayuda de una escuadra dibujes el segmento de recta que va desde el punto \(Q\)  hasta la recta \(\ell\) para que te quede más claro que el agua} te la dejo punteada y de color rojo para que la coloques de forma continua.
			
			\section*{Gráfica en el plano cartesiano}
			\begin{center}
				\begin{tikzpicture}[scale=0.6, transform shape]
					% Límites
					\def\xmin{-8}\def\xmax{6}
					\def\ymin{-6}\def\ymax{18}
					
					% Cuadrícula y ejes
					\draw[very thin,gray!50] (\xmin,\ymin) grid (\xmax,\ymax);
					\draw[-{Latex},thick] (\xmin,0)--(\xmax,0) node[right] {$x$};
					\draw[-{Latex},thick] (0,\ymin)--(0,\ymax) node[above] {$y$};
					
					% Marcas numéricas (opcional)
					\foreach \x in {-8,-7,...,6} \draw (\x,0)--(\x,0.12) node[below=6pt,scale=0.8]{\x};
					\foreach \y in {-6,-4,...,18} \draw (0,\y)--(0.12,\y) node[left=6pt,scale=0.8]{\y};
					
					% Punto P
					\coordinate (P) at (-6,5);
					\filldraw (P) circle (2pt) node[above left] {$P(-6,5)$};
					
					% Recta l: 2x + y - 1 = 0 (y = -2x + 1) recortada a la cuadrícula:
					% Intersecciones con el borde: (-8,17) y (3.5,-6)
					\draw[thick,blue] (-8,17) -- (3.5,-6);
					\node[blue,fill=white,inner sep=1pt] at (-4,9) {$\ell:\ 2x+y-1=0$};
					
					% Pie de la perpendicular Q
					\coordinate (Q) at (-14/5, 33/5);
					
					% Segmento perpendicular (distancia)
					\draw[dashed,very thick,red] (P)--(Q);
					
					% Punto Q
					\filldraw[red] (Q) circle (2pt) node[above right] {$Q$};
					
					% Etiqueta: a la derecha de la recta y alineada con la perpendicular (vector (2,1))
					\pgfmathsetmacro{\ang}{atan2(0,9)} % ≈ 26.565°
					\def\off{2.3}
					\node[red!70!black,fill=white,inner sep=2pt,rotate=\ang]
					at ($ (Q) + (-\off*2, -\off*1) $)
					{$d=\dfrac{8}{\sqrt{5}}=\dfrac{8\sqrt{5}}{5}\approx 3.58$};
					
					% Cuadrito de ángulo recto en Q
					\def\s{0.25}
					\draw[red,thick]
					($ (Q) + ({2*\s},{1*\s}) $) --
					($ (Q) + ({3*\s},{-1*\s}) $) --
					($ (Q) + ({\s},{-2*\s}) $);
				\end{tikzpicture}
			\end{center}
			
			\bigskip
			\noindent
			\textbf{Resultado final:}\quad
			\[
			\boxed{\,d\big(P(-6,5),\ 2x+y-1=0\big)=\dfrac{8}{\sqrt{5}}=\dfrac{8\sqrt{5}}{5}\,}.
			\]
			
			
	
	
\end{document}
