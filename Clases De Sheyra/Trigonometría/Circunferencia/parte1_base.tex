% PARTE 1: ESTRUCTURA BASE DE LA GUÍA
% Subagente 1 - Estructura Base
% Tema: Circunferencia - Grado 10

\documentclass[12pt,a4paper,twoside]{article}
\usepackage{fontspec}
\usepackage[spanish,es-nodecimaldot]{babel}
\usepackage{amsmath,amssymb,amsthm}
\usepackage[margin=2cm]{geometry}
\usepackage{tikz,pgfplots}
\usetikzlibrary{calc,arrows.meta,babel,patterns,shapes.geometric,intersections,through}
\usepackage{tcolorbox}
\tcbuselibrary{skins,breakable}
\usepackage{fancyhdr}
\usepackage{graphicx}
\usepackage{hyperref}
\usepackage{enumitem}
\usepackage{multicol}
\usepackage{array}
\usepackage{booktabs}
\pgfplotsset{compat=1.18}

% Configuración de hyperref
\hypersetup{
    colorlinks=true,
    linkcolor=ColorPrincipal,
    urlcolor=ColorPrincipal,
    citecolor=ColorPrincipal
}

% Definición de colores
\definecolor{ColorPrincipal}{RGB}{0,70,173}
\definecolor{ColorAcento}{RGB}{255,127,0}
\definecolor{ColorTerciario}{RGB}{0,150,0}
\definecolor{ColorFondo}{RGB}{240,240,255}

% Entornos personalizados
\newtcolorbox{definicion}[1][]{
  colback=ColorPrincipal!5,
  colframe=ColorPrincipal,
  fonttitle=\bfseries,
  title=Definición,
  sharp corners,
  boxrule=1pt,
  #1
}

\newtcolorbox{teorema}[1][]{
  colback=ColorTerciario!5,
  colframe=ColorTerciario,
  fonttitle=\bfseries,
  title=Teorema,
  sharp corners,
  boxrule=1pt,
  #1
}

\newtcolorbox{ejemplo}[1][]{
  colback=ColorAcento!5,
  colframe=ColorAcento,
  fonttitle=\bfseries,
  title=Ejemplo,
  breakable,
  sharp corners,
  boxrule=1pt,
  #1
}

\newtcolorbox{ejercicio}[1][]{
  colback=ColorAcento!5,
  colframe=ColorAcento,
  fonttitle=\bfseries,
  title=Ejercicio,
  breakable,
  sharp corners,
  boxrule=1pt,
  #1
}

\newtcolorbox{solucion}[1][]{
  colback=ColorTerciario!5,
  colframe=ColorTerciario,
  fonttitle=\bfseries,
  title=Solución,
  breakable,
  sharp corners,
  boxrule=1pt,
  #1
}

\newtcolorbox{nota}[1][]{
  colback=yellow!10,
  colframe=yellow!80!black,
  fonttitle=\bfseries,
  title=Nota,
  sharp corners,
  boxrule=1pt,
  #1
}

\newtcolorbox{importante}[1][]{
  colback=red!5,
  colframe=red!60!black,
  fonttitle=\bfseries,
  title=¡Importante!,
  sharp corners,
  boxrule=1pt,
  #1
}

% Configuración de headers y footers
\pagestyle{fancy}
\fancyhf{}
\fancyhead[LE,RO]{\thepage}
\fancyhead[LO]{\nouppercase{\rightmark}}
\fancyhead[RE]{\nouppercase{\leftmark}}
\fancyfoot[C]{\small Geometría Analítica - Circunferencia}
\renewcommand{\headrulewidth}{0.4pt}
\renewcommand{\footrulewidth}{0.4pt}

% Comandos personalizados
\newcommand{\R}{\mathbb{R}}
\newcommand{\circunf}[3]{(x-#1)^2 + (y-#2)^2 = #3^2}

\begin{document}

% ===========================================
% PORTADA
% ===========================================

\begin{titlepage}
\centering
\vspace*{1cm}

{\Huge\bfseries GEOMETRÍA ANALÍTICA\par}
\vspace{1.5cm}
{\LARGE\color{ColorPrincipal} CIRCUNFERENCIA\par}
\vspace{2cm}

\begin{tikzpicture}[scale=1.5]
  \begin{axis}[
    axis equal image,
    grid=major,
    grid style={gray!30},
    xmin=-4, xmax=4,
    ymin=-4, ymax=4,
    xlabel={$x$},
    ylabel={$y$},
    axis lines=middle,
    width=0.5\textwidth,
    height=0.5\textwidth,
  ]
  % Circunferencia principal
  \addplot[ColorPrincipal, thick, samples=100, domain=0:360]
    ({3*cos(x)}, {3*sin(x)});
  % Centro
  \addplot[mark=*, ColorAcento, mark size=3pt] coordinates {(0,0)};
  % Radio
  \addplot[ColorAcento, thick, dashed] coordinates {(0,0) (3,0)};
  \node at (axis cs:1.5,0.3) {$r$};
  \node at (axis cs:0,-0.5) {Centro};
  \end{axis}
\end{tikzpicture}

\vspace{2cm}
{\Large\bfseries Prof: Toribio De J Arrieta F\par}
\vspace{1cm}
{\large La Pruebita\par}
\vspace{0.5cm}
{\large Grado 10 -- Trigonometría\par}
\vfill
{\large \today\par}
\end{titlepage}

% Tabla de contenidos
\tableofcontents
\newpage

% ===========================================
% INTRODUCCIÓN
% ===========================================

\section{Introducción}

¡Bienvenido al fascinante mundo de las circunferencias! Si alguna vez has observado una rueda girando, has visto el sol en el horizonte o has admirado los anillos de Saturno, entonces ya has tenido un encuentro cercano con una de las figuras más perfectas y fundamentales de las matemáticas: la circunferencia.

\subsection{¿Por qué estudiar la circunferencia?}

Imagina por un momento un mundo sin ruedas. ¿Cómo sería? Sin bicicletas, sin autos, sin engranajes... La vida sería completamente diferente. La circunferencia no es solo una figura geométrica más; es una de las formas más importantes en nuestra vida cotidiana y en el desarrollo de la humanidad.

Desde los antiguos griegos que estudiaron sus propiedades místicas, hasta los ingenieros modernos que diseñan sistemas de radar y satélites, la circunferencia ha sido y sigue siendo fundamental en el desarrollo del conocimiento humano.

\subsection{La circunferencia en tu vida diaria}

Mira a tu alrededor en este momento. ¿Cuántas circunferencias puedes identificar? Están por todas partes:

\begin{itemize}[leftmargin=2cm]
  \item \textbf{En el transporte:} Las ruedas de bicicletas, autos, trenes y aviones dependen de la perfección de la circunferencia para funcionar eficientemente.

  \item \textbf{En la tecnología:} Los sistemas de radar utilizan circunferencias concéntricas para detectar objetos. Los satélites orbitan siguiendo trayectorias circulares alrededor de la Tierra.

  \item \textbf{En la arquitectura:} Desde los arcos romanos hasta los modernos domos de los estadios, la circunferencia proporciona belleza y estabilidad estructural.

  \item \textbf{En la ingeniería:} Los engranajes, poleas y sistemas de transmisión dependen de circunferencias perfectas para transmitir movimiento y fuerza.

  \item \textbf{En el diseño:} Logos, interfaces de usuario, señales de tráfico... La circunferencia comunica completitud, perfección y armonía.

  \item \textbf{En la naturaleza:} Los anillos de los árboles, las ondas en el agua, las pupilas de nuestros ojos... La naturaleza ama las formas circulares.
\end{itemize}

\subsection{¿Qué aprenderás en esta guía?}

En esta aventura matemática, vamos a descubrir los secretos de la circunferencia desde una perspectiva analítica. No solo dibujaremos círculos bonitos (aunque también lo haremos), sino que aprenderemos a describirlos con ecuaciones precisas y a resolver problemas complejos usando álgebra.

Específicamente, exploraremos:

\begin{enumerate}
  \item \textbf{La ecuación canónica de la circunferencia:} Aprenderás a escribir la ecuación de cualquier circunferencia conociendo solo su centro y radio. Es como tener la receta secreta para crear círculos perfectos matemáticamente.

  \item \textbf{La ecuación general de la circunferencia:} Descubrirás cómo identificar si una ecuación aparentemente complicada representa una circunferencia y cómo extraer información útil de ella.

  \item \textbf{Posiciones relativas entre rectas y circunferencias:} ¿Cuándo una recta toca, corta o pasa de largo a una circunferencia? Aprenderás a determinarlo algebraicamente, sin necesidad de dibujar.

  \item \textbf{Posiciones relativas entre circunferencias:} Dos circunferencias pueden relacionarse de maneras fascinantes. Pueden ser como hermanas que se abrazan, amigas que se tocan o conocidas que mantienen su distancia.

  \item \textbf{Aplicaciones prácticas:} Resolverás problemas reales donde las circunferencias son protagonistas, desde el diseño de rotondas hasta el cálculo de coberturas de antenas.
\end{enumerate}

\subsection{Conexión con tus conocimientos previos}

Esta guía no parte de cero. Se construye sobre los cimientos que ya has establecido:

\begin{itemize}
  \item Si recuerdas el \textbf{teorema de Pitágoras}, estás a medio camino de entender la ecuación de la circunferencia.

  \item Si dominas los \textbf{productos notables} y sabes \textbf{completar cuadrados}, podrás transformar ecuaciones como un mago matemático.

  \item Si entiendes el \textbf{plano cartesiano} y las \textbf{coordenadas}, ya tienes el escenario listo para ubicar circunferencias.

  \item Si has trabajado con \textbf{funciones y gráficas}, esta será una extensión natural de esos conceptos.
\end{itemize}

\subsection{Cómo aprovechar al máximo esta guía}

Para que tu aprendizaje sea exitoso, te recomiendo:

\begin{nota}
  \textbf{Consejos para el éxito:}
  \begin{itemize}
    \item Lee cada sección con calma, no hay prisa. Las matemáticas se disfrutan mejor cuando se saborean.
    \item Ten papel y lápiz a mano. Dibuja, calcula, experimenta. Las matemáticas se aprenden haciendo.
    \item No te saltes los ejemplos. Son como los entrenamientos antes del partido.
    \item Si algo no queda claro, vuelve a leerlo. A veces la segunda lectura revela detalles que se escaparon en la primera.
    \item Relaciona cada concepto con algo de tu vida diaria. Las matemáticas cobran vida cuando las conectas con tu realidad.
  \end{itemize}
\end{nota}

\subsection{Un viaje que comienza}

Estás a punto de embarcarte en un viaje fascinante donde las ecuaciones cobran forma y las figuras se convierten en álgebra. La circunferencia, esa figura perfecta que ha cautivado a la humanidad durante milenios, está a punto de revelarte sus secretos matemáticos.

¿Estás listo? ¡Comencemos esta aventura circular!

\newpage

% ===========================================
% CONCEPTOS FUNDAMENTALES
% ===========================================

\section{Conceptos Fundamentales}

\subsection{Definición de Circunferencia}

Antes de sumergirnos en las ecuaciones, necesitamos entender qué es exactamente una circunferencia. Puede parecer obvio, pero su definición matemática es elegante y precisa.

\begin{definicion}[title={Circunferencia}]
Una \textbf{circunferencia} es el lugar geométrico de todos los puntos del plano que equidistan de un punto fijo llamado \textbf{centro}. La distancia constante desde cualquier punto de la circunferencia hasta el centro se llama \textbf{radio}.

En términos más simples: si tomas un punto fijo (el centro) y marcas todos los puntos que están a la misma distancia de él, obtienes una circunferencia perfecta.
\end{definicion}

Esta definición aparentemente simple esconde una profunda verdad matemática: la circunferencia es la figura más simétrica posible en el plano. No importa desde qué dirección la mires, siempre se ve igual.

\begin{center}
\begin{tikzpicture}
  \begin{axis}[
    axis equal image,
    grid=major,
    grid style={gray!30},
    xmin=-5, xmax=5,
    ymin=-5, ymax=5,
    xlabel={$x$},
    ylabel={$y$},
    axis lines=middle,
    width=0.85\textwidth,
    height=0.85\textwidth,
    xtick={-4,-3,-2,-1,0,1,2,3,4},
    ytick={-4,-3,-2,-1,0,1,2,3,4},
  ]
  % Circunferencia
  \addplot[ColorPrincipal, ultra thick, samples=100, domain=0:360]
    ({2 + 3*cos(x)}, {1 + 3*sin(x)});

  % Centro
  \addplot[mark=*, ColorAcento, mark size=4pt] coordinates {(2,1)};
  \node[above right] at (axis cs:2,1) {Centro $(h,k)$};

  % Varios radios
  \foreach \angulo in {0, 45, 90, 135, 180, 225, 270, 315} {
    \addplot[ColorTerciario, dashed, thin]
      coordinates {(2,1) ({2 + 3*cos(\angulo)}, {1 + 3*sin(\angulo)})};
  }

  % Radio etiquetado
  \addplot[ColorAcento, thick] coordinates {(2,1) (5,1)};
  \node[above] at (axis cs:3.5,1) {$r$ (radio)};

  % Puntos sobre la circunferencia
  \addplot[mark=*, blue, mark size=2pt] coordinates {(5,1)};
  \node[right] at (axis cs:5,1) {$P_1$};

  \addplot[mark=*, blue, mark size=2pt] coordinates {(2,4)};
  \node[above] at (axis cs:2,4) {$P_2$};

  \addplot[mark=*, blue, mark size=2pt] coordinates {(-1,1)};
  \node[left] at (axis cs:-1,1) {$P_3$};

  \end{axis}
\end{tikzpicture}
\end{center}

Observa cómo todos los puntos $P_1$, $P_2$, $P_3$, etc., están exactamente a la misma distancia del centro. Esta es la esencia de la circunferencia.

\subsection{Ecuación Canónica de la Circunferencia}

Ahora viene la parte emocionante: ¿cómo expresamos matemáticamente esta idea de "todos los puntos a la misma distancia del centro"? Aquí es donde el álgebra y la geometría se dan la mano.

\begin{teorema}[title={Ecuación Canónica}]
Una circunferencia con centro en el punto $(h, k)$ y radio $r$ tiene como ecuación:
\[
(x - h)^2 + (y - k)^2 = r^2
\]
Esta es la \textbf{ecuación canónica} o \textbf{forma estándar} de la circunferencia.
\end{teorema}

\subsubsection{¿De dónde viene esta ecuación?}

La belleza de esta ecuación es que surge naturalmente de la definición. Veamos:

Si tienes un punto cualquiera $P(x, y)$ sobre la circunferencia y el centro está en $C(h, k)$, entonces la distancia entre $P$ y $C$ debe ser igual al radio $r$.

Usando la fórmula de distancia entre dos puntos (que viene del teorema de Pitágoras):
\[
\text{distancia} = \sqrt{(x - h)^2 + (y - k)^2} = r
\]

Elevando al cuadrado ambos lados:
\[
(x - h)^2 + (y - k)^2 = r^2
\]

¡Y ahí está! La ecuación canónica de la circunferencia.

\subsubsection{Interpretación de los parámetros}

Cada parte de la ecuación tiene un significado específico:

\begin{itemize}
  \item \textbf{$h$ y $k$:} Son las coordenadas del centro. Si $h > 0$, el centro está a la derecha del origen; si $h < 0$, está a la izquierda. Lo mismo aplica para $k$ con arriba y abajo.

  \item \textbf{$r$:} Es el radio, siempre positivo. Determina el tamaño de la circunferencia.

  \item \textbf{$(x - h)$ y $(y - k)$:} Representan las distancias horizontal y vertical desde cualquier punto de la circunferencia hasta el centro.
\end{itemize}

\subsubsection{Caso especial: Centro en el origen}

Cuando el centro está en el origen $(0, 0)$, la ecuación se simplifica maravillosamente:

\begin{importante}
Si el centro está en $(0, 0)$, entonces $h = 0$ y $k = 0$, por lo que la ecuación se convierte en:
\[
x^2 + y^2 = r^2
\]
Esta es la forma más simple de la ecuación de una circunferencia.
\end{importante}

Veamos algunos ejemplos visuales:

\begin{center}
\begin{tikzpicture}
  \begin{axis}[
    axis equal image,
    grid=major,
    grid style={gray!30},
    xmin=-6, xmax=6,
    ymin=-6, ymax=6,
    xlabel={$x$},
    ylabel={$y$},
    axis lines=middle,
    width=0.9\textwidth,
    height=0.9\textwidth,
    legend pos=outer north east,
  ]

  % Circunferencia con centro en origen
  \addplot[ColorPrincipal, ultra thick, samples=100, domain=0:360]
    ({4*cos(x)}, {4*sin(x)});
  \addlegendentry{$x^2 + y^2 = 16$ (centro en origen)}

  % Circunferencia desplazada
  \addplot[ColorAcento, ultra thick, samples=100, domain=0:360]
    ({3 + 2*cos(x)}, {2 + 2*sin(x)});
  \addlegendentry{$(x-3)^2 + (y-2)^2 = 4$}

  % Circunferencia pequeña
  \addplot[ColorTerciario, ultra thick, samples=100, domain=0:360]
    ({-2 + 1.5*cos(x)}, {-1 + 1.5*sin(x)});
  \addlegendentry{$(x+2)^2 + (y+1)^2 = 2.25$}

  % Centros
  \addplot[mark=*, black, mark size=3pt] coordinates {(0,0)};
  \addplot[mark=*, black, mark size=3pt] coordinates {(3,2)};
  \addplot[mark=*, black, mark size=3pt] coordinates {(-2,-1)};

  \end{axis}
\end{tikzpicture}
\end{center}

\subsection{Ecuación General de la Circunferencia}

En la vida real (y en los exámenes), no siempre te dan la ecuación en su forma bonita y ordenada. A veces aparece expandida y mezclada. Esta forma se llama ecuación general.

\begin{definicion}[title={Ecuación General}]
La \textbf{ecuación general de la circunferencia} tiene la forma:
\[
x^2 + y^2 + Dx + Ey + F = 0
\]
donde $D$, $E$ y $F$ son constantes reales.
\end{definicion}

\subsubsection{¿Cómo pasamos de la forma canónica a la general?}

Partiendo de $(x - h)^2 + (y - k)^2 = r^2$, si expandimos los binomios:

\begin{align}
(x - h)^2 + (y - k)^2 &= r^2 \\
x^2 - 2hx + h^2 + y^2 - 2ky + k^2 &= r^2 \\
x^2 + y^2 - 2hx - 2ky + h^2 + k^2 - r^2 &= 0
\end{align}

Comparando con $x^2 + y^2 + Dx + Ey + F = 0$, obtenemos:
\begin{itemize}
  \item $D = -2h$
  \item $E = -2k$
  \item $F = h^2 + k^2 - r^2$
\end{itemize}

\subsubsection{¿Toda ecuación de esta forma representa una circunferencia?}

¡Esta es una pregunta crucial! No toda ecuación de la forma $x^2 + y^2 + Dx + Ey + F = 0$ representa una circunferencia real.

\begin{teorema}[title={Condición de existencia}]
La ecuación $x^2 + y^2 + Dx + Ey + F = 0$ representa una circunferencia real si y solo si:
\[
D^2 + E^2 - 4F > 0
\]

En este caso:
\begin{itemize}
  \item Centro: $\left(-\frac{D}{2}, -\frac{E}{2}\right)$
  \item Radio: $r = \frac{1}{2}\sqrt{D^2 + E^2 - 4F}$
\end{itemize}
\end{teorema}

¿Qué pasa si $D^2 + E^2 - 4F \leq 0$?

\begin{itemize}
  \item Si $D^2 + E^2 - 4F = 0$: La ecuación representa un punto (circunferencia degenerada de radio 0).
  \item Si $D^2 + E^2 - 4F < 0$: La ecuación no tiene solución real (circunferencia imaginaria).
\end{itemize}

\subsubsection{Completando cuadrados: La técnica mágica}

Para convertir de la forma general a la canónica, usamos la técnica de completar cuadrados. Es como reorganizar un rompecabezas hasta que las piezas encajan perfectamente.

\textbf{Proceso paso a paso:}

Dada la ecuación $x^2 + y^2 + Dx + Ey + F = 0$:

\begin{enumerate}
  \item Agrupa los términos en $x$ y los términos en $y$:
  \[
  (x^2 + Dx) + (y^2 + Ey) + F = 0
  \]

  \item Para completar el cuadrado en $x$:
  \begin{itemize}
    \item Toma el coeficiente de $x$ (que es $D$)
    \item Divide entre 2: $\frac{D}{2}$
    \item Eleva al cuadrado: $\left(\frac{D}{2}\right)^2$
    \item Suma y resta este valor
  \end{itemize}

  \item Haz lo mismo para $y$ con el coeficiente $E$

  \item Reorganiza para obtener la forma canónica
\end{enumerate}

\textbf{Ejemplo ilustrativo:}

Convirtamos $x^2 + y^2 - 6x + 4y - 3 = 0$ a forma canónica:

\begin{align}
x^2 + y^2 - 6x + 4y - 3 &= 0 \\
(x^2 - 6x) + (y^2 + 4y) &= 3 \\
(x^2 - 6x + 9 - 9) + (y^2 + 4y + 4 - 4) &= 3 \\
(x^2 - 6x + 9) + (y^2 + 4y + 4) - 9 - 4 &= 3 \\
(x - 3)^2 + (y + 2)^2 &= 3 + 9 + 4 \\
(x - 3)^2 + (y + 2)^2 &= 16
\end{align}

Por lo tanto: Centro en $(3, -2)$ y radio $r = 4$.

\subsection{Posiciones Relativas de una Recta y una Circunferencia}

Cuando una recta y una circunferencia se encuentran en el mismo plano, pueden relacionarse de tres maneras distintas. Es como cuando lanzas una piedra hacia un lago circular: puede caer fuera, rozar el borde o atravesarlo.

\begin{definicion}[title={Posiciones relativas recta-circunferencia}]
Una recta $L$ y una circunferencia $C$ pueden estar en una de tres posiciones:
\begin{enumerate}
  \item \textbf{Recta exterior:} No tienen puntos en común (0 intersecciones)
  \item \textbf{Recta tangente:} Se tocan en exactamente un punto (1 intersección)
  \item \textbf{Recta secante:} Se cortan en dos puntos (2 intersecciones)
\end{enumerate}
\end{definicion}

\subsubsection{Método algebraico: El discriminante}

Para determinar la posición relativa, sustituimos la ecuación de la recta en la ecuación de la circunferencia. Esto nos da una ecuación cuadrática cuyo discriminante nos dice todo:

\begin{teorema}[title={Criterio del discriminante}]
Sea una circunferencia $(x - h)^2 + (y - k)^2 = r^2$ y una recta $y = mx + b$ (o $x = c$ si es vertical).

Al sustituir la recta en la circunferencia, obtenemos una ecuación cuadrática $At^2 + Bt + C = 0$.

Si $\Delta = B^2 - 4AC$, entonces:
\begin{itemize}
  \item $\Delta > 0$: La recta es secante (2 puntos de intersección)
  \item $\Delta = 0$: La recta es tangente (1 punto de tangencia)
  \item $\Delta < 0$: La recta es exterior (0 puntos de intersección)
\end{itemize}
\end{teorema}

\subsubsection{Método geométrico: Distancia del centro a la recta}

Hay un método más elegante usando la distancia del centro a la recta:

\begin{teorema}[title={Criterio de la distancia}]
Si $d$ es la distancia del centro de la circunferencia a la recta, y $r$ es el radio:
\begin{itemize}
  \item $d < r$: La recta es secante
  \item $d = r$: La recta es tangente
  \item $d > r$: La recta es exterior
\end{itemize}
\end{teorema}

La fórmula de distancia de un punto $(x_0, y_0)$ a una recta $Ax + By + C = 0$ es:
\[
d = \frac{|Ax_0 + By_0 + C|}{\sqrt{A^2 + B^2}}
\]

\begin{center}
\begin{tikzpicture}
  \begin{axis}[
    axis equal image,
    grid=major,
    grid style={gray!30},
    xmin=-6, xmax=6,
    ymin=-6, ymax=6,
    xlabel={$x$},
    ylabel={$y$},
    axis lines=middle,
    width=0.95\textwidth,
    height=0.95\textwidth,
    legend pos=outer north east,
  ]

  % Circunferencia
  \addplot[black, ultra thick, samples=100, domain=0:360]
    ({3*cos(x)}, {3*sin(x)});

  % Centro
  \addplot[mark=*, black, mark size=3pt] coordinates {(0,0)};
  \node[below left] at (axis cs:0,0) {Centro};

  % Recta secante
  \addplot[ColorPrincipal, ultra thick, domain=-6:6] {0.5*x + 1};
  \addlegendentry{Recta secante}

  % Puntos de intersección secante
  \addplot[mark=*, ColorPrincipal, mark size=4pt] coordinates {(-0.69,0.655)};
  \addplot[mark=*, ColorPrincipal, mark size=4pt] coordinates {(2.69,2.345)};

  % Recta tangente
  \addplot[ColorAcento, ultra thick, domain=-6:6] {-0.75*x + 3};
  \addlegendentry{Recta tangente}

  % Punto de tangencia
  \addplot[mark=*, ColorAcento, mark size=4pt] coordinates {(1.44,1.92)};

  % Recta exterior
  \addplot[ColorTerciario, ultra thick, domain=-6:6] {0.3*x + 4.5};
  \addlegendentry{Recta exterior}

  \end{axis}
\end{tikzpicture}
\end{center}

\subsubsection{Ecuación de la recta tangente}

Un caso especial importante es encontrar la ecuación de una recta tangente a una circunferencia en un punto dado.

\begin{teorema}[title={Recta tangente en un punto}]
La recta tangente a la circunferencia $(x - h)^2 + (y - k)^2 = r^2$ en el punto $(x_0, y_0)$ tiene ecuación:
\[
(x_0 - h)(x - h) + (y_0 - k)(y - k) = r^2
\]
\end{teorema}

Esta fórmula es hermosa porque muestra la simetría: simplemente "distribuyes" las variables entre los dos factores.

\subsection{Posiciones Relativas de Dos Circunferencias}

Cuando dos circunferencias comparten el mismo plano, pueden relacionarse de seis maneras diferentes. Es fascinante cómo dos figuras tan simples pueden crear patrones tan variados.

\begin{definicion}[title={Posiciones relativas entre circunferencias}]
Sean dos circunferencias $C_1$ y $C_2$ con centros $O_1$ y $O_2$, radios $r_1$ y $r_2$, y sea $d$ la distancia entre sus centros. Las posiciones posibles son:

\begin{enumerate}
  \item \textbf{Exteriores:} No se tocan ($d > r_1 + r_2$)
  \item \textbf{Tangentes exteriormente:} Se tocan en un punto por fuera ($d = r_1 + r_2$)
  \item \textbf{Secantes:} Se cortan en dos puntos ($|r_1 - r_2| < d < r_1 + r_2$)
  \item \textbf{Tangentes interiormente:} Se tocan en un punto por dentro ($d = |r_1 - r_2|$, $d \neq 0$)
  \item \textbf{Interiores:} Una dentro de la otra sin tocarse ($0 < d < |r_1 - r_2|$)
  \item \textbf{Concéntricas:} Mismo centro ($d = 0$)
\end{enumerate}
\end{definicion}

\subsubsection{Criterio de clasificación}

La clave está en comparar la distancia entre centros con la suma y diferencia de radios:

\begin{teorema}[title={Clasificación por distancia}]
Dadas dos circunferencias con radios $r_1$, $r_2$ (con $r_1 \geq r_2$) y distancia entre centros $d$:

\begin{center}
\begin{tabular}{lc}
\toprule
\textbf{Posición} & \textbf{Condición} \\
\midrule
Exteriores & $d > r_1 + r_2$ \\
Tangentes exteriormente & $d = r_1 + r_2$ \\
Secantes & $r_1 - r_2 < d < r_1 + r_2$ \\
Tangentes interiormente & $d = r_1 - r_2$ (y $d > 0$) \\
Interiores & $0 < d < r_1 - r_2$ \\
Concéntricas & $d = 0$ \\
\bottomrule
\end{tabular}
\end{center}
\end{teorema}

\begin{center}
\begin{tikzpicture}[scale=0.8]
  % Exteriores
  \begin{scope}[shift={(0,0)}]
    \draw[ColorPrincipal, ultra thick] (0,0) circle (1);
    \draw[ColorAcento, ultra thick] (3,0) circle (0.8);
    \node at (1.5,-2) {Exteriores};
    \fill (0,0) circle (2pt);
    \fill (3,0) circle (2pt);
    \draw[dashed] (0,0) -- (3,0);
  \end{scope}

  % Tangentes exteriormente
  \begin{scope}[shift={(6,0)}]
    \draw[ColorPrincipal, ultra thick] (0,0) circle (1);
    \draw[ColorAcento, ultra thick] (1.8,0) circle (0.8);
    \node at (0.9,-2) {Tang. Ext.};
    \fill (0,0) circle (2pt);
    \fill (1.8,0) circle (2pt);
    \draw[dashed] (0,0) -- (1.8,0);
    \fill[red] (1,0) circle (3pt);
  \end{scope}

  % Secantes
  \begin{scope}[shift={(11,0)}]
    \draw[ColorPrincipal, ultra thick] (0,0) circle (1.2);
    \draw[ColorAcento, ultra thick] (1.5,0) circle (1);
    \node at (0.75,-2) {Secantes};
    \fill (0,0) circle (2pt);
    \fill (1.5,0) circle (2pt);
    \draw[dashed] (0,0) -- (1.5,0);
    \fill[red] (0.65,0.95) circle (3pt);
    \fill[red] (0.65,-0.95) circle (3pt);
  \end{scope}

  % Tangentes interiormente
  \begin{scope}[shift={(0,-5)}]
    \draw[ColorPrincipal, ultra thick] (0,0) circle (1.5);
    \draw[ColorAcento, ultra thick] (0.7,0) circle (0.8);
    \node at (0.35,-2.5) {Tang. Int.};
    \fill (0,0) circle (2pt);
    \fill (0.7,0) circle (2pt);
    \draw[dashed] (0,0) -- (0.7,0);
    \fill[red] (1.5,0) circle (3pt);
  \end{scope}

  % Interiores
  \begin{scope}[shift={(6,-5)}]
    \draw[ColorPrincipal, ultra thick] (0,0) circle (1.5);
    \draw[ColorAcento, ultra thick] (0.3,0) circle (0.6);
    \node at (0.15,-2.5) {Interiores};
    \fill (0,0) circle (2pt);
    \fill (0.3,0) circle (2pt);
    \draw[dashed] (0,0) -- (0.3,0);
  \end{scope}

  % Concéntricas
  \begin{scope}[shift={(11,-5)}]
    \draw[ColorPrincipal, ultra thick] (0,0) circle (1.5);
    \draw[ColorAcento, ultra thick] (0,0) circle (0.8);
    \node at (0,-2.5) {Concéntricas};
    \fill (0,0) circle (2pt);
  \end{scope}
\end{tikzpicture}
\end{center}

\subsubsection{Puntos de intersección}

Cuando dos circunferencias se cortan (caso secante), podemos encontrar sus puntos de intersección resolviendo el sistema de ecuaciones:

\begin{align}
(x - h_1)^2 + (y - k_1)^2 &= r_1^2 \\
(x - h_2)^2 + (y - k_2)^2 &= r_2^2
\end{align}

El truco está en restar las ecuaciones para obtener una ecuación lineal (el eje radical), y luego sustituir en cualquiera de las ecuaciones originales.

\subsubsection{Aplicaciones prácticas}

Las posiciones relativas de circunferencias tienen aplicaciones fascinantes:

\begin{itemize}
  \item \textbf{Engranajes:} Dos ruedas dentadas son circunferencias tangentes exteriormente.
  \item \textbf{Poleas y correas:} Circunferencias exteriores conectadas por tangentes comunes.
  \item \textbf{Cobertura de antenas:} Áreas de señal representadas por circunferencias que pueden solaparse.
  \item \textbf{Diseño de logos:} Muchos logos famosos usan circunferencias en diferentes posiciones.
  \item \textbf{Órbitas planetarias:} Aunque son elipses, se aproximan a circunferencias en muchos casos.
\end{itemize}

\subsection{Tabla Resumen de Fórmulas}

Para tener todas las herramientas a mano, aquí está tu caja de herramientas completa:

\begin{center}
\begin{tcolorbox}[colback=ColorFondo,colframe=ColorPrincipal,title={\textbf{Formulario de la Circunferencia}}]

\begin{tabular}{p{0.35\textwidth}p{0.6\textwidth}}
\toprule
\textbf{Concepto} & \textbf{Fórmula} \\
\midrule
\textbf{Ecuación canónica} & $(x - h)^2 + (y - k)^2 = r^2$ \\
& Centro: $(h, k)$, Radio: $r$ \\
\midrule
\textbf{Centro en el origen} & $x^2 + y^2 = r^2$ \\
\midrule
\textbf{Ecuación general} & $x^2 + y^2 + Dx + Ey + F = 0$ \\
\midrule
\textbf{Conversión general a canónica} & Centro: $\left(-\frac{D}{2}, -\frac{E}{2}\right)$ \\
& Radio: $r = \frac{1}{2}\sqrt{D^2 + E^2 - 4F}$ \\
\midrule
\textbf{Condición de existencia} & $D^2 + E^2 - 4F > 0$ \\
\midrule
\textbf{Distancia entre dos puntos} & $d = \sqrt{(x_2 - x_1)^2 + (y_2 - y_1)^2}$ \\
\midrule
\textbf{Distancia punto a recta} & $d = \frac{|Ax_0 + By_0 + C|}{\sqrt{A^2 + B^2}}$ \\
& Para $Ax + By + C = 0$ y punto $(x_0, y_0)$ \\
\midrule
\textbf{Recta tangente en $(x_0, y_0)$} & $(x_0 - h)(x - h) + (y_0 - k)(y - k) = r^2$ \\
\midrule
\textbf{Posición recta-circunferencia} & $d < r$: Secante \\
& $d = r$: Tangente \\
& $d > r$: Exterior \\
\midrule
\textbf{Posición de dos circunferencias} & $d > r_1 + r_2$: Exteriores \\
& $d = r_1 + r_2$: Tang. exteriormente \\
& $|r_1 - r_2| < d < r_1 + r_2$: Secantes \\
& $d = |r_1 - r_2|$: Tang. interiormente \\
& $d < |r_1 - r_2|$: Interiores \\
& $d = 0$: Concéntricas \\
\bottomrule
\end{tabular}

\end{tcolorbox}
\end{center}

\newpage

% ===========================================
% PLACEHOLDERS PARA OTRAS SECCIONES
% ===========================================

%INSERTAR_EJEMPLOS_AQUI%

%INSERTAR_EJERCICIOS_AQUI%

% ===========================================
% CONCLUSIÓN
% ===========================================

\section{Conclusión}

\subsection{El viaje que hemos recorrido}

¡Felicitaciones! Has completado un viaje fascinante por el mundo de las circunferencias. Comenzamos con una simple idea —todos los puntos a la misma distancia de un centro— y construimos todo un edificio matemático sobre ella.

Recapitulemos lo que has aprendido:

\begin{enumerate}
  \item \textbf{La esencia de la circunferencia:} No es solo una figura redonda, sino un concepto matemático preciso con propiedades únicas. La perfección de su simetría la hace especial entre todas las figuras geométricas.

  \item \textbf{El poder de las ecuaciones:} Descubriste que una simple ecuación como $(x - h)^2 + (y - k)^2 = r^2$ contiene toda la información de una circunferencia. Es como tener el ADN matemático de la figura.

  \item \textbf{La versatilidad de las formas:} Aprendiste a trabajar tanto con la forma canónica como con la general, y a convertir entre ellas. Es como ser bilingüe en el idioma de las circunferencias.

  \item \textbf{Las relaciones geométricas:} Exploraste cómo las circunferencias interactúan con rectas y con otras circunferencias, descubriendo patrones y criterios precisos para clasificar estas relaciones.

  \item \textbf{La conexión álgebra-geometría:} Viste cómo problemas aparentemente visuales pueden resolverse algebraicamente, y viceversa. Esta dualidad es una de las bellezas de la geometría analítica.
\end{enumerate}

\subsection{Las herramientas que ahora dominas}

Has agregado herramientas poderosas a tu caja matemática:

\begin{center}
\begin{tcolorbox}[colback=ColorPrincipal!10,colframe=ColorPrincipal,title={\textbf{Tu Caja de Herramientas}}]

\textbf{Herramientas Conceptuales:}
\begin{itemize}
  \item Identificar centro y radio de cualquier circunferencia
  \item Reconocer cuándo una ecuación representa una circunferencia real
  \item Visualizar mentalmente posiciones relativas
  \item Conectar representaciones algebraicas y geométricas
\end{itemize}

\textbf{Herramientas Algebraicas:}
\begin{itemize}
  \item Escribir ecuaciones de circunferencias
  \item Completar cuadrados con confianza
  \item Resolver sistemas de ecuaciones circulares
  \item Calcular discriminantes e interpretar resultados
\end{itemize}

\textbf{Herramientas Analíticas:}
\begin{itemize}
  \item Determinar posiciones relativas algebraicamente
  \item Encontrar puntos de intersección
  \item Calcular distancias estratégicamente
  \item Resolver problemas de optimización circular
\end{itemize}

\end{tcolorbox}
\end{center}

\subsection{Aplicaciones en el mundo real}

Lo que has aprendido no se queda en el papel. Las circunferencias y sus propiedades están en todas partes:

\subsubsection{En la tecnología}

\begin{itemize}
  \item \textbf{GPS y navegación:} Los satélites calculan tu posición usando circunferencias. Cada satélite define una esfera (circunferencia en 3D) donde podrías estar, y la intersección de varias esferas determina tu ubicación exacta.

  \item \textbf{Diseño de antenas:} La cobertura de una antena de telefonía móvil se modela como circunferencias. Los ingenieros usan las posiciones relativas para garantizar cobertura sin interferencias.

  \item \textbf{Gráficos por computadora:} Cada vez que ves un círculo perfecto en tu pantalla, hay ecuaciones de circunferencia trabajando detrás de escena.
\end{itemize}

\subsubsection{En la ingeniería}

\begin{itemize}
  \item \textbf{Diseño de rotondas:} Los ingenieros de tráfico usan circunferencias tangentes para diseñar rotondas eficientes y seguras.

  \item \textbf{Sistemas mecánicos:} Engranajes, poleas, rodamientos... todos dependen de circunferencias perfectas y sus relaciones.

  \item \textbf{Arquitectura:} Desde los arcos romanos hasta los modernos domos geodésicos, la circunferencia proporciona fuerza y belleza.
\end{itemize}

\subsubsection{En las ciencias}

\begin{itemize}
  \item \textbf{Física:} El movimiento circular, las ondas, los campos... muchos fenómenos físicos se describen con circunferencias.

  \item \textbf{Astronomía:} Aunque las órbitas son elípticas, muchas se aproximan tanto a circunferencias que podemos usar lo que aprendiste para calcularlas.

  \item \textbf{Química:} Los orbitales atómicos, las estructuras moleculares cíclicas... la circunferencia aparece hasta en lo microscópico.
\end{itemize}

\subsection{Consejos para el éxito continuo}

Para seguir mejorando en este tema:

\begin{nota}
\textbf{Estrategias de estudio recomendadas:}

\begin{enumerate}
  \item \textbf{Practica la visualización:} Antes de calcular, dibuja. Un buen esquema vale más que mil cálculos.

  \item \textbf{Domina las conversiones:} Practica pasar de forma canónica a general y viceversa hasta que sea automático.

  \item \textbf{Conecta conceptos:} Relaciona la circunferencia con otros temas: funciones, trigonometría, vectores...

  \item \textbf{Resuelve problemas variados:} No te quedes con ejercicios mecánicos. Busca problemas que requieran creatividad.

  \item \textbf{Usa tecnología:} Programas como GeoGebra te permiten experimentar y verificar tus soluciones.

  \item \textbf{Enseña a otros:} Explicar estos conceptos a un compañero es la mejor forma de consolidar tu comprensión.
\end{enumerate}
\end{nota}

\subsection{Errores comunes a evitar}

Aprende de los tropiezos típicos:

\begin{importante}
\textbf{Errores frecuentes y cómo evitarlos:}

\begin{itemize}
  \item \textbf{Confundir radio con radio al cuadrado:} Recuerda que en la ecuación aparece $r^2$, no $r$.

  \item \textbf{Signos en el centro:} En $(x - h)^2 + (y - k)^2 = r^2$, si el centro es $(3, -2)$, la ecuación es $(x - 3)^2 + (y + 2)^2 = r^2$.

  \item \textbf{Olvidar verificar la condición de existencia:} No toda ecuación de segundo grado representa una circunferencia real.

  \item \textbf{Mezclar criterios de posición relativa:} Memoriza bien las condiciones para cada caso.

  \item \textbf{No simplificar al final:} Siempre simplifica tus respuestas a la forma más elegante posible.
\end{itemize}
\end{importante}

\subsection{Conexiones con temas futuros}

Lo que has aprendido aquí es la base para temas más avanzados:

\begin{itemize}
  \item \textbf{Elipse e hipérbola:} Son las hermanas de la circunferencia. Con lo que sabes, entenderlas será más fácil.

  \item \textbf{Geometría en 3D:} La esfera es la circunferencia en tres dimensiones. Las ecuaciones son sorprendentemente similares.

  \item \textbf{Cálculo:} Derivadas e integrales de funciones circulares aparecen en física e ingeniería.

  \item \textbf{Números complejos:} La circunferencia unitaria es fundamental en el plano complejo.

  \item \textbf{Trigonometría avanzada:} El círculo unitario es la base de todas las funciones trigonométricas.
\end{itemize}

\subsection{Reflexión final}

La circunferencia es mucho más que una figura redonda. Es un puente entre el álgebra y la geometría, entre lo abstracto y lo concreto, entre la teoría y la aplicación. Cada vez que veas una rueda girando, un CD brillando o la luna llena en el cielo, recuerda que ahora entiendes la matemática detrás de esa perfección.

Has demostrado que puedes tomar un concepto simple —puntos equidistantes de un centro— y construir sobre él un conocimiento profundo y útil. Esta es la esencia de las matemáticas: partir de ideas simples y construir estructuras complejas y hermosas.

\begin{center}
\begin{tcolorbox}[colback=ColorAcento!20,colframe=ColorAcento,title={\textbf{Mensaje final}}]
\centering
\Large
\textit{La circunferencia te ha enseñado que en matemáticas, como en la vida, la perfección no está en la complejidad, sino en la elegancia de lo simple.}

\vspace{0.5cm}

\textbf{¡Sigue explorando, sigue aprendiendo, sigue girando hacia el conocimiento!}
\end{tcolorbox}
\end{center}

\subsection{Tabla de referencia rápida}

Para tus futuros estudios y consultas:

\begin{center}
\begin{tcolorbox}[colback=ColorFondo,colframe=ColorPrincipal,title={\textbf{Referencia Rápida - Circunferencia}}]

\begin{tabular}{ll}
\toprule
\textbf{Si necesitas...} & \textbf{Usa...} \\
\midrule
Escribir la ecuación con centro y radio & $(x - h)^2 + (y - k)^2 = r^2$ \\
Encontrar centro desde ecuación general & $C = \left(-\frac{D}{2}, -\frac{E}{2}\right)$ \\
Encontrar radio desde ecuación general & $r = \frac{1}{2}\sqrt{D^2 + E^2 - 4F}$ \\
Verificar si es circunferencia real & $D^2 + E^2 - 4F > 0$ \\
Completar cuadrados en $x$ & $x^2 + Dx = \left(x + \frac{D}{2}\right)^2 - \frac{D^2}{4}$ \\
Distancia entre dos puntos & $d = \sqrt{(x_2 - x_1)^2 + (y_2 - y_1)^2}$ \\
Posición recta-circunferencia & Compara $d$ con $r$ \\
Posición de dos circunferencias & Compara $d$ con $r_1 + r_2$ y $|r_1 - r_2|$ \\
\bottomrule
\end{tabular}

\end{tcolorbox}
\end{center}

\vspace{1cm}

% FIN DE LA PARTE 1 - NO INCLUIR \end{document}