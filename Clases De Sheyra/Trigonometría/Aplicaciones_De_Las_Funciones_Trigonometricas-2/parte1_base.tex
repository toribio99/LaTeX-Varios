% !TEX program = lualatex
\documentclass[12pt,a4paper,twoside]{article}
\usepackage{fontspec}
\usepackage[spanish,es-noshorthands]{babel}
\usepackage{amsmath,amssymb}
\usepackage[margin=2.5cm]{geometry}
\usepackage{xcolor}
\usepackage{tikz,pgfplots}
\usetikzlibrary{calc,arrows.meta,babel,patterns,angles,quotes}
\usepackage{multicol}
\usepackage{enumitem}
\usepackage{array}
\usepackage{booktabs}
\pgfplotsset{compat=1.18}
\definecolor{maincolor}{RGB}{26,35,126}
\definecolor{accentcolor}{RGB}{255,87,34}

% Configuración de títulos y formato
\usepackage{titlesec}
\titleformat{\section}{\Large\bfseries\color{maincolor}}{\thesection}{1em}{}
\titleformat{\subsection}{\large\bfseries\color{accentcolor}}{\thesubsection}{1em}{}
\titleformat{\subsubsection}{\normalsize\bfseries\color{maincolor}}{\thesubsubsection}{1em}{}

% Configuración de cajas para ejemplos
\usepackage{tcolorbox}
\tcbuselibrary{skins,breakable}

% Headers y footers
\usepackage{fancyhdr}
\pagestyle{fancy}
\fancyhf{}
\fancyhead[LE]{\small\textcolor{maincolor}{\thepage \quad Aplicaciones de las Funciones Trigonométricas}}
\fancyhead[RO]{\small\textcolor{maincolor}{Solución de Triángulos Oblicuángulos \quad \thepage}}
\fancyhead[LO]{\small\textcolor{maincolor}{Grado 10 - Trigonometría}}
\fancyhead[RE]{\small\textcolor{maincolor}{Prof. Toribio De J Arrieta F}}
\fancyfoot[C]{}
\renewcommand{\headrulewidth}{0.5pt}
\renewcommand{\footrulewidth}{0pt}
\setlength{\headheight}{14pt}

% Entornos tcolorbox
\newtcolorbox{definicion}[1][]{
  enhanced,
  breakable,
  colback=maincolor!5,
  colframe=maincolor,
  fonttitle=\bfseries,
  title=Definición,
  #1
}

\newtcolorbox{ejemplo}[1][]{
  enhanced,
  breakable,
  colback=maincolor!5,
  colframe=maincolor,
  fonttitle=\bfseries,
  title=Ejemplo Resuelto,
  #1
}

\newtcolorbox{ejercicio}[1][]{
  enhanced,
  breakable,
  colback=accentcolor!5,
  colframe=accentcolor,
  fonttitle=\bfseries,
  title=Ejercicio,
  #1
}

\newtcolorbox{solucion}[1][]{
  enhanced,
  breakable,
  colback=green!5,
  colframe=green!60!black,
  fonttitle=\bfseries,
  title=Solución,
  #1
}

\newtcolorbox{nota}[1][]{
  enhanced,
  breakable,
  colback=yellow!10,
  colframe=orange!80!black,
  fonttitle=\bfseries,
  title=Nota Importante,
  #1
}

% Título
\title{\textbf{\Huge APLICACIONES DE LAS FUNCIONES TRIGONOMÉTRICAS}\\[0.5cm]
\Large Solución de Triángulos Oblicuángulos\\[0.3cm]
\normalsize Guía de Trigonometría}
\author{Prof. Toribio De J Arrieta F\\
\textit{La Pruebita}\\
Grado 10}
\date{\today}

\begin{document}

\maketitle
\thispagestyle{empty}

\vfill

\begin{center}
\textit{Esta guía está dedicada a todos los estudiantes que desean comprender\\
cómo las matemáticas nos ayudan a resolver problemas del mundo real.}
\end{center}

\newpage

\tableofcontents

\newpage

\section{Introducción}

¡Hola! ¿Alguna vez te has preguntado cómo los ingenieros calculan la altura de una montaña sin tener que escalarla? ¿O cómo los navegantes encuentran su camino en medio del océano? ¿Cómo los arquitectos diseñan esos puentes increíbles que parecen desafiar la gravedad? La respuesta está en lo que vamos a aprender hoy: los triángulos oblicuángulos.

Hasta ahora, probablemente has trabajado mucho con triángulos rectángulos, esos triángulos que tienen un ángulo de $90^\circ$. Son geniales y súper útiles, pero ¿qué pasa cuando ninguno de los ángulos del triángulo es recto? Ahí es donde entran los triángulos oblicuángulos, y créeme, ¡son igual de poderosos!

\subsection*{¿Qué son los triángulos oblicuángulos?}

Un triángulo oblicuángulo es simplemente cualquier triángulo que NO tiene un ángulo recto. Puede ser:
\begin{itemize}
    \item \textbf{Acutángulo}: cuando todos sus ángulos son agudos (menores de $90^\circ$)
    \item \textbf{Obtusángulo}: cuando uno de sus ángulos es obtuso (mayor de $90^\circ$)
\end{itemize}

Piénsalo así: la mayoría de los triángulos que encuentras en la vida real son oblicuángulos. La forma de un pedazo de pizza, el triángulo que forman tres ciudades en un mapa, la estructura triangular de un puente colgante... ¡todos son oblicuángulos!

\subsection*{¿Por qué no podemos usar solo el teorema de Pitágoras?}

Buena pregunta. El teorema de Pitágoras es como esa herramienta especial que solo funciona con triángulos rectángulos. Para los triángulos oblicuángulos, necesitamos herramientas nuevas y más versátiles: la Ley del Seno y la Ley del Coseno. Son como las navajas suizas de la trigonometría: funcionan con CUALQUIER triángulo.

\subsection*{Aplicaciones prácticas que veremos}

Te voy a mostrar cómo estos conceptos se usan en situaciones reales fascinantes:

\begin{enumerate}
    \item \textbf{Navegación marítima}: Los marineros usan triangulación para determinar su posición en el océano. Con la ley del seno, pueden calcular distancias a faros o puntos de referencia costeros.

    \item \textbf{Topografía}: Los topógrafos miden terrenos irregulares dividiendo el área en triángulos oblicuángulos. Así determinan alturas de montañas, pendientes de terrenos y crean mapas precisos.

    \item \textbf{Arquitectura}: Los arquitectos usan estas leyes para diseñar estructuras con formas triangulares no rectangulares, calculando fuerzas y tensiones en cada parte de la estructura.

    \item \textbf{Ingeniería}: En el diseño de puentes innovadores, especialmente los puentes colgantes y atirantados, cada cable forma triángulos oblicuángulos con la estructura. Los ingenieros deben calcular las tensiones exactas en cada cable.

    \item \textbf{Astronomía y GPS}: Para determinar la posición de satélites, calcular distancias entre estrellas, o hacer que tu GPS funcione correctamente, se utilizan constantemente estas leyes.
\end{enumerate}

\subsection*{¿Por qué deberías emocionarte con este tema?}

Mira, entiendo que a veces las matemáticas pueden parecer abstractas, pero lo que vamos a aprender aquí es súper concreto y útil. Imagínate esto:

\begin{itemize}
    \item Podrás calcular la altura de cualquier objeto alto (un árbol, un edificio, una montaña) sin necesidad de medirlo directamente
    \item Entenderás cómo funcionan los sistemas de GPS y navegación
    \item Comprenderás los principios detrás del diseño de estructuras increíbles
    \item Tendrás las herramientas matemáticas que usan ingenieros y científicos todos los días
\end{itemize}

Además, resolver triángulos oblicuángulos es como resolver puzzles. Tienes algunas piezas de información (algunos lados y ángulos) y debes descubrir las piezas faltantes. Es un desafío intelectual que, una vez que le agarras el truco, ¡es súper satisfactorio!

\subsection*{Lo que necesitas recordar antes de empezar}

Para sacarle el máximo provecho a esta guía, asegúrate de tener frescos estos conceptos:
\begin{itemize}
    \item Las funciones trigonométricas básicas: seno, coseno y tangente
    \item Cómo trabajar con ángulos en grados
    \item Álgebra básica (resolver ecuaciones)
    \item El teorema de Pitágoras (sí, todavía lo usaremos como referencia)
\end{itemize}

No te preocupes si no eres un experto en todo esto. Iremos paso a paso, y cuando necesitemos usar algún concepto previo, lo repasaremos rápidamente.

\subsection*{Nuestra meta}

Al final de esta guía, serás capaz de resolver CUALQUIER triángulo oblicuángulo. No importa qué información te den (tres lados, dos lados y un ángulo, dos ángulos y un lado), tendrás las herramientas para encontrar toda la información faltante. ¡Es como tener superpoderes matemáticos!

Así que respira profundo, prepara tu calculadora (la vas a necesitar), y vamos a sumergirnos en el fascinante mundo de los triángulos oblicuángulos. Te prometo que al final dirás: "¡Wow, no sabía que las matemáticas podían hacer todo esto!"

\newpage

\section{Conceptos Fundamentales}

\subsection{Definición de triángulo oblicuángulo}

\begin{definicion}
Un \textbf{triángulo oblicuángulo} es cualquier triángulo que no contiene un ángulo recto ($90^\circ$). Todos sus ángulos son diferentes de $90^\circ$.
\end{definicion}

Los triángulos oblicuángulos se clasifican en dos tipos:

\begin{center}
\begin{tikzpicture}[scale=1.5]
    % Triángulo acutángulo
    \coordinate (A1) at (0,0);
    \coordinate (B1) at (3,0);
    \coordinate (C1) at (1.5,2);

    \draw[thick, maincolor] (A1) -- (B1) -- (C1) -- cycle;
    \node[below] at (A1) {$A$};
    \node[below] at (B1) {$B$};
    \node[above] at (C1) {$C$};

    % Ángulos
    \pic[draw, accentcolor, angle radius=0.3cm, angle eccentricity=1.5] {angle = B1--A1--C1};
    \pic[draw, accentcolor, angle radius=0.3cm, angle eccentricity=1.5] {angle = C1--B1--A1};
    \pic[draw, accentcolor, angle radius=0.3cm, angle eccentricity=1.5] {angle = A1--C1--B1};

    \node[below] at (1.5,-0.5) {\textbf{Triángulo Acutángulo}};
    \node[below] at (1.5,-0.8) {\small Todos los ángulos $< 90^\circ$};

    % Triángulo obtusángulo
    \begin{scope}[xshift=5cm]
        \coordinate (A2) at (0,0);
        \coordinate (B2) at (3,0);
        \coordinate (C2) at (0.5,1.5);

        \draw[thick, maincolor] (A2) -- (B2) -- (C2) -- cycle;
        \node[below] at (A2) {$D$};
        \node[below] at (B2) {$E$};
        \node[above] at (C2) {$F$};

        % Ángulos
        \pic[draw, accentcolor, angle radius=0.3cm, angle eccentricity=1.5] {angle = B2--A2--C2};
        \pic[draw, red, angle radius=0.3cm, angle eccentricity=1.5] {angle = C2--B2--A2};
        \pic[draw, accentcolor, angle radius=0.3cm, angle eccentricity=1.5] {angle = A2--C2--B2};

        \node[below] at (1.5,-0.5) {\textbf{Triángulo Obtusángulo}};
        \node[below] at (1.5,-0.8) {\small Un ángulo $> 90^\circ$};
    \end{scope}
\end{tikzpicture}
\end{center}

En cualquier triángulo oblicuángulo, usamos la siguiente notación estándar:
\begin{itemize}
    \item Los vértices se nombran con letras mayúsculas: $A$, $B$, $C$
    \item Los lados opuestos a cada vértice se nombran con letras minúsculas: $a$, $b$, $c$
    \item Los ángulos en cada vértice se pueden denotar con la letra del vértice o con letras griegas
\end{itemize}

\subsection{Ley del Seno}

La Ley del Seno es nuestra primera herramienta poderosa para resolver triángulos oblicuángulos.

\begin{definicion}[title=Ley del Seno]
En cualquier triángulo, la razón entre la longitud de un lado y el seno del ángulo opuesto es constante:
\[
\frac{a}{\sin A} = \frac{b}{\sin B} = \frac{c}{\sin C} = 2R
\]
donde $R$ es el radio del círculo circunscrito al triángulo.
\end{definicion}

\subsubsection{Demostración visual de la Ley del Seno}

Vamos a ver por qué esta ley funciona. Imagina un triángulo inscrito en un círculo:

\begin{center}
\begin{tikzpicture}[scale=2]
    % Círculo
    \draw[dashed, gray] (0,0) circle (2);
    \coordinate (O) at (0,0);

    % Triángulo
    \coordinate (A) at (140:2);
    \coordinate (B) at (20:2);
    \coordinate (C) at (260:2);

    \draw[thick, maincolor] (A) -- (B) -- (C) -- cycle;

    % Etiquetas de vértices
    \node[above left] at (A) {$A$};
    \node[right] at (B) {$B$};
    \node[below left] at (C) {$C$};

    % Lados
    \node[above right] at ($(A)!0.5!(B)$) {$c$};
    \node[below] at ($(B)!0.5!(C)$) {$a$};
    \node[left] at ($(C)!0.5!(A)$) {$b$};

    % Radio y centro
    \draw[accentcolor, dashed] (O) -- (A) node[midway, left] {$R$};
    \draw[accentcolor, dashed] (O) -- (B) node[midway, right] {$R$};
    \draw[accentcolor, dashed] (O) -- (C) node[midway, below] {$R$};
    \filldraw (O) circle (0.03) node[below right] {$O$};

    % Diámetro auxiliar
    \coordinate (D) at (-140:2);
    \draw[dotted, thick] (A) -- (D);
    \node[below right] at (D) {$D$};

    % Ángulo inscrito
    \pic[draw, accentcolor, angle radius=0.4cm, angle eccentricity=1.3] {angle = C--A--B};

    \node[below] at (0,-2.5) {El triángulo inscrito en un círculo de radio $R$};
\end{tikzpicture}
\end{center}

La demostración se basa en que el ángulo inscrito en una semicircunferencia es recto, y usando propiedades del círculo, llegamos a que cada razón es igual a $2R$.

\subsubsection{Casos de uso de la Ley del Seno}

La Ley del Seno es especialmente útil cuando conocemos:
\begin{enumerate}
    \item \textbf{Caso ALA (Ángulo-Lado-Ángulo)}: Dos ángulos y el lado entre ellos
    \item \textbf{Caso AAL (Ángulo-Ángulo-Lado)}: Dos ángulos y un lado cualquiera
    \item \textbf{Caso LLA (Lado-Lado-Ángulo)}: Dos lados y un ángulo opuesto a uno de ellos (¡cuidado con el caso ambiguo!)
\end{enumerate}

\begin{nota}
El caso LLA puede tener 0, 1 o 2 soluciones. Esto se conoce como el caso ambiguo de la ley del seno.
\end{nota}

\subsection{Ley del Coseno}

La Ley del Coseno es como una versión generalizada del Teorema de Pitágoras que funciona para cualquier triángulo.

\begin{definicion}[title=Ley del Coseno]
En cualquier triángulo, el cuadrado de un lado es igual a la suma de los cuadrados de los otros dos lados menos el doble producto de estos lados por el coseno del ángulo comprendido entre ellos:
\begin{align*}
a^2 &= b^2 + c^2 - 2bc \cos A \\
b^2 &= a^2 + c^2 - 2ac \cos B \\
c^2 &= a^2 + b^2 - 2ab \cos C
\end{align*}
\end{definicion}

\subsubsection{Demostración visual de la Ley del Coseno}

Veamos geométricamente por qué funciona esta ley:

\begin{center}
\begin{tikzpicture}[scale=2.5]
    % Triángulo
    \coordinate (B) at (0,0);
    \coordinate (C) at (3,0);
    \coordinate (A) at (2,1.5);

    \draw[thick, maincolor] (B) -- (C) -- (A) -- cycle;

    % Altura
    \coordinate (H) at (2,0);
    \draw[dashed, gray] (A) -- (H);
    \draw (H) rectangle +(-0.1,0.1);

    % Etiquetas
    \node[below] at (B) {$B$};
    \node[below] at (C) {$C$};
    \node[above] at (A) {$A$};
    \node[below] at (H) {$H$};

    % Lados
    \node[above left] at ($(B)!0.5!(A)$) {$c$};
    \node[above right] at ($(A)!0.5!(C)$) {$b$};
    \node[below] at ($(B)!0.5!(C)$) {$a$};

    % Segmentos de la base
    \draw[|-|, accentcolor] ($(B)+(0,-0.3)$) -- ($(H)+(0,-0.3)$) node[midway, below] {$c\cos B$};
    \draw[|-|, accentcolor] ($(H)+(0,-0.4)$) -- ($(C)+(0,-0.4)$) node[midway, below] {$a - c\cos B$};

    % Altura
    \node[right] at ($(H)!0.5!(A)$) {$h = c\sin B$};

    % Ángulo
    \pic[draw, accentcolor, angle radius=0.4cm, angle eccentricity=1.3] {angle = C--B--A};
\end{tikzpicture}
\end{center}

Aplicando el Teorema de Pitágoras al triángulo rectángulo $AHC$ y manipulando algebraicamente, obtenemos la Ley del Coseno.

\subsubsection{Casos de uso de la Ley del Coseno}

La Ley del Coseno es perfecta cuando conocemos:
\begin{enumerate}
    \item \textbf{Caso LAL (Lado-Ángulo-Lado)}: Dos lados y el ángulo comprendido entre ellos
    \item \textbf{Caso LLL (Lado-Lado-Lado)}: Los tres lados del triángulo
\end{enumerate}

\begin{nota}
Cuando todos los ángulos son agudos y uno de ellos es de $90^\circ$, la Ley del Coseno se reduce al Teorema de Pitágoras, ya que $\cos(90^\circ) = 0$.
\end{nota}

\subsection{Área de triángulos oblicuángulos}

Para calcular el área de un triángulo oblicuángulo, tenemos varias fórmulas útiles:

\subsubsection{Fórmula del área con seno}

\begin{definicion}[title=Área con dos lados y el ángulo comprendido]
El área de un triángulo con lados $a$ y $b$ y ángulo comprendido $C$ es:
\[
\text{Área} = \frac{1}{2}ab\sin C
\]
De manera general:
\begin{align*}
\text{Área} &= \frac{1}{2}ab\sin C \\
&= \frac{1}{2}ac\sin B \\
&= \frac{1}{2}bc\sin A
\end{align*}
\end{definicion}

Esta fórmula es súper práctica porque solo necesitas conocer dos lados y el ángulo entre ellos.

\begin{center}
\begin{tikzpicture}[scale=2]
    % Triángulo
    \coordinate (A) at (0,0);
    \coordinate (B) at (3.5,0);
    \coordinate (C) at (1.5,2);

    \draw[thick, maincolor] (A) -- (B) -- (C) -- cycle;

    % Altura
    \coordinate (H) at (1.5,0);
    \draw[dashed, gray] (C) -- (H);

    % Etiquetas
    \node[below] at (A) {$A$};
    \node[below] at (B) {$B$};
    \node[above] at (C) {$C$};

    % Lados
    \node[below] at ($(A)!0.5!(B)$) {$c$};
    \node[above right] at ($(B)!0.5!(C)$) {$a$};
    \node[above left] at ($(A)!0.5!(C)$) {$b$};

    % Altura
    \node[right] at ($(H)!0.5!(C)$) {$h = b\sin A$};

    % Ángulo
    \pic[draw, accentcolor, angle radius=0.5cm, angle eccentricity=1.3] {angle = B--A--C};

    % Área sombreada
    \fill[maincolor!20, opacity=0.5] (A) -- (B) -- (C) -- cycle;

    \node[maincolor] at (1.75,0.8) {Área $= \frac{1}{2}ch$};
    \node[maincolor] at (1.75,0.4) {$= \frac{1}{2}c(b\sin A)$};
\end{tikzpicture}
\end{center}

\subsubsection{Fórmula de Herón}

Cuando conoces los tres lados del triángulo pero ningún ángulo, la Fórmula de Herón es tu mejor amiga.

\begin{definicion}[title=Fórmula de Herón]
Si conocemos los tres lados $a$, $b$ y $c$ de un triángulo, su área es:
\[
\text{Área} = \sqrt{s(s-a)(s-b)(s-c)}
\]
donde $s = \frac{a+b+c}{2}$ es el semiperímetro del triángulo.
\end{definicion}

Esta fórmula es elegante y poderosa. El semiperímetro $s$ es simplemente la mitad del perímetro total.

\subsection{Casos de resolución de triángulos oblicuángulos}

Ahora vamos a organizar todo lo que hemos aprendido. Dependiendo de qué información tengas sobre el triángulo, usarás diferentes estrategias:

\subsubsection{Caso LAL (Lado-Ángulo-Lado)}

\textbf{Conocemos}: Dos lados y el ángulo comprendido entre ellos.

\textbf{Estrategia}:
\begin{enumerate}
    \item Usar la Ley del Coseno para encontrar el tercer lado
    \item Usar la Ley del Seno para encontrar uno de los ángulos restantes
    \item Calcular el tercer ángulo sabiendo que la suma de ángulos es $180^\circ$
\end{enumerate}

\begin{center}
\begin{tikzpicture}[scale=1.8]
    \coordinate (A) at (0,0);
    \coordinate (B) at (3,0);
    \coordinate (C) at (1.2,1.8);

    \draw[thick, maincolor] (A) -- (B) -- (C) -- cycle;

    % Datos conocidos en rojo
    \draw[ultra thick, red] (A) -- (B) node[midway, below] {conocido};
    \draw[ultra thick, red] (A) -- (C) node[midway, left] {conocido};

    % Ángulo conocido
    \pic[draw, red, ultra thick, angle radius=0.5cm, angle eccentricity=1.5] {angle = B--A--C};

    % Incógnita
    \draw[dashed, thick, blue] (B) -- (C) node[midway, right] {?};

    \node[below] at (1.5,-0.5) {\textbf{Caso LAL}};
\end{tikzpicture}
\end{center}

\subsubsection{Caso ALA (Ángulo-Lado-Ángulo)}

\textbf{Conocemos}: Dos ángulos y el lado comprendido entre ellos.

\textbf{Estrategia}:
\begin{enumerate}
    \item Calcular el tercer ángulo (suma = $180^\circ$)
    \item Usar la Ley del Seno para encontrar los otros dos lados
\end{enumerate}

\begin{center}
\begin{tikzpicture}[scale=1.8]
    \coordinate (A) at (0,0);
    \coordinate (B) at (3,0);
    \coordinate (C) at (1.8,1.5);

    \draw[thick, maincolor] (A) -- (B) -- (C) -- cycle;

    % Lado conocido
    \draw[ultra thick, red] (A) -- (B) node[midway, below] {conocido};

    % Ángulos conocidos
    \pic[draw, red, ultra thick, angle radius=0.4cm, angle eccentricity=1.5] {angle = B--A--C};
    \pic[draw, red, ultra thick, angle radius=0.4cm, angle eccentricity=1.5] {angle = A--B--C};

    % Incógnitas
    \draw[dashed, thick, blue] (A) -- (C) node[midway, left] {?};
    \draw[dashed, thick, blue] (B) -- (C) node[midway, right] {?};

    \node[below] at (1.5,-0.5) {\textbf{Caso ALA}};
\end{tikzpicture}
\end{center}

\subsubsection{Caso LLL (Lado-Lado-Lado)}

\textbf{Conocemos}: Los tres lados del triángulo.

\textbf{Estrategia}:
\begin{enumerate}
    \item Usar la Ley del Coseno para encontrar uno de los ángulos
    \item Usar la Ley del Coseno o del Seno para encontrar otro ángulo
    \item Calcular el tercer ángulo por diferencia
\end{enumerate}

\begin{center}
\begin{tikzpicture}[scale=1.8]
    \coordinate (A) at (0,0);
    \coordinate (B) at (3.2,0);
    \coordinate (C) at (1.5,1.7);

    \draw[thick, maincolor] (A) -- (B) -- (C) -- cycle;

    % Todos los lados conocidos
    \draw[ultra thick, red] (A) -- (B) node[midway, below] {conocido};
    \draw[ultra thick, red] (B) -- (C) node[midway, right] {conocido};
    \draw[ultra thick, red] (A) -- (C) node[midway, left] {conocido};

    % Ángulos desconocidos
    \pic[draw, blue, dashed, thick, angle radius=0.4cm, angle eccentricity=1.3] {angle = B--A--C};
    \pic[draw, blue, dashed, thick, angle radius=0.4cm, angle eccentricity=1.3] {angle = A--B--C};
    \pic[draw, blue, dashed, thick, angle radius=0.4cm, angle eccentricity=1.3] {angle = B--C--A};

    \node[below] at (1.6,-0.5) {\textbf{Caso LLL}};
\end{tikzpicture}
\end{center}

\subsubsection{Caso LLA (Lado-Lado-Ángulo) - El caso ambiguo}

\textbf{Conocemos}: Dos lados y un ángulo opuesto a uno de ellos.

\textbf{¡Cuidado!} Este es el caso más complicado porque puede tener:
\begin{itemize}
    \item \textbf{Ninguna solución}: El triángulo no existe
    \item \textbf{Una solución}: Un único triángulo posible
    \item \textbf{Dos soluciones}: Dos triángulos diferentes posibles
\end{itemize}

\textbf{Estrategia}:
\begin{enumerate}
    \item Usar la Ley del Seno para encontrar otro ángulo
    \item Verificar si la solución es válida (el seno debe estar entre -1 y 1)
    \item Si hay solución, verificar si existe una segunda solución válida
    \item Completar el triángulo con la información restante
\end{enumerate}

\begin{center}
\begin{tikzpicture}[scale=1.5]
    % Primer caso - dos soluciones
    \coordinate (A1) at (0,0);
    \coordinate (B1) at (3,0);
    \coordinate (C1) at (1.8,1.5);
    \coordinate (C1p) at (1.8,-0.8);

    % Triángulos
    \draw[thick, maincolor] (A1) -- (B1);
    \draw[thick, maincolor] (A1) -- (C1);
    \draw[thick, maincolor] (B1) -- (C1);

    \draw[thick, blue, dashed] (A1) -- (C1p);
    \draw[thick, blue, dashed] (B1) -- (C1p);

    % Arco mostrando las dos posibilidades
    \draw[accentcolor, dashed] (B1) ++(150:2) arc (150:210:2);

    % Etiquetas
    \node[below] at (A1) {$A$};
    \node[below] at (B1) {$B$};
    \node[above] at (C1) {$C$};
    \node[below] at (C1p) {$C'$};

    % Datos conocidos
    \draw[ultra thick, red] (A1) -- (B1) node[midway, below] {\small conocido};
    \node[red, left] at ($(A1)!0.5!(C1)$) {\small conocido};

    % Ángulo conocido
    \pic[draw, red, ultra thick, angle radius=0.3cm, angle eccentricity=1.3] {angle = C1--B1--A1};

    \node[below] at (1.5,-1.5) {\textbf{Caso ambiguo: 2 soluciones}};
\end{tikzpicture}
\end{center}

\subsection{Resumen de estrategias}

Para que tengas todo claro, aquí está el resumen de qué herramienta usar en cada caso:

\begin{center}
\renewcommand{\arraystretch}{1.5}
\begin{tabular}{|c|l|l|c|}
\hline
\textbf{Caso} & \textbf{Información conocida} & \textbf{Herramienta principal} & \textbf{Soluciones} \\
\hline
LAL & 2 lados y ángulo entre ellos & Ley del Coseno & 1 \\
\hline
ALA & 2 ángulos y lado entre ellos & Ley del Seno & 1 \\
\hline
AAL & 2 ángulos y 1 lado cualquiera & Ley del Seno & 1 \\
\hline
LLL & 3 lados & Ley del Coseno & 0 o 1 \\
\hline
LLA & 2 lados y ángulo opuesto a uno & Ley del Seno & 0, 1 o 2 \\
\hline
\end{tabular}
\end{center}

\subsection{Aplicación: Vectores en triángulos oblicuángulos}

Los vectores son súper importantes en física e ingeniería, y los triángulos oblicuángulos aparecen constantemente cuando trabajamos con ellos.

\subsubsection{Suma de vectores}

Cuando sumas dos vectores que no son perpendiculares, formas un triángulo oblicuángulo:

\begin{center}
\begin{tikzpicture}[scale=2]
    % Vectores
    \coordinate (O) at (0,0);
    \coordinate (A) at (3,0);
    \coordinate (B) at (1.5,2);

    % Vector A
    \draw[ultra thick, blue, -latex] (O) -- (A) node[midway, below] {$\vec{A}$};

    % Vector B desde el final de A
    \draw[ultra thick, red, -latex] (A) -- (B) node[midway, right] {$\vec{B}$};

    % Resultante
    \draw[ultra thick, maincolor, -latex] (O) -- (B) node[midway, left] {$\vec{R} = \vec{A} + \vec{B}$};

    % Vector B desde el origen (para mostrar el paralelogramo)
    \draw[dashed, red, -latex] (O) -- +($(B)-(A)$) node[midway, above] {$\vec{B}$};

    % Ángulo entre vectores
    \pic[draw, accentcolor, angle radius=0.5cm, angle eccentricity=1.3] {angle = A--O--B};

    \node[below] at (1.5,-0.5) {La suma de vectores forma un triángulo};
\end{tikzpicture}
\end{center}

Para encontrar la magnitud del vector resultante $\vec{R}$, usamos la Ley del Coseno:
\[
|\vec{R}|^2 = |\vec{A}|^2 + |\vec{B}|^2 - 2|\vec{A}||\vec{B}|\cos(180^\circ - \theta)
\]

\subsubsection{Descomposición de fuerzas}

En ingeniería, frecuentemente necesitamos descomponer una fuerza en componentes que no son perpendiculares:

\begin{center}
\begin{tikzpicture}[scale=2]
    % Sistema de coordenadas inclinado
    \coordinate (O) at (0,0);

    % Ejes inclinados
    \draw[thick, ->] (O) -- (30:3) node[right] {$x'$};
    \draw[thick, ->] (O) -- (120:2.5) node[above] {$y'$};

    % Fuerza
    \draw[ultra thick, maincolor, -latex] (O) -- (75:2.5) node[right] {$\vec{F}$};

    % Componentes
    \draw[dashed, blue] (75:2.5) -- ($(O)+(30:2.165)$);
    \draw[dashed, red] (75:2.5) -- ($(O)+(120:1.25)$);

    \draw[thick, blue, -latex] (O) -- (30:2.165) node[below] {$F_{x'}$};
    \draw[thick, red, -latex] (O) -- (120:1.25) node[left] {$F_{y'}$};

    % Ángulos
    \pic[draw, angle radius=0.4cm, angle eccentricity=1.5] {angle = 30:3--O--0:3};
    \pic[draw, angle radius=0.6cm, angle eccentricity=1.3] {angle = 75:2.5--O--30:3};

    \node[below] at (0,-0.5) {Descomposición en ejes no perpendiculares};
\end{tikzpicture}
\end{center}

\subsection{Aplicación: Diseño de puentes innovadores}

Los puentes modernos, especialmente los atirantados y colgantes, son maravillas de la aplicación de triángulos oblicuángulos.

\begin{center}
\begin{tikzpicture}[scale=0.8]
    % Torre del puente
    \draw[ultra thick] (0,-1) -- (0,4);
    \node[left] at (0,4) {Torre};

    % Tablero del puente
    \draw[ultra thick] (-4,0) -- (4,0);
    \node[below] at (0,0) {Tablero};

    % Cables (tirantes)
    \foreach \x/\y in {-3/0.5, -2/1, -1/1.5, 1/1.5, 2/1, 3/0.5} {
        \draw[thick, accentcolor] (0,3.5) -- (\x,0);
    }

    % Triángulo ejemplo
    \draw[ultra thick, blue] (0,3.5) -- (2,0) -- (0,0) -- cycle;

    % Etiquetas del triángulo
    \node[above] at (0,3.5) {$A$};
    \node[below] at (2,0) {$B$};
    \node[below] at (0,0) {$C$};

    % Fuerzas
    \draw[thick, red, -latex] (2,0) -- (2,-0.8) node[right] {Peso};
    \draw[thick, red, -latex] (0,3.5) -- (-0.5,3.8) node[above] {Tensión};

    \node[below] at (0,-1.5) {\textbf{Puente atirantado: cada cable forma un triángulo}};

    % Anotación
    \node[text width=6cm, right] at (5,2) {
        Cada cable debe soportar una tensión específica.
        Usando la ley del seno y coseno, los ingenieros calculan:
        \begin{itemize}
            \item Longitud exacta de cada cable
            \item Ángulo óptimo de inclinación
            \item Distribución de fuerzas
            \item Tensión en cada punto
        \end{itemize}
    };
\end{tikzpicture}
\end{center}

Los ingenieros deben resolver cientos de triángulos oblicuángulos para diseñar un puente seguro y eficiente. Cada cable, cada viga, cada conexión forma triángulos que deben ser analizados cuidadosamente.

%INSERTAR_EJEMPLOS_AQUI%

%INSERTAR_EJERCICIOS_INVERSOS_AQUI%

%INSERTAR_EJERCICIOS_PROPUESTOS_AQUI%

%INSERTAR_SOLUCIONES_AQUI%

\newpage

\section{Conclusión}

¡Felicitaciones! Has completado esta primera parte de la guía sobre triángulos oblicuángulos. Ahora tienes en tu caja de herramientas matemáticas dos leyes súper poderosas que te permiten resolver cualquier triángulo, no solo los rectángulos.

\subsection*{Lo que has aprendido}

En esta guía hemos cubierto:
\begin{itemize}
    \item La diferencia entre triángulos rectángulos y oblicuángulos
    \item La Ley del Seno y cuándo usarla
    \item La Ley del Coseno y sus aplicaciones
    \item Cómo calcular el área de cualquier triángulo
    \item La famosa Fórmula de Herón
    \item Los diferentes casos de resolución: LAL, ALA, LLL, LLA
    \item El caso ambiguo y por qué es especial
    \item Aplicaciones reales en navegación, topografía, arquitectura e ingeniería
\end{itemize}

\subsection*{Tabla resumen de fórmulas importantes}

\begin{center}
\renewcommand{\arraystretch}{1.8}
\begin{tabular}{|l|c|}
\hline
\textbf{Concepto} & \textbf{Fórmula} \\
\hline
Ley del Seno & $\displaystyle\frac{a}{\sin A} = \frac{b}{\sin B} = \frac{c}{\sin C} = 2R$ \\
\hline
Ley del Coseno & $a^2 = b^2 + c^2 - 2bc\cos A$ \\
\hline
Área (con seno) & $\text{Área} = \frac{1}{2}ab\sin C$ \\
\hline
Fórmula de Herón & $\text{Área} = \sqrt{s(s-a)(s-b)(s-c)}$ \\
& donde $s = \frac{a+b+c}{2}$ \\
\hline
Suma de ángulos & $A + B + C = 180^\circ$ \\
\hline
\end{tabular}
\end{center}

\subsection*{Consejos para resolver triángulos oblicuángulos}

\begin{enumerate}
    \item \textbf{Identifica qué información tienes}: Antes de empezar a calcular, lista claramente qué lados y ángulos conoces.

    \item \textbf{Elige la herramienta correcta}:
    \begin{itemize}
        \item ¿Tienes un ángulo y su lado opuesto? → Ley del Seno
        \item ¿Tienes dos lados y el ángulo entre ellos? → Ley del Coseno
        \item ¿Tienes los tres lados? → Ley del Coseno (o Herón para el área)
    \end{itemize}

    \item \textbf{Dibuja siempre el triángulo}: Un dibujo te ayuda a visualizar el problema y evitar errores.

    \item \textbf{Verifica tus respuestas}:
    \begin{itemize}
        \item Los ángulos deben sumar $180^\circ$
        \item Ningún ángulo puede ser negativo o mayor a $180^\circ$
        \item El lado más largo está opuesto al ángulo más grande
    \end{itemize}

    \item \textbf{Ten cuidado con el caso ambiguo}: Cuando uses la ley del seno con dos lados y un ángulo opuesto, recuerda que puede haber dos soluciones.

    \item \textbf{Usa la calculadora correctamente}:
    \begin{itemize}
        \item Asegúrate de que esté en modo grados (DEG)
        \item Redondea solo al final
        \item Guarda valores intermedios en la memoria
    \end{itemize}
\end{enumerate}

\subsection*{Aplicaciones en el mundo real}

Ahora que dominas estas herramientas, puedes entender y resolver problemas como:

\begin{itemize}
    \item \textbf{GPS y navegación}: Tu teléfono usa triangulación (con triángulos oblicuángulos) para determinar tu posición exacta usando señales de al menos tres satélites.

    \item \textbf{Videojuegos y animación}: Los gráficos 3D se construyen con millones de triángulos. Las leyes que aprendiste se usan para calcular iluminación, sombras y perspectiva.

    \item \textbf{Astronomía}: Los astrónomos usan estas leyes para medir distancias a estrellas y planetas usando la técnica del paralaje.

    \item \textbf{Rescate y emergencias}: Los equipos de rescate usan triangulación para localizar señales de emergencia y planificar rutas de búsqueda.

    \item \textbf{Deportes}: En deportes como el golf o el billar, entender los ángulos y las distancias es crucial para la estrategia.
\end{itemize}

\subsection*{Recomendaciones para el éxito}

Para dominar completamente los triángulos oblicuángulos:

\begin{enumerate}
    \item \textbf{Practica con problemas variados}: Cada tipo de problema (LAL, ALA, LLL, etc.) tiene sus trucos.

    \item \textbf{Conecta con la física}: Muchos problemas de fuerzas, velocidades y aceleraciones involucran triángulos oblicuángulos.

    \item \textbf{Explora aplicaciones}: Busca ejemplos en tu entorno. ¿Ves una grúa? Hay triángulos oblicuángulos. ¿Un techo inclinado? Más triángulos.

    \item \textbf{Trabaja en equipo}: Resolver problemas con compañeros te ayuda a ver diferentes enfoques.

    \item \textbf{No memorices, comprende}: Es mejor entender por qué funcionan las leyes que memorizar fórmulas sin sentido.
\end{enumerate}

\subsection*{Un último pensamiento}

Los triángulos oblicuángulos pueden parecer más complicados que los rectángulos, pero en realidad son más representativos del mundo real. La naturaleza raramente nos da ángulos rectos perfectos. Las montañas, los árboles, las trayectorias de vuelo, las órbitas planetarias... todo involucra triángulos oblicuángulos.

Al dominar estas herramientas matemáticas, no solo estás aprendiendo a resolver problemas en papel. Estás desarrollando una forma de pensar que te permite entender y modelar el mundo que te rodea. Desde el diseño de un simple techo hasta la navegación espacial, los conceptos que has aprendido hoy son fundamentales.

\subsection*{¿Qué sigue?}

En las siguientes partes de esta guía, pondremos en práctica todo lo aprendido con:
\begin{itemize}
    \item Ejemplos resueltos paso a paso de cada tipo de problema
    \item Ejercicios progresivos para desarrollar tu habilidad
    \item Problemas de aplicación del mundo real
    \item Proyectos integradores que combinen todos los conceptos
\end{itemize}

Recuerda: las matemáticas son como un deporte mental. Mientras más practiques, más fuerte y ágil será tu mente matemática. ¡Sigue adelante y conquista esos triángulos!

\vspace{2cm}

\begin{center}
\Large\textit{``En cada triángulo hay una historia de ángulos y distancias\\
esperando ser descubierta.''}\\
\vspace{0.5cm}
\normalsize
--- Adaptado para estudiantes de Grado 10
\end{center}

\end{document}