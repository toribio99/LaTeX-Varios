% !TEX program = lualatex
\documentclass[12pt,a4paper]{article}
\usepackage{fontspec}
\usepackage[spanish,es-nodecimaldot]{babel}
\usepackage{amsmath,amssymb}
\usepackage[margin=2.5cm]{geometry}
\usepackage{xcolor}
\usepackage{tikz,pgfplots}
\usetikzlibrary{calc,arrows.meta,babel}
\usepackage{multicol}
\usepackage{enumitem}
\pgfplotsset{compat=1.18}
\definecolor{maincolor}{RGB}{26,35,126}
\definecolor{accentcolor}{RGB}{255,87,34}

% Configuración de títulos y formato
\usepackage{titlesec}
\titleformat{\section}{\Large\bfseries\color{maincolor}}{\thesection}{1em}{}
\titleformat{\subsection}{\large\bfseries\color{accentcolor}}{\thesubsection}{1em}{}

% Configuración de cajas para ejemplos
\usepackage{tcolorbox}
\tcbuselibrary{skins,breakable}

\usepackage{fancyhdr}

\pagestyle{fancy}
\fancyhf{}
\fancyhead[LE]{\small\textcolor{maincolor}{\thepage \quad Funciones Trigonométricas}}
\fancyhead[RO]{\small\textcolor{maincolor}{Funciones Trigonométricas \quad \thepage}}
\fancyhead[LO]{\small\textcolor{maincolor}{Grado 10 - Trigonometría}}
\fancyhead[RE]{\small\textcolor{maincolor}{Prof. Toribio De J Arrieta F}}
\fancyfoot[C]{}
\renewcommand{\headrulewidth}{0.5pt}
\renewcommand{\footrulewidth}{0pt}
\setlength{\headheight}{14pt}

\newtcolorbox{ejemplo}[1][]{
  enhanced,
  breakable,
  colback=maincolor!5,
  colframe=maincolor,
  fonttitle=\bfseries,
  title=Ejemplo Resuelto,
  #1
}

\newtcolorbox{ejercicio}[1][]{
  enhanced,
  breakable,
  colback=accentcolor!5,
  colframe=accentcolor,
  fonttitle=\bfseries,
  title=Ejercicio,
  #1
}

\newtcolorbox{solucion}[1][]{
  enhanced,
  breakable,
  colback=green!5,
  colframe=green!60!black,
  fonttitle=\bfseries,
  title=Solución,
  #1
}

\newtcolorbox{nota}[1][]{
  enhanced,
  colback=yellow!10,
  colframe=orange!80!black,
  fonttitle=\bfseries,
  title=Nota Importante,
  #1
}

% Título
\title{\textbf{\Huge Funciones Trigonométricas}\\[0.5cm]
\Large Guía de Trigonometría}
\author{Prof. Toribio De J Arrieta F\\
\textit{La Pruebita}\\
Grado 10}
\date{\today}

\begin{document}

\maketitle

\tableofcontents
\newpage

\section{Introducción}

¡Bienvenidos al fascinante mundo de las funciones trigonométricas! Si alguna vez te has preguntado cómo los ingenieros diseñan puentes, cómo los músicos analizan ondas sonoras, o cómo los astrónomos predicen el movimiento de los planetas, la respuesta está en la trigonometría.

Las funciones trigonométricas son herramientas matemáticas super poderosas que nos permiten relacionar ángulos con razones numéricas. Aunque nacieron estudiando triángulos (de ahí viene su nombre: \textit{trigon} = triángulo, \textit{metría} = medida), hoy en día se usan para modelar cualquier fenómeno que se repite periódicamente: las olas del mar, el movimiento de un péndulo, las señales de radio, ¡y hasta el ritmo de tu corazón!

\subsection*{¿Por qué son importantes?}

Las funciones trigonométricas aparecen en todas partes:

\begin{itemize}
    \item \textbf{Física:} Movimiento circular, ondas, oscilaciones, luz, sonido
    \item \textbf{Ingeniería:} Diseño de estructuras, análisis de señales eléctricas, navegación
    \item \textbf{Astronomía:} Órbitas planetarias, posición de estrellas
    \item \textbf{Medicina:} Electrocardiogramas, análisis de ritmos biológicos
    \item \textbf{Música:} Análisis de frecuencias, síntesis de sonido
    \item \textbf{Tecnología:} GPS, telecomunicaciones, procesamiento de imágenes
\end{itemize}

\subsection*{¿Qué vamos a aprender?}

En esta guía vamos a explorar:
\begin{enumerate}
    \item La \textbf{circunferencia unitaria}: el círculo mágico de radio 1
    \item Las \textbf{seis funciones trigonométricas}: seno, coseno, tangente y sus recíprocas
    \item Cómo calcular valores para ángulos especiales
    \item Las propiedades y relaciones entre estas funciones
    \item Aplicaciones prácticas que te harán decir: ``¡Eso se usa para esto!''
\end{enumerate}

Prepárate para cambiar tu forma de ver los ángulos. No solo son medidas abstractas, ¡son las llaves que abren las puertas del universo matemático!

\newpage

\section{Conceptos Fundamentales}

\subsection{La Circunferencia Unitaria}

La circunferencia unitaria es el corazón de la trigonometría moderna. Es simplemente un círculo con radio igual a 1, centrado en el origen del plano cartesiano.

\begin{center}
\begin{tikzpicture}[scale=3]
    % Ejes
    \draw[-{Latex},thick] (-1.3,0) -- (1.3,0) node[right] {$x$};
    \draw[-{Latex},thick] (0,-1.3) -- (0,1.3) node[above] {$y$};

    % Circunferencia unitaria
    \draw[maincolor,very thick] (0,0) circle (1);

    % Radio = 1
    \draw[accentcolor,thick,-{Latex}] (0,0) -- (1,0) node[midway,below] {$r=1$};

    % Punto en la circunferencia
    \def\angulo{40}
    \coordinate (P) at ({\angulo}:1);
    \filldraw[maincolor] (P) circle (0.03) node[above right] {$(x,y)$};

    % Radio hacia el punto
    \draw[accentcolor,thick,-{Latex}] (0,0) -- (P);

    % Ángulo
    \draw[accentcolor,-{Latex}] (0.3,0) arc (0:\angulo:0.3) node[midway,right] {$\theta$};

    % Proyecciones
    \draw[dashed,gray] (P) -- ({\angulo}:1 |- 0,0) node[below] {$x$};
    \draw[dashed,gray] (P) -- (0,0 -| {\angulo}:1) node[left] {$y$};

    % Marcas en los ejes
    \foreach \x in {-1,1}
        \draw (\x,0.05) -- (\x,-0.05) node[below] {$\x$};
    \foreach \y in {-1,1}
        \draw (0.05,\y) -- (-0.05,\y) node[left] {$\y$};
\end{tikzpicture}
\end{center}

\begin{nota}
La ecuación de la circunferencia unitaria es: $x^2 + y^2 = 1$
\end{nota}

\subsection{Puntos en la Circunferencia Unitaria}

Aquí viene la parte genial: cualquier punto $(x,y)$ que esté sobre la circunferencia unitaria tiene coordenadas especiales. Si trazamos un radio desde el origen hasta ese punto, y ese radio forma un ángulo $\theta$ con el eje $x$ positivo, entonces:

\[
\boxed{(x,y) = (\cos\theta, \sin\theta)}
\]

¡Sí, leíste bien! La coordenada $x$ del punto ES el coseno del ángulo, y la coordenada $y$ ES el seno del ángulo. Esta es la definición más fundamental de las funciones trigonométricas.

\begin{center}
\begin{tikzpicture}[scale=3.5]
    % Circunferencia
    \draw[maincolor,very thick] (0,0) circle (1);
    \draw[-{Latex},thick] (-1.2,0) -- (1.2,0) node[right] {$x$};
    \draw[-{Latex},thick] (0,-1.2) -- (0,1.2) node[above] {$y$};

    % Varios puntos importantes
    \foreach \ang/\etiqueta/\pos in {0/{(1,0)}/below, 30/{$(\cos 30°, \sin 30°)$}/right,
    60/{$(\cos 60°, \sin 60°)$}/right,
    90/{$(0,1)$}/left, 120/{$(\cos 120°, \sin 120°)$}/left,
    150/{$(\cos 150°, \sin 150°)$}/left} {
        \filldraw[accentcolor] (\ang:1) circle (0.02);
        \node[\pos] at (\ang:1.15) {\tiny \etiqueta};
    }

    % Ángulo de 30 grados destacado
    \draw[accentcolor,thick,-{Latex}] (0,0) -- (30:1);
    \draw[accentcolor,-{Latex}] (0.3,0) arc (0:30:0.3) node[midway,right] {\tiny $30°$};

    % Proyecciones
    \draw[dashed,gray] (30:1) -- (30:1 |- 0,0) node[below left] at(0.7,-0.025){\tiny $\cos 30°$};
    \draw[dashed,gray] (30:1) -- (0,0 -| 30:1) node[right] at(0.6,0.2){\tiny $\sin 30°$};
\end{tikzpicture}
\end{center}

\subsection{Definición de las Funciones Trigonométricas}

Las funciones trigonométricas se pueden definir de dos maneras equivalentes: usando triángulos rectángulos o usando la circunferencia unitaria. Veamos ambas.

\subsubsection{Usando el Triángulo Rectángulo}

Para un ángulo agudo $\theta$ en un triángulo rectángulo:

\begin{center}
\begin{tikzpicture}[scale=2]
    % Triángulo
    \coordinate (A) at (0,0);
    \coordinate (B) at (3,0);
    \coordinate (C) at (3,2);

    \draw[thick] (A) -- (B) node[midway,below] {cateto adyacente} -- (C) node[midway,right] {cateto opuesto} -- cycle node[midway,above left] {hipotenusa};

    % Ángulo recto
    \draw (2.8,0) -- (2.8,0.2) -- (3,0.2);

    % Ángulo theta
    \draw[accentcolor,-{Latex}] (0.5,0) arc (0:33.69:0.5) node[midway,right] {$\theta$};

    % Etiquetas
    \node at (A) [below left] {$A$};
    \node at (B) [below right] {$B$};
    \node at (C) [above right] {$C$};
\end{tikzpicture}
\end{center}

\begin{align*}
\sin\theta &= \frac{\text{cateto opuesto}}{\text{hipotenusa}} \\[0.3cm]
\cos\theta &= \frac{\text{cateto adyacente}}{\text{hipotenusa}} \\[0.3cm]
\tan\theta &= \frac{\text{cateto opuesto}}{\text{cateto adyacente}} = \frac{\sin\theta}{\cos\theta}
\end{align*}

\subsubsection{Usando la Circunferencia Unitaria}

En la circunferencia unitaria, para cualquier ángulo $\theta$ (no solo agudos):

\begin{center}
\begin{tikzpicture}[scale=3]
    \draw[maincolor,very thick] (0,0) circle (1);
    \draw[-{Latex},thick] (-1.2,0) -- (1.2,0) node[right] {$x$};
    \draw[-{Latex},thick] (0,-1.2) -- (0,1.2) node[above] {$y$};

    \def\angulo{50}
    \coordinate (P) at ({\angulo}:1);

    % Triángulo en el círculo
    \draw[thick] (0,0) -- ({\angulo}:1 |- 0,0) -- (P) -- cycle;

    % Radio
    \draw[accentcolor,very thick,-{Latex}] (0,0) -- (P) node[midway,above,sloped] {$r=1$};

    % Punto
    \filldraw[maincolor] (P) circle (0.03) node[above right] {$P(\cos\theta, \sin\theta)$};

    % Ángulo
    \draw[accentcolor,-{Latex}] (0.3,0) arc (0:\angulo:0.3) node[midway,right] {$\theta$};

    % Proyecciones con etiquetas
    \draw[very thick,blue,-{Latex}] (0,0) -- ({\angulo}:1 |- 0,0) node[midway,below] {$\cos\theta$};
    \draw[very thick,red,-{Latex}] ({\angulo}:1 |- 0,0) -- (P) node[midway,right] {$\sin\theta$};

    % Tangente (línea desde (1,0) hasta que toca el radio extendido)
    \coordinate (T) at (1,{tan(\angulo)});
    \draw[dashed] (0,0) -- (1.3,{1.3*tan(\angulo)});
    \draw[very thick,green!60!black] (1,0) -- (T) node[midway,right] {$\tan\theta$};
    \filldraw[green!60!black] (T) circle (0.02);
\end{tikzpicture}
\end{center}

\begin{align*}
\sin\theta &= \text{coordenada } y \text{ del punto en el círculo} \\[0.3cm]
\cos\theta &= \text{coordenada } x \text{ del punto en el círculo} \\[0.3cm]
\tan\theta &= \frac{\sin\theta}{\cos\theta} = \frac{y}{x}
\end{align*}

\subsection{Las Seis Funciones Trigonométricas}

Además de seno, coseno y tangente, existen tres funciones más que son sus recíprocas:

\begin{center}
\renewcommand{\arraystretch}{2}
\begin{tabular}{|c|c|c|}
\hline
\textbf{Función} & \textbf{Definición} & \textbf{Relación} \\
\hline
Seno & $\sin\theta = \frac{y}{r}$ & --- \\
\hline
Coseno & $\cos\theta = \frac{x}{r}$ & --- \\
\hline
Tangente & $\tan\theta = \frac{y}{x}$ & $\tan\theta = \frac{\sin\theta}{\cos\theta}$ \\
\hline
\hline
Cosecante & $\csc\theta = \frac{r}{y}$ & $\csc\theta = \frac{1}{\sin\theta}$ \\
\hline
Secante & $\sec\theta = \frac{r}{x}$ & $\sec\theta = \frac{1}{\cos\theta}$ \\
\hline
Cotangente & $\cot\theta = \frac{x}{y}$ & $\cot\theta = \frac{1}{\tan\theta} = \frac{\cos\theta}{\sin\theta}$ \\
\hline
\end{tabular}
\end{center}

En la circunferencia unitaria, donde $r=1$, estas definiciones se simplifican aún más.

\subsection{Identidades Recíprocas}

Las relaciones entre las funciones y sus recíprocas son fundamentales:

\[
\boxed{
\begin{aligned}
\sin\theta \cdot \csc\theta &= 1 \\[0.2cm]
\cos\theta \cdot \sec\theta &= 1 \\[0.2cm]
\tan\theta \cdot \cot\theta &= 1
\end{aligned}
}
\]

\textbf{Demostración:}
\begin{align*}
\sin\theta \cdot \csc\theta &= \sin\theta \cdot \frac{1}{\sin\theta} = 1 \quad \checkmark \\[0.2cm]
\cos\theta \cdot \sec\theta &= \cos\theta \cdot \frac{1}{\cos\theta} = 1 \quad \checkmark \\[0.2cm]
\tan\theta \cdot \cot\theta &= \frac{\sin\theta}{\cos\theta} \cdot \frac{\cos\theta}{\sin\theta} = 1 \quad \checkmark
\end{align*}

\subsection{Ángulos Cuadrantales}

Los ángulos cuadrantales son aquellos cuyos lados terminales coinciden con los ejes coordenados: $0°$, $90°$, $180°$, $270°$ y $360°$.

\begin{center}
\begin{tikzpicture}[scale=3]
    \draw[maincolor,very thick] (0,0) circle (1);
    \draw[-{Latex},thick] (-1.3,0) -- (1.3,0) node[right] {$x$};
    \draw[-{Latex},thick] (0,-1.3) -- (0,1.3) node[above] {$y$};

    % Puntos cuadrantales
    \filldraw[red] (1,0) circle (0.03) node[below right] {$0°$ $(360°)$};
    \filldraw[red] (0,1) circle (0.03) node[above left] {$90°$};
    \filldraw[red] (-1,0) circle (0.03) node[above left] {$180°$};
    \filldraw[red] (0,-1) circle (0.03) node[below left] {$270°$};

    % Cuadrantes
    \node[maincolor] at (0.5,0.5) {\small I};
    \node[maincolor] at (-0.5,0.5) {\small II};
    \node[maincolor] at (-0.5,-0.5) {\small III};
    \node[maincolor] at (0.5,-0.5) {\small IV};
\end{tikzpicture}
\end{center}

Veamos los valores de las funciones trigonométricas para estos ángulos:

\begin{center}
\renewcommand{\arraystretch}{1.5}
\begin{tabular}{|c|c|c|c|c|c|c|c|}
\hline
$\theta$ & $(x,y)$ & $\sin\theta$ & $\cos\theta$ & $\tan\theta$ & $\csc\theta$ & $\sec\theta$ & $\cot\theta$ \\
\hline
$0°$ & $(1,0)$ & $0$ & $1$ & $0$ & Indef. & $1$ & Indef. \\
\hline
$90°$ & $(0,1)$ & $1$ & $0$ & Indef. & $1$ & Indef. & $0$ \\
\hline
$180°$ & $(-1,0)$ & $0$ & $-1$ & $0$ & Indef. & $-1$ & Indef. \\
\hline
$270°$ & $(0,-1)$ & $-1$ & $0$ & Indef. & $-1$ & Indef. & $0$ \\
\hline
$360°$ & $(1,0)$ & $0$ & $1$ & $0$ & Indef. & $1$ & Indef. \\
\hline
\end{tabular}
\end{center}

\begin{nota}
``Indef.'' significa que la función no está definida para ese ángulo (porque implicaría dividir por cero).
\end{nota}

\subsection{Signos de las Funciones en Cada Cuadrante}

El signo de cada función trigonométrica depende del cuadrante en el que se encuentre el ángulo. Recuerda que en la circunferencia unitaria:
\begin{itemize}
    \item $\sin\theta = y$ (positivo arriba, negativo abajo)
    \item $\cos\theta = x$ (positivo a la derecha, negativo a la izquierda)
    \item $\tan\theta = \frac{y}{x}$ (depende de los signos de $x$ e $y$)
\end{itemize}

\begin{center}
\begin{tikzpicture}[scale=3.5]
    \draw[-{Latex},thick] (-1.3,0) -- (1.3,0) node[right] {$x$};
    \draw[-{Latex},thick] (0,-1.3) -- (0,1.3) node[above] {$y$};
    \draw[maincolor,very thick] (0,0) circle (1);

    % Cuadrante I
    \node[align=center,scale=.85] at (0.35,0.35) {
        \textbf{Cuadrante I} \\
        $x>0, y>0$ \\
        \color{green!60!black}$\sin(+)$ \\
        \color{green!60!black}$\cos(+)$ \\
        \color{green!60!black}$\tan(+)$
    };

    % Cuadrante II
    \node[align=center, scale=.85] at (-0.35,0.35) {
        \textbf{Cuadrante II} \\
        $x<0, y>0$ \\
        \color{green!60!black}$\sin(+)$ \\
        \color{red}$\cos(-)$ \\
        \color{red}$\tan(-)$
    };

    % Cuadrante III
    \node[align=center,scale=.85] at (-0.35,-0.35) {
        \textbf{Cuadrante III} \\
        $x<0, y<0$ \\
        \color{red}$\sin(-)$ \\
        \color{red}$\cos(-)$ \\
        \color{green!60!black}$\tan(+)$
    };

    % Cuadrante IV
    \node[align=center,scale=.85] at (0.37,-0.35) {
        \textbf{Cuadrante IV} \\
        $x>0, y<0$ \\
        \color{red}$\sin(-)$ \\
        \color{green!60!black}$\cos(+)$ \\
        \color{red}$\tan(-)$
    };
\end{tikzpicture}
\end{center}

\textbf{Truco mnemotécnico:} ``\textbf{T}odo \textbf{S}tudent \textbf{T}akes \textbf{C}alculus''
\begin{itemize}
    \item Cuadrante I: \textbf{T}odo es positivo
    \item Cuadrante II: Solo \textbf{S}eno (y su recíproca cosecante) es positivo
    \item Cuadrante III: Solo \textbf{T}angente (y su recíproca cotangente) es positivo
    \item Cuadrante IV: Solo \textbf{C}oseno (y su recíproca secante) es positivo
\end{itemize}

\subsection{Funciones Pares e Impares}

Las funciones trigonométricas tienen propiedades de simetría importantes:

\begin{nota}[title=Función Par]
Una función $f$ es PAR si $f(-x) = f(x)$ para todo $x$. Geométricamente, es simétrica respecto al eje $y$.

\textbf{El coseno es una función par:}
\[
\boxed{\cos(-\theta) = \cos\theta}
\]
\end{nota}

\begin{nota}[title=Función Impar]
Una función $f$ es IMPAR si $f(-x) = -f(x)$ para todo $x$. Geométricamente, es simétrica respecto al origen.

\textbf{El seno es una función impar:}
\[
\boxed{\sin(-\theta) = -\sin\theta}
\]
\end{nota}

\textbf{Verificación geométrica:}

\begin{center}
\begin{tikzpicture}[scale=3]
    \draw[maincolor,very thick] (0,0) circle (1);
    \draw[-{Latex},thick] (-1.3,0) -- (1.3,0) node[right] {$x$};
    \draw[-{Latex},thick] (0,-1.3) -- (0,1.3) node[above] {$y$};

    \def\angulo{40}

    % Ángulo positivo
    \coordinate (P1) at ({\angulo}:1);
    \filldraw[blue] (P1) circle (0.02);
    \draw[blue,thick,-{Latex}] (0,0) -- (P1);
    \draw[blue,-{Latex}] (0.3,0) arc (0:\angulo:0.3) node[midway,right] {\tiny $\theta$};
    \node[blue,above right] at (P1) {\tiny $(\cos\theta, \sin\theta)$};

    % Ángulo negativo
    \coordinate (P2) at ({-\angulo}:1);
    \filldraw[red] (P2) circle (0.02);
    \draw[red,thick,-{Latex}] (0,0) -- (P2);
    \draw[red,-{Latex}] (0.3,0) arc (0:-\angulo:0.3) node[midway,right] {\tiny $-\theta$};
    \node[red,below right] at (P2) {\tiny $(\cos\theta, -\sin\theta)$};

    % Proyecciones
    \draw[dashed,blue] (P1) -- ({\angulo}:1 |- 0,0);
    \draw[dashed,red] (P2) -- ({-\angulo}:1 |- 0,0);
\end{tikzpicture}
\end{center}

Observa que los puntos $(\cos\theta, \sin\theta)$ y $(\cos(-\theta), \sin(-\theta))$ tienen:
\begin{itemize}
    \item La misma coordenada $x$: $\cos(-\theta) = \cos\theta$
    \item Coordenadas $y$ opuestas: $\sin(-\theta) = -\sin\theta$
\end{itemize}

Para la tangente (función impar):
\[
\tan(-\theta) = \frac{\sin(-\theta)}{\cos(-\theta)} = \frac{-\sin\theta}{\cos\theta} = -\tan\theta
\]

\subsection{Aplicaciones Prácticas}

Las funciones trigonométricas modelan fenómenos periódicos en múltiples disciplinas:

\subsubsection{Ondas Sonoras}

El sonido viaja como ondas que pueden representarse mediante funciones seno o coseno:
\[
y(t) = A\sin(2\pi ft + \phi)
\]
donde $A$ es la amplitud (volumen), $f$ es la frecuencia (tono), y $\phi$ es la fase inicial.

\subsubsection{Movimiento Circular}

Un objeto que se mueve en círculo tiene coordenadas:
\[
x(t) = r\cos(\omega t), \quad y(t) = r\sin(\omega t)
\]
donde $r$ es el radio y $\omega$ es la velocidad angular.

\subsubsection{Mareas}

La altura del agua en función del tiempo se modela con:
\[
h(t) = h_0 + A\cos\left(\frac{2\pi}{T}t\right)
\]
donde $h_0$ es el nivel medio, $A$ es la amplitud de la marea, y $T$ es el período (aproximadamente 12.4 horas).

\subsubsection{Señales Eléctricas}

La corriente alterna (AC) que llega a tu casa varía según:
\[
I(t) = I_0\sin(2\pi \cdot 60 \cdot t)
\]
En la mayoría de países, la frecuencia es 60 Hz.

\subsubsection{Movimiento Armónico Simple}

Un péndulo o un resorte oscilando siguen:
\[
x(t) = A\cos(\omega t + \phi)
\]

\subsubsection{Astronomía}

La posición de planetas en órbitas elípticas se calcula usando funciones trigonométricas. La ecuación del tiempo (diferencia entre tiempo solar real y tiempo medio) también es una combinación de funciones trigonométricas.

\subsubsection{Ingeniería Civil}

El diseño de puentes colgantes, la distribución de fuerzas en estructuras, y el análisis de vibraciones requieren funciones trigonométricas.

\newpage

\section{Ejemplos Resueltos}

Ahora vamos a poner en práctica todo lo que hemos aprendido. Cada ejemplo está completamente desarrollado para que entiendas el proceso.

\begin{ejemplo}[title=Ejemplo 1: Encontrar las seis funciones trigonométricas]
Dado que el punto $P\left(\frac{3}{5}, \frac{4}{5}\right)$ está sobre la circunferencia unitaria, encuentra los valores de las seis funciones trigonométricas del ángulo $\theta$ formado por el radio $\overline{OP}$ y el eje $x$ positivo.

\vspace{0.3cm}
\textbf{Solución:}

\textbf{Paso 1:} Verificar que el punto está en la circunferencia unitaria.
\begin{align*}
x^2 + y^2 &= \left(\frac{3}{5}\right)^2 + \left(\frac{4}{5}\right)^2 \\
&= \frac{9}{25} + \frac{16}{25} \\
&= \frac{25}{25} = 1 \quad \checkmark
\end{align*}

\textbf{Paso 2:} Identificar las coordenadas.

En la circunferencia unitaria: $(x,y) = (\cos\theta, \sin\theta)$

Por lo tanto:
\[
\cos\theta = \frac{3}{5}, \quad \sin\theta = \frac{4}{5}
\]

\textbf{Paso 3:} Calcular la tangente.
\[
\tan\theta = \frac{\sin\theta}{\cos\theta} = \frac{4/5}{3/5} = \frac{4}{5} \cdot \frac{5}{3} = \frac{4}{3}
\]

\textbf{Paso 4:} Calcular las funciones recíprocas.
\begin{align*}
\csc\theta &= \frac{1}{\sin\theta} = \frac{1}{4/5} = \frac{5}{4} \\[0.2cm]
\sec\theta &= \frac{1}{\cos\theta} = \frac{1}{3/5} = \frac{5}{3} \\[0.2cm]
\cot\theta &= \frac{1}{\tan\theta} = \frac{1}{4/3} = \frac{3}{4}
\end{align*}

\textbf{Respuesta final:}
\[
\boxed{
\begin{aligned}
\sin\theta &= \frac{4}{5}, \quad \cos\theta = \frac{3}{5}, \quad \tan\theta = \frac{4}{3} \\[0.2cm]
\csc\theta &= \frac{5}{4}, \quad \sec\theta = \frac{5}{3}, \quad \cot\theta = \frac{3}{4}
\end{aligned}
}
\]
\end{ejemplo}

\begin{ejemplo}[title=Ejemplo 2: Determinar el cuadrante y signos]
Si $\sin\theta = \frac{2}{3}$ y $\cos\theta < 0$, determina:
\begin{itemize}
    \item[a)] En qué cuadrante está el ángulo $\theta$
    \item[b)] El valor de $\cos\theta$
    \item[c)] Los valores de las otras cuatro funciones trigonométricas
\end{itemize}

\vspace{0.3cm}
\textbf{Solución:}

\textbf{Parte a):} Determinar el cuadrante.

\begin{itemize}
    \item $\sin\theta = \frac{2}{3} > 0$ (positivo)
    \item $\cos\theta < 0$ (negativo)
\end{itemize}

El seno es positivo en los cuadrantes I y II.\\
El coseno es negativo en los cuadrantes II y III.

Por lo tanto: $\theta$ está en el \textbf{Cuadrante II}.

\textbf{Parte b):} Encontrar $\cos\theta$.

Usamos la identidad fundamental: $\sin^2\theta + \cos^2\theta = 1$

\begin{align*}
\left(\frac{2}{3}\right)^2 + \cos^2\theta &= 1 \\
\frac{4}{9} + \cos^2\theta &= 1 \\
\cos^2\theta &= 1 - \frac{4}{9} = \frac{9-4}{9} = \frac{5}{9} \\
\cos\theta &= \pm\frac{\sqrt{5}}{3}
\end{align*}

Como sabemos que $\cos\theta < 0$:
\[
\boxed{\cos\theta = -\frac{\sqrt{5}}{3}}
\]

\textbf{Parte c):} Calcular las otras funciones.

\begin{align*}
\tan\theta &= \frac{\sin\theta}{\cos\theta} = \frac{2/3}{-\sqrt{5}/3} = \frac{2}{3} \cdot \frac{3}{-\sqrt{5}} = -\frac{2}{\sqrt{5}} = -\frac{2\sqrt{5}}{5} \\[0.3cm]
\csc\theta &= \frac{1}{\sin\theta} = \frac{1}{2/3} = \frac{3}{2} \\[0.3cm]
\sec\theta &= \frac{1}{\cos\theta} = \frac{1}{-\sqrt{5}/3} = -\frac{3}{\sqrt{5}} = -\frac{3\sqrt{5}}{5} \\[0.3cm]
\cot\theta &= \frac{1}{\tan\theta} = -\frac{\sqrt{5}}{2} \quad \text{o} \quad \cot\theta = \frac{\cos\theta}{\sin\theta} = \frac{-\sqrt{5}/3}{2/3} = -\frac{\sqrt{5}}{2}
\end{align*}

\textbf{Respuesta final:}
\[
\boxed{
\begin{aligned}
&\text{a) Cuadrante II} \\
&\text{b) } \cos\theta = -\frac{\sqrt{5}}{3} \\
&\text{c) } \tan\theta = -\frac{2\sqrt{5}}{5}, \quad \csc\theta = \frac{3}{2}, \quad \sec\theta = -\frac{3\sqrt{5}}{5}, \quad \cot\theta = -\frac{\sqrt{5}}{2}
\end{aligned}
}
\]
\end{ejemplo}

\begin{ejemplo}[title=Ejemplo 3: Usar funciones pares e impares]
Simplifica las siguientes expresiones usando las propiedades de funciones pares e impares:

\begin{itemize}
    \item[a)] $\sin(-45°) + \cos(-45°)$
    \item[b)] $\frac{\sin(-\theta)\cos(-\theta)}{\tan(-\theta)}$
\end{itemize}

\vspace{0.3cm}
\textbf{Solución:}

\textbf{Parte a):} $\sin(-45°) + \cos(-45°)$

Recordemos que:
\begin{itemize}
    \item El seno es impar: $\sin(-\theta) = -\sin\theta$
    \item El coseno es par: $\cos(-\theta) = \cos\theta$
\end{itemize}

Entonces:
\begin{align*}
\sin(-45°) + \cos(-45°) &= -\sin(45°) + \cos(45°) \\
&= -\frac{\sqrt{2}}{2} + \frac{\sqrt{2}}{2} \\
&= 0
\end{align*}

\textbf{Respuesta:} $\boxed{0}$

\textbf{Parte b):} $\frac{\sin(-\theta)\cos(-\theta)}{\tan(-\theta)}$

Aplicamos las propiedades:
\begin{align*}
\frac{\sin(-\theta)\cos(-\theta)}{\tan(-\theta)} &= \frac{(-\sin\theta)(\cos\theta)}{-\tan\theta} \\
&= \frac{-\sin\theta \cos\theta}{-\tan\theta} \\
&= \frac{\sin\theta \cos\theta}{\tan\theta}
\end{align*}

Ahora sustituimos $\tan\theta = \frac{\sin\theta}{\cos\theta}$:
\begin{align*}
\frac{\sin\theta \cos\theta}{\tan\theta} &= \frac{\sin\theta \cos\theta}{\sin\theta/\cos\theta} \\
&= \sin\theta \cos\theta \cdot \frac{\cos\theta}{\sin\theta} \\
&= \cos^2\theta
\end{align*}

\textbf{Respuesta:} $\boxed{\cos^2\theta}$
\end{ejemplo}

\begin{ejemplo}[title=Ejemplo 4: Verificar una identidad trigonométrica]
Verifica la identidad: $\tan\theta \cdot \cos\theta = \sin\theta$

\vspace{0.3cm}
\textbf{Solución:}

Para verificar una identidad, trabajamos con uno de los lados (generalmente el más complicado) hasta transformarlo en el otro lado.

\textbf{Partimos del lado izquierdo:}
\begin{align*}
\tan\theta \cdot \cos\theta &= \frac{\sin\theta}{\cos\theta} \cdot \cos\theta \quad \text{(definición de tangente)} \\
&= \sin\theta \cdot \frac{\cos\theta}{\cos\theta} \\
&= \sin\theta \cdot 1 \\
&= \sin\theta \quad \checkmark
\end{align*}

Hemos llegado al lado derecho, por lo tanto la identidad es verdadera.

\textbf{Conclusión:} $\boxed{\text{Identidad verificada}}$
\end{ejemplo}

\begin{ejemplo}[title=Ejemplo 5: Problema aplicado - Movimiento circular]
Una rueda de la fortuna tiene un radio de 20 metros y gira en sentido antihorario. Un pasajero inicia su recorrido en el punto más a la derecha (el punto $(20, 0)$ si ponemos el centro en el origen). Después de girar un ángulo de $120°$, determina:

\begin{itemize}
    \item[a)] La posición $(x, y)$ del pasajero
    \item[b)] La altura del pasajero sobre el centro de la rueda
\end{itemize}

\vspace{0.3cm}
\textbf{Solución:}

\textbf{Paso 1:} Entender el problema.

El pasajero se mueve en un círculo de radio $r = 20$ m. La posición en cualquier punto está dada por:
\[
(x, y) = (r\cos\theta, r\sin\theta)
\]

\textbf{Parte a):} Posición después de $120°$.

\begin{align*}
x &= 20\cos(120°) \\
y &= 20\sin(120°)
\end{align*}

Necesitamos los valores de $\sin(120°)$ y $\cos(120°)$.

$120° = 180° - 60°$ (ángulo en el segundo cuadrante)

En el segundo cuadrante:
\begin{itemize}
    \item $\sin(120°) = \sin(180° - 60°) = \sin(60°) = \frac{\sqrt{3}}{2}$
    \item $\cos(120°) = \cos(180° - 60°) = -\cos(60°) = -\frac{1}{2}$
\end{itemize}

Por lo tanto:
\begin{align*}
x &= 20 \cdot \left(-\frac{1}{2}\right) = -10 \text{ m} \\
y &= 20 \cdot \frac{\sqrt{3}}{2} = 10\sqrt{3} \approx 17.32 \text{ m}
\end{align*}

\textbf{Respuesta a):} $\boxed{(x, y) = (-10, 10\sqrt{3}) \text{ metros}}$

\textbf{Parte b):} Altura sobre el centro.

La altura sobre el centro es simplemente la coordenada $y$:

\textbf{Respuesta b):} $\boxed{h = 10\sqrt{3} \approx 17.32 \text{ metros}}$

\textbf{Interpretación:} El pasajero está a la izquierda del centro (porque $x < 0$) y arriba del centro (porque $y > 0$), a una altura de aproximadamente 17.32 metros.
\end{ejemplo}

\newpage

\section{Ejercicios Propuestos}

Ahora es tu turno. Resuelve los siguientes ejercicios aplicando lo que has aprendido. Las soluciones detalladas están en la siguiente sección, pero intenta resolverlos primero por tu cuenta.

\begin{ejercicio}[title=Ejercicio 1]
El punto $P\left(-\frac{5}{13}, \frac{12}{13}\right)$ está sobre la circunferencia unitaria. Encuentra los valores de las seis funciones trigonométricas del ángulo $\theta$ formado por el radio $\overline{OP}$ y el eje $x$ positivo.
\end{ejercicio}

\begin{ejercicio}[title=Ejercicio 2]
Si $\tan\theta = -\frac{4}{3}$ y $\sin\theta > 0$, determina:
\begin{itemize}
    \item[a)] En qué cuadrante está $\theta$
    \item[b)] Los valores de $\sin\theta$ y $\cos\theta$
    \item[c)] Los valores de $\csc\theta$, $\sec\theta$ y $\cot\theta$
\end{itemize}
\end{ejercicio}

\begin{ejercicio}[title=Ejercicio 3]
Evalúa sin usar calculadora:
\[
\sin(90°) \cdot \cos(0°) + \tan(180°) - \cos(270°)
\]
\end{ejercicio}

\begin{ejercicio}[title=Ejercicio 4]
Simplifica la expresión usando propiedades de funciones pares e impares:
\[
\frac{\cos(-\theta) \cdot \sin(-\theta)}{\tan(-\theta)} + \cos^2(-\theta)
\]
\end{ejercicio}

\begin{ejercicio}[title=Ejercicio 5]
Verifica la identidad:
\[
\frac{\sin\theta}{\cos\theta} + \frac{\cos\theta}{\sin\theta} = \frac{1}{\sin\theta \cos\theta}
\]
\end{ejercicio}

\begin{ejercicio}[title=Ejercicio 6]
Si $\sec\theta = \frac{5}{3}$ y $\theta$ está en el cuarto cuadrante, encuentra los valores de las otras cinco funciones trigonométricas.
\end{ejercicio}

\begin{ejercicio}[title=Ejercicio 7: Problema aplicado]
Un satélite orbita la Tierra en una órbita circular de radio 8000 km (medido desde el centro de la Tierra). Si el satélite inicia en la posición $(8000, 0)$ y gira $210°$ en sentido antihorario, determina:
\begin{itemize}
    \item[a)] Su posición $(x, y)$ en kilómetros
    \item[b)] Su distancia vertical por debajo del centro de la Tierra
\end{itemize}
\end{ejercicio}

\newpage

\section{Soluciones Detalladas}

\begin{solucion}[title=Solución Ejercicio 1]
\textbf{Dado:} $P\left(-\frac{5}{13}, \frac{12}{13}\right)$ en la circunferencia unitaria.

\textbf{Paso 1:} Verificar que está en la circunferencia unitaria.
\begin{align*}
x^2 + y^2 &= \left(-\frac{5}{13}\right)^2 + \left(\frac{12}{13}\right)^2 \\
&= \frac{25}{169} + \frac{144}{169} = \frac{169}{169} = 1 \quad \checkmark
\end{align*}

\textbf{Paso 2:} En la circunferencia unitaria: $(x,y) = (\cos\theta, \sin\theta)$
\[
\cos\theta = -\frac{5}{13}, \quad \sin\theta = \frac{12}{13}
\]

\textbf{Paso 3:} Calcular tangente.
\[
\tan\theta = \frac{\sin\theta}{\cos\theta} = \frac{12/13}{-5/13} = \frac{12}{13} \cdot \frac{13}{-5} = -\frac{12}{5}
\]

\textbf{Paso 4:} Funciones recíprocas.
\begin{align*}
\csc\theta &= \frac{1}{\sin\theta} = \frac{13}{12} \\[0.2cm]
\sec\theta &= \frac{1}{\cos\theta} = \frac{1}{-5/13} = -\frac{13}{5} \\[0.2cm]
\cot\theta &= \frac{1}{\tan\theta} = \frac{1}{-12/5} = -\frac{5}{12}
\end{align*}

\textbf{Respuesta:}
\[
\boxed{
\begin{aligned}
\sin\theta &= \frac{12}{13}, \quad \cos\theta = -\frac{5}{13}, \quad \tan\theta = -\frac{12}{5} \\[0.2cm]
\csc\theta &= \frac{13}{12}, \quad \sec\theta = -\frac{13}{5}, \quad \cot\theta = -\frac{5}{12}
\end{aligned}
}
\]

\textbf{Nota:} El ángulo está en el cuadrante II porque $\sin\theta > 0$ y $\cos\theta < 0$.
\end{solucion}

\begin{solucion}[title=Solución Ejercicio 2]
\textbf{Dado:} $\tan\theta = -\frac{4}{3}$ y $\sin\theta > 0$

\textbf{Parte a):} Determinar el cuadrante.
\begin{itemize}
    \item $\tan\theta = -\frac{4}{3} < 0$ (negativo)
    \item $\sin\theta > 0$ (positivo)
\end{itemize}

La tangente es negativa en los cuadrantes II y IV.\\
El seno es positivo en los cuadrantes I y II.

Por lo tanto: $\boxed{\theta \text{ está en el Cuadrante II}}$

\textbf{Parte b):} Encontrar $\sin\theta$ y $\cos\theta$.

Sabemos que $\tan\theta = \frac{\sin\theta}{\cos\theta} = -\frac{4}{3}$

Esto significa: $\frac{\sin\theta}{\cos\theta} = -\frac{4}{3}$, entonces $\sin\theta = -\frac{4}{3}\cos\theta$

Pero sabemos que $\sin\theta > 0$, así que $\sin\theta = \frac{4}{3}|\cos\theta|$

Como estamos en el cuadrante II, $\cos\theta < 0$, entonces $|\cos\theta| = -\cos\theta$

Por lo tanto: $\sin\theta = -\frac{4}{3}\cos\theta$

Usando $\sin^2\theta + \cos^2\theta = 1$:
\begin{align*}
\left(-\frac{4}{3}\cos\theta\right)^2 + \cos^2\theta &= 1 \\
\frac{16}{9}\cos^2\theta + \cos^2\theta &= 1 \\
\cos^2\theta\left(\frac{16}{9} + 1\right) &= 1 \\
\cos^2\theta \cdot \frac{25}{9} &= 1 \\
\cos^2\theta &= \frac{9}{25} \\
\cos\theta &= \pm\frac{3}{5}
\end{align*}

Como estamos en el cuadrante II: $\cos\theta = -\frac{3}{5}$

Ahora:
\[
\sin\theta = -\frac{4}{3}\cos\theta = -\frac{4}{3} \cdot \left(-\frac{3}{5}\right) = \frac{4}{5}
\]

\textbf{Respuesta b):} $\boxed{\sin\theta = \frac{4}{5}, \quad \cos\theta = -\frac{3}{5}}$

\textbf{Parte c):} Funciones recíprocas.
\begin{align*}
\csc\theta &= \frac{1}{\sin\theta} = \frac{5}{4} \\[0.2cm]
\sec\theta &= \frac{1}{\cos\theta} = -\frac{5}{3} \\[0.2cm]
\cot\theta &= \frac{1}{\tan\theta} = -\frac{3}{4}
\end{align*}

\textbf{Respuesta c):} $\boxed{\csc\theta = \frac{5}{4}, \quad \sec\theta = -\frac{5}{3}, \quad \cot\theta = -\frac{3}{4}}$
\end{solucion}

\begin{solucion}[title=Solución Ejercicio 3]
\textbf{Evaluar:} $\sin(90°) \cdot \cos(0°) + \tan(180°) - \cos(270°)$

Usando la tabla de valores de ángulos cuadrantales:
\begin{align*}
\sin(90°) &= 1 \\
\cos(0°) &= 1 \\
\tan(180°) &= 0 \\
\cos(270°) &= 0
\end{align*}

Sustituyendo:
\begin{align*}
\sin(90°) \cdot \cos(0°) + \tan(180°) - \cos(270°) &= 1 \cdot 1 + 0 - 0 \\
&= 1
\end{align*}

\textbf{Respuesta:} $\boxed{1}$
\end{solucion}

\begin{solucion}[title=Solución Ejercicio 4]
\textbf{Simplificar:} $\frac{\cos(-\theta) \cdot \sin(-\theta)}{\tan(-\theta)} + \cos^2(-\theta)$

\textbf{Paso 1:} Aplicar propiedades de funciones pares e impares.
\begin{itemize}
    \item $\cos(-\theta) = \cos\theta$ (par)
    \item $\sin(-\theta) = -\sin\theta$ (impar)
    \item $\tan(-\theta) = -\tan\theta$ (impar)
\end{itemize}

\begin{align*}
\frac{\cos(-\theta) \cdot \sin(-\theta)}{\tan(-\theta)} + \cos^2(-\theta) &= \frac{\cos\theta \cdot (-\sin\theta)}{-\tan\theta} + \cos^2\theta \\
&= \frac{-\cos\theta \sin\theta}{-\tan\theta} + \cos^2\theta \\
&= \frac{\cos\theta \sin\theta}{\tan\theta} + \cos^2\theta
\end{align*}

\textbf{Paso 2:} Sustituir $\tan\theta = \frac{\sin\theta}{\cos\theta}$
\begin{align*}
\frac{\cos\theta \sin\theta}{\tan\theta} + \cos^2\theta &= \frac{\cos\theta \sin\theta}{\sin\theta/\cos\theta} + \cos^2\theta \\
&= \cos\theta \sin\theta \cdot \frac{\cos\theta}{\sin\theta} + \cos^2\theta \\
&= \cos^2\theta + \cos^2\theta \\
&= 2\cos^2\theta
\end{align*}

\textbf{Respuesta:} $\boxed{2\cos^2\theta}$
\end{solucion}

\begin{solucion}[title=Solución Ejercicio 5]
\textbf{Verificar:} $\frac{\sin\theta}{\cos\theta} + \frac{\cos\theta}{\sin\theta} = \frac{1}{\sin\theta \cos\theta}$

\textbf{Trabajamos con el lado izquierdo:}
\begin{align*}
\frac{\sin\theta}{\cos\theta} + \frac{\cos\theta}{\sin\theta} &= \frac{\sin\theta}{\cos\theta} \cdot \frac{\sin\theta}{\sin\theta} + \frac{\cos\theta}{\sin\theta} \cdot \frac{\cos\theta}{\cos\theta} \\
&= \frac{\sin^2\theta}{\sin\theta\cos\theta} + \frac{\cos^2\theta}{\sin\theta\cos\theta} \\
&= \frac{\sin^2\theta + \cos^2\theta}{\sin\theta\cos\theta}
\end{align*}

Usando la identidad fundamental $\sin^2\theta + \cos^2\theta = 1$:
\[
= \frac{1}{\sin\theta\cos\theta} \quad \checkmark
\]

Hemos llegado al lado derecho.

\textbf{Conclusión:} $\boxed{\text{Identidad verificada}}$
\end{solucion}

\begin{solucion}[title=Solución Ejercicio 6]
\textbf{Dado:} $\sec\theta = \frac{5}{3}$ y $\theta$ en el cuarto cuadrante.

\textbf{Paso 1:} Encontrar $\cos\theta$.
\[
\sec\theta = \frac{1}{\cos\theta} = \frac{5}{3}
\]
Por lo tanto:
\[
\cos\theta = \frac{3}{5}
\]

Esto es consistente con el cuarto cuadrante donde $\cos\theta > 0$. $\checkmark$

\textbf{Paso 2:} Encontrar $\sin\theta$ usando $\sin^2\theta + \cos^2\theta = 1$
\begin{align*}
\sin^2\theta + \left(\frac{3}{5}\right)^2 &= 1 \\
\sin^2\theta + \frac{9}{25} &= 1 \\
\sin^2\theta &= 1 - \frac{9}{25} = \frac{16}{25} \\
\sin\theta &= \pm\frac{4}{5}
\end{align*}

Como estamos en el cuarto cuadrante, $\sin\theta < 0$:
\[
\sin\theta = -\frac{4}{5}
\]

\textbf{Paso 3:} Calcular las otras funciones.
\begin{align*}
\tan\theta &= \frac{\sin\theta}{\cos\theta} = \frac{-4/5}{3/5} = -\frac{4}{3} \\[0.2cm]
\csc\theta &= \frac{1}{\sin\theta} = \frac{1}{-4/5} = -\frac{5}{4} \\[0.2cm]
\cot\theta &= \frac{1}{\tan\theta} = -\frac{3}{4}
\end{align*}

\textbf{Respuesta:}
\[
\boxed{
\begin{aligned}
\sin\theta &= -\frac{4}{5}, \quad \cos\theta = \frac{3}{5}, \quad \tan\theta = -\frac{4}{3} \\[0.2cm]
\csc\theta &= -\frac{5}{4}, \quad \cot\theta = -\frac{3}{4}
\end{aligned}
}
\]
\end{solucion}

\begin{solucion}[title=Solución Ejercicio 7]
\textbf{Dado:} Satélite en órbita circular de radio $r = 8000$ km, gira $210°$ desde $(8000, 0)$.

\textbf{Parte a):} Posición $(x, y)$.

La posición está dada por:
\begin{align*}
x &= r\cos\theta = 8000\cos(210°) \\
y &= r\sin\theta = 8000\sin(210°)
\end{align*}

Necesitamos evaluar $\sin(210°)$ y $\cos(210°)$.

$210° = 180° + 30°$ (ángulo en el tercer cuadrante)

En el tercer cuadrante, tanto seno como coseno son negativos:
\begin{align*}
\sin(210°) &= \sin(180° + 30°) = -\sin(30°) = -\frac{1}{2} \\
\cos(210°) &= \cos(180° + 30°) = -\cos(30°) = -\frac{\sqrt{3}}{2}
\end{align*}

Por lo tanto:
\begin{align*}
x &= 8000 \cdot \left(-\frac{\sqrt{3}}{2}\right) = -4000\sqrt{3} \approx -6928.2 \text{ km} \\
y &= 8000 \cdot \left(-\frac{1}{2}\right) = -4000 \text{ km}
\end{align*}

\textbf{Respuesta a):} $\boxed{(x, y) = (-4000\sqrt{3}, -4000) \approx (-6928.2, -4000) \text{ km}}$

\textbf{Parte b):} Distancia vertical por debajo del centro.

La coordenada $y$ nos da la posición vertical. Como $y = -4000$ km, el satélite está 4000 km por debajo del centro de la Tierra.

\textbf{Respuesta b):} $\boxed{4000 \text{ km por debajo del centro}}$

\textbf{Interpretación:} El satélite está en el tercer cuadrante, a la izquierda y por debajo del centro de la Tierra.
\end{solucion}

\newpage

\section{Ejercicios Inversos}

Los ejercicios inversos te desafían a trabajar ``al revés'': en lugar de darte un ángulo y pedirte las funciones, te dan información sobre las funciones y tú debes encontrar el ángulo o determinar otras propiedades.

\begin{ejercicio}[title=Ejercicio Inverso 1]
Encuentra todos los ángulos $\theta$ en el intervalo $[0°, 360°)$ tales que:
\[
\cos\theta = \frac{1}{2}
\]
\end{ejercicio}

\begin{ejercicio}[title=Ejercicio Inverso 2]
Si un punto $P(x, y)$ está en la circunferencia unitaria en el tercer cuadrante, y su coordenada $x$ es el triple de su coordenada $y$, encuentra:
\begin{itemize}
    \item[a)] Las coordenadas exactas del punto
    \item[b)] Los valores de las seis funciones trigonométricas del ángulo correspondiente
\end{itemize}
\end{ejercicio}

\begin{ejercicio}[title=Ejercicio Inverso 3]
Determina todos los ángulos cuadrantales $\theta$ en $[0°, 360°]$ para los cuales:
\[
\sin\theta \cdot \cos\theta = 0
\]
\end{ejercicio}

\begin{ejercicio}[title=Ejercicio Inverso 4]
Un ingeniero necesita diseñar una rampa circular. El punto final de la rampa debe estar a una altura de 15 metros y a una distancia horizontal de 8 metros del centro (hacia la izquierda). Si la rampa sigue un arco de circunferencia centrado en el origen, determina:
\begin{itemize}
    \item[a)] El radio de la circunferencia
    \item[b)] El ángulo que forma el punto final con el eje $x$ positivo
    \item[c)] Los valores de $\sin\theta$ y $\cos\theta$ para ese ángulo
\end{itemize}
\end{ejercicio}

\newpage

\section{Soluciones de Ejercicios Inversos}

\begin{solucion}[title=Solución Ejercicio Inverso 1]
\textbf{Encontrar:} Todos los $\theta \in [0°, 360°)$ tales que $\cos\theta = \frac{1}{2}$

\textbf{Paso 1:} Identificar el ángulo de referencia.

Sabemos que $\cos(60°) = \frac{1}{2}$. Entonces $60°$ es nuestro ángulo de referencia.

\textbf{Paso 2:} Determinar en qué cuadrantes el coseno es positivo.

El coseno es la coordenada $x$ en la circunferencia unitaria. Es positivo cuando $x > 0$, es decir, en los cuadrantes I y IV.

\textbf{Paso 3:} Encontrar los ángulos en cada cuadrante.

\textbf{Cuadrante I:} El ángulo es directamente el ángulo de referencia:
\[
\theta_1 = 60°
\]

\textbf{Cuadrante IV:} El ángulo se calcula como $360° - \text{ángulo de referencia}$:
\[
\theta_2 = 360° - 60° = 300°
\]

\textbf{Verificación:}
\begin{align*}
\cos(60°) &= \frac{1}{2} \quad \checkmark \\
\cos(300°) &= \cos(360° - 60°) = \cos(60°) = \frac{1}{2} \quad \checkmark
\end{align*}

\textbf{Respuesta:} $\boxed{\theta = 60° \text{ y } \theta = 300°}$

\begin{center}
\begin{tikzpicture}[scale=2.5]
    \draw[maincolor,very thick] (0,0) circle (1);
    \draw[-{Latex},thick] (-1.2,0) -- (1.2,0) node[right] {$x$};
    \draw[-{Latex},thick] (0,-1.2) -- (0,1.2) node[above] {$y$};

    % 60 grados
    \draw[blue,thick,-{Latex}] (0,0) -- (60:1);
    \filldraw[blue] (60:1) circle (0.02) node[above right] {$60°$};
    \draw[blue,-{Latex}] (0.3,0) arc (0:60:0.3) node[midway,right] {\tiny $60°$};

    % 300 grados
    \draw[red,thick,-{Latex}] (0,0) -- (300:1);
    \filldraw[red] (300:1) circle (0.02) node[below right] {$300°$};
    \draw[red,-{Latex}] (0.3,0) arc (0:-60:0.3) node[midway,right] {\tiny $-60°$};

    % Línea vertical en x = 1/2
    \draw[dashed,green!60!black] (0.5,-1) -- (0.5,1) node[above] {$x=\frac{1}{2}$};
\end{tikzpicture}
\end{center}
\end{solucion}

\begin{solucion}[title=Solución Ejercicio Inverso 2]
\textbf{Dado:} Punto $P(x,y)$ en la circunferencia unitaria, tercer cuadrante, $x = 3y$

\textbf{Parte a):} Encontrar las coordenadas.

\textbf{Condiciones:}
\begin{enumerate}
    \item $x^2 + y^2 = 1$ (está en la circunferencia unitaria)
    \item $x = 3y$ (relación dada)
    \item Tercer cuadrante: $x < 0$ y $y < 0$
\end{enumerate}

Sustituyendo la condición 2 en la condición 1:
\begin{align*}
(3y)^2 + y^2 &= 1 \\
9y^2 + y^2 &= 1 \\
10y^2 &= 1 \\
y^2 &= \frac{1}{10} \\
y &= \pm\frac{1}{\sqrt{10}} = \pm\frac{\sqrt{10}}{10}
\end{align*}

Como estamos en el tercer cuadrante, $y < 0$:
\[
y = -\frac{\sqrt{10}}{10}
\]

Entonces:
\[
x = 3y = 3 \cdot \left(-\frac{\sqrt{10}}{10}\right) = -\frac{3\sqrt{10}}{10}
\]

\textbf{Verificación:}
\begin{align*}
x^2 + y^2 &= \left(-\frac{3\sqrt{10}}{10}\right)^2 + \left(-\frac{\sqrt{10}}{10}\right)^2 \\
&= \frac{9 \cdot 10}{100} + \frac{10}{100} = \frac{90 + 10}{100} = \frac{100}{100} = 1 \quad \checkmark
\end{align*}

\textbf{Respuesta a):} $\boxed{P\left(-\frac{3\sqrt{10}}{10}, -\frac{\sqrt{10}}{10}\right)}$

\textbf{Parte b):} Funciones trigonométricas.

En la circunferencia unitaria: $(x,y) = (\cos\theta, \sin\theta)$
\begin{align*}
\cos\theta &= -\frac{3\sqrt{10}}{10} \\[0.2cm]
\sin\theta &= -\frac{\sqrt{10}}{10}
\end{align*}

\begin{align*}
\tan\theta &= \frac{\sin\theta}{\cos\theta} = \frac{-\sqrt{10}/10}{-3\sqrt{10}/10} = \frac{1}{3} \\[0.3cm]
\csc\theta &= \frac{1}{\sin\theta} = \frac{1}{-\sqrt{10}/10} = -\frac{10}{\sqrt{10}} = -\sqrt{10} \\[0.3cm]
\sec\theta &= \frac{1}{\cos\theta} = \frac{1}{-3\sqrt{10}/10} = -\frac{10}{3\sqrt{10}} = -\frac{\sqrt{10}}{3} \\[0.3cm]
\cot\theta &= \frac{1}{\tan\theta} = 3
\end{align*}

\textbf{Respuesta b):}
\[
\boxed{
\begin{aligned}
\sin\theta &= -\frac{\sqrt{10}}{10}, \quad \cos\theta = -\frac{3\sqrt{10}}{10}, \quad \tan\theta = \frac{1}{3} \\[0.2cm]
\csc\theta &= -\sqrt{10}, \quad \sec\theta = -\frac{\sqrt{10}}{3}, \quad \cot\theta = 3
\end{aligned}
}
\]
\end{solucion}

\begin{solucion}[title=Solución Ejercicio Inverso 3]
\textbf{Encontrar:} Ángulos cuadrantales en $[0°, 360°]$ tales que $\sin\theta \cdot \cos\theta = 0$

\textbf{Análisis:}

Un producto es cero si y solo si al menos uno de los factores es cero:
\[
\sin\theta \cdot \cos\theta = 0 \quad \Leftrightarrow \quad \sin\theta = 0 \text{ o } \cos\theta = 0
\]

\textbf{Caso 1:} $\sin\theta = 0$

El seno es la coordenada $y$ en la circunferencia unitaria. Es cero cuando el punto está en el eje $x$.

Ángulos cuadrantales con $\sin\theta = 0$:
\[
\theta = 0°, \quad \theta = 180°, \quad \theta = 360°
\]

\textbf{Caso 2:} $\cos\theta = 0$

El coseno es la coordenada $x$ en la circunferencia unitaria. Es cero cuando el punto está en el eje $y$.

Ángulos cuadrantales con $\cos\theta = 0$:
\[
\theta = 90°, \quad \theta = 270°
\]

\textbf{Unión de ambos casos:}
\[
\boxed{\theta = 0°, 90°, 180°, 270°, 360°}
\]

\textbf{Verificación:}
\begin{align*}
\theta = 0°: \quad &\sin(0°)\cos(0°) = 0 \cdot 1 = 0 \quad \checkmark \\
\theta = 90°: \quad &\sin(90°)\cos(90°) = 1 \cdot 0 = 0 \quad \checkmark \\
\theta = 180°: \quad &\sin(180°)\cos(180°) = 0 \cdot (-1) = 0 \quad \checkmark \\
\theta = 270°: \quad &\sin(270°)\cos(270°) = (-1) \cdot 0 = 0 \quad \checkmark \\
\theta = 360°: \quad &\sin(360°)\cos(360°) = 0 \cdot 1 = 0 \quad \checkmark
\end{align*}

\textbf{Conclusión:} Todos los ángulos cuadrantales satisfacen la ecuación.
\end{solucion}

\begin{solucion}[title=Solución Ejercicio Inverso 4]
\textbf{Dado:} Punto a altura 15 m, distancia horizontal 8 m a la izquierda del centro.

Las coordenadas del punto son: $(x, y) = (-8, 15)$

\textbf{Parte a):} Radio de la circunferencia.

El radio es la distancia del origen al punto:
\begin{align*}
r &= \sqrt{x^2 + y^2} \\
&= \sqrt{(-8)^2 + 15^2} \\
&= \sqrt{64 + 225} \\
&= \sqrt{289} \\
&= 17 \text{ m}
\end{align*}

\textbf{Respuesta a):} $\boxed{r = 17 \text{ metros}}$

\textbf{Parte b):} Ángulo con el eje $x$ positivo.

El punto $(-8, 15)$ está en el segundo cuadrante (porque $x < 0$ y $y > 0$).

Para encontrar el ángulo, primero encontramos el ángulo de referencia usando:
\[
\tan(\alpha) = \frac{|y|}{|x|} = \frac{15}{8}
\]

Entonces: $\alpha = \arctan\left(\frac{15}{8}\right)$

Calculando (o usando tablas): $\alpha \approx 61.93°$

Como el punto está en el segundo cuadrante, el ángulo $\theta$ medido desde el eje $x$ positivo es:
\[
\theta = 180° - \alpha = 180° - 61.93° \approx 118.07°
\]

\textbf{Respuesta b):} $\boxed{\theta \approx 118.07°}$ (o en forma exacta: $\theta = 180° - \arctan\left(\frac{15}{8}\right)$)

\textbf{Parte c):} Valores de $\sin\theta$ y $\cos\theta$.

En una circunferencia de radio $r = 17$, las coordenadas de un punto son:
\[
(x, y) = (r\cos\theta, r\sin\theta)
\]

Por lo tanto:
\begin{align*}
r\cos\theta &= -8 \quad \Rightarrow \quad \cos\theta = -\frac{8}{17} \\[0.2cm]
r\sin\theta &= 15 \quad \Rightarrow \quad \sin\theta = \frac{15}{17}
\end{align*}

\textbf{Verificación:}
\[
\sin^2\theta + \cos^2\theta = \left(\frac{15}{17}\right)^2 + \left(-\frac{8}{17}\right)^2 = \frac{225 + 64}{289} = \frac{289}{289} = 1 \quad \checkmark
\]

\textbf{Respuesta c):} $\boxed{\sin\theta = \frac{15}{17}, \quad \cos\theta = -\frac{8}{17}}$

\textbf{Diagrama:}
\begin{center}
\begin{tikzpicture}[scale=0.3]
    \draw[-{Latex},thick] (-10,0) -- (10,0) node[right] {$x$};
    \draw[-{Latex},thick] (0,-2) -- (0,18) node[above] {$y$};

    % Punto
    \coordinate (P) at (-8,15);
    \filldraw[maincolor] (P) circle (0.2) node[above left] {$(-8, 15)$};

    % Radio
    \draw[accentcolor,very thick,-{Latex}] (0,0) -- (P) node[midway,above,sloped] {$r=17$ m};

    % Ángulo
    \draw[accentcolor,-{Latex}] (3,0) arc (0:118:3) node[midway,above] {$\theta$};

    % Proyecciones
    \draw[dashed] (P) -- (-8,0) node[below] {$-8$};
    \draw[dashed] (P) -- (0,15) node[left] {$15$};
\end{tikzpicture}
\end{center}
\end{solucion}

\newpage

\section{Conclusión}

¡Felicitaciones! Has completado esta guía sobre funciones trigonométricas. Ahora tienes las herramientas para:

\begin{itemize}
    \item Entender la circunferencia unitaria y su relación con las funciones trigonométricas
    \item Calcular las seis funciones trigonométricas para cualquier ángulo
    \item Determinar signos según el cuadrante
    \item Usar propiedades de funciones pares e impares
    \item Aplicar las funciones trigonométricas a problemas reales
\end{itemize}

\subsection*{Resumen de Conceptos Clave}

\begin{tcolorbox}[enhanced,colback=maincolor!10,colframe=maincolor,title=Fórmulas Fundamentales]
\textbf{En la circunferencia unitaria:}
\[
(x, y) = (\cos\theta, \sin\theta)
\]

\textbf{Identidad pitagórica:}
\[
\sin^2\theta + \cos^2\theta = 1
\]

\textbf{Funciones recíprocas:}
\[
\csc\theta = \frac{1}{\sin\theta}, \quad \sec\theta = \frac{1}{\cos\theta}, \quad \cot\theta = \frac{1}{\tan\theta}
\]

\textbf{Funciones pares e impares:}
\[
\cos(-\theta) = \cos\theta, \quad \sin(-\theta) = -\sin\theta, \quad \tan(-\theta) = -\tan\theta
\]
\end{tcolorbox}

\subsection*{Próximos Pasos}

Para continuar tu aprendizaje en trigonometría, los siguientes temas serían:

\begin{enumerate}
    \item Identidades trigonométricas (suma, diferencia, ángulo doble)
    \item Gráficas de funciones trigonométricas
    \item Ecuaciones trigonométricas
    \item Funciones trigonométricas inversas
    \item Aplicaciones avanzadas: vectores, números complejos
\end{enumerate}

\subsection*{Consejo Final}

La trigonometría es como un idioma: se aprende mejor con práctica constante. No te desanimes si algunos problemas parecen difíciles al principio. Con el tiempo, los patrones se vuelven más claros y las soluciones más naturales.

¡Sigue practicando y explorando el fascinante mundo de las funciones trigonométricas!

\vspace{1cm}

\begin{center}
\textit{``La matemática es el alfabeto con el cual Dios ha escrito el universo.''} \\
--- Galileo Galilei
\end{center}

\end{document}
