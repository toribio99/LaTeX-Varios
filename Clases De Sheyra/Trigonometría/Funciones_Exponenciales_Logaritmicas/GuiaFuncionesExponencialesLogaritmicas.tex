% !TEX program = lualatex
\documentclass[12pt,a4paper,twoside]{article}
\usepackage{fontspec}
\usepackage[spanish,es-nodecimaldot]{babel}
\usepackage{amsmath,amssymb}
\usepackage[margin=2.5cm]{geometry}
\usepackage{xcolor}
\usepackage{tikz,pgfplots}
\usetikzlibrary{calc,arrows.meta,babel}
\usepackage{multicol}
\usepackage{enumitem}
\pgfplotsset{compat=1.18}
\definecolor{maincolor}{RGB}{26,35,126}
\definecolor{accentcolor}{RGB}{255,87,34}

\usepackage{tcolorbox}
\tcbuselibrary{breakable,skins}
\usepackage{fancyhdr}

\pagestyle{fancy}
\fancyhf{}
\fancyhead[LE]{\small\textcolor{maincolor}{\thepage \quad Funciones Exponenciales y Logarítmicas}}
\fancyhead[RO]{\small\textcolor{maincolor}{Funciones Exponenciales y Logarítmicas \quad \thepage}}
\fancyhead[LO]{\small\textcolor{maincolor}{Grado 10 - Trigonometría}}
\fancyhead[RE]{\small\textcolor{maincolor}{Prof. Toribio De J Arrieta F}}
\fancyfoot[C]{}
\renewcommand{\headrulewidth}{0.5pt}
\renewcommand{\footrulewidth}{0pt}
\setlength{\headheight}{14pt}

\newtcolorbox{definicion}{
    colback=maincolor!5,
    colframe=maincolor,
    fonttitle=\bfseries,
    title=Definición,
    breakable
}

\newtcolorbox{teorema}{
    colback=accentcolor!5,
    colframe=accentcolor,
    fonttitle=\bfseries,
    title=Propiedad/Teorema,
    breakable
}

\newtcolorbox{ejemplo}{
    colback=green!5,
    colframe=green!70!black,
    fonttitle=\bfseries,
    title=Ejemplo,
    breakable
}

\newtcolorbox{nota}{
    colback=yellow!10,
    colframe=yellow!80!black,
    fonttitle=\bfseries,
    title=Nota Importante,
    breakable
}

\title{
    \vspace{-2cm}
    \Huge\textcolor{maincolor}{\textbf{Funciones Exponenciales y Logarítmicas}}\\
    \vspace{0.5cm}
    \Large Explorando el crecimiento y la escala
}
\author{
    \textbf{Prof. Toribio De J Arrieta F}\\
    \textit{La Pruebita}
}
\date{\today}

\begin{document}

\maketitle

\begin{center}
\textcolor{maincolor}{\textbf{Grado 10 --- Trigonometría}}
\end{center}

\vspace{1cm}

\begin{center}
\begin{tikzpicture}
    \begin{axis}[
        width=10cm,
        height=6cm,
        axis lines=middle,
        xlabel={$x$},
        ylabel={$y$},
        domain=-2:3,
        samples=100,
        grid=major,
        legend pos=north east,
        ymin=-1,
        ymax=8
    ]
    \addplot[blue, thick, smooth] {exp(x)};
    \addplot[red, thick, smooth, domain=0.01:8] {ln(x)};
    \addplot[dashed, gray] {x};
    \legend{$y=e^x$, $y=\ln(x)$, $y=x$}
    \end{axis}
\end{tikzpicture}
\end{center}

\newpage

\tableofcontents

\newpage

\section{Introducción}

¿Alguna vez te has preguntado cómo se propaga un virus, cómo crece el dinero en una cuenta de ahorros, o por qué los terremotos se miden con la escala de Richter? Todas estas situaciones tienen algo en común: involucran funciones exponenciales y logarítmicas.

Las funciones exponenciales son especiales porque describen procesos donde la tasa de cambio es proporcional a la cantidad actual. Imagínate que tienes una colonia de bacterias: entre más bacterias haya, más rápido se reproducen. Ese tipo de crecimiento \textit{acelerado} se modela con funciones exponenciales.

Por otro lado, las funciones logarítmicas son las ``inversas'' de las exponenciales. Mientras que las exponenciales responden preguntas como ``¿A qué valor llega $2^{10}$?'', los logaritmos responden ``¿A qué potencia debo elevar 2 para obtener 1024?''. Los logaritmos son fundamentales para comprimir escalas enormes (como en los decibeles o el pH) en rangos manejables.

\subsection{¿Qué vamos a aprender?}

En esta guía vas a dominar:

\begin{itemize}[leftmargin=*]
    \item \textbf{Funciones exponenciales}: Su definición, propiedades y gráficas.
    \item \textbf{El número $e$}: Un número mágico (aproximadamente 2.71828...) que aparece en todo: desde el interés compuesto hasta la probabilidad.
    \item \textbf{Funciones logarítmicas}: La herramienta para ``deshacer'' exponenciales.
    \item \textbf{Propiedades de logaritmos}: Cómo convertir multiplicaciones en sumas, divisiones en restas, y potencias en productos.
    \item \textbf{Ecuaciones exponenciales y logarítmicas}: Técnicas para resolver problemas que involucran estas funciones.
    \item \textbf{Aplicaciones reales}: Desde el crecimiento poblacional hasta la datación con carbono-14.
\end{itemize}

\subsection{¿Por qué son importantes?}

Las funciones exponenciales y logarítmicas no son solo conceptos abstractos. Están por todas partes:

\begin{itemize}[leftmargin=*]
    \item \textbf{Finanzas}: El interés compuesto hace que tu dinero crezca exponencialmente.
    \item \textbf{Biología}: Las poblaciones crecen exponencialmente bajo condiciones ideales.
    \item \textbf{Física}: El decaimiento radiactivo sigue una función exponencial.
    \item \textbf{Geología}: La escala de Richter usa logaritmos para medir terremotos.
    \item \textbf{Química}: El pH es una escala logarítmica.
    \item \textbf{Acústica}: Los decibeles también usan logaritmos.
\end{itemize}

Prepárate para explorar uno de los temas más fascinantes y útiles de las matemáticas. ¡Vamos a ello!

\newpage

\section{Conceptos Fundamentales}

\subsection{Función Exponencial}

\begin{definicion}
Una \textbf{función exponencial} es una función de la forma:
\[
f(x) = a^x
\]
donde $a$ es una constante positiva llamada \textbf{base} ($a > 0$, $a \neq 1$), y $x$ es el exponente (la variable independiente).
\end{definicion}

\begin{nota}
\textbf{¿Por qué $a > 0$ y $a \neq 1$?}
\begin{itemize}
    \item Si $a \leq 0$, tendríamos problemas con exponentes fraccionarios (como $(-2)^{1/2}$, que no es un número real).
    \item Si $a = 1$, la función sería $f(x) = 1^x = 1$ para todo $x$, lo cual es una función constante y no exponencial.
\end{itemize}
\end{nota}

\subsubsection{Características de las funciones exponenciales}

Para $f(x) = a^x$ con $a > 1$:

\begin{itemize}[leftmargin=*]
    \item \textbf{Dominio}: Todos los números reales, $\mathbb{R}$.
    \item \textbf{Rango}: Todos los números positivos, $(0, \infty)$.
    \item \textbf{Intercepto con el eje $y$}: El punto $(0, 1)$, porque $a^0 = 1$.
    \item \textbf{Comportamiento}:
    \begin{itemize}
        \item Cuando $x \to \infty$, $f(x) \to \infty$ (crece sin límite).
        \item Cuando $x \to -\infty$, $f(x) \to 0$ (se acerca al eje $x$ pero nunca lo toca).
    \end{itemize}
    \item \textbf{Crecimiento}: La función es \textit{creciente} en todo su dominio.
    \item \textbf{Asíntota horizontal}: El eje $x$ (la recta $y = 0$) es una asíntota horizontal.
\end{itemize}

Para $f(x) = a^x$ con $0 < a < 1$:

\begin{itemize}[leftmargin=*]
    \item Las características de dominio, rango e intercepto son las mismas.
    \item \textbf{Comportamiento}:
    \begin{itemize}
        \item Cuando $x \to \infty$, $f(x) \to 0$.
        \item Cuando $x \to -\infty$, $f(x) \to \infty$.
    \end{itemize}
    \item \textbf{Decrecimiento}: La función es \textit{decreciente} en todo su dominio.
\end{itemize}

\subsubsection{Gráficas de funciones exponenciales}

Veamos algunas gráficas para entender mejor el comportamiento:

\begin{center}
\begin{tikzpicture}
    \begin{axis}[
        width=12cm,
        height=8cm,
        axis lines=middle,
        xlabel={$x$},
        ylabel={$y$},
        domain=-3:3,
        samples=100,
        grid=major,
        legend pos=north west,
        ymin=-0.5,
        ymax=8,
        xmin=-3,
        xmax=3
    ]
    \addplot[blue, ultra thick, smooth] {2^x};
    \addplot[red, ultra thick, smooth] {3^x};
    \addplot[green!70!black, ultra thick, smooth] {(1/2)^x};
    \addplot[dashed, gray] coordinates {(-3,1) (3,1)};
    \node[blue] at (axis cs:2.3,3.3) {$y=2^x$};
    \node[red] at (axis cs:1,5) {$y=3^x$};
    \node[green!70!black] at (axis cs:-1.3,4.5) {$y=(1/2)^x$};
    \end{axis}
\end{tikzpicture}
\end{center}

\textbf{Observaciones}:
\begin{itemize}
    \item Todas pasan por $(0,1)$.
    \item $y=2^x$ y $y=3^x$ son crecientes (bases mayores que 1).
    \item $y=(1/2)^x$ es decreciente (base menor que 1).
    \item Entre mayor sea la base, más rápido crece la función.
\end{itemize}

\subsection{El número $e$ (número de Euler)}

\begin{definicion}
El \textbf{número de Euler}, denotado por $e$, es un número irracional aproximadamente igual a:
\[
e \approx 2.718281828459045\ldots
\]
Se define como el límite:
\[
e = \lim_{n \to \infty} \left(1 + \frac{1}{n}\right)^n
\]
\end{definicion}

El número $e$ es la base más importante en matemáticas, especialmente en cálculo. La función $f(x) = e^x$ es conocida como la \textbf{función exponencial natural}.

\subsubsection{¿Por qué es especial $e$?}

\begin{itemize}[leftmargin=*]
    \item En cálculo, la derivada de $e^x$ es ella misma: $\frac{d}{dx}(e^x) = e^x$. ¡Es la única función exponencial con esta propiedad!
    \item Aparece naturalmente en problemas de crecimiento continuo, como el interés compuesto continuo.
    \item Es fundamental en probabilidad, estadística y muchas áreas de la ciencia.
\end{itemize}

\begin{center}
\begin{tikzpicture}
    \begin{axis}[
        width=12cm,
        height=8cm,
        axis lines=middle,
        xlabel={$x$},
        ylabel={$y$},
        domain=-2:3,
        samples=100,
        grid=major,
        legend pos=north west,
        ymin=-1,
        ymax=10,
        xmin=-2,
        xmax=3
    ]
    \addplot[blue, ultra thick, smooth] {exp(x)};
    \addplot[dashed, gray] coordinates {(-2,1) (3,1)};
    \node[blue] at (axis cs:1.5,6) {$y=e^x$};
    \addplot[only marks, red, mark=*, mark size=3pt] coordinates {(0,1) (1,2.718)};
    \end{axis}
\end{tikzpicture}
\end{center}

\subsection{Propiedades de las funciones exponenciales}

\begin{teorema}
Para $a > 0$, $a \neq 1$, y para cualesquiera números reales $x$ e $y$:

\begin{enumerate}[leftmargin=*]
    \item $a^x \cdot a^y = a^{x+y}$ \quad (Producto de potencias de igual base)
    \item $\displaystyle\frac{a^x}{a^y} = a^{x-y}$ \quad (Cociente de potencias de igual base)
    \item $(a^x)^y = a^{xy}$ \quad (Potencia de una potencia)
    \item $(ab)^x = a^x \cdot b^x$ \quad (Potencia de un producto)
    \item $\displaystyle\left(\frac{a}{b}\right)^x = \frac{a^x}{b^x}$ \quad (Potencia de un cociente)
    \item $a^0 = 1$ \quad (Exponente cero)
    \item $a^{-x} = \displaystyle\frac{1}{a^x}$ \quad (Exponente negativo)
\end{enumerate}
\end{teorema}

Estas propiedades son fundamentales para simplificar expresiones y resolver ecuaciones exponenciales.

\newpage

\subsection{Función Logarítmica}

Ya vimos que la función exponencial $f(x) = a^x$ toma un exponente $x$ y nos da un resultado. Ahora, ¿qué pasa si queremos el proceso inverso? Es decir, si sabemos el resultado y queremos encontrar el exponente. Ahí entran los logaritmos.

\begin{definicion}
El \textbf{logaritmo en base $a$} de un número positivo $x$, denotado $\log_a(x)$, es el exponente al que hay que elevar la base $a$ para obtener $x$.

Matemáticamente:
\[
y = \log_a(x) \quad \Longleftrightarrow \quad a^y = x
\]

donde $a > 0$, $a \neq 1$, y $x > 0$.
\end{definicion}

\textbf{En palabras simples}: $\log_a(x)$ responde la pregunta: ``¿A qué potencia debo elevar $a$ para obtener $x$?''

\textbf{Ejemplos}:
\begin{itemize}
    \item $\log_2(8) = 3$ porque $2^3 = 8$.
    \item $\log_{10}(100) = 2$ porque $10^2 = 100$.
    \item $\log_5(25) = 2$ porque $5^2 = 25$.
    \item $\log_3(1/9) = -2$ porque $3^{-2} = 1/9$.
\end{itemize}

\subsubsection{Logaritmo natural y logaritmo común}

Hay dos bases de logaritmos que son especialmente importantes:

\begin{definicion}
\begin{itemize}
    \item \textbf{Logaritmo natural}: Es el logaritmo en base $e$. Se denota $\ln(x)$ en lugar de $\log_e(x)$.
    \[
    \ln(x) = \log_e(x)
    \]

    \item \textbf{Logaritmo común} (o decimal): Es el logaritmo en base 10. A menudo se escribe simplemente como $\log(x)$ (sin especificar la base).
    \[
    \log(x) = \log_{10}(x)
    \]
\end{itemize}
\end{definicion}

\textbf{Nota}: En matemáticas avanzadas y ciencias, $\log(x)$ suele referirse al logaritmo natural. En ingeniería y ciencias aplicadas, $\log(x)$ generalmente significa logaritmo en base 10. Siempre verifica el contexto.

\subsubsection{Características de las funciones logarítmicas}

Para $f(x) = \log_a(x)$ con $a > 1$:

\begin{itemize}[leftmargin=*]
    \item \textbf{Dominio}: Todos los números positivos, $(0, \infty)$.
    \item \textbf{Rango}: Todos los números reales, $\mathbb{R}$.
    \item \textbf{Intercepto con el eje $x$}: El punto $(1, 0)$, porque $\log_a(1) = 0$ (ya que $a^0 = 1$).
    \item \textbf{Comportamiento}:
    \begin{itemize}
        \item Cuando $x \to \infty$, $f(x) \to \infty$ (crece, pero muy lentamente).
        \item Cuando $x \to 0^+$, $f(x) \to -\infty$ (decrece sin límite).
    \end{itemize}
    \item \textbf{Crecimiento}: La función es \textit{creciente} en todo su dominio.
    \item \textbf{Asíntota vertical}: El eje $y$ (la recta $x = 0$) es una asíntota vertical.
\end{itemize}

Para $f(x) = \log_a(x)$ con $0 < a < 1$:

\begin{itemize}[leftmargin=*]
    \item Las características de dominio, rango e intercepto son las mismas.
    \item \textbf{Decrecimiento}: La función es \textit{decreciente} en todo su dominio.
\end{itemize}

\subsubsection{Gráficas de funciones logarítmicas}

\begin{center}
\begin{tikzpicture}
    \begin{axis}[
        width=12cm,
        height=8cm,
        axis lines=middle,
        xlabel={$x$},
        ylabel={$y$},
        domain=0.01:10,
        samples=100,
        grid=major,
        legend pos=south east,
        ymin=-3,
        ymax=3,
        xmin=-0.5,
        xmax=10
    ]
    \addplot[blue, ultra thick, smooth] {ln(x)/ln(2)};
    \addplot[red, ultra thick, smooth] {ln(x)};
    \addplot[green!70!black, ultra thick, smooth] {ln(x)/ln(10)};
    \addplot[dashed, gray] coordinates {(1,-3) (1,3)};
    \node[blue] at (axis cs:8,2.5) {$y=\log_2(x)$};
    \node[red] at (axis cs:8,1.7) {$y=\ln(x)$};
    \node[green!70!black] at (axis cs:8,0.6) {$y=\log_{10}(x)$};
    \end{axis}
\end{tikzpicture}
\end{center}

\textbf{Observaciones}:
\begin{itemize}
    \item Todas pasan por $(1,0)$.
    \item Todas tienen asíntota vertical en $x=0$.
    \item Crecen lentamente para valores grandes de $x$.
    \item Entre mayor la base, más lentamente crece el logaritmo.
\end{itemize}

\subsection{Relación entre funciones exponenciales y logarítmicas}

Las funciones exponencial y logarítmica son \textbf{funciones inversas}. Esto significa que una ``deshace'' lo que la otra hace.

\begin{teorema}[Propiedades de inversas]
Para $a > 0$, $a \neq 1$:

\begin{enumerate}[leftmargin=*]
    \item $a^{\log_a(x)} = x$ para todo $x > 0$
    \item $\log_a(a^x) = x$ para todo $x \in \mathbb{R}$
\end{enumerate}
\end{teorema}

\textbf{En particular para la base $e$}:
\begin{itemize}
    \item $e^{\ln(x)} = x$ para todo $x > 0$
    \item $\ln(e^x) = x$ para todo $x \in \mathbb{R}$
\end{itemize}

\subsubsection{Visualización geométrica}

Las gráficas de funciones inversas son simétricas respecto a la recta $y = x$:

\begin{center}
\begin{tikzpicture}
    \begin{axis}[
        width=12cm,
        height=10cm,
        axis lines=middle,
        xlabel={$x$},
        ylabel={$y$},
        domain=-2:4,
        samples=100,
        grid=major,
        legend pos=south east,
        ymin=-2,
        ymax=4,
        xmin=-2,
        xmax=4,
        axis equal image
    ]
    \addplot[blue, ultra thick, smooth, domain=-2:3] {exp(x)};
    \addplot[red, ultra thick, smooth, domain=0.01:4] {ln(x)};
    \addplot[dashed, gray, domain=-2:4] {x};
    \legend{$y=e^x$, $y=\ln(x)$, $y=x$}
    \end{axis}
\end{tikzpicture}
\end{center}

Observa cómo las gráficas de $y=e^x$ y $y=\ln(x)$ son reflejos una de la otra respecto a la recta $y=x$.

\subsection{Propiedades de los logaritmos}

Los logaritmos tienen propiedades algebraicas muy útiles que nos permiten simplificar expresiones complicadas.

\begin{teorema}[Propiedades fundamentales de logaritmos]
Para $a > 0$, $a \neq 1$, y para $x, y > 0$:

\begin{enumerate}[leftmargin=*]
    \item \textbf{Logaritmo de un producto}:
    \[
    \log_a(xy) = \log_a(x) + \log_a(y)
    \]

    \item \textbf{Logaritmo de un cociente}:
    \[
    \log_a\left(\frac{x}{y}\right) = \log_a(x) - \log_a(y)
    \]

    \item \textbf{Logaritmo de una potencia}:
    \[
    \log_a(x^r) = r \cdot \log_a(x) \quad \text{para cualquier } r \in \mathbb{R}
    \]

    \item \textbf{Cambio de base}:
    \[
    \log_a(x) = \frac{\log_b(x)}{\log_b(a)} \quad \text{para cualquier base } b > 0, b \neq 1
    \]

    \item \textbf{Identidades básicas}:
    \begin{itemize}
        \item $\log_a(1) = 0$ (porque $a^0 = 1$)
        \item $\log_a(a) = 1$ (porque $a^1 = a$)
        \item $\log_a(a^x) = x$
        \item $a^{\log_a(x)} = x$
    \end{itemize}
\end{enumerate}
\end{teorema}

\subsubsection{Demostraciones de las propiedades}

\textbf{Propiedad 1} (Logaritmo del producto):

Sea $\log_a(x) = m$ y $\log_a(y) = n$. Entonces:
\begin{align*}
a^m &= x\\
a^n &= y
\end{align*}

Multiplicando ambas ecuaciones:
\[
xy = a^m \cdot a^n = a^{m+n}
\]

Por definición de logaritmo:
\[
\log_a(xy) = m + n = \log_a(x) + \log_a(y)
\]

\textbf{Propiedad 2} (Logaritmo del cociente):

De manera similar a la propiedad 1, dividiendo en lugar de multiplicar:
\[
\frac{x}{y} = \frac{a^m}{a^n} = a^{m-n}
\]

Por lo tanto:
\[
\log_a\left(\frac{x}{y}\right) = m - n = \log_a(x) - \log_a(y)
\]

\textbf{Propiedad 3} (Logaritmo de una potencia):

Sea $\log_a(x) = m$, entonces $a^m = x$. Elevando ambos lados a la potencia $r$:
\[
x^r = (a^m)^r = a^{mr}
\]

Por definición de logaritmo:
\[
\log_a(x^r) = mr = r \cdot \log_a(x)
\]

\textbf{Propiedad 4} (Cambio de base):

Sea $y = \log_a(x)$, entonces $a^y = x$. Aplicando $\log_b$ a ambos lados:
\[
\log_b(a^y) = \log_b(x)
\]

Usando la propiedad 3:
\[
y \cdot \log_b(a) = \log_b(x)
\]

Despejando $y$:
\[
y = \frac{\log_b(x)}{\log_b(a)} \quad \Rightarrow \quad \log_a(x) = \frac{\log_b(x)}{\log_b(a)}
\]

\subsection{Cambio de base}

La fórmula de cambio de base es especialmente útil porque las calculadoras generalmente solo tienen botones para $\log$ (base 10) y $\ln$ (base $e$).

\begin{nota}
Para calcular $\log_a(x)$ usando una calculadora:
\[
\log_a(x) = \frac{\ln(x)}{\ln(a)} \quad \text{o} \quad \log_a(x) = \frac{\log(x)}{\log(a)}
\]
\end{nota}

\textbf{Ejemplo}: Calcular $\log_2(50)$ usando una calculadora.

\textbf{Solución}:
\[
\log_2(50) = \frac{\ln(50)}{\ln(2)} = \frac{3.912}{0.693} \approx 5.644
\]

O usando logaritmo común:
\[
\log_2(50) = \frac{\log(50)}{\log(2)} = \frac{1.699}{0.301} \approx 5.644
\]

\subsection{Ecuaciones exponenciales}

Una ecuación exponencial es aquella donde la incógnita aparece en el exponente.

\textbf{Estrategias para resolver ecuaciones exponenciales}:

\begin{enumerate}[leftmargin=*]
    \item \textbf{Igualar las bases}: Si puedes expresar ambos lados con la misma base, iguala los exponentes.

    \textbf{Ejemplo}: Resolver $2^{x+1} = 8$

    Como $8 = 2^3$:
    \[
    2^{x+1} = 2^3 \quad \Rightarrow \quad x+1 = 3 \quad \Rightarrow \quad x = 2
    \]

    \item \textbf{Aplicar logaritmos}: Cuando no puedes igualar las bases, aplica logaritmos a ambos lados.

    \textbf{Ejemplo}: Resolver $3^x = 10$

    Aplicando logaritmo natural a ambos lados:
    \[
    \ln(3^x) = \ln(10)
    \]
    \[
    x \cdot \ln(3) = \ln(10)
    \]
    \[
    x = \frac{\ln(10)}{\ln(3)} \approx 2.096
    \]
\end{enumerate}

\subsection{Ecuaciones logarítmicas}

Una ecuación logarítmica es aquella que contiene logaritmos de expresiones con la incógnita.

\textbf{Estrategias para resolver ecuaciones logarítmicas}:

\begin{enumerate}[leftmargin=*]
    \item \textbf{Usar propiedades de logaritmos}: Simplifica usando las propiedades de suma, resta, etc.

    \item \textbf{Convertir a forma exponencial}: Una vez simplificado, convierte la ecuación logarítmica a su forma exponencial.

    \textbf{Ejemplo}: Resolver $\log_2(x) + \log_2(x-3) = 2$

    Usando la propiedad del producto:
    \[
    \log_2[x(x-3)] = 2
    \]

    Convirtiendo a forma exponencial:
    \[
    x(x-3) = 2^2 = 4
    \]
    \[
    x^2 - 3x - 4 = 0
    \]
    \[
    (x-4)(x+1) = 0
    \]

    Por lo tanto $x = 4$ o $x = -1$. Como el logaritmo solo está definido para números positivos, y además $x-3$ debe ser positivo, la única solución válida es $x = 4$.
\end{enumerate}

\begin{nota}
\textbf{¡Importante!} Siempre verifica tus soluciones en ecuaciones logarítmicas. Puede haber soluciones ``extrañas'' que no son válidas porque hacen que tomes logaritmos de números negativos o cero.
\end{nota}

\subsection{Aplicaciones de funciones exponenciales y logarítmicas}

Las funciones exponenciales y logarítmicas modelan numerosos fenómenos del mundo real:

\subsubsection{Crecimiento y decrecimiento exponencial}

\textbf{Modelo general}:
\[
P(t) = P_0 \cdot a^{kt}
\]

o equivalentemente:
\[
P(t) = P_0 \cdot e^{kt}
\]

donde:
\begin{itemize}
    \item $P(t)$ es la cantidad en el tiempo $t$
    \item $P_0$ es la cantidad inicial (cuando $t=0$)
    \item $k$ es la tasa de crecimiento (si $k > 0$) o decrecimiento (si $k < 0$)
\end{itemize}

\textbf{Aplicaciones}:

\begin{itemize}[leftmargin=*]
    \item \textbf{Crecimiento poblacional}: Si $k > 0$, modela cómo crece una población.
    \item \textbf{Interés compuesto}: El dinero en una cuenta crece exponencialmente.
    \[
    A = P\left(1 + \frac{r}{n}\right)^{nt}
    \]
    donde $P$ es el capital inicial, $r$ es la tasa de interés anual, $n$ es el número de veces que se capitaliza por año, y $t$ es el tiempo en años.

    Para interés compuesto continuo:
    \[
    A = Pe^{rt}
    \]

    \item \textbf{Decaimiento radiactivo}: Los elementos radiactivos decaen exponencialmente con $k < 0$.
    \[
    N(t) = N_0 e^{-\lambda t}
    \]
    donde $\lambda$ es la constante de decaimiento.

    La \textbf{vida media} $t_{1/2}$ es el tiempo que tarda la sustancia en reducirse a la mitad:
    \[
    t_{1/2} = \frac{\ln(2)}{\lambda}
    \]

    \item \textbf{Crecimiento bacteriano}: Las bacterias se reproducen exponencialmente bajo condiciones ideales.
\end{itemize}

\subsubsection{Escalas logarítmicas}

Los logaritmos se usan para comprimir rangos enormes en escalas manejables:

\begin{itemize}[leftmargin=*]
    \item \textbf{Escala de Richter} (terremotos): Mide la magnitud de un terremoto.
    \[
    M = \log_{10}\left(\frac{I}{I_0}\right)
    \]
    donde $I$ es la intensidad del terremoto e $I_0$ es una intensidad de referencia.

    Un aumento de 1 unidad en la escala de Richter representa un aumento de 10 veces en la amplitud de las ondas sísmicas.

    \item \textbf{Escala de pH} (acidez): Mide la concentración de iones de hidrógeno.
    \[
    \text{pH} = -\log_{10}[\text{H}^+]
    \]
    donde $[\text{H}^+]$ es la concentración de iones de hidrógeno en moles por litro.

    Un pH de 7 es neutro, menor que 7 es ácido, mayor que 7 es básico.

    \item \textbf{Decibeles} (intensidad del sonido):
    \[
    \beta = 10\log_{10}\left(\frac{I}{I_0}\right)
    \]
    donde $I$ es la intensidad del sonido e $I_0$ es la intensidad de referencia (umbral de audición).

    Un aumento de 10 dB representa un aumento de 10 veces en la intensidad del sonido.
\end{itemize}

\newpage

\section{Ejemplos Resueltos}

\begin{ejemplo}
\textbf{Ejemplo 1: Crecimiento poblacional}

Una ciudad tiene una población de 50,000 habitantes en el año 2020. Si la población crece a una tasa del 3\% anual, ¿cuántos habitantes tendrá en el año 2030? ¿En qué año la población alcanzará los 100,000 habitantes?

\vspace{0.3cm}

\textbf{Solución}:

\textbf{Parte 1}: Población en 2030.

El modelo de crecimiento exponencial es:
\[
P(t) = P_0 \cdot (1 + r)^t
\]

donde:
\begin{itemize}
    \item $P_0 = 50000$ (población inicial en 2020)
    \item $r = 0.03$ (tasa de crecimiento del 3\%)
    \item $t$ es el número de años desde 2020
\end{itemize}

Para el año 2030, $t = 10$ años:
\[
P(10) = 50000 \cdot (1.03)^{10}
\]

Calculando:
\[
(1.03)^{10} \approx 1.3439
\]
\[
P(10) = 50000 \cdot 1.3439 \approx 67195 \text{ habitantes}
\]

\textbf{Respuesta}: En 2030 la ciudad tendrá aproximadamente 67,195 habitantes.

\vspace{0.3cm}

\textbf{Parte 2}: ¿Cuándo alcanzará 100,000 habitantes?

Necesitamos resolver:
\[
100000 = 50000 \cdot (1.03)^t
\]

Dividiendo ambos lados por 50000:
\[
2 = (1.03)^t
\]

Aplicando logaritmo natural a ambos lados:
\[
\ln(2) = \ln[(1.03)^t]
\]
\[
\ln(2) = t \cdot \ln(1.03)
\]
\[
t = \frac{\ln(2)}{\ln(1.03)} = \frac{0.6931}{0.0296} \approx 23.45 \text{ años}
\]

\textbf{Respuesta}: La población alcanzará los 100,000 habitantes aproximadamente 23.45 años después de 2020, es decir, a mediados del año 2043.

\vspace{0.3cm}

\textbf{Gráfica del crecimiento}:

\begin{center}
\begin{tikzpicture}
    \begin{axis}[
        width=12cm,
        height=8cm,
        axis lines=left,
        xlabel={Años desde 2020},
        ylabel={Población (miles)},
        domain=0:30,
        samples=100,
        grid=major,
        ymin=40,
        ymax=110,
        xmin=0,
        xmax=30
    ]
    \addplot[blue, ultra thick, smooth] {50*(1.03)^x};
    \addplot[dashed, red] coordinates {(0,50) (30,50)};
    \addplot[dashed, red] coordinates {(0,100) (30,100)};
    \addplot[only marks, red, mark=*, mark size=3pt] coordinates {(10,67.195) (23.45,100)};
    \node[red, below right] at (axis cs:12,70) {2030};
    \node[red, below right] at (axis cs:25,103) {2043};
    \end{axis}
\end{tikzpicture}
\end{center}
\end{ejemplo}

\newpage

\begin{ejemplo}
\textbf{Ejemplo 2: Interés compuesto continuo}

María invierte \$5,000 en una cuenta de ahorros que ofrece un 4.5\% de interés anual compuesto continuamente.

a) ¿Cuánto dinero tendrá después de 10 años?

b) ¿Cuánto tiempo tomará para que su inversión se duplique?

c) Compare con el interés compuesto mensual.

\vspace{0.3cm}

\textbf{Solución}:

\textbf{Parte a}: Dinero después de 10 años (interés continuo).

La fórmula para interés compuesto continuo es:
\[
A = Pe^{rt}
\]

donde:
\begin{itemize}
    \item $P = 5000$ (capital inicial)
    \item $r = 0.045$ (tasa de interés del 4.5\%)
    \item $t = 10$ (años)
\end{itemize}

Sustituyendo:
\[
A = 5000 \cdot e^{0.045 \cdot 10} = 5000 \cdot e^{0.45}
\]

Calculando:
\[
e^{0.45} \approx 1.5683
\]
\[
A = 5000 \cdot 1.5683 \approx 7841.56
\]

\textbf{Respuesta}: Después de 10 años, María tendrá aproximadamente \$7,841.56.

\vspace{0.3cm}

\textbf{Parte b}: Tiempo para duplicar la inversión.

Necesitamos resolver:
\[
10000 = 5000 \cdot e^{0.045t}
\]

Dividiendo por 5000:
\[
2 = e^{0.045t}
\]

Aplicando logaritmo natural:
\[
\ln(2) = 0.045t
\]
\[
t = \frac{\ln(2)}{0.045} = \frac{0.6931}{0.045} \approx 15.40 \text{ años}
\]

\textbf{Respuesta}: La inversión se duplicará en aproximadamente 15.4 años.

\textbf{Nota}: Existe una regla práctica llamada ``Regla del 72'' que dice que el tiempo de duplicación es aproximadamente $72/r$ donde $r$ es el porcentaje de interés. En este caso: $72/4.5 = 16$ años, muy cercano a nuestro cálculo exacto.

\vspace{0.3cm}

\textbf{Parte c}: Comparación con interés compuesto mensual.

Para interés compuesto $n$ veces al año:
\[
A = P\left(1 + \frac{r}{n}\right)^{nt}
\]

Con $n = 12$ (mensual), $t = 10$:
\[
A = 5000\left(1 + \frac{0.045}{12}\right)^{12 \cdot 10} = 5000(1.00375)^{120}
\]

Calculando:
\[
(1.00375)^{120} \approx 1.5657
\]
\[
A = 5000 \cdot 1.5657 \approx 7828.50
\]

\textbf{Comparación}:
\begin{itemize}
    \item Interés continuo: \$7,841.56
    \item Interés mensual: \$7,828.50
    \item Diferencia: \$13.06
\end{itemize}

La diferencia es pequeña, pero el interés continuo siempre genera un poco más.

\vspace{0.3cm}

\textbf{Gráfica comparativa}:

\begin{center}
\begin{tikzpicture}
    \begin{axis}[
        width=12cm,
        height=8cm,
        axis lines=left,
        xlabel={Años},
        ylabel={Dinero (\$)},
        domain=0:20,
        samples=100,
        grid=major,
        legend pos=south east,
        ymin=4000,
        ymax=11000
    ]
    \addplot[blue, ultra thick, smooth] {5000*exp(0.045*x)};
    \addplot[red, thick, smooth] {5000*(1+0.045/12)^(12*x)};
    \addplot[dashed, gray] coordinates {(0,10000) (20,10000)};
    \legend{Continuo, Mensual, Doble}
    \end{axis}
\end{tikzpicture}
\end{center}
\end{ejemplo}

\newpage

\begin{ejemplo}
\textbf{Ejemplo 3: Decaimiento radiactivo del Carbono-14}

El Carbono-14 es un isótopo radiactivo usado para datar fósiles. Su vida media es de 5,730 años. Un fósil tiene actualmente el 25\% del Carbono-14 que tenía cuando el organismo estaba vivo.

a) ¿Qué edad tiene el fósil?

b) ¿Qué porcentaje de Carbono-14 quedará después de 10,000 años?

\vspace{0.3cm}

\textbf{Solución}:

Primero, necesitamos encontrar la constante de decaimiento $\lambda$ usando la vida media.

La fórmula del decaimiento es:
\[
N(t) = N_0 e^{-\lambda t}
\]

La vida media $t_{1/2}$ satisface:
\[
\frac{N_0}{2} = N_0 e^{-\lambda t_{1/2}}
\]

Dividiendo por $N_0$:
\[
\frac{1}{2} = e^{-\lambda t_{1/2}}
\]

Aplicando logaritmo natural:
\[
\ln(1/2) = -\lambda t_{1/2}
\]
\[
-\ln(2) = -\lambda t_{1/2}
\]
\[
\lambda = \frac{\ln(2)}{t_{1/2}} = \frac{0.6931}{5730} \approx 0.000121 \text{ año}^{-1}
\]

\vspace{0.3cm}

\textbf{Parte a}: Edad del fósil.

Si actualmente tiene 25\% del Carbono-14 original:
\[
0.25N_0 = N_0 e^{-0.000121t}
\]

Dividiendo por $N_0$:
\[
0.25 = e^{-0.000121t}
\]

Aplicando logaritmo natural:
\[
\ln(0.25) = -0.000121t
\]
\[
t = \frac{\ln(0.25)}{-0.000121} = \frac{-1.3863}{-0.000121} \approx 11460 \text{ años}
\]

\textbf{Respuesta}: El fósil tiene aproximadamente 11,460 años de antigüedad.

\textbf{Verificación}: Observa que 11,460 años son aproximadamente dos vidas medias ($2 \times 5730 = 11460$). Después de una vida media queda 50\%, después de dos vidas medias queda 25\%. ¡Tiene sentido!

\vspace{0.3cm}

\textbf{Parte b}: Porcentaje después de 10,000 años.

\[
N(10000) = N_0 e^{-0.000121 \cdot 10000} = N_0 e^{-1.21}
\]

Calculando:
\[
e^{-1.21} \approx 0.2982
\]

\textbf{Respuesta}: Después de 10,000 años quedará aproximadamente el 29.82\% del Carbono-14 original.

\vspace{0.3cm}

\textbf{Gráfica del decaimiento}:

\begin{center}
\begin{tikzpicture}
    \begin{axis}[
        width=12cm,
        height=8cm,
        axis lines=left,
        xlabel={Tiempo (años)},
        ylabel={Porcentaje de C-14 (\%)},
        domain=0:20000,
        samples=100,
        grid=major,
        ymin=0,
        ymax=100
    ]
    \addplot[blue, ultra thick, smooth] {100*exp(-0.000121*x)};
    \addplot[dashed, red] coordinates {(5730,0) (5730,50)};
    \addplot[dashed, red] coordinates {(0,50) (5730,50)};
    \addplot[dashed, green!70!black] coordinates {(11460,0) (11460,25)};
    \addplot[dashed, green!70!black] coordinates {(0,25) (11460,25)};
    \node[red] at (axis cs:6050,55) {$t_{1/2}$};
    \node[green!70!black] at (axis cs:12460,30) {$2t_{1/2}$};
    \end{axis}
\end{tikzpicture}
\end{center}
\end{ejemplo}

\newpage

\begin{ejemplo}
\textbf{Ejemplo 4: Escala de Richter y terremotos}

El terremoto de San Francisco de 1906 tuvo una magnitud de 7.9 en la escala de Richter. El terremoto de Loma Prieta de 1989 tuvo una magnitud de 6.9.

a) ¿Cuántas veces más intensa fue la amplitud de las ondas sísmicas del terremoto de 1906 comparado con el de 1989?

b) Si un terremoto tiene una intensidad 1000 veces mayor que otro, ¿cuál es la diferencia en la escala de Richter?

\vspace{0.3cm}

\textbf{Solución}:

La escala de Richter se define como:
\[
M = \log_{10}\left(\frac{I}{I_0}\right)
\]

donde $I$ es la intensidad del terremoto e $I_0$ es una intensidad de referencia.

\vspace{0.3cm}

\textbf{Parte a}: Comparación de intensidades.

Para el terremoto de 1906 ($M_1 = 7.9$):
\[
7.9 = \log_{10}\left(\frac{I_1}{I_0}\right)
\]

Convirtiendo a forma exponencial:
\[
10^{7.9} = \frac{I_1}{I_0} \quad \Rightarrow \quad I_1 = I_0 \cdot 10^{7.9}
\]

Para el terremoto de 1989 ($M_2 = 6.9$):
\[
6.9 = \log_{10}\left(\frac{I_2}{I_0}\right)
\]
\[
10^{6.9} = \frac{I_2}{I_0} \quad \Rightarrow \quad I_2 = I_0 \cdot 10^{6.9}
\]

La razón entre las intensidades es:
\[
\frac{I_1}{I_2} = \frac{I_0 \cdot 10^{7.9}}{I_0 \cdot 10^{6.9}} = 10^{7.9-6.9} = 10^{1.0} = 10
\]

\textbf{Respuesta}: El terremoto de 1906 fue 10 veces más intenso que el de 1989.

\textbf{Regla general}: Cada unidad de diferencia en la escala de Richter representa un factor de 10 en la intensidad.

\vspace{0.3cm}

\textbf{Parte b}: Diferencia en magnitud.

Si $I_1 = 1000 \cdot I_2$, entonces:
\[
\frac{I_1}{I_2} = 1000
\]

Las magnitudes son:
\[
M_1 = \log_{10}\left(\frac{I_1}{I_0}\right) \quad \text{y} \quad M_2 = \log_{10}\left(\frac{I_2}{I_0}\right)
\]

La diferencia es:
\begin{align*}
M_1 - M_2 &= \log_{10}\left(\frac{I_1}{I_0}\right) - \log_{10}\left(\frac{I_2}{I_0}\right)\\
&= \log_{10}\left(\frac{I_1/I_0}{I_2/I_0}\right)\\
&= \log_{10}\left(\frac{I_1}{I_2}\right)\\
&= \log_{10}(1000)\\
&= \log_{10}(10^3)\\
&= 3
\end{align*}

\textbf{Respuesta}: La diferencia en la escala de Richter es 3 unidades.

\vspace{0.3cm}

\textbf{Visualización}:

\begin{center}
\begin{tikzpicture}
    \begin{axis}[
        width=12cm,
        height=8cm,
        axis lines=left,
        xlabel={Magnitud de Richter},
        ylabel={Intensidad relativa (escala log)},
        domain=1:9,
        samples=50,
        grid=major,
        ymode=log,
        ymin=10,
        ymax=1e8,
        ytick={1e1,1e2,1e3,1e4,1e5,1e6,1e7,1e8}
    ]
    \addplot[blue, ultra thick, smooth] {10^x};
    \addplot[only marks, red, mark=*, mark size=4pt] coordinates {(6.9,10^6.9) (7.9,10^7.9)};
    \node[red] at (axis cs:6.9,2e6) {1989};
    \node[red] at (axis cs:7.9,2e7) {1906};
    \end{axis}
\end{tikzpicture}
\end{center}
\end{ejemplo}

\newpage

\begin{ejemplo}
\textbf{Ejemplo 5: Ecuaciones exponenciales y logarítmicas}

Resolver las siguientes ecuaciones:

a) $5^{2x-1} = 25$

b) $e^{3x} = 20$

c) $\log_3(x+2) + \log_3(x-1) = 2$

d) $\ln(x) + \ln(x-3) = \ln(4)$

\vspace{0.3cm}

\textbf{Solución}:

\textbf{Parte a}: $5^{2x-1} = 25$

Expresamos 25 como potencia de 5:
\[
5^{2x-1} = 5^2
\]

Como las bases son iguales, los exponentes deben ser iguales:
\[
2x - 1 = 2
\]
\[
2x = 3
\]
\[
x = \frac{3}{2}
\]

\textbf{Verificación}: $5^{2(3/2)-1} = 5^{3-1} = 5^2 = 25$ ✓

\textbf{Respuesta}: $x = \frac{3}{2}$

\vspace{0.3cm}

\textbf{Parte b}: $e^{3x} = 20$

Aplicamos logaritmo natural a ambos lados:
\[
\ln(e^{3x}) = \ln(20)
\]

Usando la propiedad $\ln(e^a) = a$:
\[
3x = \ln(20)
\]
\[
x = \frac{\ln(20)}{3} = \frac{2.9957}{3} \approx 0.9986
\]

\textbf{Verificación}: $e^{3(0.9986)} = e^{2.9958} \approx 20$ ✓

\textbf{Respuesta}: $x \approx 0.9986$

\vspace{0.3cm}

\textbf{Parte c}: $\log_3(x+2) + \log_3(x-1) = 2$

Usando la propiedad del producto de logaritmos:
\[
\log_3[(x+2)(x-1)] = 2
\]

Convirtiendo a forma exponencial:
\[
(x+2)(x-1) = 3^2 = 9
\]

Expandiendo:
\[
x^2 + x - 2 = 9
\]
\[
x^2 + x - 11 = 0
\]

Usando la fórmula cuadrática:
\[
x = \frac{-1 \pm \sqrt{1 + 44}}{2} = \frac{-1 \pm \sqrt{45}}{2} = \frac{-1 \pm 6.708}{2}
\]

Por lo tanto:
\[
x_1 = \frac{-1 + 6.708}{2} = 2.854 \quad \text{o} \quad x_2 = \frac{-1 - 6.708}{2} = -3.854
\]

\textbf{Verificación de dominios}:
\begin{itemize}
    \item Para $x_1 = 2.854$: $x+2 = 4.854 > 0$ ✓ y $x-1 = 1.854 > 0$ ✓
    \item Para $x_2 = -3.854$: $x+2 = -1.854 < 0$ ✗ (logaritmo indefinido)
\end{itemize}

\textbf{Respuesta}: $x \approx 2.854$ (la solución negativa no es válida)

\vspace{0.3cm}

\textbf{Parte d}: $\ln(x) + \ln(x-3) = \ln(4)$

Usando la propiedad del producto:
\[
\ln[x(x-3)] = \ln(4)
\]

Como $\ln$ es una función uno-a-uno, si $\ln(a) = \ln(b)$, entonces $a = b$:
\[
x(x-3) = 4
\]
\[
x^2 - 3x - 4 = 0
\]
\[
(x-4)(x+1) = 0
\]

Por lo tanto:
\[
x = 4 \quad \text{o} \quad x = -1
\]

\textbf{Verificación de dominios}:
\begin{itemize}
    \item Para $x = 4$: $\ln(4)$ está definido ✓ y $\ln(4-3) = \ln(1) = 0$ está definido ✓
    \item Para $x = -1$: $\ln(-1)$ no está definido ✗
\end{itemize}

\textbf{Verificación completa para $x=4$}:
\[
\ln(4) + \ln(1) = \ln(4) + 0 = \ln(4) = \ln(4)
\]

\textbf{Respuesta}: $x = 4$ (la solución negativa no es válida)
\end{ejemplo}

\newpage

\section{Ejercicios Propuestos}

Resuelve los siguientes ejercicios. Las soluciones detalladas se encuentran en la siguiente sección.

\begin{enumerate}[leftmargin=*]

\item \textbf{Simplificación de expresiones}

Simplifica las siguientes expresiones usando propiedades de exponentes y logaritmos:

\begin{enumerate}
    \item $\log_5(125) + \log_5(1/25)$
    \item $\ln(e^3) - 2\ln(e)$
    \item $e^{\ln(7)}$
    \item $\log_2(8) \cdot \log_3(9)$
\end{enumerate}

\item \textbf{Ecuaciones exponenciales}

Resuelve las siguientes ecuaciones:

\begin{enumerate}
    \item $3^{x+2} = 81$
    \item $2^{3x} = 16^{x-1}$
    \item $5^x = 12$
    \item $e^{2x} - 6e^x + 5 = 0$
\end{enumerate}

\item \textbf{Ecuaciones logarítmicas}

Resuelve las siguientes ecuaciones (verifica el dominio):

\begin{enumerate}
    \item $\log_2(x) = 5$
    \item $\log_4(x+3) - \log_4(x-1) = 1$
    \item $\ln(x^2) = 4$
    \item $\log(x) + \log(x+3) = 1$
\end{enumerate}

\item \textbf{Cambio de base}

Calcula usando la fórmula de cambio de base (usa calculadora si es necesario):

\begin{enumerate}
    \item $\log_7(50)$
    \item $\log_4(1000)$
\end{enumerate}

\item \textbf{Interés compuesto}

Pedro invierte \$8,000 en un fondo que paga 5.2\% de interés anual compuesto mensualmente. ¿Cuánto dinero tendrá después de 15 años?

\item \textbf{Decaimiento radiactivo}

El Yodo-131 tiene una vida media de 8 días. Si un paciente recibe una dosis de 100 mg:

\begin{enumerate}
    \item ¿Cuánto Yodo-131 quedará después de 24 días?
    \item ¿Cuánto tiempo tomará para que quede solo 5 mg?
\end{enumerate}

\item \textbf{Escala de pH}

El jugo de limón tiene una concentración de iones de hidrógeno de $[\text{H}^+] = 0.01$ moles/litro. El agua pura tiene $[\text{H}^+] = 10^{-7}$ moles/litro.

\begin{enumerate}
    \item Calcula el pH del jugo de limón.
    \item Calcula el pH del agua pura.
    \item ¿Cuántas veces más ácido es el jugo de limón que el agua pura?
\end{enumerate}

\end{enumerate}

\newpage

\section{Soluciones Detalladas}

\subsection{Solución Ejercicio 1: Simplificación de expresiones}

\textbf{a) $\log_5(125) + \log_5(1/25)$}

Primero, expresamos 125 y 25 como potencias de 5:
\[
125 = 5^3 \quad \text{y} \quad 25 = 5^2 \quad \Rightarrow \quad \frac{1}{25} = 5^{-2}
\]

Usando la propiedad $\log_a(a^x) = x$:
\[
\log_5(125) + \log_5(1/25) = \log_5(5^3) + \log_5(5^{-2}) = 3 + (-2) = 1
\]

\textbf{Método alternativo} usando la propiedad del producto:
\[
\log_5(125) + \log_5(1/25) = \log_5\left(125 \cdot \frac{1}{25}\right) = \log_5(5) = 1
\]

\textbf{Respuesta}: 1

\vspace{0.3cm}

\textbf{b) $\ln(e^3) - 2\ln(e)$}

Usando la propiedad $\ln(e^x) = x$:
\[
\ln(e^3) - 2\ln(e) = 3 - 2(1) = 3 - 2 = 1
\]

\textbf{Respuesta}: 1

\vspace{0.3cm}

\textbf{c) $e^{\ln(7)}$}

Usando la propiedad de funciones inversas $e^{\ln(x)} = x$:
\[
e^{\ln(7)} = 7
\]

\textbf{Respuesta}: 7

\vspace{0.3cm}

\textbf{d) $\log_2(8) \cdot \log_3(9)$}

Calculamos cada logaritmo por separado:
\[
\log_2(8) = \log_2(2^3) = 3
\]
\[
\log_3(9) = \log_3(3^2) = 2
\]

Por lo tanto:
\[
\log_2(8) \cdot \log_3(9) = 3 \cdot 2 = 6
\]

\textbf{Respuesta}: 6

\subsection{Solución Ejercicio 2: Ecuaciones exponenciales}

\textbf{a) $3^{x+2} = 81$}

Expresamos 81 como potencia de 3:
\[
81 = 3^4
\]

Por lo tanto:
\[
3^{x+2} = 3^4
\]

Igualando exponentes:
\[
x + 2 = 4
\]
\[
x = 2
\]

\textbf{Respuesta}: $x = 2$

\vspace{0.3cm}

\textbf{b) $2^{3x} = 16^{x-1}$}

Expresamos 16 como potencia de 2:
\[
16 = 2^4
\]

Por lo tanto:
\[
2^{3x} = (2^4)^{x-1} = 2^{4(x-1)} = 2^{4x-4}
\]

Igualando exponentes:
\[
3x = 4x - 4
\]
\[
-x = -4
\]
\[
x = 4
\]

\textbf{Respuesta}: $x = 4$

\vspace{0.3cm}

\textbf{c) $5^x = 12$}

No podemos expresar 12 como potencia de 5, así que aplicamos logaritmos:
\[
\ln(5^x) = \ln(12)
\]
\[
x \cdot \ln(5) = \ln(12)
\]
\[
x = \frac{\ln(12)}{\ln(5)} = \frac{2.4849}{1.6094} \approx 1.544
\]

\textbf{Respuesta}: $x \approx 1.544$

\vspace{0.3cm}

\textbf{d) $e^{2x} - 6e^x + 5 = 0$}

Esta es una ecuación cuadrática en $e^x$. Hagamos la sustitución $u = e^x$:
\[
u^2 - 6u + 5 = 0
\]

Factorizando:
\[
(u-5)(u-1) = 0
\]

Por lo tanto:
\[
u = 5 \quad \text{o} \quad u = 1
\]

Regresando a $x$:
\begin{itemize}
    \item Si $e^x = 5$, entonces $x = \ln(5) \approx 1.609$
    \item Si $e^x = 1$, entonces $x = \ln(1) = 0$
\end{itemize}

\textbf{Respuesta}: $x = 0$ o $x = \ln(5) \approx 1.609$

\subsection{Solución Ejercicio 3: Ecuaciones logarítmicas}

\textbf{a) $\log_2(x) = 5$}

Convirtiendo a forma exponencial:
\[
x = 2^5 = 32
\]

\textbf{Respuesta}: $x = 32$

\vspace{0.3cm}

\textbf{b) $\log_4(x+3) - \log_4(x-1) = 1$}

Usando la propiedad del cociente:
\[
\log_4\left(\frac{x+3}{x-1}\right) = 1
\]

Convirtiendo a forma exponencial:
\[
\frac{x+3}{x-1} = 4^1 = 4
\]

Resolviendo:
\[
x + 3 = 4(x-1)
\]
\[
x + 3 = 4x - 4
\]
\[
7 = 3x
\]
\[
x = \frac{7}{3} \approx 2.333
\]

\textbf{Verificación del dominio}:
\begin{itemize}
    \item $x+3 = 7/3 + 3 = 16/3 > 0$ ✓
    \item $x-1 = 7/3 - 1 = 4/3 > 0$ ✓
\end{itemize}

\textbf{Respuesta}: $x = \frac{7}{3}$

\vspace{0.3cm}

\textbf{c) $\ln(x^2) = 4$}

Usando la propiedad de la potencia:
\[
2\ln(x) = 4
\]
\[
\ln(x) = 2
\]

Convirtiendo a forma exponencial:
\[
x = e^2 \approx 7.389
\]

\textbf{Nota}: También podríamos tener $x = -e^2$, pero solo $x = e^2$ es válido si interpretamos $\ln(x^2)$ estrictamente (requiere $x^2 > 0$, lo cual es cierto para cualquier $x \neq 0$).

Si usamos $\ln(x^2) = e^4$, entonces $x^2 = e^4$, y $x = \pm e^2$. Sin embargo, típicamente en este contexto buscamos la solución positiva.

\textbf{Respuesta}: $x = e^2 \approx 7.389$

\vspace{0.3cm}

\textbf{d) $\log(x) + \log(x+3) = 1$}

Usando la propiedad del producto:
\[
\log[x(x+3)] = 1
\]

Convirtiendo a forma exponencial (asumiendo base 10):
\[
x(x+3) = 10^1 = 10
\]
\[
x^2 + 3x - 10 = 0
\]
\[
(x+5)(x-2) = 0
\]

Por lo tanto:
\[
x = -5 \quad \text{o} \quad x = 2
\]

\textbf{Verificación del dominio}:
\begin{itemize}
    \item Para $x = -5$: $\log(-5)$ no está definido ✗
    \item Para $x = 2$: $\log(2)$ está definido ✓ y $\log(5)$ está definido ✓
\end{itemize}

\textbf{Respuesta}: $x = 2$

\subsection{Solución Ejercicio 4: Cambio de base}

\textbf{a) $\log_7(50)$}

Usando la fórmula de cambio de base:
\[
\log_7(50) = \frac{\ln(50)}{\ln(7)} = \frac{3.912}{1.946} \approx 2.010
\]

O usando base 10:
\[
\log_7(50) = \frac{\log(50)}{\log(7)} = \frac{1.699}{0.845} \approx 2.010
\]

\textbf{Respuesta}: $\log_7(50) \approx 2.010$

\vspace{0.3cm}

\textbf{b) $\log_4(1000)$}

Usando la fórmula de cambio de base:
\[
\log_4(1000) = \frac{\ln(1000)}{\ln(4)} = \frac{6.908}{1.386} \approx 4.983
\]

O usando base 10:
\[
\log_4(1000) = \frac{\log(1000)}{\log(4)} = \frac{3}{0.602} \approx 4.983
\]

\textbf{Respuesta}: $\log_4(1000) \approx 4.983$

\subsection{Solución Ejercicio 5: Interés compuesto}

Pedro invierte \$8,000 al 5.2\% anual compuesto mensualmente por 15 años.

La fórmula es:
\[
A = P\left(1 + \frac{r}{n}\right)^{nt}
\]

donde:
\begin{itemize}
    \item $P = 8000$
    \item $r = 0.052$
    \item $n = 12$ (mensual)
    \item $t = 15$
\end{itemize}

Sustituyendo:
\[
A = 8000\left(1 + \frac{0.052}{12}\right)^{12 \cdot 15} = 8000(1.004333)^{180}
\]

Calculando:
\[
(1.004333)^{180} \approx 2.1700
\]
\[
A = 8000 \cdot 2.1700 \approx 17360
\]

\textbf{Respuesta}: Pedro tendrá aproximadamente \$17,360 después de 15 años.

\subsection{Solución Ejercicio 6: Decaimiento radiactivo}

El Yodo-131 tiene vida media de 8 días, dosis inicial de 100 mg.

Primero calculamos la constante de decaimiento:
\[
\lambda = \frac{\ln(2)}{t_{1/2}} = \frac{0.6931}{8} \approx 0.0866 \text{ día}^{-1}
\]

La fórmula del decaimiento es:
\[
N(t) = N_0 e^{-\lambda t} = 100e^{-0.0866t}
\]

\textbf{a) Después de 24 días}:
\[
N(24) = 100e^{-0.0866 \cdot 24} = 100e^{-2.078} \approx 100 \cdot 0.125 = 12.5 \text{ mg}
\]

\textbf{Verificación}: 24 días son 3 vidas medias ($3 \times 8 = 24$). Después de 1 vida media queda 50 mg, después de 2 queda 25 mg, después de 3 queda 12.5 mg. ✓

\textbf{Respuesta a)}: Quedarán 12.5 mg.

\vspace{0.3cm}

\textbf{b) ¿Cuándo quedan 5 mg?}
\[
5 = 100e^{-0.0866t}
\]
\[
0.05 = e^{-0.0866t}
\]
\[
\ln(0.05) = -0.0866t
\]
\[
t = \frac{\ln(0.05)}{-0.0866} = \frac{-2.996}{-0.0866} \approx 34.6 \text{ días}
\]

\textbf{Respuesta b)}: Tomarán aproximadamente 34.6 días para que queden 5 mg.

\subsection{Solución Ejercicio 7: Escala de pH}

La fórmula del pH es:
\[
\text{pH} = -\log_{10}[\text{H}^+]
\]

\textbf{a) pH del jugo de limón} ($[\text{H}^+] = 0.01 = 10^{-2}$):
\[
\text{pH} = -\log_{10}(10^{-2}) = -(-2) = 2
\]

\textbf{Respuesta a)}: El pH del jugo de limón es 2 (ácido).

\vspace{0.3cm}

\textbf{b) pH del agua pura} ($[\text{H}^+] = 10^{-7}$):
\[
\text{pH} = -\log_{10}(10^{-7}) = -(-7) = 7
\]

\textbf{Respuesta b)}: El pH del agua pura es 7 (neutro).

\vspace{0.3cm}

\textbf{c) ¿Cuántas veces más ácido es el jugo de limón?}

La acidez está determinada por la concentración de iones H$^+$:
\[
\frac{[\text{H}^+]_{\text{limón}}}{[\text{H}^+]_{\text{agua}}} = \frac{10^{-2}}{10^{-7}} = 10^{-2-(-7)} = 10^5 = 100000
\]

\textbf{Respuesta c)}: El jugo de limón es 100,000 veces más ácido que el agua pura.

\textbf{Nota}: Cada unidad de diferencia en pH representa un factor de 10 en acidez. La diferencia aquí es $7 - 2 = 5$ unidades, por lo que el factor es $10^5 = 100,000$.

\newpage

\section{Ejercicios Inversos (Modelado)}

Los siguientes ejercicios requieren que plantees un modelo matemático usando funciones exponenciales o logarítmicas, y luego lo resuelvas.

\begin{enumerate}[leftmargin=*]

\item \textbf{Crecimiento bacteriano}

Una colonia de bacterias tiene inicialmente 500 individuos. Después de observarla durante 3 horas, se determina que la población se duplica cada hora.

\begin{enumerate}
    \item Plantea un modelo matemático para la población $P(t)$ en función del tiempo $t$ (en horas).
    \item ¿Cuántas bacterias habrá después de 6 horas?
    \item ¿Cuánto tiempo tomará para que la población alcance 50,000 bacterias?
\end{enumerate}

\item \textbf{Temperatura de enfriamiento}

Una taza de café se saca del microondas a 90°C. La temperatura ambiente es de 20°C. Según la Ley de Enfriamiento de Newton, la temperatura $T(t)$ del café después de $t$ minutos está dada por:
\[
T(t) = T_a + (T_0 - T_a)e^{-kt}
\]
donde $T_a$ es la temperatura ambiente, $T_0$ es la temperatura inicial, y $k$ es una constante.

Se observa que después de 5 minutos, el café está a 70°C.

\begin{enumerate}
    \item Determina el valor de la constante $k$.
    \item ¿Cuál será la temperatura del café después de 15 minutos?
    \item ¿Cuánto tiempo tomará para que el café alcance 40°C?
\end{enumerate}

\item \textbf{Nivel de ruido}

En un concierto, el nivel de ruido cerca del escenario es de 110 dB (decibeles). La fórmula de los decibeles es:
\[
\beta = 10\log_{10}\left(\frac{I}{I_0}\right)
\]
donde $I$ es la intensidad del sonido e $I_0 = 10^{-12}$ W/m² es la intensidad de referencia (umbral de audición).

\begin{enumerate}
    \item ¿Cuál es la intensidad del sonido cerca del escenario (en W/m²)?
    \item Si te alejas del escenario y el nivel baja a 80 dB, ¿cuántas veces menos intensa es la intensidad del sonido?
    \item ¿A qué nivel de decibeles corresponde una intensidad de $10^{-5}$ W/m²?
\end{enumerate}

\item \textbf{Inversión y tasa de interés}

Lucía quiere tener \$20,000 en 8 años para comprar un auto. Un banco le ofrece una cuenta de ahorros con interés compuesto anualmente.

\begin{enumerate}
    \item Si Lucía invierte \$10,000 hoy, ¿qué tasa de interés anual necesita para alcanzar su meta?
    \item Si la tasa de interés disponible es del 6\% anual, ¿cuánto debe invertir hoy para tener \$20,000 en 8 años?
\end{enumerate}

\end{enumerate}

\newpage

\section{Soluciones de Ejercicios Inversos}

\subsection{Solución Ejercicio Inverso 1: Crecimiento bacteriano}

\textbf{a) Modelo matemático}

Sabemos que:
\begin{itemize}
    \item Población inicial: $P_0 = 500$
    \item La población se duplica cada hora: período de duplicación $T_d = 1$ hora
\end{itemize}

El modelo exponencial general es:
\[
P(t) = P_0 \cdot 2^{t/T_d}
\]

Sustituyendo:
\[
P(t) = 500 \cdot 2^{t/1} = 500 \cdot 2^t
\]

\textbf{Respuesta a)}: $P(t) = 500 \cdot 2^t$ bacterias después de $t$ horas.

\vspace{0.3cm}

\textbf{b) Población después de 6 horas}
\[
P(6) = 500 \cdot 2^6 = 500 \cdot 64 = 32000 \text{ bacterias}
\]

\textbf{Respuesta b)}: Habrá 32,000 bacterias.

\vspace{0.3cm}

\textbf{c) Tiempo para alcanzar 50,000 bacterias}

Necesitamos resolver:
\[
50000 = 500 \cdot 2^t
\]
\[
100 = 2^t
\]

Aplicando logaritmo:
\[
\ln(100) = t \cdot \ln(2)
\]
\[
t = \frac{\ln(100)}{\ln(2)} = \frac{4.605}{0.693} \approx 6.64 \text{ horas}
\]

\textbf{Respuesta c)}: Tomará aproximadamente 6.64 horas (o 6 horas y 38 minutos).

\vspace{0.3cm}

\textbf{Gráfica del crecimiento}:

\begin{center}
\begin{tikzpicture}
    \begin{axis}[
        width=12cm,
        height=8cm,
        axis lines=left,
        xlabel={Tiempo (horas)},
        ylabel={Población (bacterias)},
        domain=0:8,
        samples=100,
        grid=major,
        ymin=0,
        ymax=60000,
        legend pos=north west
    ]
    \addplot[blue, ultra thick, smooth] {500*2^x};
    \addplot[dashed, red] coordinates {(6,0) (6,32000)};
    \addplot[dashed, green!70!black] coordinates {(6.64,0) (6.64,50000)};
    \addplot[only marks, red, mark=*, mark size=3pt] coordinates {(6,32000)};
    \addplot[only marks, green!70!black, mark=*, mark size=3pt] coordinates {(6.64,50000)};
    \end{axis}
\end{tikzpicture}
\end{center}

\subsection{Solución Ejercicio Inverso 2: Temperatura de enfriamiento}

Datos:
\begin{itemize}
    \item $T_0 = 90$°C (temperatura inicial)
    \item $T_a = 20$°C (temperatura ambiente)
    \item $T(5) = 70$°C (temperatura después de 5 minutos)
\end{itemize}

Modelo:
\[
T(t) = 20 + (90-20)e^{-kt} = 20 + 70e^{-kt}
\]

\textbf{a) Determinar $k$}

Usamos la condición $T(5) = 70$:
\[
70 = 20 + 70e^{-5k}
\]
\[
50 = 70e^{-5k}
\]
\[
\frac{50}{70} = e^{-5k}
\]
\[
\frac{5}{7} = e^{-5k}
\]

Aplicando logaritmo natural:
\[
\ln\left(\frac{5}{7}\right) = -5k
\]
\[
k = -\frac{1}{5}\ln\left(\frac{5}{7}\right) = -\frac{1}{5}(\ln(5) - \ln(7))
\]
\[
k = -\frac{1}{5}(1.609 - 1.946) = -\frac{-0.337}{5} \approx 0.0673 \text{ min}^{-1}
\]

\textbf{Respuesta a)}: $k \approx 0.0673$ min$^{-1}$

\vspace{0.3cm}

\textbf{b) Temperatura después de 15 minutos}

Usando $k = 0.0673$:
\[
T(15) = 20 + 70e^{-0.0673 \cdot 15} = 20 + 70e^{-1.010}
\]
\[
e^{-1.010} \approx 0.364
\]
\[
T(15) = 20 + 70(0.364) = 20 + 25.5 = 45.5\text{°C}
\]

\textbf{Respuesta b)}: La temperatura será aproximadamente 45.5°C.

\vspace{0.3cm}

\textbf{c) Tiempo para alcanzar 40°C}

Necesitamos resolver:
\[
40 = 20 + 70e^{-0.0673t}
\]
\[
20 = 70e^{-0.0673t}
\]
\[
\frac{20}{70} = e^{-0.0673t}
\]
\[
\frac{2}{7} = e^{-0.0673t}
\]

Aplicando logaritmo:
\[
\ln\left(\frac{2}{7}\right) = -0.0673t
\]
\[
t = \frac{\ln(2/7)}{-0.0673} = \frac{-1.253}{-0.0673} \approx 18.6 \text{ minutos}
\]

\textbf{Respuesta c)}: Tomará aproximadamente 18.6 minutos.

\vspace{0.3cm}

\textbf{Gráfica del enfriamiento}:

\begin{center}
\begin{tikzpicture}
    \begin{axis}[
        width=12cm,
        height=8cm,
        axis lines=left,
        xlabel={Tiempo (minutos)},
        ylabel={Temperatura (°C)},
        domain=0:30,
        samples=100,
        grid=major,
        ymin=15,
        ymax=95
    ]
    \addplot[blue, ultra thick, smooth] {20+70*exp(-0.0673*x)};
    \addplot[dashed, red] coordinates {(0,20) (30,20)};
    \addplot[only marks, green!70!black, mark=*, mark size=3pt] coordinates {(5,70) (15,45.5) (18.6,40)};
    \node[green!70!black,right=1mm] at (axis cs:5,73) {5 min};
    \node[green!70!black,right=1mm] at (axis cs:15,48) {15 min};
    \node[green!70!black,right=1mm] at (axis cs:18.6,43) {18.6 min};
    \end{axis}
\end{tikzpicture}
\end{center}

\subsection{Solución Ejercicio Inverso 3: Nivel de ruido}

La fórmula de decibeles es:
\[
\beta = 10\log_{10}\left(\frac{I}{I_0}\right)
\]
donde $I_0 = 10^{-12}$ W/m².

\textbf{a) Intensidad cerca del escenario} ($\beta = 110$ dB)

\[
110 = 10\log_{10}\left(\frac{I}{10^{-12}}\right)
\]
\[
11 = \log_{10}\left(\frac{I}{10^{-12}}\right)
\]

Convirtiendo a forma exponencial:
\[
10^{11} = \frac{I}{10^{-12}}
\]
\[
I = 10^{11} \cdot 10^{-12} = 10^{-1} = 0.1 \text{ W/m}^2
\]

\textbf{Respuesta a)}: La intensidad es 0.1 W/m².

\vspace{0.3cm}

\textbf{b) Comparación con 80 dB}

Para $\beta = 80$ dB:
\[
80 = 10\log_{10}\left(\frac{I'}{10^{-12}}\right)
\]
\[
8 = \log_{10}\left(\frac{I'}{10^{-12}}\right)
\]
\[
10^8 = \frac{I'}{10^{-12}}
\]
\[
I' = 10^8 \cdot 10^{-12} = 10^{-4} \text{ W/m}^2
\]

La razón es:
\[
\frac{I}{I'} = \frac{10^{-1}}{10^{-4}} = 10^3 = 1000
\]

\textbf{Respuesta b)}: A 80 dB, la intensidad es 1000 veces menor.

\textbf{Nota}: Una diferencia de 30 dB ($110 - 80 = 30$) corresponde a un factor de $10^{30/10} = 10^3 = 1000$ en intensidad.

\vspace{0.3cm}

\textbf{c) Decibeles para $I = 10^{-5}$ W/m²}

\[
\beta = 10\log_{10}\left(\frac{10^{-5}}{10^{-12}}\right) = 10\log_{10}(10^7) = 10 \cdot 7 = 70 \text{ dB}
\]

\textbf{Respuesta c)}: Corresponde a 70 dB.

\subsection{Solución Ejercicio Inverso 4: Inversión y tasa de interés}

\textbf{a) Tasa de interés necesaria}

Datos:
\begin{itemize}
    \item Capital inicial: $P = 10000$
    \item Monto final: $A = 20000$
    \item Tiempo: $t = 8$ años
    \item Interés compuesto anualmente: $n = 1$
\end{itemize}

Fórmula:
\[
A = P(1+r)^t
\]
\[
20000 = 10000(1+r)^8
\]
\[
2 = (1+r)^8
\]

Aplicando raíz octava (o logaritmo):
\[
1+r = 2^{1/8}
\]

Calculando:
\[
2^{1/8} = 2^{0.125} \approx 1.0905
\]
\[
r = 0.0905 = 9.05\%
\]

\textbf{Respuesta a)}: Necesita una tasa de interés del 9.05\% anual.

\textbf{Método alternativo usando logaritmos}:
\[
\ln(2) = 8\ln(1+r)
\]
\[
\ln(1+r) = \frac{\ln(2)}{8} = \frac{0.6931}{8} = 0.0866
\]
\[
1+r = e^{0.0866} \approx 1.0905
\]
\[
r \approx 0.0905 = 9.05\%
\]

\vspace{0.3cm}

\textbf{b) Capital inicial necesario con $r = 6\%$}

Necesitamos encontrar $P$ tal que:
\[
20000 = P(1.06)^8
\]

Calculando:
\[
(1.06)^8 \approx 1.5938
\]
\[
P = \frac{20000}{1.5938} \approx 12549
\]

\textbf{Respuesta b)}: Debe invertir aproximadamente \$12,549 hoy.

\vspace{1cm}

\begin{center}
\textcolor{maincolor}{\rule{10cm}{0.5pt}}

\vspace{0.5cm}

\textbf{\Large ¡Felicitaciones!}

\vspace{0.3cm}

Has completado la guía de Funciones Exponenciales y Logarítmicas.

Ahora dominas herramientas matemáticas que modelan desde el crecimiento de poblaciones hasta la medición de terremotos.

\vspace{0.3cm}

\textcolor{maincolor}{\rule{10cm}{0.5pt}}
\end{center}

\end{document}
