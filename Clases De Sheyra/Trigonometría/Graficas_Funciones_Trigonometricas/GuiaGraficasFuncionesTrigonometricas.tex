% !TEX program = lualatex
\documentclass[12pt,a4paper,twoside]{article}
\usepackage{fontspec}
\usepackage[spanish,es-nodecimaldot]{babel}
\usepackage{amsmath,amssymb}
\usepackage[margin=2.5cm]{geometry}
\usepackage{xcolor}
\usepackage{tikz,pgfplots}
\usetikzlibrary{calc,arrows.meta,babel}
\usepackage{multicol}
\usepackage{enumitem}
\pgfplotsset{compat=1.18}
\definecolor{maincolor}{RGB}{26,35,126}
\definecolor{accentcolor}{RGB}{255,87,34}

% Configuración de títulos y formato
\usepackage{titlesec}
\titleformat{\section}{\Large\bfseries\color{maincolor}}{\thesection}{1em}{}
\titleformat{\subsection}{\large\bfseries\color{accentcolor}}{\thesubsection}{1em}{}

% Configuración de cajas para ejemplos
\usepackage{tcolorbox}
\tcbuselibrary{skins,breakable}

\usepackage{fancyhdr}

\pagestyle{fancy}
\fancyhf{}
\fancyhead[LE]{\small\textcolor{maincolor}{\thepage \quad Líneas trigonométricas}}
\fancyhead[RO]{\small\textcolor{maincolor}{Líneas trigonométricas \quad \thepage}}
\fancyhead[LO]{\small\textcolor{maincolor}{Grado 10 - Trigonometría}}
\fancyhead[RE]{\small\textcolor{maincolor}{Prof. Toribio De J Arrieta F}}
\fancyfoot[C]{}
\renewcommand{\headrulewidth}{0.5pt}
\renewcommand{\footrulewidth}{0pt}
\setlength{\headheight}{14pt}

\newtcolorbox{ejemplo}[1][]{
  enhanced,
  breakable,
  colback=maincolor!5,
  colframe=maincolor,
  fonttitle=\bfseries,
  title=Ejemplo Resuelto,
  #1
}

\newtcolorbox{ejercicio}[1][]{
  enhanced,
  breakable,
  colback=accentcolor!5,
  colframe=accentcolor,
  fonttitle=\bfseries,
  title=Ejercicio,
  #1
}

\newtcolorbox{solucion}[1][]{
  enhanced,
  breakable,
  colback=green!5,
  colframe=green!60!black,
  fonttitle=\bfseries,
  title=Solución,
  #1
}

\newtcolorbox{nota}[1][]{
  enhanced,
  colback=yellow!10,
  colframe=orange!80!black,
  fonttitle=\bfseries,
  title=Nota Importante,
  #1
}

% Título
\title{\textbf{\Huge Gráficas de las funciones trigonométricas}\\[0.5cm]
\Large Guía de Trigonometría}
\author{Prof. Toribio De J Arrieta F\\
\textit{La Pruebita}\\
Grado 10}
\date{\today}

\begin{document}

\maketitle

\tableofcontents
\newpage

\section{Introducción}

Bienvenidos a esta guía sobre líneas trigonométricas. En matemáticas, las funciones trigonométricas no solo son números abstractos; también se pueden representar como segmentos de recta dentro del círculo unitario. Estas representaciones geométricas se conocen como \textbf{líneas trigonométricas}.

Las líneas trigonométricas nos permiten visualizar de manera directa las funciones seno, coseno y tangente como segmentos que podemos dibujar y medir. Esta perspectiva geométrica es fundamental para entender profundamente la trigonometría y sus aplicaciones.

\subsection*{¿Qué son las líneas trigonométricas?}

Las líneas trigonométricas son segmentos de recta que representan los valores de las funciones trigonométricas en el círculo unitario. Específicamente:

\begin{itemize}
    \item \textbf{Línea seno:} Es el segmento perpendicular al eje $x$ que va desde un punto en el círculo hasta el eje $x$.
    \item \textbf{Línea coseno:} Es el segmento sobre el eje $x$ que va desde el origen hasta la proyección del punto.
    \item \textbf{Línea tangente:} Es el segmento sobre la tangente geométrica al círculo que va desde el punto $(1,0)$ hasta donde esta tangente intercepta el radio prolongado.
\end{itemize}

\subsection*{¿Por qué son importantes?}

Las líneas trigonométricas nos ayudan a:
\begin{itemize}
    \item Visualizar geométricamente las funciones trigonométricas
    \item Entender los signos de las funciones en cada cuadrante
    \item Comprender las relaciones entre las diferentes funciones
    \item Resolver problemas de manera gráfica
    \item Desarrollar intuición geométrica sobre la trigonometría
\end{itemize}

\subsection*{¿Qué vamos a aprender?}

En esta guía exploraremos:
\begin{enumerate}
    \item Cómo identificar y dibujar las líneas seno, coseno y tangente
    \item Las propiedades de estas líneas en diferentes cuadrantes
    \item Cómo los signos de las funciones se relacionan con las direcciones de los segmentos
    \item Aplicaciones prácticas de esta representación geométrica
\end{enumerate}

Prepárate para ver la trigonometría desde una nueva perspectiva visual.

\newpage

\section{Conceptos Fundamentales}

\subsection{El Círculo Unitario}

El círculo unitario es la base de las líneas trigonométricas. Es un círculo con radio igual a 1, centrado en el origen del plano cartesiano.

\begin{center}
\begin{tikzpicture}[scale=3]
    % Ejes
    \draw[-{Latex},thick] (-1.3,0) -- (1.3,0) node[right] {$x$};
    \draw[-{Latex},thick] (0,-1.3) -- (0,1.3) node[above] {$y$};

    % Circunferencia unitaria
    \draw[maincolor,very thick] (0,0) circle (1);

    % Radio = 1
    \draw[accentcolor,thick,-{Latex}] (0,0) -- (1,0) node[midway,below] {$r=1$};

    % Cuadrantes
    \node[maincolor] at (0.5,0.5) {\small I};
    \node[maincolor] at (-0.5,0.5) {\small II};
    \node[maincolor] at (-0.5,-0.5) {\small III};
    \node[maincolor] at (0.5,-0.5) {\small IV};

    % Marcas en los ejes
    \foreach \x in {-1,1}
        \draw (\x,0.05) -- (\x,-0.05) node[below] {$\x$};
    \foreach \y in {-1,1}
        \draw (0.05,\y) -- (-0.05,\y) node[left] {$\y$};
\end{tikzpicture}
\end{center}

\begin{nota}
La ecuación del círculo unitario es: $x^2 + y^2 = 1$
\end{nota}

\subsection{Las Tres Líneas Trigonométricas Principales}

Para un ángulo $\theta$ en posición estándar (con vértice en el origen y lado inicial en el eje $x$ positivo), definimos:

\subsubsection{Línea Seno}

La línea seno es el segmento perpendicular al eje $x$ que va desde el punto $P$ en el círculo hasta su proyección en el eje $x$. Su longitud es $|\sin\theta|$ y su signo depende de si está arriba (positivo) o abajo (negativo) del eje $x$.

\subsubsection{Línea Coseno}

La línea coseno es el segmento sobre el eje $x$ desde el origen hasta la proyección del punto $P$. Su longitud es $|\cos\theta|$ y su signo depende de si está a la derecha (positivo) o izquierda (negativo) del origen.

\subsubsection{Línea Tangente}

La línea tangente es el segmento sobre la recta tangente al círculo en el punto $(1,0)$, desde este punto hasta donde la tangente intercepta la prolongación del radio que forma el ángulo $\theta$. Su longitud es $|\tan\theta|$.

\begin{center}
\begin{tikzpicture}[scale=3]
    \draw[maincolor,very thick] (0,0) circle (1);
    \draw[-{Latex},thick] (-1.3,0) -- (1.3,0) node[right] {$x$};
    \draw[-{Latex},thick] (0,-1.3) -- (0,1.3) node[above] {$y$};

    \def\angulo{50}
    \coordinate (P) at ({\angulo}:1);

    % Radio
    \draw[gray,dashed] (0,0) -- (1.5,{1.5*tan(\angulo)});
    \draw[maincolor,very thick,-{Latex}] (0,0) -- (P);

    % Punto P
    \filldraw[maincolor] (P) circle (0.03) node[above right] {$P$};

    % Ángulo
    \draw[accentcolor,-{Latex}] (0.3,0) arc (0:\angulo:0.3) node[midway,right] {$\theta$};

    % Línea COSENO (horizontal)
    \draw[blue,very thick,-{Latex}] (0,0) -- ({\angulo}:1 |- 0,0) node[midway,below] {\small coseno};

    % Línea SENO (vertical)
    \draw[red,very thick,-{Latex}] ({\angulo}:1 |- 0,0) -- (P) node[midway,right] {\small seno};

    % Línea TANGENTE
    \coordinate (T) at (1,{tan(\angulo)});
    \draw[green!60!black,very thick] (1,0) -- (T) node[midway,right] {\small tangente};
    \filldraw[green!60!black] (T) circle (0.02);

    % Recta tangente al círculo en (1,0)
    \draw[green!60!black,dashed] (1,-0.5) -- (1,1.5);
\end{tikzpicture}
\end{center}

\subsection{Signos de las Líneas en Cada Cuadrante}

Los signos de las líneas trigonométricas varían según el cuadrante:

\begin{center}
\renewcommand{\arraystretch}{1.8}
\begin{tabular}{|c|c|c|c|}
\hline
\textbf{Cuadrante} & \textbf{Seno} & \textbf{Coseno} & \textbf{Tangente} \\
\hline
I & $+$ & $+$ & $+$ \\
\hline
II & $+$ & $-$ & $-$ \\
\hline
III & $-$ & $-$ & $+$ \\
\hline
IV & $-$ & $+$ & $-$ \\
\hline
\end{tabular}
\end{center}

\begin{nota}
Recuerda: \textbf{T}odo \textbf{S}tudent \textbf{T}akes \textbf{C}alculus
\begin{itemize}
    \item Cuadrante I: \textbf{T}odo es positivo
    \item Cuadrante II: Solo \textbf{S}eno es positivo
    \item Cuadrante III: Solo \textbf{T}angente es positivo
    \item Cuadrante IV: Solo \textbf{C}oseno es positivo
\end{itemize}
\end{nota}

\newpage

\section{Ejemplos Resueltos}

A continuación se presentan ejemplos detallados sobre cómo identificar y dibujar líneas trigonométricas.

\begin{ejemplo}[title=Ejemplo 1: Lineas para 30 grados]
Dibuja las líneas seno, coseno y tangente para el ángulo de $30^\circ$ en el círculo unitario.

\vspace{0.3cm}
\textbf{Solución:}

\textbf{Paso 1:} Ubicar el ángulo.

El ángulo $30^\circ$ está en el primer cuadrante. El punto $P$ en el círculo unitario tiene coordenadas:
\[
P = (\cos 30^\circ, \sin 30^\circ) = \left(\frac{\sqrt{3}}{2}, \frac{1}{2}\right)
\]

\textbf{Paso 2:} Identificar cada línea.

\begin{itemize}
    \item \textbf{Línea coseno:} Segmento horizontal desde el origen hasta $\left(\frac{\sqrt{3}}{2}, 0\right)$
    \item Longitud: $\frac{\sqrt{3}}{2} \approx 0.866$
    \item \textbf{Línea seno:} Segmento vertical desde $\left(\frac{\sqrt{3}}{2}, 0\right)$ hasta $P$
    \item Longitud: $\frac{1}{2} = 0.5$
    \item \textbf{Línea tangente:} Segmento desde $(1,0)$ hasta la intersección con el radio prolongado
    \item Longitud: $\tan 30^\circ = \frac{1}{\sqrt{3}} = \frac{\sqrt{3}}{3} \approx 0.577$
\end{itemize}

\textbf{Paso 3:} Dibujo.

\begin{center}
\begin{tikzpicture}[scale=3.5]
    \draw[maincolor,very thick] (0,0) circle (1);
    \draw[-{Latex},thick] (-0.3,0) -- (1.3,0) node[right] {$x$};
    \draw[-{Latex},thick] (0,-0.3) -- (0,1.3) node[above] {$y$};

    \def\angulo{30}
    \coordinate (P) at ({\angulo}:1);

    % Radio
    \draw[gray,dashed] (0,0) -- (1.2,{1.2*tan(\angulo)});
    \draw[maincolor,very thick,-{Latex}] (0,0) -- (P);

    % Punto P
    \filldraw[maincolor] (P) circle (0.02) node[above right,scale=0.8] {$P\left(\frac{\sqrt{3}}{2}, \frac{1}{2}\right)$};

    % Ángulo
    \draw[accentcolor,-{Latex}] (0.25,0) arc (0:\angulo:0.25) node[midway,right,scale=0.7] {$30^\circ$};

    % Línea COSENO
    \draw[blue,very thick,-{Latex}] (0,0) -- ({\angulo}:1 |- 0,0);
    \node[blue,below,scale=0.7] at (0.433,0) {$\cos 30^\circ = \frac{\sqrt{3}}{2}$};

    % Línea SENO
    \draw[red,very thick,-{Latex}] ({\angulo}:1 |- 0,0) -- (P);
    \node[red,right,scale=0.7] at (0.866,0.25) {$\sin 30^\circ = \frac{1}{2}$};

    % Línea TANGENTE
    \coordinate (T) at (1,{tan(\angulo)});
    \draw[green!60!black,very thick,-{Latex}] (1,0) -- (T);
    \node[green!60!black,right,scale=0.7] at (1,0.289) {$\tan 30^\circ = \frac{\sqrt{3}}{3}$};
    \filldraw[green!60!black] (T) circle (0.015);

    % Recta tangente
    \draw[green!60!black,dashed,thin] (1,-0.2) -- (1,0.7);
\end{tikzpicture}
\end{center}

\textbf{Conclusión:} En el primer cuadrante, todas las líneas son positivas y se dirigen hacia arriba (seno) y hacia la derecha (coseno y tangente).
\end{ejemplo}

\begin{ejemplo}[title=Ejemplo 2: Lineas para 120 grados]
Dibuja las líneas seno, coseno y tangente para el ángulo de $120^\circ$.

\vspace{0.3cm}
\textbf{Solución:}

\textbf{Paso 1:} Ubicar el ángulo.

El ángulo $120^\circ$ está en el segundo cuadrante. Usando $120^\circ = 180^\circ - 60^\circ$:
\[
P = (\cos 120^\circ, \sin 120^\circ) = \left(-\frac{1}{2}, \frac{\sqrt{3}}{2}\right)
\]

\textbf{Paso 2:} Identificar signos.

En el segundo cuadrante:
\begin{itemize}
    \item $\sin 120^\circ = \frac{\sqrt{3}}{2} > 0$ (positivo)
    \item $\cos 120^\circ = -\frac{1}{2} < 0$ (negativo)
    \item $\tan 120^\circ = \frac{\sqrt{3}/2}{-1/2} = -\sqrt{3} < 0$ (negativo)
\end{itemize}

\textbf{Paso 3:} Dibujo.

\begin{center}
\begin{tikzpicture}[scale=3.5]
    \draw[maincolor,very thick] (0,0) circle (1);
    \draw[-{Latex},thick] (-1.3,0) -- (1.3,0) node[right] {$x$};
    \draw[-{Latex},thick] (0,-0.3) -- (0,1.3) node[above] {$y$};

    \def\angulo{120}
    \coordinate (P) at ({\angulo}:1);

    % Radio
    \draw[gray,dashed] (0,0) -- (1.5,{1.5*tan(\angulo)});
    \draw[maincolor,very thick,-{Latex}] (0,0) -- (P);

    % Punto P
    \filldraw[maincolor] (P) circle (0.02) node[above left,scale=0.8] {$P$};

    % Ángulo
    \draw[accentcolor,-{Latex}] (0.25,0) arc (0:\angulo:0.25) node[midway,above,scale=0.7] {$120^\circ$};

    % Línea COSENO (negativo)
    \draw[blue,very thick,{Latex}-] (0,0) -- ({\angulo}:1 |- 0,0);
    \node[blue,below,scale=0.7] at (-0.25,0) {$\cos 120^\circ = -\frac{1}{2}$};

    % Línea SENO (positivo)
    \draw[red,very thick,-{Latex}] ({\angulo}:1 |- 0,0) -- (P);
    \node[red,left,scale=0.7] at (-0.5,0.433) {$\sin 120^\circ = \frac{\sqrt{3}}{2}$};

    % Línea TANGENTE (negativo)
    \coordinate (T) at (1,{tan(\angulo)});
    \draw[green!60!black,very thick,{Latex}-] (1,0) -- (T);
    \node[green!60!black,right,scale=0.6] at (1,-0.866) {$\tan 120^\circ = -\sqrt{3}$};
    \filldraw[green!60!black] (T) circle (0.015);

    % Recta tangente
    \draw[green!60!black,dashed,thin] (1,-2) -- (1,0.3);
\end{tikzpicture}
\end{center}

\textbf{Conclusión:} En el segundo cuadrante, el seno es positivo (hacia arriba) mientras que el coseno y la tangente son negativos.
\end{ejemplo}

\begin{ejemplo}[title=Ejemplo 3: Lineas para 225 grados]
Dibuja las líneas trigonométricas para el ángulo de $225^\circ$.

\vspace{0.3cm}
\textbf{Solución:}

\textbf{Paso 1:} Ubicar el ángulo.

El ángulo $225^\circ$ está en el tercer cuadrante. Usando $225^\circ = 180^\circ + 45^\circ$:
\[
P = (\cos 225^\circ, \sin 225^\circ) = \left(-\frac{\sqrt{2}}{2}, -\frac{\sqrt{2}}{2}\right)
\]

\textbf{Paso 2:} Identificar signos.

En el tercer cuadrante:
\begin{itemize}
    \item $\sin 225^\circ = -\frac{\sqrt{2}}{2} < 0$ (negativo)
    \item $\cos 225^\circ = -\frac{\sqrt{2}}{2} < 0$ (negativo)
    \item $\tan 225^\circ = \frac{-\sqrt{2}/2}{-\sqrt{2}/2} = 1 > 0$ (positivo)
\end{itemize}

\textbf{Paso 3:} Dibujo.

\begin{center}
\begin{tikzpicture}[scale=3.5]
    \draw[maincolor,very thick] (0,0) circle (1);
    \draw[-{Latex},thick] (-1.3,0) -- (1.3,0) node[right] {$x$};
    \draw[-{Latex},thick] (0,-1.3) -- (0,1.3) node[above] {$y$};

    \def\angulo{225}
    \coordinate (P) at ({\angulo}:1);

    % Radio
    \draw[gray,dashed] (0,0) -- (1.5,{1.5*tan(\angulo)});
    \draw[maincolor,very thick,-{Latex}] (0,0) -- (P);

    % Punto P
    \filldraw[maincolor] (P) circle (0.02) node[below left,scale=0.8] {$P$};

    % Ángulo
    \draw[accentcolor,-{Latex}] (0.25,0) arc (0:\angulo:0.25);
    \node[accentcolor,scale=0.7] at (0.15,0.15) {$225^\circ$};

    % Línea COSENO (negativo)
    \draw[blue,very thick,{Latex}-] (0,0) -- ({\angulo}:1 |- 0,0);
    \node[blue,above,scale=0.7] at (-0.35,0) {$\cos 225^\circ = -\frac{\sqrt{2}}{2}$};

    % Línea SENO (negativo)
    \draw[red,very thick,{Latex}-] ({\angulo}:1 |- 0,0) -- (P);
    \node[red,right,scale=0.7] at (-0.707,-0.35) {$\sin 225^\circ = -\frac{\sqrt{2}}{2}$};

    % Línea TANGENTE (positivo)
    \coordinate (T) at (1,{tan(\angulo)});
    \draw[green!60!black,very thick,-{Latex}] (1,0) -- (T);
    \node[green!60!black,right,scale=0.7] at (1,-0.5) {$\tan 225^\circ = 1$};
    \filldraw[green!60!black] (T) circle (0.015);

    % Recta tangente
    \draw[green!60!black,dashed,thin] (1,-1.3) -- (1,0.3);
\end{tikzpicture}
\end{center}

\textbf{Conclusión:} En el tercer cuadrante, seno y coseno son negativos, pero la tangente es positiva.
\end{ejemplo}

\begin{ejemplo}[title=Ejemplo 4: Lineas para 300 grados]
Dibuja las líneas trigonométricas para el ángulo de $300^\circ$.

\vspace{0.3cm}
\textbf{Solución:}

\textbf{Paso 1:} Ubicar el ángulo.

El ángulo $300^\circ$ está en el cuarto cuadrante. Usando $300^\circ = 360^\circ - 60^\circ$:
\[
P = (\cos 300^\circ, \sin 300^\circ) = \left(\frac{1}{2}, -\frac{\sqrt{3}}{2}\right)
\]

\textbf{Paso 2:} Identificar signos.

En el cuarto cuadrante:
\begin{itemize}
    \item $\sin 300^\circ = -\frac{\sqrt{3}}{2} < 0$ (negativo)
    \item $\cos 300^\circ = \frac{1}{2} > 0$ (positivo)
    \item $\tan 300^\circ = \frac{-\sqrt{3}/2}{1/2} = -\sqrt{3} < 0$ (negativo)
\end{itemize}

\textbf{Paso 3:} Dibujo.

\begin{center}
\begin{tikzpicture}[scale=3.5]
    \draw[maincolor,very thick] (0,0) circle (1);
    \draw[-{Latex},thick] (-0.3,0) -- (1.3,0) node[right] {$x$};
    \draw[-{Latex},thick] (0,-1.3) -- (0,1.3) node[above] {$y$};

    \def\angulo{300}
    \coordinate (P) at ({\angulo}:1);

    % Radio
    \draw[gray,dashed] (0,0) -- (1.5,{1.5*tan(\angulo)});
    \draw[maincolor,very thick,-{Latex}] (0,0) -- (P);

    % Punto P
    \filldraw[maincolor] (P) circle (0.02) node[below right,scale=0.8] {$P$};

    % Ángulo
    \draw[accentcolor,-{Latex}] (0.25,0) arc (0:\angulo:0.25);
    \node[accentcolor,scale=0.7] at (0.3,-0.15) {$300^\circ$};

    % Línea COSENO (positivo)
    \draw[blue,very thick,-{Latex}] (0,0) -- ({\angulo}:1 |- 0,0);
    \node[blue,above,scale=0.7] at (0.25,0) {$\cos 300^\circ = \frac{1}{2}$};

    % Línea SENO (negativo)
    \draw[red,very thick,{Latex}-] ({\angulo}:1 |- 0,0) -- (P);
    \node[red,left,scale=0.7] at (0.5,-0.433) {$\sin 300^\circ = -\frac{\sqrt{3}}{2}$};

    % Línea TANGENTE (negativo)
    \coordinate (T) at (1,{tan(\angulo)});
    \draw[green!60!black,very thick,{Latex}-] (1,0) -- (T);
    \node[green!60!black,right,scale=0.6] at (1,-0.866) {$\tan 300^\circ = -\sqrt{3}$};
    \filldraw[green!60!black] (T) circle (0.015);

    % Recta tangente
    \draw[green!60!black,dashed,thin] (1,-2) -- (1,0.3);
\end{tikzpicture}
\end{center}

\textbf{Conclusión:} En el cuarto cuadrante, solo el coseno es positivo, mientras que seno y tangente son negativos.
\end{ejemplo}

\begin{ejemplo}[title=Ejemplo 5: Comparacion entre angulos]
Compara las líneas trigonométricas de $45^\circ$ y $135^\circ$.

\vspace{0.3cm}
\textbf{Solución:}

\textbf{Para $45^\circ$ (Primer cuadrante):}
\[
\sin 45^\circ = \frac{\sqrt{2}}{2}, \quad \cos 45^\circ = \frac{\sqrt{2}}{2}, \quad \tan 45^\circ = 1
\]

\textbf{Para $135^\circ$ (Segundo cuadrante):}

Usando $135^\circ = 180^\circ - 45^\circ$:
\[
\sin 135^\circ = \frac{\sqrt{2}}{2}, \quad \cos 135^\circ = -\frac{\sqrt{2}}{2}, \quad \tan 135^\circ = -1
\]

\textbf{Observaciones:}

\begin{itemize}
    \item Ambos ángulos tienen la misma línea seno (misma longitud positiva)
    \item Las líneas coseno tienen la misma longitud, pero signos opuestos
    \item Las líneas tangente tienen la misma longitud, pero signos opuestos
\end{itemize}

\begin{center}
\begin{tikzpicture}[scale=2.5]
    \draw[maincolor,very thick] (0,0) circle (1);
    \draw[-{Latex},thick] (-1.3,0) -- (1.3,0) node[right] {$x$};
    \draw[-{Latex},thick] (0,-0.3) -- (0,1.3) node[above] {$y$};

    % Ángulo 45 grados
    \coordinate (P1) at (45:1);
    \draw[blue,very thick,-{Latex}] (0,0) -- (P1);
    \filldraw[blue] (P1) circle (0.02) node[above right,scale=0.7] {$45^\circ$};

    % Ángulo 135 grados
    \coordinate (P2) at (135:1);
    \draw[red,very thick,-{Latex}] (0,0) -- (P2);
    \filldraw[red] (P2) circle (0.02) node[above left,scale=0.7] {$135^\circ$};

    % Líneas seno (misma altura)
    \draw[green!60!black,dashed] (-1,0.707) -- (1,0.707);
    \node[green!60!black,right,scale=0.6] at (1,0.707) {Mismo seno};
\end{tikzpicture}
\end{center}

\textbf{Conclusión:} Los ángulos suplementarios (que suman $180^\circ$) tienen senos iguales y cosenos opuestos.
\end{ejemplo}

\begin{ejemplo}[title=Ejemplo 6: Angulos cuadrantales]
Dibuja las líneas trigonométricas para el ángulo de $90^\circ$.

\vspace{0.3cm}
\textbf{Solución:}

\textbf{Paso 1:} Valores para $90^\circ$.

El punto en el círculo es $P = (0, 1)$, por lo tanto:
\[
\sin 90^\circ = 1, \quad \cos 90^\circ = 0, \quad \tan 90^\circ = \text{indefinido}
\]

\textbf{Paso 2:} Interpretación geométrica.

\begin{itemize}
    \item La línea seno tiene longitud máxima: 1
    \item La línea coseno tiene longitud cero (no existe como segmento)
    \item La línea tangente no está definida (el radio es paralelo a la recta tangente en $(1,0)$)
\end{itemize}

\begin{center}
\begin{tikzpicture}[scale=3.5]
    \draw[maincolor,very thick] (0,0) circle (1);
    \draw[-{Latex},thick] (-0.3,0) -- (1.3,0) node[right] {$x$};
    \draw[-{Latex},thick] (0,-0.3) -- (0,1.3) node[above] {$y$};

    \def\angulo{90}
    \coordinate (P) at ({\angulo}:1);

    % Radio
    \draw[maincolor,very thick,-{Latex}] (0,0) -- (P);

    % Punto P
    \filldraw[maincolor] (P) circle (0.02) node[above right,scale=0.8] {$P(0,1)$};

    % Ángulo
    \draw[accentcolor,-{Latex}] (0.25,0) arc (0:\angulo:0.25);
    \node[accentcolor,scale=0.7] at (0.35,0.35) {$90^\circ$};

    % Línea COSENO (cero)
    \filldraw[blue] (0,0) circle (0.02);
    \node[blue,below,scale=0.7] at (0,-0.1) {$\cos 90^\circ = 0$};

    % Línea SENO (máximo)
    \draw[red,very thick,-{Latex}] (0,0) -- (P);
    \node[red,left,scale=0.7] at (0,0.5) {$\sin 90^\circ = 1$};

    % Recta tangente (paralela al radio)
    \draw[green!60!black,dashed,thin] (1,-0.3) -- (1,1.3);
    \node[green!60!black,right,scale=0.6] at (1,0.65) {$\tan 90^\circ$ indefinido};
\end{tikzpicture}
\end{center}

\textbf{Conclusión:} Para ángulos cuadrantales, algunas líneas pueden tener longitud cero o no estar definidas.
\end{ejemplo}

\begin{ejemplo}[title=Ejemplo 7: Relacion entre lineas]
Demuestra geométricamente que $\tan\theta = \frac{\sin\theta}{\cos\theta}$ usando las líneas trigonométricas.

\vspace{0.3cm}
\textbf{Solución:}

\textbf{Análisis geométrico:}

Consideremos un ángulo $\theta$ en el primer cuadrante. En el círculo unitario:

\begin{itemize}
    \item El punto $P$ tiene coordenadas $(\cos\theta, \sin\theta)$
    \item La línea seno tiene longitud $\sin\theta$
    \item La línea coseno tiene longitud $\cos\theta$
\end{itemize}

\textbf{Triángulos semejantes:}

El triángulo formado por el origen, la proyección de $P$ en el eje $x$, y el punto $P$ es semejante al triángulo formado por el origen, el punto $(1,0)$, y el punto donde la tangente intercepta el radio prolongado.

Por semejanza:
\[
\frac{\text{línea tangente}}{1} = \frac{\sin\theta}{\cos\theta}
\]

Por lo tanto:
\[
\tan\theta = \frac{\sin\theta}{\cos\theta}
\]

\begin{center}
\begin{tikzpicture}[scale=3.5]
    \draw[maincolor,very thick] (0,0) circle (1);
    \draw[-{Latex},thick] (-0.3,0) -- (1.5,0) node[right] {$x$};
    \draw[-{Latex},thick] (0,-0.3) -- (0,1.3) node[above] {$y$};

    \def\angulo{40}
    \coordinate (P) at ({\angulo}:1);
    \coordinate (T) at (1,{tan(\angulo)});

    % Radios
    \draw[gray,dashed] (0,0) -- (1.3,{1.3*tan(\angulo)});
    \draw[maincolor,very thick] (0,0) -- (P);

    % Triángulo principal
    \draw[blue,thick] (0,0) -- ({\angulo}:1 |- 0,0) -- (P) -- cycle;

    % Triángulo semejante
    \draw[red,thick] (0,0) -- (1,0) -- (T) -- cycle;

    % Puntos
    \filldraw[maincolor] (P) circle (0.02) node[above right,scale=0.7] {$P$};
    \filldraw[green!60!black] (T) circle (0.02);

    % Etiquetas
    \node[blue,below,scale=0.7] at (0.4,0) {$\cos\theta$};
    \node[blue,right,scale=0.7] at (0.766,0.32) {$\sin\theta$};
    \node[red,below,scale=0.7] at (0.5,0) {$1$};
    \node[red,right,scale=0.7] at (1,0.42) {$\tan\theta$};

    % Recta tangente
    \draw[green!60!black,dashed,thin] (1,-0.2) -- (1,1);
\end{tikzpicture}
\end{center}

\textbf{Conclusión:} La relación $\tan\theta = \frac{\sin\theta}{\cos\theta}$ se puede demostrar usando la semejanza de triángulos en el círculo unitario.
\end{ejemplo}

\newpage

\section{Ejercicios Propuestos}

Resuelve los siguientes ejercicios aplicando los conceptos de líneas trigonométricas.

\begin{ejercicio}[title=Ejercicio 1]
Dibuja las líneas seno, coseno y tangente para el ángulo de $60^\circ$ e indica sus valores numéricos.
\end{ejercicio}

\begin{ejercicio}[title=Ejercicio 2]
Para el ángulo de $150^\circ$, determina:
\begin{itemize}
    \item[a)] En qué cuadrante se encuentra
    \item[b)] El signo de cada línea trigonométrica
    \item[c)] Los valores de $\sin 150^\circ$, $\cos 150^\circ$ y $\tan 150^\circ$
\end{itemize}
\end{ejercicio}

\begin{ejercicio}[title=Ejercicio 3]
Dibuja las líneas trigonométricas para $210^\circ$ y explica por qué el seno y el coseno son ambos negativos en este cuadrante.
\end{ejercicio}

\begin{ejercicio}[title=Ejercicio 4]
Compara las líneas trigonométricas de $45^\circ$ y $315^\circ$. ¿Qué relación existe entre ellas?
\end{ejercicio}

\begin{ejercicio}[title=Ejercicio 5]
Para el ángulo de $270^\circ$:
\begin{itemize}
    \item[a)] Indica el valor de cada función trigonométrica
    \item[b)] Explica qué sucede con la línea tangente
    \item[c)] Dibuja el diagrama completo
\end{itemize}
\end{ejercicio}

\begin{ejercicio}[title=Ejercicio 6]
Demuestra geométricamente usando las líneas trigonométricas que $\sin^2\theta + \cos^2\theta = 1$ para cualquier ángulo $\theta$.
\end{ejercicio}

\begin{ejercicio}[title=Ejercicio 7]
Dibuja en un mismo círculo unitario las líneas para $30^\circ$, $150^\circ$, $210^\circ$ y $330^\circ$. ¿Qué patrón observas en las líneas seno?
\end{ejercicio}

\begin{ejercicio}[title=Ejercicio 8]
Si la línea seno de un ángulo $\theta$ en el segundo cuadrante tiene longitud $\frac{3}{5}$, determina:
\begin{itemize}
    \item[a)] El valor de $\sin\theta$
    \item[b)] El valor de $\cos\theta$ usando la identidad pitagórica
    \item[c)] El valor de $\tan\theta$
\end{itemize}
\end{ejercicio}

\newpage

\section{Soluciones Detalladas}

\begin{solucion}[title=Solucion Ejercicio 1]
\textbf{Líneas para $60^\circ$}

\textbf{Paso 1:} El ángulo $60^\circ$ está en el primer cuadrante.

El punto $P$ tiene coordenadas:
\[
P = \left(\cos 60^\circ, \sin 60^\circ\right) = \left(\frac{1}{2}, \frac{\sqrt{3}}{2}\right)
\]

\textbf{Paso 2:} Valores de las líneas:

\begin{itemize}
    \item Línea coseno: longitud $= \frac{1}{2} = 0.5$
    \item Línea seno: longitud $= \frac{\sqrt{3}}{2} \approx 0.866$
    \item Línea tangente: longitud $= \tan 60^\circ = \sqrt{3} \approx 1.732$
\end{itemize}

\textbf{Paso 3:} Dibujo:

\begin{center}
\begin{tikzpicture}[scale=3.5]
    \draw[maincolor,very thick] (0,0) circle (1);
    \draw[-{Latex},thick] (-0.3,0) -- (1.5,0) node[right] {$x$};
    \draw[-{Latex},thick] (0,-0.3) -- (0,1.3) node[above] {$y$};

    \def\angulo{60}
    \coordinate (P) at ({\angulo}:1);
    \coordinate (T) at (1,{tan(\angulo)});

    \draw[gray,dashed] (0,0) -- (1.3,{1.3*tan(\angulo)});
    \draw[maincolor,very thick,-{Latex}] (0,0) -- (P);
    \filldraw[maincolor] (P) circle (0.02) node[above right,scale=0.8] {$P$};

    \draw[accentcolor,-{Latex}] (0.25,0) arc (0:\angulo:0.25) node[midway,right,scale=0.7] {$60^\circ$};

    \draw[blue,very thick,-{Latex}] (0,0) -- ({\angulo}:1 |- 0,0);
    \node[blue,below,scale=0.7] at (0.25,-0.05) {$\frac{1}{2}$};

    \draw[red,very thick,-{Latex}] ({\angulo}:1 |- 0,0) -- (P);
    \node[red,left,scale=0.7] at (0.5,0.433) {$\frac{\sqrt{3}}{2}$};

    \draw[green!60!black,very thick,-{Latex}] (1,0) -- (T);
    \node[green!60!black,right,scale=0.7] at (1,0.866) {$\sqrt{3}$};
    \filldraw[green!60!black] (T) circle (0.015);

    \draw[green!60!black,dashed,thin] (1,-0.2) -- (1,2);
\end{tikzpicture}
\end{center}

\textbf{Respuesta:}
\[
\boxed{\sin 60^\circ = \frac{\sqrt{3}}{2}, \quad \cos 60^\circ = \frac{1}{2}, \quad \tan 60^\circ = \sqrt{3}}
\]
\end{solucion}

\begin{solucion}[title=Solucion Ejercicio 2]
\textbf{Ángulo $150^\circ$}

\textbf{Parte a):} Cuadrante.

Como $90^\circ < 150^\circ < 180^\circ$, el ángulo está en el \textbf{segundo cuadrante}.

\textbf{Parte b):} Signos.

En el segundo cuadrante:
\begin{itemize}
    \item $\sin 150^\circ > 0$ (positivo)
    \item $\cos 150^\circ < 0$ (negativo)
    \item $\tan 150^\circ < 0$ (negativo)
\end{itemize}

\textbf{Parte c):} Valores.

Usando $150^\circ = 180^\circ - 30^\circ$:
\begin{align*}
\sin 150^\circ &= \sin 30^\circ = \frac{1}{2} \\
\cos 150^\circ &= -\cos 30^\circ = -\frac{\sqrt{3}}{2} \\
\tan 150^\circ &= \frac{\sin 150^\circ}{\cos 150^\circ} = \frac{1/2}{-\sqrt{3}/2} = -\frac{1}{\sqrt{3}} = -\frac{\sqrt{3}}{3}
\end{align*}

\textbf{Respuesta:}
\[
\boxed{\sin 150^\circ = \frac{1}{2}, \quad \cos 150^\circ = -\frac{\sqrt{3}}{2}, \quad \tan 150^\circ = -\frac{\sqrt{3}}{3}}
\]
\end{solucion}

\begin{solucion}[title=Solucion Ejercicio 3]
\textbf{Ángulo $210^\circ$}

El ángulo $210^\circ = 180^\circ + 30^\circ$ está en el tercer cuadrante.

\textbf{Coordenadas del punto:}
\[
P = \left(\cos 210^\circ, \sin 210^\circ\right) = \left(-\frac{\sqrt{3}}{2}, -\frac{1}{2}\right)
\]

\textbf{Explicación de signos:}

En el tercer cuadrante:
\begin{itemize}
    \item La coordenada $x$ es negativa (punto a la izquierda del eje $y$), por tanto $\cos 210^\circ < 0$
    \item La coordenada $y$ es negativa (punto debajo del eje $x$), por tanto $\sin 210^\circ < 0$
    \item Como ambos son negativos, $\tan 210^\circ = \frac{\text{negativo}}{\text{negativo}} > 0$
\end{itemize}

\begin{center}
\begin{tikzpicture}[scale=3.5]
    \draw[maincolor,very thick] (0,0) circle (1);
    \draw[-{Latex},thick] (-1.3,0) -- (1.3,0) node[right] {$x$};
    \draw[-{Latex},thick] (0,-1.3) -- (0,1.3) node[above] {$y$};

    \def\angulo{210}
    \coordinate (P) at ({\angulo}:1);

    \draw[gray,dashed] (0,0) -- (1.5,{1.5*tan(\angulo)});
    \draw[maincolor,very thick,-{Latex}] (0,0) -- (P);
    \filldraw[maincolor] (P) circle (0.02) node[below left,scale=0.8] {$P$};

    \draw[accentcolor,-{Latex}] (0.25,0) arc (0:\angulo:0.25);
    \node[accentcolor,scale=0.7] at (0.15,0.15) {$210^\circ$};

    \draw[blue,very thick,{Latex}-] (0,0) -- ({\angulo}:1 |- 0,0);
    \node[blue,above,scale=0.7] at (-0.433,0) {$-\frac{\sqrt{3}}{2}$};

    \draw[red,very thick,{Latex}-] ({\angulo}:1 |- 0,0) -- (P);
    \node[red,left,scale=0.7] at (-0.866,-0.25) {$-\frac{1}{2}$};
\end{tikzpicture}
\end{center}

\textbf{Respuesta:} Ambas líneas apuntan en direcciones negativas (izquierda y abajo) porque el punto está en el tercer cuadrante.
\end{solucion}

\begin{solucion}[title=Solucion Ejercicio 4]
\textbf{Comparación: $45^\circ$ y $315^\circ$}

\textbf{Para $45^\circ$ (primer cuadrante):}
\[
\sin 45^\circ = \frac{\sqrt{2}}{2}, \quad \cos 45^\circ = \frac{\sqrt{2}}{2}, \quad \tan 45^\circ = 1
\]

\textbf{Para $315^\circ$ (cuarto cuadrante):}

Usando $315^\circ = 360^\circ - 45^\circ$:
\[
\sin 315^\circ = -\frac{\sqrt{2}}{2}, \quad \cos 315^\circ = \frac{\sqrt{2}}{2}, \quad \tan 315^\circ = -1
\]

\textbf{Relación:}

\begin{itemize}
    \item Ambos ángulos tienen la misma línea coseno (positiva, misma longitud)
    \item Las líneas seno tienen la misma longitud pero signos opuestos
    \item Las líneas tangente tienen la misma longitud pero signos opuestos
\end{itemize}

Los ángulos $45^\circ$ y $315^\circ$ son \textbf{simétricos respecto al eje $x$}.

\textbf{Respuesta:}
\[
\boxed{\sin 315^\circ = -\sin 45^\circ, \quad \cos 315^\circ = \cos 45^\circ, \quad \tan 315^\circ = -\tan 45^\circ}
\]
\end{solucion}

\begin{solucion}[title=Solucion Ejercicio 5]
\textbf{Ángulo $270^\circ$}

\textbf{Parte a):} Valores.

El punto en el círculo es $P = (0, -1)$:
\begin{align*}
\sin 270^\circ &= -1 \\
\cos 270^\circ &= 0 \\
\tan 270^\circ &= \text{indefinido}
\end{align*}

\textbf{Parte b):} Línea tangente.

La línea tangente no está definida porque el radio que forma el ángulo de $270^\circ$ es vertical (paralelo al eje $y$), y nunca intercepta la recta tangente vertical en $(1,0)$. Matemáticamente, esto se refleja en que $\tan 270^\circ = \frac{-1}{0}$ es indefinido.

\textbf{Parte c):} Diagrama.

\begin{center}
\begin{tikzpicture}[scale=3.5]
    \draw[maincolor,very thick] (0,0) circle (1);
    \draw[-{Latex},thick] (-0.3,0) -- (1.3,0) node[right] {$x$};
    \draw[-{Latex},thick] (0,-1.3) -- (0,1.3) node[above] {$y$};

    \def\angulo{270}
    \coordinate (P) at ({\angulo}:1);

    \draw[maincolor,very thick,-{Latex}] (0,0) -- (P);
    \filldraw[maincolor] (P) circle (0.02) node[below right,scale=0.8] {$P(0,-1)$};

    \draw[accentcolor,-{Latex}] (0.25,0) arc (0:\angulo:0.25);
    \node[accentcolor,scale=0.7] at (0.35,-0.25) {$270^\circ$};

    \filldraw[blue] (0,0) circle (0.02);
    \node[blue,right,scale=0.7] at (0.1,0) {$\cos 270^\circ = 0$};

    \draw[red,very thick,{Latex}-] (0,0) -- (P);
    \node[red,right,scale=0.7] at (0,-0.5) {$\sin 270^\circ = -1$};

    \draw[green!60!black,dashed,thin] (1,-1.3) -- (1,1.3);
    \node[green!60!black,right,scale=0.6] at (1,-0.65) {$\tan 270^\circ$ indefinido};
\end{tikzpicture}
\end{center}

\textbf{Respuesta:} Para ángulos cuadrantales como $270^\circ$, algunas líneas pueden no estar definidas.
\end{solucion}

\begin{solucion}[title=Solucion Ejercicio 6]
\textbf{Demostración geométrica de $\sin^2\theta + \cos^2\theta = 1$}

\textbf{Usando el círculo unitario:}

Sea $P = (\cos\theta, \sin\theta)$ un punto en el círculo unitario.

\textbf{Observación geométrica:}

\begin{itemize}
    \item La línea coseno tiene longitud $|\cos\theta|$ y está sobre el eje $x$
    \item La línea seno tiene longitud $|\sin\theta|$ y es perpendicular al eje $x$
    \item Estas dos líneas forman los catetos de un triángulo rectángulo
    \item La hipotenusa de este triángulo es el radio del círculo, que vale 1
\end{itemize}

\textbf{Por el Teorema de Pitágoras:}
\[
(\text{cateto horizontal})^2 + (\text{cateto vertical})^2 = (\text{hipotenusa})^2
\]

\[
\cos^2\theta + \sin^2\theta = 1^2 = 1
\]

\begin{center}
\begin{tikzpicture}[scale=3.5]
    \draw[maincolor,very thick] (0,0) circle (1);
    \draw[-{Latex},thick] (-0.3,0) -- (1.3,0) node[right] {$x$};
    \draw[-{Latex},thick] (0,-0.3) -- (0,1.3) node[above] {$y$};

    \def\angulo{35}
    \coordinate (P) at ({\angulo}:1);

    \draw[thick,blue] (0,0) -- ({\angulo}:1 |- 0,0) node[midway,below,scale=0.7] {$\cos\theta$};
    \draw[thick,red] ({\angulo}:1 |- 0,0) -- (P) node[midway,right,scale=0.7] {$\sin\theta$};
    \draw[very thick,maincolor,-{Latex}] (0,0) -- (P) node[midway,above left,scale=0.7] {$r = 1$};

    \filldraw[maincolor] (P) circle (0.02) node[above right,scale=0.8] {$P$};

    \draw[accentcolor,-{Latex}] (0.25,0) arc (0:\angulo:0.25) node[midway,right,scale=0.7] {$\theta$};

    % Ángulo recto
    \draw[thick] ({\angulo}:1 |- 0,0) ++(-0.05,0) -- ++(-0.05,0.05) -- ++(0.05,0.05);
\end{tikzpicture}
\end{center}

\textbf{Conclusión:} La identidad pitagórica es una consecuencia directa del Teorema de Pitágoras aplicado al triángulo formado por las líneas seno, coseno y el radio unitario.
\end{solucion}

\begin{solucion}[title=Solucion Ejercicio 7]
\textbf{Ángulos: $30^\circ$, $150^\circ$, $210^\circ$ y $330^\circ$}

\textbf{Valores de seno:}

\begin{align*}
\sin 30^\circ &= \frac{1}{2} \\
\sin 150^\circ &= \sin(180^\circ - 30^\circ) = \frac{1}{2} \\
\sin 210^\circ &= \sin(180^\circ + 30^\circ) = -\frac{1}{2} \\
\sin 330^\circ &= \sin(360^\circ - 30^\circ) = -\frac{1}{2}
\end{align*}

\textbf{Patrón observado:}

\begin{itemize}
    \item Los ángulos en cuadrantes superiores ($30^\circ$ y $150^\circ$) tienen seno $= +\frac{1}{2}$
    \item Los ángulos en cuadrantes inferiores ($210^\circ$ y $330^\circ$) tienen seno $= -\frac{1}{2}$
    \item Todos tienen la misma magnitud: $|\sin\theta| = \frac{1}{2}$
\end{itemize}

\begin{center}
\begin{tikzpicture}[scale=2.5]
    \draw[maincolor,very thick] (0,0) circle (1);
    \draw[-{Latex},thick] (-1.3,0) -- (1.3,0) node[right] {$x$};
    \draw[-{Latex},thick] (0,-1.3) -- (0,1.3) node[above] {$y$};

    % Línea horizontal en y = 0.5
    \draw[green!60!black,dashed] (-1,0.5) -- (1,0.5);
    \node[green!60!black,right,scale=0.6] at (1,0.5) {$y = \frac{1}{2}$};

    % Línea horizontal en y = -0.5
    \draw[orange,dashed] (-1,-0.5) -- (1,-0.5);
    \node[orange,right,scale=0.6] at (1,-0.5) {$y = -\frac{1}{2}$};

    % Ángulos
    \foreach \ang/\col in {30/blue, 150/red, 210/purple, 330/brown} {
        \draw[\col,thick,-{Latex}] (0,0) -- (\ang:1);
        \filldraw[\col] (\ang:1) circle (0.02);
    }

    \node[blue,above right,scale=0.6] at (30:1.1) {$30^\circ$};
    \node[red,above left,scale=0.6] at (150:1.1) {$150^\circ$};
    \node[purple,below left,scale=0.6] at (210:1.1) {$210^\circ$};
    \node[brown,below right,scale=0.6] at (330:1.1) {$330^\circ$};
\end{tikzpicture}
\end{center}

\textbf{Respuesta:} Los cuatro ángulos tienen líneas seno con la misma magnitud, pero los signos dependen de si están en la mitad superior o inferior del círculo.
\end{solucion}

\begin{solucion}[title=Solucion Ejercicio 8]
\textbf{Dado:} $\theta$ en el segundo cuadrante, línea seno con longitud $\frac{3}{5}$.

\textbf{Parte a):} Valor de $\sin\theta$.

Como el ángulo está en el segundo cuadrante, el seno es positivo:
\[
\boxed{\sin\theta = \frac{3}{5}}
\]

\textbf{Parte b):} Valor de $\cos\theta$.

Usando la identidad pitagórica: $\sin^2\theta + \cos^2\theta = 1$

\begin{align*}
\left(\frac{3}{5}\right)^2 + \cos^2\theta &= 1 \\
\frac{9}{25} + \cos^2\theta &= 1 \\
\cos^2\theta &= 1 - \frac{9}{25} = \frac{16}{25} \\
\cos\theta &= \pm\frac{4}{5}
\end{align*}

Como estamos en el segundo cuadrante, el coseno es negativo:
\[
\boxed{\cos\theta = -\frac{4}{5}}
\]

\textbf{Parte c):} Valor de $\tan\theta$.

\[
\tan\theta = \frac{\sin\theta}{\cos\theta} = \frac{3/5}{-4/5} = \frac{3}{5} \cdot \frac{5}{-4} = \boxed{-\frac{3}{4}}
\]

\textbf{Verificación:} En el segundo cuadrante, la tangente debe ser negativa. ✓
\end{solucion}

\newpage

\section{Ejercicios Inversos}

Los ejercicios inversos te desafían a pensar de manera diferente sobre las líneas trigonométricas.

\begin{ejercicio}[title=Ejercicio Inverso 1]
Si la línea tangente para un ángulo $\theta$ en el tercer cuadrante tiene longitud $2$, encuentra:
\begin{itemize}
    \item[a)] El valor de $\tan\theta$
    \item[b)] Los valores de $\sin\theta$ y $\cos\theta$
\end{itemize}
\end{ejercicio}

\begin{ejercicio}[title=Ejercicio Inverso 2]
Encuentra todos los ángulos entre $0^\circ$ y $360^\circ$ cuya línea seno tenga longitud $\frac{\sqrt{2}}{2}$.
\end{ejercicio}

\begin{ejercicio}[title=Ejercicio Inverso 3]
Si un ángulo $\theta$ en el cuarto cuadrante tiene línea coseno de longitud $0.6$, determina:
\begin{itemize}
    \item[a)] El valor de $\cos\theta$
    \item[b)] El valor de $\sin\theta$
    \item[c)] Dibuja las líneas trigonométricas aproximadas
\end{itemize}
\end{ejercicio}

\begin{ejercicio}[title=Ejercicio Inverso 4]
Demuestra que para cualquier ángulo $\theta$, las líneas seno y coseno nunca pueden tener ambas longitud mayor que $1$.
\end{ejercicio}

\begin{ejercicio}[title=Ejercicio Inverso 5]
Si las líneas seno y coseno de un ángulo tienen la misma longitud, ¿cuáles son los posibles valores del ángulo entre $0^\circ$ y $360^\circ$?
\end{ejercicio}

\newpage

\section{Soluciones de Ejercicios Inversos}

\begin{solucion}[title=Solucion Ejercicio Inverso 1]
\textbf{Dado:} $\theta$ en el tercer cuadrante, línea tangente con longitud $2$.

\textbf{Parte a):} Valor de $\tan\theta$.

En el tercer cuadrante, la tangente es positiva:
\[
\boxed{\tan\theta = 2}
\]

\textbf{Parte b):} Valores de $\sin\theta$ y $\cos\theta$.

Sabemos que $\tan\theta = \frac{\sin\theta}{\cos\theta} = 2$, entonces:
\[
\sin\theta = 2\cos\theta
\]

Usando la identidad pitagórica:
\begin{align*}
\sin^2\theta + \cos^2\theta &= 1 \\
(2\cos\theta)^2 + \cos^2\theta &= 1 \\
4\cos^2\theta + \cos^2\theta &= 1 \\
5\cos^2\theta &= 1 \\
\cos^2\theta &= \frac{1}{5} \\
\cos\theta &= \pm\frac{1}{\sqrt{5}} = \pm\frac{\sqrt{5}}{5}
\end{align*}

Como estamos en el tercer cuadrante, $\cos\theta < 0$:
\[
\cos\theta = -\frac{\sqrt{5}}{5}
\]

Por lo tanto:
\[
\sin\theta = 2\cos\theta = 2 \cdot \left(-\frac{\sqrt{5}}{5}\right) = -\frac{2\sqrt{5}}{5}
\]

\textbf{Respuesta:}
\[
\boxed{\sin\theta = -\frac{2\sqrt{5}}{5}, \quad \cos\theta = -\frac{\sqrt{5}}{5}}
\]
\end{solucion}

\begin{solucion}[title=Solucion Ejercicio Inverso 2]
\textbf{Encontrar ángulos con $|\sin\theta| = \frac{\sqrt{2}}{2}$}

\textbf{Ángulos de referencia:}

Sabemos que $\sin 45^\circ = \frac{\sqrt{2}}{2}$

\textbf{Casos:}

\textbf{Caso 1:} $\sin\theta = +\frac{\sqrt{2}}{2}$ (cuadrantes I y II)

\begin{itemize}
    \item Cuadrante I: $\theta = 45^\circ$
    \item Cuadrante II: $\theta = 180^\circ - 45^\circ = 135^\circ$
\end{itemize}

\textbf{Caso 2:} $\sin\theta = -\frac{\sqrt{2}}{2}$ (cuadrantes III y IV)

\begin{itemize}
    \item Cuadrante III: $\theta = 180^\circ + 45^\circ = 225^\circ$
    \item Cuadrante IV: $\theta = 360^\circ - 45^\circ = 315^\circ$
\end{itemize}

\textbf{Respuesta:}
\[
\boxed{\theta = 45^\circ, \quad 135^\circ, \quad 225^\circ, \quad 315^\circ}
\]
\end{solucion}

\begin{solucion}[title=Solucion Ejercicio Inverso 3]
\textbf{Dado:} $\theta$ en el cuarto cuadrante, línea coseno con longitud $0.6$.

\textbf{Parte a):} Valor de $\cos\theta$.

En el cuarto cuadrante, el coseno es positivo:
\[
\boxed{\cos\theta = 0.6 = \frac{3}{5}}
\]

\textbf{Parte b):} Valor de $\sin\theta$.

Usando la identidad pitagórica:
\begin{align*}
\sin^2\theta + \cos^2\theta &= 1 \\
\sin^2\theta + \left(\frac{3}{5}\right)^2 &= 1 \\
\sin^2\theta + \frac{9}{25} &= 1 \\
\sin^2\theta &= \frac{16}{25} \\
\sin\theta &= \pm\frac{4}{5}
\end{align*}

Como estamos en el cuarto cuadrante, $\sin\theta < 0$:
\[
\boxed{\sin\theta = -\frac{4}{5} = -0.8}
\]

\textbf{Parte c):} Diagrama aproximado.

\begin{center}
\begin{tikzpicture}[scale=3.5]
    \draw[maincolor,very thick] (0,0) circle (1);
    \draw[-{Latex},thick] (-0.3,0) -- (1.3,0) node[right] {$x$};
    \draw[-{Latex},thick] (0,-1.3) -- (0,1.3) node[above] {$y$};

    % Calcular ángulo: arcsin(-0.8)
    \def\angulo{323.13}
    \coordinate (P) at ({\angulo}:1);

    \draw[maincolor,very thick,-{Latex}] (0,0) -- (P);
    \filldraw[maincolor] (P) circle (0.02) node[below right,scale=0.8] {$P$};

    \draw[blue,very thick,-{Latex}] (0,0) -- ({\angulo}:1 |- 0,0);
    \node[blue,above,scale=0.7] at (0.3,0) {$0.6$};

    \draw[red,very thick,{Latex}-] ({\angulo}:1 |- 0,0) -- (P);
    \node[red,right,scale=0.7] at (0.6,-0.4) {$-0.8$};
\end{tikzpicture}
\end{center}
\end{solucion}

\begin{solucion}[title=Solucion Ejercicio Inverso 4]
\textbf{Demostración:} Las líneas seno y coseno no pueden tener ambas longitud mayor que $1$.

\textbf{Por contradicción:}

Supongamos que existe un ángulo $\theta$ tal que:
\[
|\sin\theta| > 1 \quad \text{y} \quad |\cos\theta| > 1
\]

Entonces:
\begin{align*}
\sin^2\theta &> 1 \\
\cos^2\theta &> 1
\end{align*}

Sumando:
\[
\sin^2\theta + \cos^2\theta > 1 + 1 = 2
\]

Pero por la identidad pitagórica:
\[
\sin^2\theta + \cos^2\theta = 1
\]

Tenemos una contradicción: $1 > 2$ (falso).

\textbf{Interpretación geométrica:}

Las líneas seno y coseno son las proyecciones de un punto en el círculo unitario (radio $= 1$). Una proyección nunca puede ser más larga que el segmento original, por lo tanto:
\[
|\sin\theta| \leq 1 \quad \text{y} \quad |\cos\theta| \leq 1
\]

\textbf{Conclusión:} Es imposible que ambas líneas tengan longitud mayor que $1$ simultáneamente.
\end{solucion}

\begin{solucion}[title=Solucion Ejercicio Inverso 5]
\textbf{Encontrar ángulos con $|\sin\theta| = |\cos\theta|$}

\textbf{Condición:}
\[
|\sin\theta| = |\cos\theta|
\]

Esto implica:
\[
\sin^2\theta = \cos^2\theta
\]

Usando la identidad pitagórica:
\[
\sin^2\theta + \cos^2\theta = 1
\]

Sustituyendo $\sin^2\theta = \cos^2\theta$:
\begin{align*}
\cos^2\theta + \cos^2\theta &= 1 \\
2\cos^2\theta &= 1 \\
\cos^2\theta &= \frac{1}{2} \\
|\cos\theta| &= \frac{1}{\sqrt{2}} = \frac{\sqrt{2}}{2}
\end{align*}

Por lo tanto: $|\sin\theta| = \frac{\sqrt{2}}{2}$

\textbf{Ángulos que satisfacen esta condición:}

En cada cuadrante hay un ángulo donde $|\sin\theta| = |\cos\theta|$:

\begin{itemize}
    \item Cuadrante I: $45^\circ$ (ambos positivos)
    \item Cuadrante II: $135^\circ$ (seno positivo, coseno negativo)
    \item Cuadrante III: $225^\circ$ (ambos negativos)
    \item Cuadrante IV: $315^\circ$ (seno negativo, coseno positivo)
\end{itemize}

\textbf{Respuesta:}
\[
\boxed{\theta = 45^\circ, \quad 135^\circ, \quad 225^\circ, \quad 315^\circ}
\]

\textbf{Observación:} Estos ángulos forman las diagonales del cuadrado inscrito en el círculo unitario.
\end{solucion}

\newpage

\section{Conclusión}

Has completado esta guía sobre líneas trigonométricas. Ahora tienes las herramientas para:

\begin{itemize}
    \item Visualizar geométricamente las funciones seno, coseno y tangente
    \item Entender cómo los signos de las funciones se relacionan con las posiciones en el círculo
    \item Identificar y dibujar líneas trigonométricas para cualquier ángulo
    \item Usar esta representación geométrica para resolver problemas
\end{itemize}

\subsection*{Conceptos Clave}

\begin{nota}[title=Resumen de Lineas Trigonometricas]
\textbf{Línea Seno:} Segmento vertical desde la proyección en el eje $x$ hasta el punto en el círculo.

\textbf{Línea Coseno:} Segmento horizontal desde el origen hasta la proyección en el eje $x$.

\textbf{Línea Tangente:} Segmento sobre la tangente geométrica en $(1,0)$ hasta el radio prolongado.

\textbf{Signos por cuadrante:}
\begin{itemize}
    \item I: Todas positivas
    \item II: Solo seno positivo
    \item III: Solo tangente positiva
    \item IV: Solo coseno positivo
\end{itemize}
\end{nota}

\subsection*{Próximos Pasos}

Para continuar tu aprendizaje en trigonometría:

\begin{enumerate}
    \item Estudiar las gráficas de funciones trigonométricas (curvas sinusoidales)
    \item Aprender sobre identidades trigonométricas
    \item Explorar funciones trigonométricas inversas
    \item Aplicar la trigonometría a problemas de física y geometría
\end{enumerate}

\vspace{1cm}

\begin{center}
\textit{``La geometría es el arte de razonar bien sobre figuras mal trazadas.''} \\
--- Henri Poincaré
\end{center}

\end{document}
