% !TEX program = lualatex
\documentclass[12pt,a4paper,twoside]{article}
\usepackage{fontspec}
\usepackage[spanish,es-nodecimaldot]{babel}
\usepackage{amsmath,amssymb}
\usepackage[margin=2.5cm]{geometry}
\usepackage{xcolor}
\usepackage{tikz,pgfplots}
\usetikzlibrary{calc,arrows.meta,babel}
\usepackage{multicol}
\usepackage{enumitem}
\pgfplotsset{compat=1.18}
\definecolor{maincolor}{RGB}{26,35,126}
\definecolor{accentcolor}{RGB}{255,87,34}

% Configuración de títulos y formato
\usepackage{titlesec}
\titleformat{\section}{\Large\bfseries\color{maincolor}}{\thesection}{1em}{}
\titleformat{\subsection}{\large\bfseries\color{accentcolor}}{\thesubsection}{1em}{}

% Configuración de cajas para ejemplos
\usepackage{tcolorbox}
\tcbuselibrary{skins,breakable}

\usepackage{fancyhdr}

\pagestyle{fancy}
\fancyhf{}
\fancyhead[LE]{\small\textcolor{maincolor}{\thepage \quad Graficas de Funciones Trigonometricas}}
\fancyhead[RO]{\small\textcolor{maincolor}{Analisis y Elaboracion de Graficas \quad \thepage}}
\fancyhead[LO]{\small\textcolor{maincolor}{Grado 10 - Trigonometría}}
\fancyhead[RE]{\small\textcolor{maincolor}{Prof. Toribio De J Arrieta F}}
\fancyfoot[C]{}
\renewcommand{\headrulewidth}{0.5pt}
\renewcommand{\footrulewidth}{0pt}
\setlength{\headheight}{14pt}

\newtcolorbox{ejemplo}[1][]{
  enhanced,
  breakable,
  colback=maincolor!5,
  colframe=maincolor,
  fonttitle=\bfseries,
  title=Ejemplo Resuelto,
  #1
}

\newtcolorbox{ejercicio}[1][]{
  enhanced,
  breakable,
  colback=accentcolor!5,
  colframe=accentcolor,
  fonttitle=\bfseries,
  title=Ejercicio,
  #1
}

\newtcolorbox{solucion}[1][]{
  enhanced,
  breakable,
  colback=green!5,
  colframe=green!60!black,
  fonttitle=\bfseries,
  title=Solución,
  #1
}

\newtcolorbox{nota}[1][]{
  enhanced,
  colback=yellow!10,
  colframe=orange!80!black,
  fonttitle=\bfseries,
  title=Nota Importante,
  #1
}

\newtcolorbox{definicion}[1][]{
  enhanced,
  colback=blue!5,
  colframe=blue!60!black,
  fonttitle=\bfseries,
  title=Definición,
  #1
}

% Título
\title{\textbf{\Huge Graficas de las Funciones Trigonometricas}\\[0.5cm]
\Large Analisis y Elaboracion de Graficas}
\author{Prof. Toribio De J Arrieta F\\
\textit{La Pruebita}\\
Grado 10}
\date{\today}

\begin{document}

\maketitle

\tableofcontents
\newpage

\section{Introducción}

¡Bienvenidos a uno de los temas más visuales y aplicados de la trigonometría! Si alguna vez te has preguntado cómo funciona el sonido en tus audífonos, cómo se transmiten las señales de radio, o cómo los ingenieros predicen el comportamiento de edificios durante un terremoto, la respuesta está en las gráficas de las funciones trigonométricas.

Hasta ahora has trabajado con las funciones trigonométricas en el círculo unitario, calculando valores para ángulos específicos. Pero ahora vamos a dar un paso más allá: vamos a ver cómo estas funciones se comportan de manera continua, cómo se transforman sus gráficas, y cómo podemos usar estas transformaciones para modelar fenómenos del mundo real.

\subsection*{¿Por qué estudiar las gráficas de funciones trigonométricas?}

Las gráficas de las funciones trigonométricas no son solo dibujos bonitos en un papel. Son herramientas fundamentales en:

\begin{itemize}
    \item \textbf{Ondas sonoras:} La música que escuchas, tu voz al hablar, todo son ondas que se representan con funciones seno y coseno
    \item \textbf{Señales eléctricas:} La corriente alterna que alimenta tu casa oscila siguiendo una función seno
    \item \textbf{Mareas:} El nivel del mar sube y baja periódicamente, modelado por funciones trigonométricas
    \item \textbf{Movimiento armónico:} Péndulos, resortes, y sistemas vibratorios siguen patrones trigonométricos
    \item \textbf{Análisis de vibraciones:} Los ingenieros usan estas gráficas para diseñar edificios resistentes
    \item \textbf{Ingeniería civil:} Diseño de puentes, análisis de estructuras, predicción de comportamiento dinámico
    \item \textbf{Física de ondas:} Luz, sonido, ondas electromagnéticas, todas se modelan con funciones trigonométricas
\end{itemize}

\subsection*{¿Qué vamos a aprender?}

En esta guía vamos a dominar las transformaciones de las funciones trigonométricas:

\begin{enumerate}
    \item \textbf{Traslaciones:} Mover las gráficas horizontal y verticalmente
    \item \textbf{Reflexiones:} Voltear las gráficas sobre los ejes
    \item \textbf{Compresión y alargamiento:} Cambiar la forma de las ondas
    \item \textbf{Amplitud:} La ``altura'' de la onda
    \item \textbf{Período:} Qué tan rápido se repite el patrón
    \item \textbf{Desfase o desplazamiento de fase:} Dónde empieza el patrón
\end{enumerate}

No te preocupes si estos términos suenan complicados. Con las explicaciones paso a paso y las gráficas interactivas que vamos a ver, todo va a quedar clarísimo.

\newpage

\subsection*{La importancia de visualizar}

Una de las ventajas más grandes de trabajar con funciones trigonométricas es que podemos \textit{verlas}. A diferencia de ecuaciones abstractas, estas funciones tienen formas reconocibles que puedes dibujar, transformar y entender visualmente.

Imagina que eres un DJ controlando el volumen y el tono de la música. Lo que estás haciendo, matemáticamente, es ajustar la amplitud y la frecuencia de ondas sonoras. O piensa en un ingeniero diseñando un puente: necesita entender cómo las fuerzas oscilantes (como el viento o el tráfico) pueden hacer vibrar la estructura. Todo esto se reduce a entender las transformaciones de funciones trigonométricas.

\subsection*{Conexión con tus conocimientos previos}

Ya conoces las funciones básicas:
\begin{itemize}
    \item $y = \sin(x)$: La función seno oscila entre $-1$ y $1$
    \item $y = \cos(x)$: La función coseno también oscila entre $-1$ y $1$, pero desplazada
    \item $y = \tan(x)$: La función tangente tiene un comportamiento diferente, con asíntotas verticales
\end{itemize}

Ahora vamos a aprender a modificar estas funciones. ¿Qué pasa si escribimos $y = 2\sin(x)$? ¿O $y = \sin(2x)$? ¿Y qué tal $y = \sin(x - \frac{\pi}{4}) + 3$? Cada pequeño cambio en la ecuación produce un cambio específico y predecible en la gráfica.

¡Prepárate para convertirte en un maestro de las transformaciones trigonométricas!

\newpage

\section{Conceptos Fundamentales}

Antes de sumergirnos en las transformaciones, necesitamos asegurarnos de que tienes una base sólida sobre las gráficas básicas de las funciones trigonométricas y los conceptos clave que vamos a usar.

\subsection{Repaso de las Gráficas Básicas}

\subsubsection{Función Seno: $y = \sin(x)$}

La función seno es periódica, lo que significa que repite su patrón indefinidamente. Aquí están sus características principales:

\begin{center}
\begin{tikzpicture}
\begin{axis}[
    width=12cm,
    height=6cm,
    axis lines=middle,
    xlabel={$x$},
    ylabel={$y$},
    xmin=-0.5, xmax=7,
    ymin=-1.5, ymax=1.5,
    xtick={0,1.57,3.14,4.71,6.28},
    xticklabels={$0$,$\frac{\pi}{2}$,$\pi$,$\frac{3\pi}{2}$,$2\pi$},
    ytick={-1,0,1},
    grid=major,
    samples=200,
    domain=0:6.28,
]
\addplot[maincolor, very thick] {sin(deg(x))};
\node[maincolor] at (axis cs:5,1.3) {\large $y = \sin(x)$};
\end{axis}
\end{tikzpicture}
\end{center}

\begin{nota}
Características de $y = \sin(x)$:
\begin{itemize}
    \item Dominio: todos los números reales $(-\infty, \infty)$
    \item Rango: $[-1, 1]$
    \item Período: $2\pi$ (se repite cada $2\pi$ unidades)
    \item Amplitud: $1$ (altura máxima desde el eje central)
    \item Pasa por el origen: $\sin(0) = 0$
    \item Máximos en $x = \frac{\pi}{2} + 2\pi k$, donde $k$ es entero
    \item Mínimos en $x = \frac{3\pi}{2} + 2\pi k$
\end{itemize}
\end{nota}

\subsubsection{Función Coseno: $y = \cos(x)$}

La función coseno es muy similar al seno, pero está desplazada horizontalmente:

\begin{center}
\begin{tikzpicture}
\begin{axis}[
    width=12cm,
    height=6cm,
    axis lines=middle,
    xlabel={$x$},
    ylabel={$y$},
    xmin=-0.5, xmax=7,
    ymin=-1.5, ymax=1.5,
    xtick={0,1.57,3.14,4.71,6.28},
    xticklabels={$0$,$\frac{\pi}{2}$,$\pi$,$\frac{3\pi}{2}$,$2\pi$},
    ytick={-1,0,1},
    grid=major,
    samples=200,
    domain=0:6.28,
]
\addplot[accentcolor, very thick] {cos(deg(x))};
\node[accentcolor] at (axis cs:5,1.3) {\large $y = \cos(x)$};
\end{axis}
\end{tikzpicture}
\end{center}

\begin{nota}
Características de $y = \cos(x)$:
\begin{itemize}
    \item Dominio: todos los números reales $(-\infty, \infty)$
    \item Rango: $[-1, 1]$
    \item Período: $2\pi$
    \item Amplitud: $1$
    \item Empieza en su máximo: $\cos(0) = 1$
    \item Máximos en $x = 2\pi k$
    \item Mínimos en $x = \pi + 2\pi k$
\end{itemize}
\end{nota}

\subsubsection{Función Tangente: $y = \tan(x)$}

La tangente tiene un comportamiento muy diferente a seno y coseno:

\begin{center}
\begin{tikzpicture}
\begin{axis}[
    width=12cm,
    height=6cm,
    axis lines=middle,
    xlabel={$x$},
    ylabel={$y$},
    xmin=-0.5, xmax=7,
    ymin=-5, ymax=5,
    xtick={0,1.57,3.14,4.71,6.28},
    xticklabels={$0$,$\frac{\pi}{2}$,$\pi$,$\frac{3\pi}{2}$,$2\pi$},
    ytick={-4,-2,0,2,4},
    grid=major,
    samples=200,
    domain=0:1.4,
]
\addplot[maincolor, very thick] {tan(deg(x))};
\addplot[maincolor, very thick, domain=1.75:4.56] {tan(deg(x))};
\addplot[maincolor, very thick, domain=4.85:6.28] {tan(deg(x))};

% Asíntotas
\addplot[dashed, gray, thick] coordinates {(1.57,-5) (1.57,5)};
\addplot[dashed, gray, thick] coordinates {(4.71,-5) (4.71,5)};

\node[maincolor] at (axis cs:5,4) {\large $y = \tan(x)$};
\end{axis}
\end{tikzpicture}
\end{center}

\begin{nota}
Características de $y = \tan(x)$:
\begin{itemize}
    \item Dominio: todos los reales excepto $x = \frac{\pi}{2} + \pi k$
    \item Rango: todos los números reales $(-\infty, \infty)$
    \item Período: $\pi$ (más corto que seno y coseno)
    \item No tiene amplitud definida (crece sin límite)
    \item Pasa por el origen: $\tan(0) = 0$
    \item Asíntotas verticales en $x = \frac{\pi}{2} + \pi k$
\end{itemize}
\end{nota}

\newpage

\subsection{Amplitud}

\begin{definicion}
La \textbf{amplitud} es la distancia desde la línea central (eje medio) de la función hasta su valor máximo. Es la ``altura'' de la onda.

Para funciones de la forma $y = A\sin(x)$ o $y = A\cos(x)$, la amplitud es $|A|$.
\end{definicion}

Cuando modificamos el coeficiente $A$ en $y = A\sin(x)$, estamos cambiando la amplitud:

\begin{center}
\begin{tikzpicture}
\begin{axis}[
    width=12cm,
    height=7cm,
    axis lines=middle,
    xlabel={$x$},
    ylabel={$y$},
    xmin=-0.5, xmax=7,
    ymin=-3.5, ymax=3.5,
    xtick={0,1.57,3.14,4.71,6.28},
    xticklabels={$0$,$\frac{\pi}{2}$,$\pi$,$\frac{3\pi}{2}$,$2\pi$},
    ytick={-3,-2,-1,0,1,2,3},
    grid=major,
    samples=200,
    domain=0:6.28,
    legend pos=north east,
]
\addplot[blue, very thick] {sin(deg(x))};
\addplot[red, very thick] {2*sin(deg(x))};
\addplot[green!60!black, very thick] {0.5*sin(deg(x))};
\legend{$y = \sin(x)$ (A=1), $y = 2\sin(x)$ (A=2), $y = 0.5\sin(x)$ (A=0.5)}
\end{axis}
\end{tikzpicture}
\end{center}

\begin{nota}
Observa que:
\begin{itemize}
    \item $A > 1$: La gráfica se \textbf{estira verticalmente} (onda más alta)
    \item $0 < A < 1$: La gráfica se \textbf{comprime verticalmente} (onda más baja)
    \item $A < 0$: La gráfica se \textbf{refleja} sobre el eje $x$ (voltea verticalmente)
\end{itemize}
\end{nota}

\textbf{Ejemplo del mundo real:} En una onda sonora, la amplitud representa el volumen. Un sonido más fuerte tiene mayor amplitud.

\subsection{Período}

\begin{definicion}
El \textbf{período} es la longitud del intervalo más pequeño después del cual la función comienza a repetirse.

Para funciones de la forma $y = \sin(Bx)$ o $y = \cos(Bx)$, el período es $\frac{2\pi}{|B|}$.

Para $y = \tan(Bx)$, el período es $\frac{\pi}{|B|}$.
\end{definicion}

Cuando modificamos el coeficiente $B$ en $y = \sin(Bx)$, estamos cambiando el período:

\begin{center}
\begin{tikzpicture}
\begin{axis}[
    width=12cm,
    height=6cm,
    axis lines=middle,
    xlabel={$x$},
    ylabel={$y$},
    xmin=-0.5, xmax=7,
    ymin=-1.5, ymax=1.5,
    xtick={0,1.57,3.14,4.71,6.28},
    xticklabels={$0$,$\frac{\pi}{2}$,$\pi$,$\frac{3\pi}{2}$,$2\pi$},
    ytick={-1,0,1},
    grid=major,
    samples=300,
    domain=0:6.28,
    legend pos=north east,
]
\addplot[blue, very thick] {sin(deg(x))};
\addplot[red, very thick] {sin(2*deg(x))};
\addplot[green!60!black, very thick] {sin(0.5*deg(x))};
\legend{$y = \sin(x)$ (T=$2\pi$), $y = \sin(2x)$ (T=$\pi$), $y = \sin(0.5x)$ (T=$4\pi$)}
\end{axis}
\end{tikzpicture}
\end{center}

\begin{nota}
Observa que:
\begin{itemize}
    \item $B > 1$: La gráfica se \textbf{comprime horizontalmente} (oscila más rápido)
    \item $0 < B < 1$: La gráfica se \textbf{estira horizontalmente} (oscila más lento)
    \item El período y $B$ son inversamente proporcionales: $T = \frac{2\pi}{B}$
\end{itemize}
\end{nota}

\textbf{Ejemplo del mundo real:} En una onda sonora, el período está relacionado con la frecuencia (tono). Un período más corto significa una frecuencia más alta (sonido más agudo).

\subsection{Frecuencia}

\begin{definicion}
La \textbf{frecuencia} es el número de ciclos completos que la función realiza en un intervalo de longitud $2\pi$.

La frecuencia es el recíproco del período:
\[
f = \frac{1}{T} = \frac{|B|}{2\pi}
\]
\end{definicion}

La frecuencia nos dice qué tan ``rápido'' oscila la función. En física e ingeniería, se mide en Hertz (Hz), que son ciclos por segundo.

\subsection{Desfase o Desplazamiento de Fase}

\begin{definicion}
El \textbf{desfase} (también llamado \textbf{desplazamiento de fase}) es un desplazamiento horizontal de la gráfica.

Para funciones de la forma $y = \sin(x - C)$ o $y = \cos(x - C)$, el desfase es $C$.

\begin{itemize}
    \item Si $C > 0$: La gráfica se desplaza $C$ unidades a la \textbf{derecha}
    \item Si $C < 0$: La gráfica se desplaza $|C|$ unidades a la \textbf{izquierda}
\end{itemize}
\end{definicion}

\begin{center}
\begin{tikzpicture}
\begin{axis}[
    width=12cm,
    height=6cm,
    axis lines=middle,
    xlabel={$x$},
    ylabel={$y$},
    xmin=-0.5, xmax=7,
    ymin=-1.5, ymax=1.5,
    xtick={0,1.57,3.14,4.71,6.28},
    xticklabels={$0$,$\frac{\pi}{2}$,$\pi$,$\frac{3\pi}{2}$,$2\pi$},
    ytick={-1,0,1},
    grid=major,
    samples=200,
    domain=0:6.28,
    legend pos=north east,
]
\addplot[blue, very thick] {sin(deg(x))};
\addplot[red, very thick] {sin(deg(x-1.57))};
\addplot[green!60!black, very thick] {sin(deg(x+1.57))};
\legend{$y = \sin(x)$, $y = \sin(x-\frac{\pi}{2})$, $y = \sin(x+\frac{\pi}{2})$}
\end{axis}
\end{tikzpicture}
\end{center}

\textbf{Nota importante:} Cuando la función está en la forma $y = \sin(B(x - C))$, el desplazamiento real es $C$ (no $BC$). Siempre factoriza primero el coeficiente $B$ antes de identificar el desfase.

\subsection{Desplazamiento Vertical}

\begin{definicion}
Un \textbf{desplazamiento vertical} mueve toda la gráfica hacia arriba o hacia abajo.

Para funciones de la forma $y = \sin(x) + D$ o $y = \cos(x) + D$:
\begin{itemize}
    \item Si $D > 0$: La gráfica se desplaza $D$ unidades \textbf{hacia arriba}
    \item Si $D < 0$: La gráfica se desplaza $|D|$ unidades \textbf{hacia abajo}
\end{itemize}

El valor $D$ también se conoce como la \textbf{línea media} o \textbf{eje central} de la función.
\end{definicion}

\begin{center}
\begin{tikzpicture}
\begin{axis}[
    width=12cm,
    height=7cm,
    axis lines=middle,
    xlabel={$x$},
    ylabel={$y$},
    xmin=-0.5, xmax=7,
    ymin=-2.5, ymax=3.5,
    xtick={0,1.57,3.14,4.71,6.28},
    xticklabels={$0$,$\frac{\pi}{2}$,$\pi$,$\frac{3\pi}{2}$,$2\pi$},
    ytick={-2,-1,0,1,2,3},
    grid=major,
    samples=200,
    domain=0:6.28,
    legend pos=north east,
]
\addplot[blue, very thick] {sin(deg(x))};
\addplot[red, very thick] {sin(deg(x))+2};
\addplot[green!60!black, very thick] {sin(deg(x))-1};

% Líneas medias
\addplot[blue, dashed, thin] {0};
\addplot[red, dashed, thin] {2};
\addplot[green!60!black, dashed, thin] {-1};

\legend{$y = \sin(x)$ (D=0), $y = \sin(x)+2$ (D=2), $y = \sin(x)-1$ (D=-1)}
\end{axis}
\end{tikzpicture}
\end{center}

\begin{nota}
Con desplazamiento vertical $D$:
\begin{itemize}
    \item El nuevo rango es $[D-A, D+A]$ donde $A$ es la amplitud
    \item La línea media está en $y = D$
    \item Los máximos están en $y = D + A$
    \item Los mínimos están en $y = D - A$
\end{itemize}
\end{nota}

\textbf{Ejemplo del mundo real:} Las mareas oceanográficas tienen un nivel medio (altura promedio del agua) y oscilan arriba y abajo de ese nivel. El desplazamiento vertical representa ese nivel medio.

\newpage

\subsection{Reflexiones}

\subsubsection{Reflexión sobre el eje x}

Cuando multiplicamos la función completa por $-1$, obtenemos una reflexión sobre el eje $x$:

\begin{center}
\begin{tikzpicture}
\begin{axis}[
    width=12cm,
    height=6cm,
    axis lines=middle,
    xlabel={$x$},
    ylabel={$y$},
    xmin=-0.5, xmax=7,
    ymin=-1.5, ymax=1.5,
    xtick={0,1.57,3.14,4.71,6.28},
    xticklabels={$0$,$\frac{\pi}{2}$,$\pi$,$\frac{3\pi}{2}$,$2\pi$},
    ytick={-1,0,1},
    grid=major,
    samples=200,
    domain=0:6.28,
    legend pos=north east,
]
\addplot[blue, very thick] {sin(deg(x))};
\addplot[red, very thick, dashed] {-sin(deg(x))};
\legend{$y = \sin(x)$, $y = -\sin(x)$}
\end{axis}
\end{tikzpicture}
\end{center}

La función $y = -\sin(x)$ es la reflexión de $y = \sin(x)$ sobre el eje $x$. Los puntos que estaban arriba ahora están abajo, y viceversa.

\subsubsection{Reflexión sobre el eje y}

Cuando reemplazamos $x$ por $-x$, obtenemos una reflexión sobre el eje $y$:

\begin{center}
\begin{tikzpicture}
\begin{axis}[
    width=12cm,
    height=6cm,
    axis lines=middle,
    xlabel={$x$},
    ylabel={$y$},
    xmin=-0.5, xmax=7,
    ymin=-1.5, ymax=1.5,
    xtick={0,1.57,3.14,4.71,6.28},
    xticklabels={$0$,$\frac{\pi}{2}$,$\pi$,$\frac{3\pi}{2}$,$2\pi$},
    ytick={-1,0,1},
    grid=major,
    samples=200,
    domain=0:6.28,
    legend pos=north east,
]
\addplot[blue, very thick] {sin(deg(x))};
\addplot[red, very thick, dashed] {sin(deg(-x))};
\legend{$y = \sin(x)$, $y = \sin(-x) = -\sin(x)$}
\end{axis}
\end{tikzpicture}
\end{center}

\begin{nota}
Recuerda las propiedades de paridad:
\begin{itemize}
    \item $\sin(-x) = -\sin(x)$ (función impar): reflexión sobre $y$ = reflexión sobre $x$
    \item $\cos(-x) = \cos(x)$ (función par): reflexión sobre $y$ no cambia la gráfica
    \item $\tan(-x) = -\tan(x)$ (función impar)
\end{itemize}
\end{nota}

\subsection{La Forma General}

Todas las transformaciones que hemos visto se pueden combinar en una sola fórmula general:

\begin{tcolorbox}[enhanced,colback=maincolor!10,colframe=maincolor,title=Forma General de las Funciones Trigonométricas Transformadas]
\[
y = A\sin(B(x - C)) + D
\]
\[
y = A\cos(B(x - C)) + D
\]

Donde:
\begin{itemize}
    \item $|A|$ = Amplitud (estiramiento/compresión vertical)
    \item $B$ determina el período: $T = \frac{2\pi}{|B|}$
    \item $C$ = Desfase (desplazamiento horizontal)
    \item $D$ = Desplazamiento vertical (línea media)
    \item Si $A < 0$: hay reflexión sobre el eje $x$
    \item Si $B < 0$: hay reflexión sobre el eje $y$
\end{itemize}

\textbf{Rango:} $[D - |A|, D + |A|]$
\end{tcolorbox}

\textbf{Orden de las transformaciones:}

Para graficar $y = A\sin(B(x - C)) + D$ correctamente, aplica las transformaciones en este orden:

\begin{enumerate}
    \item Empieza con la gráfica básica: $y = \sin(x)$
    \item Aplica el estiramiento/compresión horizontal (modifica $B$): $y = \sin(Bx)$
    \item Aplica el desfase horizontal (modifica $C$): $y = \sin(B(x - C))$
    \item Aplica el estiramiento/compresión vertical (modifica $A$): $y = A\sin(B(x - C))$
    \item Aplica el desplazamiento vertical (modifica $D$): $y = A\sin(B(x - C)) + D$
\end{enumerate}

\subsection{Cómo Identificar los Parámetros}

Dado una función como $y = 3\sin(2x - \pi) + 1$, ¿cómo identificamos cada parámetro?

\textbf{Paso importante:} Primero factoriza el coeficiente de $x$ dentro del argumento:

\begin{align*}
y &= 3\sin(2x - \pi) + 1 \\
  &= 3\sin\left(2\left(x - \frac{\pi}{2}\right)\right) + 1
\end{align*}

Ahora podemos identificar:
\begin{itemize}
    \item $A = 3$ (amplitud)
    \item $B = 2$ (afecta el período: $T = \frac{2\pi}{2} = \pi$)
    \item $C = \frac{\pi}{2}$ (desfase a la derecha)
    \item $D = 1$ (desplazamiento vertical arriba)
\end{itemize}

\begin{nota}[title=Cuidado con el Desfase]
El desfase NO es el término constante que aparece originalmente. Debes factorizar primero.

Ejemplo: En $y = \sin(2x - \pi)$, el desfase NO es $\pi$.

Correcto: $y = \sin(2(x - \frac{\pi}{2}))$, entonces el desfase es $C = \frac{\pi}{2}$.
\end{nota}

\newpage

\subsection{Compresión y Alargamiento}

Vamos a profundizar más en cómo los coeficientes $A$ y $B$ afectan la forma de la gráfica.

\subsubsection{Compresión y Alargamiento Vertical (Coeficiente $A$)}

El coeficiente $A$ en $y = A\sin(x)$ controla el estiramiento o compresión en la dirección vertical:

\begin{itemize}
    \item Si $|A| > 1$: La gráfica se \textbf{estira verticalmente} (se hace más alta)
    \item Si $0 < |A| < 1$: La gráfica se \textbf{comprime verticalmente} (se hace más baja)
    \item Si $A < 0$: Además del estiramiento/compresión, hay una \textbf{reflexión sobre el eje x}
\end{itemize}

\begin{center}
\begin{tikzpicture}
\begin{axis}[
    width=12cm,
    height=7cm,
    axis lines=middle,
    xlabel={$x$},
    ylabel={$y$},
    xmin=-0.5, xmax=7,
    ymin=-4, ymax=4,
    xtick={0,1.57,3.14,4.71,6.28},
    xticklabels={$0$,$\frac{\pi}{2}$,$\pi$,$\frac{3\pi}{2}$,$2\pi$},
    ytick={-3,-2,-1,0,1,2,3},
    grid=major,
    samples=200,
    domain=0:6.28,
    legend pos=north east,
]
\addplot[blue, very thick] {sin(deg(x))};
\addplot[red, very thick] {3*sin(deg(x))};
\addplot[green!60!black, very thick] {-2*sin(deg(x))};
\legend{$A=1$, $A=3$ (estiramiento), $A=-2$ (estiramiento + reflexión)}
\end{axis}
\end{tikzpicture}
\end{center}

\subsubsection{Compresión y Alargamiento Horizontal (Coeficiente $B$)}

El coeficiente $B$ en $y = \sin(Bx)$ controla el estiramiento o compresión en la dirección horizontal:

\begin{itemize}
    \item Si $|B| > 1$: La gráfica se \textbf{comprime horizontalmente} (oscila más veces en el mismo intervalo)
    \item Si $0 < |B| < 1$: La gráfica se \textbf{estira horizontalmente} (oscila menos veces en el mismo intervalo)
    \item El período se calcula como $T = \frac{2\pi}{|B|}$
\end{itemize}

\begin{center}
\begin{tikzpicture}
\begin{axis}[
    width=12cm,
    height=6cm,
    axis lines=middle,
    xlabel={$x$},
    ylabel={$y$},
    xmin=-0.5, xmax=7,
    ymin=-1.5, ymax=1.5,
    xtick={0,1.57,3.14,4.71,6.28},
    xticklabels={$0$,$\frac{\pi}{2}$,$\pi$,$\frac{3\pi}{2}$,$2\pi$},
    ytick={-1,0,1},
    grid=major,
    samples=300,
    domain=0:6.28,
    legend pos=north east,
]
\addplot[blue, very thick] {sin(deg(x))};
\addplot[red, very thick] {sin(3*deg(x))};
\addplot[green!60!black, very thick] {sin(0.5*deg(x))};
\legend{$B=1$ (T=$2\pi$), $B=3$ (T=$\frac{2\pi}{3}$), $B=0.5$ (T=$4\pi$)}
\end{axis}
\end{tikzpicture}
\end{center}

\begin{nota}[title=Intuición sobre B]
Piensa en $B$ como un ``acelerador de tiempo'':
\begin{itemize}
    \item $B = 2$: La función oscila al doble de velocidad (dos ciclos completos en $2\pi$)
    \item $B = 0.5$: La función oscila a la mitad de velocidad (medio ciclo en $2\pi$)
\end{itemize}
\end{nota}

\subsection{Resumen Visual de Transformaciones}

A continuación, una tabla resumen de cómo cada parámetro afecta la gráfica:

\begin{center}
\renewcommand{\arraystretch}{1.8}
\begin{tabular}{|c|c|c|c|}
\hline
\textbf{Parámetro} & \textbf{Efecto} & \textbf{Fórmula} & \textbf{Qué controla} \\
\hline
$A$ & Estiramiento vertical & $y = A\sin(x)$ & Amplitud = $|A|$ \\
\hline
$B$ & Compresión horizontal & $y = \sin(Bx)$ & Período = $\frac{2\pi}{|B|}$ \\
\hline
$C$ & Desplazamiento horizontal & $y = \sin(x - C)$ & Desfase = $C$ \\
\hline
$D$ & Desplazamiento vertical & $y = \sin(x) + D$ & Línea media = $D$ \\
\hline
$-A$ & Reflexión sobre eje $x$ & $y = -A\sin(x)$ & Voltea verticalmente \\
\hline
$-B$ & Reflexión sobre eje $y$ & $y = \sin(-Bx)$ & Voltea horizontalmente \\
\hline
\end{tabular}
\end{center}

\newpage

\section{Conclusión}

¡Excelente trabajo! Has completado la primera parte de esta guía sobre gráficas de funciones trigonométricas. Ahora tienes una base sólida sobre:

\begin{itemize}
    \item Las gráficas básicas de seno, coseno y tangente
    \item Qué es la amplitud y cómo afecta la gráfica
    \item Qué es el período y cómo se relaciona con la frecuencia
    \item Cómo funcionan los desplazamientos horizontales (desfase) y verticales
    \item Cómo las reflexiones voltean las gráficas
    \item La forma general de las funciones trigonométricas transformadas
    \item Cómo la compresión y el alargamiento modifican las ondas
\end{itemize}

\subsection*{Conceptos Clave para Recordar}

\begin{tcolorbox}[enhanced,colback=maincolor!10,colframe=maincolor,title=Fórmulas Esenciales]
\textbf{Forma general:}
\[
y = A\sin(B(x - C)) + D \quad \text{o} \quad y = A\cos(B(x - C)) + D
\]

\textbf{Parámetros:}
\begin{itemize}
    \item Amplitud: $|A|$
    \item Período: $T = \frac{2\pi}{|B|}$
    \item Frecuencia: $f = \frac{|B|}{2\pi}$
    \item Desfase: $C$ (positivo = derecha, negativo = izquierda)
    \item Línea media: $D$
    \item Rango: $[D - |A|, D + |A|]$
\end{itemize}

\textbf{Para la tangente:}
\[
y = A\tan(B(x - C)) + D
\]
Período: $T = \frac{\pi}{|B|}$ (nota: $\pi$, no $2\pi$)
\end{tcolorbox}

\subsection*{Conexiones con el Mundo Real}

Recuerda que estas transformaciones no son solo ejercicios matemáticos abstractos. Son herramientas que se usan todos los días en:

\begin{itemize}
    \item \textbf{Música y audio:} Cada nota musical es una onda con amplitud (volumen), frecuencia (tono) y fase específicos
    \item \textbf{Ingeniería eléctrica:} La corriente alterna en tu casa sigue una función seno con frecuencia de 60 Hz (en América) o 50 Hz (en Europa)
    \item \textbf{Oceanografía:} Las mareas suben y bajan siguiendo patrones sinusoidales con períodos de aproximadamente 12.4 horas
    \item \textbf{Medicina:} Los electrocardiogramas y los ritmos circadianos se pueden modelar con funciones trigonométricas
    \item \textbf{Física:} El movimiento de un péndulo, las vibraciones de un resorte, las ondas de luz, todo sigue patrones sinusoidales
    \item \textbf{Ingeniería civil:} El análisis de vibraciones en puentes y edificios usa estas funciones para predecir comportamientos dinámicos
    \item \textbf{Telecomunicaciones:} Las señales de radio, televisión, WiFi y celular son ondas electromagnéticas que se modelan con funciones trigonométricas
\end{itemize}

\subsection*{Consejos para el Éxito}

\begin{enumerate}
    \item \textbf{Practica identificar parámetros:} Cuando veas una función como $y = 2\sin(3x - \pi) + 1$, acostúmbrate a factorizar y extraer cada parámetro inmediatamente

    \item \textbf{Dibuja las gráficas básicas de memoria:} Debes poder dibujar $y = \sin(x)$, $y = \cos(x)$ y $y = \tan(x)$ sin pensarlo. Son tu punto de partida para todas las transformaciones

    \item \textbf{Piensa en el orden:} Cuando apliques múltiples transformaciones, hazlo en el orden correcto: horizontal primero (B, C), luego vertical (A, D)

    \item \textbf{Verifica con puntos clave:} Identifica dónde están los máximos, mínimos, y cruces con el eje $x$ para verificar que tu gráfica es correcta

    \item \textbf{Usa la tecnología:} Herramientas como Desmos, GeoGebra o calculadoras gráficas te permiten verificar tus respuestas y desarrollar intuición visual
\end{enumerate}

\subsection*{Lo que Viene a Continuación}

En las siguientes partes de esta guía verás:

\begin{itemize}
    \item \textbf{Parte 2:} Ejemplos resueltos paso a paso que te mostrarán cómo aplicar todos estos conceptos
    \item \textbf{Parte 3:} Ejercicios propuestos con soluciones detalladas para que practiques y domines las transformaciones
\end{itemize}

%INSERTAR_EJEMPLOS_AQUI%

%INSERTAR_EJERCICIOS_PROPUESTOS_AQUI%

%INSERTAR_SOLUCIONES_AQUI%

%INSERTAR_EJERCICIOS_INVERSOS_AQUI%

%INSERTAR_SOLUCIONES_INVERSOS_AQUI%

\subsection*{Reflexión Final}

Las funciones trigonométricas son el lenguaje universal del movimiento periódico y las ondas. Desde el latido de tu corazón hasta las ondas de luz que te permiten ver estas palabras, todo está conectado por estos patrones matemáticos hermosos y elegantes.

Dominar las transformaciones de estas funciones te abre las puertas a entender fenómenos complejos del mundo real. No se trata solo de memorizar fórmulas, sino de desarrollar una intuición visual y analítica que te servirá en física, ingeniería, ciencias de la computación, y muchas otras áreas.

¡Sigue adelante con confianza! Los ejemplos y ejercicios que vienen te darán toda la práctica que necesitas para convertirte en un experto en gráficas trigonométricas.

\vspace{1cm}

\begin{center}
\textit{``Las matemáticas son la música de la razón.''} \\
--- James Joseph Sylvester
\end{center}

\end{document}
