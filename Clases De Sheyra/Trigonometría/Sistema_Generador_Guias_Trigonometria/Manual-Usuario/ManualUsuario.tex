% Manual de Usuario: Sistema Generador de Guías de Trigonometría v1.0
\documentclass[12pt,a4paper]{article}

% Paquetes esenciales
\usepackage{fontspec}
\usepackage[spanish,es-nodecimaldot]{babel}
\usepackage[margin=2.5cm]{geometry}
\usepackage{amsmath,amssymb}
\usepackage{xcolor}
\usepackage{tcolorbox}
\tcbuselibrary{breakable,skins}
\usepackage{fancyhdr}
\usepackage{titlesec}
\usepackage{enumitem}
\usepackage{hyperref}

% Colores institucionales
\definecolor{maincolor}{RGB}{26,35,126}
\definecolor{accentcolor}{RGB}{255,87,34}
\definecolor{successcolor}{RGB}{76,175,80}
\definecolor{warningcolor}{RGB}{255,193,7}

% Configuración de hipervínculos
\hypersetup{
    colorlinks=true,
    linkcolor=maincolor,
    urlcolor=accentcolor,
    citecolor=maincolor
}

% Configuración de headers
\pagestyle{fancy}
\fancyhf{}
\fancyhead[L]{Manual de Usuario}
\fancyhead[R]{Sistema Generador v1.0}
\fancyfoot[C]{\thepage}
\renewcommand{\headrulewidth}{0.5pt}

% Entornos personalizados
\newtcolorbox{instruccion}{
    colback=maincolor!5,
    colframe=maincolor,
    fonttitle=\bfseries,
    title=Instrucción,
    breakable
}

\newtcolorbox{ejemplo}{
    colback=successcolor!5,
    colframe=successcolor,
    fonttitle=\bfseries,
    title=Ejemplo de Uso,
    breakable
}

\newtcolorbox{importante}{
    colback=accentcolor!5,
    colframe=accentcolor,
    fonttitle=\bfseries,
    title=Importante,
    breakable
}

\newtcolorbox{consejo}{
    colback=warningcolor!5,
    colframe=warningcolor,
    fonttitle=\bfseries,
    title=Consejo,
    breakable
}

% Título
\title{
    \Huge\bfseries Manual de Usuario\\
    \LARGE Sistema Generador de Guías\\
    \large Trigonometría v1.0
}
\author{Prof. Toribio De J Arrieta F}
\date{Noviembre 2025}

\begin{document}

\maketitle
\thispagestyle{empty}
\newpage

\tableofcontents
\newpage

\section{Introducción}

Este manual te guiará paso a paso en el uso del \textbf{Sistema Generador de Guías de Trigonometría v1.0}, optimizado con tecnología multi-agente para crear guías educativas de alta calidad en tiempo récord.

\subsection{Características del Sistema}

\begin{itemize}
    \item \textbf{Optimización Multi-Agente:} Utiliza 3 subagentes trabajando en paralelo
    \item \textbf{Rapidez:} Genera guías de 25--35 páginas en solo 2 minutos
    \item \textbf{Eficiencia:} Ahorro de 60\% en tiempo vs. sistema tradicional
    \item \textbf{Calidad:} Basado en plantilla verificada de 30 páginas
    \item \textbf{Automatización:} Compilación y guardado en git automático
\end{itemize}

\subsection{¿Qué Genera el Sistema?}

El sistema crea automáticamente:

\begin{enumerate}
    \item Archivo \texttt{.tex} completo (25--35 páginas)
    \item PDF compilado profesional
    \item Archivo \texttt{README.md} de documentación
    \item Commit en repositorio git local
    \item Push al repositorio remoto
\end{enumerate}

\newpage

\section{Iniciar el Sistema}

\begin{instruccion}
Para iniciar el sistema, envía exactamente este mensaje al asistente de IA (Claude Code):
\end{instruccion}

\begin{tcolorbox}[colback=gray!10, colframe=gray!50, boxrule=1pt]
\texttt{Quiero crear una nueva guía de Trigonometría usando el sistema en:\\
Sistema\_Generador\_Guias\_Trigonometria/PROMPT\_TRIGONOMETRIA\_v1.0.md}
\end{tcolorbox}

\begin{importante}
El asistente automáticamente:
\begin{itemize}
    \item Leerá el PROMPT con todas las especificaciones
    \item Iniciará el proceso interactivo de 14 preguntas
    \item Configurará los 3 subagentes para trabajar en paralelo
\end{itemize}
\end{importante}

\newpage

\section{Las 14 Preguntas}

El sistema te hará 14 preguntas para configurar tu guía. A continuación se describe cada una:

\subsection{Pregunta 1: Título de la Guía}

\begin{instruccion}
\textbf{Pregunta:} ¿Cuál es el TÍTULO de la guía?
\end{instruccion}

\begin{consejo}
Usa un título descriptivo y claro. Ejemplos:
\begin{itemize}
    \item ``Identidades Trigonométricas''
    \item ``Ecuaciones Trigonométricas''
    \item ``Ley de Senos y Cosenos''
\end{itemize}
\end{consejo}

\subsection{Pregunta 2: Autor}

\begin{instruccion}
\textbf{Pregunta:} ¿Quién es el AUTOR?
\end{instruccion}

\begin{ejemplo}
\textbf{Respuesta típica:}\\
\texttt{Prof. Toribio De J Arrieta F}
\end{ejemplo}

\subsection{Pregunta 3: Institución}

\begin{instruccion}
\textbf{Pregunta:} ¿Cuál es la INSTITUCIÓN?
\end{instruccion}

\begin{ejemplo}
\textbf{Respuesta típica:}\\
\texttt{La Pruebita}
\end{ejemplo}

\subsection{Pregunta 4: Fecha}

\begin{instruccion}
\textbf{Pregunta:} ¿Fecha de creación?
\end{instruccion}

\begin{consejo}
Responde: \texttt{Toma la fecha de compilación}\\
Esto hará que LaTeX use \texttt{\textbackslash today} automáticamente.
\end{consejo}

\subsection{Pregunta 5: Título Corto}

\begin{instruccion}
\textbf{Pregunta:} ¿Título corto para encabezados?
\end{instruccion}

\begin{consejo}
Usa una versión corta del título (máximo 40 caracteres).\\
Ejemplo: Si el título es ``Identidades Trigonométricas Fundamentales'',\\
el título corto puede ser: ``Identidades Trigonométricas''
\end{consejo}

\subsection{Pregunta 6: Grado}

\begin{instruccion}
\textbf{Pregunta:} ¿Para qué GRADO es la guía?
\end{instruccion}

\begin{ejemplo}
\textbf{Respuesta típica:}\\
\texttt{10} (para Grado 10)
\end{ejemplo}

\subsection{Pregunta 7: Asignatura}

\begin{instruccion}
\textbf{Pregunta:} ¿Qué ASIGNATURA/ÁREA?
\end{instruccion}

\begin{ejemplo}
\textbf{Respuesta típica:}\\
\texttt{Trigonometría}
\end{ejemplo}

\subsection{Pregunta 8: Elementos Clave}

\begin{instruccion}
\textbf{Pregunta:} ¿Qué ELEMENTOS CLAVE debe incluir?
\end{instruccion}

\begin{importante}
Esta es la pregunta MÁS IMPORTANTE. Aquí defines el contenido técnico de tu guía.
\end{importante}

\begin{consejo}
Sé específico y enumera en orden lógico. Ejemplos:

\textbf{Para Identidades Trigonométricas:}
\begin{itemize}[leftmargin=1cm]
    \item Identidades pitagóricas
    \item Identidades recíprocas
    \item Identidades de cociente
    \item Identidades de suma y diferencia
    \item Identidades de ángulo doble
    \item Identidades de ángulo mitad
    \item Simplificación de expresiones
    \item Verificación de identidades
\end{itemize}

\textbf{Para Ecuaciones Trigonométricas:}
\begin{itemize}[leftmargin=1cm]
    \item Ecuaciones básicas
    \item Ecuaciones cuadráticas en funciones trigonométricas
    \item Ecuaciones con múltiples ángulos
    \item Soluciones generales
    \item Restricciones de dominio
\end{itemize}
\end{consejo}

\subsection{Pregunta 9: Aplicaciones}

\begin{instruccion}
\textbf{Pregunta:} ¿Qué APLICACIONES de la vida real mencionar?
\end{instruccion}

\begin{consejo}
Menciona al menos 3--5 aplicaciones reales. Ejemplos:
\begin{itemize}
    \item Análisis de ondas
    \item Ingeniería eléctrica
    \item Física ondulatoria
    \item Procesamiento de señales
    \item Astronomía
    \item Navegación
    \item Arquitectura
    \item Movimiento armónico simple
\end{itemize}
\end{consejo}

\subsection{Pregunta 10: Ejemplos Resueltos}

\begin{instruccion}
\textbf{Pregunta:} ¿Cuántos EJEMPLOS RESUELTOS deseas?
\end{instruccion}

\begin{consejo}
Rango recomendado: 5--7 ejemplos\\
Cada ejemplo será resuelto paso a paso con gráficas donde aplique.
\end{consejo}

\subsection{Pregunta 11: Ejercicios Propuestos}

\begin{instruccion}
\textbf{Pregunta:} ¿Cuántos EJERCICIOS PROPUESTOS deseas?
\end{instruccion}

\begin{consejo}
Rango recomendado: 7--8 ejercicios\\
Cada ejercicio tendrá su solución detallada al final de la guía.
\end{consejo}

\subsection{Pregunta 12: Ejercicios Inversos}

\begin{instruccion}
\textbf{Pregunta:} ¿Necesitas ejercicios inversos?
\end{instruccion}

\begin{consejo}
Responde: \texttt{Sí}

Los ejercicios inversos son creativos y ayudan a profundizar:
\begin{itemize}
    \item Construcción de funciones
    \item Diseño de problemas
    \item Análisis inverso
\end{itemize}
\end{consejo}

\subsection{Pregunta 13: Carpeta}

\begin{instruccion}
\textbf{Pregunta:} ¿En qué CARPETA se guardará la guía?
\end{instruccion}

\begin{importante}
Usa el formato: \texttt{Clases De Sheyra/Trigonometría/Nombre\_Tema}

Donde \texttt{Nombre\_Tema} usa:
\begin{itemize}
    \item Mayúsculas en cada palabra
    \item Guiones bajos (\_) en lugar de espacios
    \item Sin tildes ni caracteres especiales
\end{itemize}
\end{importante}

\begin{ejemplo}
\textbf{Ejemplos correctos:}
\begin{itemize}
    \item \texttt{Clases De Sheyra/Trigonometría/Identidades\_Trigonometricas}
    \item \texttt{Clases De Sheyra/Trigonometría/Ley\_Senos\_Cosenos}
    \item \texttt{Clases De Sheyra/Trigonometría/Ecuaciones\_Trigonometricas}
\end{itemize}

\textbf{Ejemplos incorrectos:}
\begin{itemize}
    \item \texttt{identidades trigonométricas} (minúsculas y espacios)
    \item \texttt{Identidades-Trig} (guión medio en lugar de bajo)
\end{itemize}
\end{ejemplo}

\subsection{Pregunta 14: Nombre de Archivo}

\begin{instruccion}
\textbf{Pregunta:} ¿Nombre del archivo .tex?
\end{instruccion}

\begin{importante}
Usa el formato: \texttt{GuiaTemaEnCamelCase.tex}

Donde:
\begin{itemize}
    \item Inicia con ``Guia''
    \item Usa CamelCase (primera letra de cada palabra en mayúscula)
    \item Sin espacios ni guiones
    \item Extensión \texttt{.tex}
\end{itemize}
\end{importante}

\begin{ejemplo}
\textbf{Ejemplos correctos:}
\begin{itemize}
    \item \texttt{GuiaIdentidadesTrigonometricas.tex}
    \item \texttt{GuiaLeySenos.tex}
    \item \texttt{GuiaEcuacionesTrigonometricas.tex}
\end{itemize}

\textbf{Ejemplos incorrectos:}
\begin{itemize}
    \item \texttt{guia\_identidades.tex} (minúsculas y guiones bajos)
    \item \texttt{Identidades.tex} (falta prefijo ``Guia'')
\end{itemize}
\end{ejemplo}

\newpage

\section{Confirmación}

\subsection{Resumen}

Después de responder las 14 preguntas, el asistente mostrará un resumen completo:

\begin{tcolorbox}[colback=gray!10, colframe=gray!50, boxrule=1pt, breakable]
\textbf{=== RESUMEN DE CONFIGURACIÓN ===}

\textbf{DOCUMENTO:}
\begin{itemize}[leftmargin=1cm]
    \item Título: [tu título]
    \item Autor: [tu nombre]
    \item Institución: [tu institución]
    \item Fecha: \textbackslash today
\end{itemize}

\textbf{ACADÉMICO:}
\begin{itemize}[leftmargin=1cm]
    \item Tema: [tu tema]
    \item Grado: [tu grado]
    \item Asignatura: [tu asignatura]
\end{itemize}

\textbf{CONTENIDO:}
\begin{itemize}[leftmargin=1cm]
    \item Elementos clave: [tus elementos]
    \item Aplicaciones: [tus aplicaciones]
    \item Ejemplos resueltos: [cantidad]
    \item Ejercicios propuestos: [cantidad]
    \item Ejercicios inversos: [Sí/No]
\end{itemize}

\textbf{UBICACIÓN:}
\begin{itemize}[leftmargin=1cm]
    \item Carpeta: [ruta de carpeta]
    \item Archivo: [nombre.tex]
    \item Ruta completa: [ruta completa]
\end{itemize}

¿Es correcta esta información? (Sí/No)
\end{tcolorbox}

\subsection{Confirmación}

\begin{instruccion}
Si todo está correcto, responde: \texttt{Sí}\\
Si necesitas corregir algo, responde: \texttt{No} y el sistema reiniciará las preguntas.
\end{instruccion}

\newpage

\section{Generación Automática}

\subsection{Proceso Multi-Agente}

Una vez confirmado, el sistema ejecuta automáticamente:

\begin{enumerate}
    \item \textbf{Fase 1: Invocación Paralela (30 segundos)}
    \begin{itemize}
        \item El asistente principal invoca 3 subagentes simultáneamente
        \item Usa UN SOLO MENSAJE con 3 llamadas a \texttt{Task tool}
    \end{itemize}

    \item \textbf{Fase 2: Generación Paralela (1--1.5 minutos)}
    \begin{itemize}
        \item \textbf{Subagente 1:} Genera estructura base (preámbulo, intro, conceptos, conclusión)
        \item \textbf{Subagente 2:} Genera ejemplos resueltos con gráficas + ejercicios inversos
        \item \textbf{Subagente 3:} Genera ejercicios propuestos + soluciones detalladas
    \end{itemize}

    \item \textbf{Fase 3: Ensamblaje y Finalización (1 minuto)}
    \begin{itemize}
        \item El asistente principal ensambla las 3 partes
        \item Compila con \texttt{lualatex} (2 pasadas)
        \item Crea \texttt{README.md}
        \item Guarda en git local
        \item Envía a repositorio remoto
    \end{itemize}
\end{enumerate}

\subsection{Tiempo Total}

\begin{tcolorbox}[colback=successcolor!10, colframe=successcolor, boxrule=1pt]
\textbf{Tiempo total estimado:} 2--3 minutos

\textbf{Distribución:}
\begin{itemize}
    \item Preguntas y respuestas: 1--2 min (depende de ti)
    \item Generación automática: 2 min (sistema paralelo)
\end{itemize}
\end{tcolorbox}

\subsection{Resultados}

Al finalizar, verás mensajes de confirmación:

\begin{tcolorbox}[colback=gray!10, colframe=gray!50, boxrule=1pt]
✅ Directorio creado: [nombre carpeta]/\\
✅ Archivo generado: [nombre.tex]\\
✅ Compilación exitosa: [X] páginas, [Y] KB\\
✅ README.md creado\\
✅ Guardado en git local (commit [hash])\\
✅ Enviado a repositorio remoto
\end{tcolorbox}

\newpage

\section{Consejos y Mejores Prácticas}

\subsection{Elementos Clave (Pregunta 8)}

\begin{consejo}
\textbf{Sé específico y ordenado:}
\begin{itemize}
    \item ✅ ``Identidades pitagóricas, de suma y diferencia, ángulo doble''
    \item ❌ ``Identidades trigonométricas en general''
\end{itemize}

El orden que proporciones será el orden en que se presentarán los conceptos.
\end{consejo}

\subsection{Aplicaciones (Pregunta 9)}

\begin{consejo}
\textbf{Menciona al menos 3--5 aplicaciones:}

El sistema las desarrollará en la introducción y conclusión, haciendo la guía más práctica y relevante.
\end{consejo}

\subsection{Número de Ejemplos y Ejercicios}

\begin{consejo}
\textbf{Cantidades recomendadas:}
\begin{itemize}
    \item \textbf{5--7 ejemplos} resueltos (muy detallados con gráficas)
    \item \textbf{7--8 ejercicios} propuestos (con soluciones completas)
    \item \textbf{Sí} a ejercicios inversos (agregan profundidad)
\end{itemize}

Esto genera guías de 25--35 páginas con contenido balanceado.
\end{consejo}

\subsection{Nombres de Archivos y Carpetas}

\begin{importante}
\textbf{Formato de carpetas:}
\begin{itemize}
    \item Usar: \texttt{Tema\_Con\_Mayusculas\_Y\_Guiones\_Bajos}
    \item Evitar: espacios, tildes, guiones medios
\end{itemize}

\textbf{Formato de archivos .tex:}
\begin{itemize}
    \item Usar: \texttt{GuiaTemaEnCamelCase.tex}
    \item Siempre iniciar con ``Guia''
    \item Sin espacios ni guiones
\end{itemize}
\end{importante}

\newpage

\section{Verificación de Calidad}

\subsection{Checklist Post-Generación}

Después de que el sistema genere tu guía, verifica:

\begin{itemize}
    \item[$\square$] PDF compilado sin errores
    \item[$\square$] Nombre del profesor visible en páginas pares
    \item[$\square$] Todas las secciones presentes:
    \begin{itemize}
        \item[$\square$] Portada
        \item[$\square$] Tabla de contenidos
        \item[$\square$] Introducción
        \item[$\square$] Conceptos fundamentales
        \item[$\square$] Ejemplos resueltos
        \item[$\square$] Ejercicios propuestos
        \item[$\square$] Soluciones detalladas
        \item[$\square$] Ejercicios inversos (si solicitados)
        \item[$\square$] Soluciones de inversos
        \item[$\square$] Conclusión
    \end{itemize}
    \item[$\square$] Gráficas correctamente renderizadas
    \item[$\square$] README.md creado en el directorio
    \item[$\square$] Commit creado en git
    \item[$\square$] Push exitoso al repositorio remoto
\end{itemize}

\subsection{Estadísticas Típicas}

Una guía generada típicamente tiene:

\begin{tcolorbox}[colback=gray!10, colframe=gray!50, boxrule=1pt]
\textbf{Páginas:} 25--35 páginas\\
\textbf{Tamaño PDF:} 150--250 KB\\
\textbf{Tiempo de generación:} 2 minutos\\
\textbf{Archivos creados:} 2 (.tex y README.md)\\
\textbf{Commits git:} 1 commit automático
\end{tcolorbox}

\newpage

\section{Solución de Problemas}

\subsection{Error de Compilación}

\begin{importante}
Si el PDF no se genera correctamente:
\begin{enumerate}
    \item Revisa el archivo \texttt{.log} para ver el error específico
    \item El asistente puede corregir automáticamente errores comunes
    \item Pide al asistente: ``Revisa el archivo .log y corrige los errores''
\end{enumerate}
\end{importante}

\subsection{Contenido Incorrecto}

Si el contenido no es lo esperado:

\begin{consejo}
Puedes pedir ajustes específicos:
\begin{itemize}
    \item ``Agrega más ejemplos sobre identidades de suma''
    \item ``Cambia el tono a más formal''
    \item ``Agrega una sección sobre aplicaciones en física''
\end{itemize}
\end{consejo}

\subsection{Sistema No Encuentra el PROMPT}

\begin{importante}
Asegúrate de estar en el directorio correcto:

\texttt{cd "/Users/toribioarrieta/Documents/LaTeX-GitHub/LaTeX-Varios/\\
Clases De Sheyra/Trigonometría"}

Luego especifica la ruta completa del PROMPT:

\texttt{Sistema\_Generador\_Guias\_Trigonometria/PROMPT\_TRIGONOMETRIA\_v1.0.md}
\end{importante}

\newpage

\section{Ejemplo Completo}

A continuación se presenta un ejemplo completo de una sesión de generación:

\subsection{Paso 1: Iniciar}

\begin{tcolorbox}[colback=gray!10, colframe=gray!50, boxrule=1pt]
\textbf{Usuario:}\\
Quiero crear una nueva guía de Trigonometría usando el sistema en:\\
Sistema\_Generador\_Guias\_Trigonometria/PROMPT\_TRIGONOMETRIA\_v1.0.md
\end{tcolorbox}

\subsection{Paso 2: Responder Preguntas}

\begin{tcolorbox}[colback=gray!10, colframe=gray!50, boxrule=1pt, breakable]
\textbf{Pregunta 1:} ¿Cuál es el TÍTULO de la guía?\\
\textbf{Respuesta:} Identidades Trigonométricas

\textbf{Pregunta 2:} ¿Quién es el AUTOR?\\
\textbf{Respuesta:} Prof. Toribio De J Arrieta F

\textbf{Pregunta 3:} ¿Cuál es la INSTITUCIÓN?\\
\textbf{Respuesta:} La Pruebita

\textbf{Pregunta 4:} ¿Fecha de creación?\\
\textbf{Respuesta:} Toma la fecha de compilación

\textbf{Pregunta 5:} ¿Título corto para encabezados?\\
\textbf{Respuesta:} Identidades Trigonométricas

\textbf{Pregunta 6:} ¿Para qué GRADO es la guía?\\
\textbf{Respuesta:} 10

\textbf{Pregunta 7:} ¿Qué ASIGNATURA/ÁREA?\\
\textbf{Respuesta:} Trigonometría

\textbf{Pregunta 8:} ¿Qué ELEMENTOS CLAVE debe incluir?\\
\textbf{Respuesta:} Identidades pitagóricas, identidades recíprocas, identidades de cociente, identidades de suma y diferencia, identidades de ángulo doble, identidades de ángulo mitad, simplificación de expresiones, verificación de identidades

\textbf{Pregunta 9:} ¿Qué APLICACIONES de la vida real mencionar?\\
\textbf{Respuesta:} Análisis de ondas, ingeniería eléctrica, física ondulatoria, procesamiento de señales, astronomía

\textbf{Pregunta 10:} ¿Cuántos EJEMPLOS RESUELTOS deseas?\\
\textbf{Respuesta:} 7

\textbf{Pregunta 11:} ¿Cuántos EJERCICIOS PROPUESTOS deseas?\\
\textbf{Respuesta:} 8

\textbf{Pregunta 12:} ¿Necesitas ejercicios inversos?\\
\textbf{Respuesta:} Sí

\textbf{Pregunta 13:} ¿En qué CARPETA se guardará la guía?\\
\textbf{Respuesta:} Clases De Sheyra/Trigonometría/Identidades\_Trigonometricas

\textbf{Pregunta 14:} ¿Nombre del archivo .tex?\\
\textbf{Respuesta:} GuiaIdentidadesTrigonometricas.tex
\end{tcolorbox}

\subsection{Paso 3: Confirmar}

\begin{tcolorbox}[colback=gray!10, colframe=gray!50, boxrule=1pt]
\textbf{Asistente muestra resumen completo}

\textbf{Usuario:}\\
Sí. Es correcta esta información
\end{tcolorbox}

\subsection{Paso 4: Generación Automática}

\begin{tcolorbox}[colback=successcolor!10, colframe=successcolor, boxrule=1pt, breakable]
El sistema automáticamente:
\begin{itemize}
    \item Invoca 3 subagentes en paralelo
    \item Genera las 3 partes simultáneamente (2 min)
    \item Ensambla el documento completo
    \item Compila con lualatex
    \item Crea README.md
    \item Guarda en git local y remoto
\end{itemize}

\textbf{Resultado:}\\
✅ GuiaIdentidadesTrigonometricas.tex (código fuente)\\
✅ GuiaIdentidadesTrigonometricas.pdf (32 páginas, 185 KB)\\
✅ README.md (documentación)\\
✅ Commit en git (guardado)\\
✅ Push remoto (respaldo)
\end{tcolorbox}

\newpage

\section{Ventajas del Sistema}

\subsection{Comparación con Método Manual}

\begin{center}
\begin{tabular}{|l|c|c|}
\hline
\textbf{Aspecto} & \textbf{Manual} & \textbf{Sistema v1.0} \\
\hline
Tiempo total & 5--8 horas & 2 minutos \\
\hline
Páginas generadas & Variable & 25--35 \\
\hline
Calidad & Variable & Estandarizada \\
\hline
Formato & Inconsistente & Uniforme \\
\hline
Compilación & Manual & Automática \\
\hline
Git & Manual & Automático \\
\hline
Documentación & Opcional & Incluida \\
\hline
\end{tabular}
\end{center}

\subsection{Beneficios Clave}

\begin{enumerate}
    \item \textbf{Ahorro de Tiempo:} De horas a minutos
    \item \textbf{Consistencia:} Todas las guías con mismo formato profesional
    \item \textbf{Calidad:} Basado en plantilla verificada de 30 páginas
    \item \textbf{Eficiencia:} Sistema multi-agente paralelo
    \item \textbf{Automatización:} Compilación, documentación y git
    \item \textbf{Escalabilidad:} Fácil crear múltiples guías
    \item \textbf{Mantenibilidad:} Control de versiones con git
\end{enumerate}

\newpage

\section{Preguntas Frecuentes}

\subsection{¿Puedo modificar la guía después de generarla?}

Sí, el archivo \texttt{.tex} es completamente editable. Puedes:
\begin{itemize}
    \item Agregar o quitar secciones
    \item Modificar ejemplos
    \item Cambiar gráficas
    \item Ajustar el formato
\end{itemize}

Después de modificar, solo compila de nuevo:
\begin{tcolorbox}[colback=gray!10, colframe=gray!50, boxrule=1pt]
\texttt{lualatex NombreArchivo.tex}\\
\texttt{lualatex NombreArchivo.tex}
\end{tcolorbox}

\subsection{¿Qué hago si necesito más ejemplos?}

Puedes pedirle al asistente:
\begin{tcolorbox}[colback=gray!10, colframe=gray!50, boxrule=1pt]
\texttt{Agrega 3 ejemplos más sobre [tema específico] en la guía}
\end{tcolorbox}

\subsection{¿Puedo usar el sistema para otras asignaturas?}

Este sistema está optimizado para Trigonometría (Grado 10). Para otras asignaturas, consulta el sistema general en:
\begin{tcolorbox}[colback=gray!10, colframe=gray!50, boxrule=1pt]
\texttt{Sistema\_Generador\_Guias\_v3.1/}
\end{tcolorbox}

\subsection{¿Cómo compilo el PDF manualmente?}

Si necesitas compilar manualmente:
\begin{tcolorbox}[colback=gray!10, colframe=gray!50, boxrule=1pt]
\texttt{cd [directorio de la guía]}\\
\texttt{lualatex NombreArchivo.tex}\\
\texttt{lualatex NombreArchivo.tex}  \# Segunda pasada para TOC
\end{tcolorbox}

\subsection{¿Qué motor de LaTeX debo usar?}

\begin{importante}
Debes usar \textbf{LuaLaTeX}.

No uses:
\begin{itemize}
    \item pdflatex (no soporta fontspec)
    \item xelatex (puede tener problemas con pgfplots)
\end{itemize}
\end{importante}

\newpage

\section{Soporte y Contacto}

\subsection{Reportar Problemas}

Si encuentras algún problema con el sistema:

\begin{enumerate}
    \item Verifica que seguiste todos los pasos del manual
    \item Revisa el archivo \texttt{.log} si hay errores de compilación
    \item Contacta al autor con los detalles del error
\end{enumerate}

\subsection{Sugerencias de Mejora}

Si tienes sugerencias para mejorar el sistema:
\begin{itemize}
    \item Crea un issue en el repositorio
    \item Contacta directamente al autor
    \item Documenta tu propuesta de mejora
\end{itemize}

\subsection{Información del Autor}

\begin{tcolorbox}[colback=maincolor!5, colframe=maincolor, boxrule=1pt]
\textbf{Autor:} Prof. Toribio De J Arrieta F\\
\textbf{Institución:} La Pruebita\\
\textbf{Repositorio:} \url{https://github.com/toribio99/LaTeX-Varios}\\
\textbf{Versión del Sistema:} 1.0 - Trigonometría\\
\textbf{Fecha:} Noviembre 2025
\end{tcolorbox}

\newpage

\section{Recursos Adicionales}

\subsection{Documentación del Sistema}

El sistema incluye documentación completa:

\begin{itemize}
    \item \texttt{README.md} - Instrucciones generales
    \item \texttt{PROMPT\_TRIGONOMETRIA\_v1.0.md} - Especificaciones técnicas
    \item \texttt{EJEMPLO\_USO.md} - Ejemplos paso a paso
    \item \texttt{ANALISIS\_MULTI\_AGENTE.md} - Análisis técnico
    \item \texttt{DIAGRAMA\_FLUJO.md} - Diagramas visuales
    \item \texttt{Manual-Usuario/ManualUsuario.pdf} - Este manual
\end{itemize}

\subsection{Archivo de Referencia}

El archivo de referencia verificado está en:
\begin{tcolorbox}[colback=gray!10, colframe=gray!50, boxrule=1pt]
\texttt{Referencia/GuiaFuncionesTrigonometricas.tex}
\end{tcolorbox}

Este archivo de 30 páginas es la plantilla base que el sistema usa.

\subsection{Guías Generadas Previamente}

Puedes consultar las 9 guías ya generadas como ejemplos:
\begin{enumerate}
    \item Funciones: Concepto y Aplicaciones
    \item Propiedades de las Funciones
    \item Funciones de Variable Real
    \item Funciones Exponenciales y Logarítmicas
    \item Ángulos
    \item Triángulos
    \item Funciones Trigonométricas
    \item Funciones Trigonométricas Segunda Parte
    \item Gráficas de Funciones Trigonométricas
\end{enumerate}

\newpage

\section{Conclusión}

El \textbf{Sistema Generador de Guías de Trigonometría v1.0} es una herramienta poderosa que te permite crear guías educativas profesionales en minutos.

\subsection{Resumen de Pasos}

\begin{enumerate}
    \item Inicia el sistema con el comando correcto
    \item Responde las 14 preguntas interactivas
    \item Confirma la información
    \item Espera 2 minutos mientras el sistema genera todo
    \item Obtén tu guía completa lista para usar
\end{enumerate}

\subsection{Ventajas Principales}

\begin{itemize}
    \item ✅ \textbf{60\% más rápido} que sistema tradicional
    \item ✅ \textbf{3 agentes en paralelo} para máxima eficiencia
    \item ✅ \textbf{Calidad garantizada} basada en plantilla verificada
    \item ✅ \textbf{Completamente automático} hasta el guardado en git
    \item ✅ \textbf{Documentación incluida} con README.md
\end{itemize}

\subsection{Próximos Pasos}

Ahora estás listo para:
\begin{enumerate}
    \item Crear tu primera guía usando el sistema
    \item Explorar las opciones de personalización
    \item Generar múltiples guías rápidamente
    \item Contribuir al mejoramiento del sistema
\end{enumerate}

\begin{center}
\rule{10cm}{0.5pt}

\vspace{1cm}

\Large\textbf{¡Listo para crear guías de alta calidad!}

\vspace{0.5cm}

\normalsize
Sistema Generador de Guías - Trigonometría v1.0\\
Noviembre 2025
\end{center}

\end{document}
