% !TEX program = lualatex
\documentclass[12pt,a4paper,twoside]{article}
\usepackage{fontspec}
\usepackage[spanish,es-nodecimaldot]{babel}
\usepackage{amsmath,amssymb}
\usepackage[margin=2.5cm]{geometry}
\usepackage{xcolor}
\usepackage{tikz,pgfplots}
\usetikzlibrary{calc,arrows.meta,babel,trees}
\usepackage{multicol}
\usepackage{enumitem}
\usepackage{array}
\usepackage{booktabs}
\usepackage{graphicx}
\pgfplotsset{compat=1.18}
\definecolor{maincolor}{RGB}{26,35,126}
\definecolor{accentcolor}{RGB}{255,87,34}

% Configuración de títulos y formato
\usepackage{titlesec}
\titleformat{\section}{\Large\bfseries\color{maincolor}}{\thesection}{1em}{}
\titleformat{\subsection}{\large\bfseries\color{accentcolor}}{\thesubsection}{1em}{}

% Configuración de cajas para ejemplos
\usepackage{tcolorbox}
\tcbuselibrary{skins,breakable}

\usepackage{fancyhdr}

\pagestyle{fancy}
\fancyhf{}
\fancyhead[LE]{\small\textcolor{maincolor}{\thepage \quad PROBABILIDAD}}
\fancyhead[RO]{\small\textcolor{maincolor}{PROBABILIDAD \quad \thepage}}
\fancyhead[LO]{\small\textcolor{maincolor}{Grado 10 - Trigonometría}}
\fancyhead[RE]{\small\textcolor{maincolor}{Prof: Toribio De J Arrieta F}}
\fancyfoot[C]{}
\renewcommand{\headrulewidth}{0.5pt}
\renewcommand{\footrulewidth}{0pt}
\setlength{\headheight}{14pt}

\newtcolorbox{ejemplo}[1][]{
  enhanced,
  breakable,
  colback=maincolor!5,
  colframe=maincolor,
  fonttitle=\bfseries,
  title=Ejemplo Resuelto,
  #1
}

\newtcolorbox{ejercicio}[1][]{
  enhanced,
  breakable,
  colback=accentcolor!5,
  colframe=accentcolor,
  fonttitle=\bfseries,
  title=Ejercicio,
  #1
}

\newtcolorbox{solucion}[1][]{
  enhanced,
  breakable,
  colback=green!5,
  colframe=green!60!black,
  fonttitle=\bfseries,
  title=Solución,
  #1
}

\newtcolorbox{nota}[1][]{
  enhanced,
  colback=yellow!10,
  colframe=orange!80!black,
  fonttitle=\bfseries,
  title=Nota Importante,
  #1
}

\newtcolorbox{definicion}[1][]{
  enhanced,
  breakable,
  colback=blue!5,
  colframe=blue!60!black,
  fonttitle=\bfseries,
  title=Definición,
  #1
}

% Título
\title{\textbf{\Huge ESTADÍSTICA Y PROBABILIDAD}\\[0.5cm]
\Large Guía de Trigonometría}
\author{Prof: Toribio De J Arrieta F\\
\textit{La Pruebita}\\
Grado 10}
\date{\today}

\begin{document}

\maketitle

\tableofcontents
\newpage

\section{Introducción}

¡Bienvenidos al fascinante mundo de la probabilidad! ¿Te has preguntado alguna vez por qué los meteorólogos dicen que hay un 70\% de probabilidad de lluvia? ¿O cómo las compañías de seguros calculan el riesgo de un accidente? ¿Por qué algunos números salen más que otros en los juegos de azar? La respuesta a todas estas preguntas está en la teoría de probabilidad.

La probabilidad es como el lenguaje matemático de la incertidumbre. Vivimos en un mundo donde no todo es seguro: no sabemos si mañana lloverá, si nuestro equipo favorito ganará el partido, o qué carta saldrá después en un juego de naipes. Pero aunque no podemos predecir el futuro con certeza, ¡sí podemos medir qué tan probable es que algo ocurra!

\subsection*{¿Dónde aparece la probabilidad en tu vida diaria?}

La probabilidad está en todas partes, aunque no siempre nos demos cuenta:

\begin{itemize}
    \item \textbf{Juegos de azar:} Dados, cartas, lotería, dominó - todos tienen probabilidades calculables
    \item \textbf{Pronósticos del tiempo:} Los meteorólogos usan modelos probabilísticos complejos
    \item \textbf{Medicina:} Los doctores evalúan la probabilidad de que un tratamiento funcione
    \item \textbf{Deportes:} Las estadísticas de jugadores ayudan a predecir resultados
    \item \textbf{Tecnología:} Tu celular usa probabilidad para corregir errores en las señales
    \item \textbf{Redes sociales:} Los algoritmos calculan la probabilidad de que te guste cierto contenido
    \item \textbf{Finanzas:} Los bancos evalúan el riesgo de prestar dinero
    \item \textbf{Control de calidad:} Las fábricas usan probabilidad para detectar productos defectuosos
\end{itemize}

\subsection*{Una historia para comenzar}

Imagina que estás jugando dominó con tus amigos. Te toca sacar una ficha del montón que está boca abajo. ¿Cuál es la probabilidad de que saques un doble? ¿Y de que saques una ficha que tenga al menos un 6? Estas preguntas parecen simples, pero esconden conceptos matemáticos profundos que han fascinado a la humanidad durante siglos.

La teoría de probabilidad nació en el siglo XVII cuando dos matemáticos franceses, Blaise Pascal y Pierre de Fermat, comenzaron a intercambiar cartas sobre problemas de juegos de azar. Un jugador llamado Chevalier de Méré les había planteado preguntas sobre las apuestas en los dados. ¡Imagínate! Los fundamentos matemáticos que hoy usamos para predecir el clima, diseñar medicamentos y hasta explorar el espacio, comenzaron con preguntas sobre juegos de apuestas.

\newpage

\subsection*{¿Por qué es importante aprender probabilidad?}

Aprender probabilidad te da superpoderes mentales:

\begin{enumerate}
    \item \textbf{Tomas mejores decisiones:} Evalúas riesgos y beneficios de forma más inteligente
    \item \textbf{No te dejas engañar:} Reconoces cuando alguien usa estadísticas de forma incorrecta
    \item \textbf{Entiendes mejor el mundo:} Desde la genética hasta la economía, todo tiene probabilidad
    \item \textbf{Desarrollas pensamiento crítico:} Aprendes a distinguir entre coincidencia y causalidad
    \item \textbf{Te preparas para el futuro:} La inteligencia artificial y el big data se basan en probabilidad
\end{enumerate}

\subsection*{¿Qué vamos a aprender en esta guía?}

En esta aventura matemática vamos a explorar:

\begin{itemize}
    \item Los \textbf{experimentos aleatorios} y cómo describirlos matemáticamente
    \item Las \textbf{técnicas de conteo}: el arte de contar sin contar uno por uno
    \item Las \textbf{permutaciones y combinaciones}: formas ordenadas y desordenadas de organizar cosas
    \item El \textbf{cálculo de probabilidades}: la fórmula mágica de Laplace y más allá
    \item La \textbf{probabilidad condicional}: cómo cambian las probabilidades con nueva información
    \item Aplicación especial: \textbf{La combinatoria en el dominó} - ¡vas a ser invencible en el juego!
\end{itemize}

\subsection*{Un consejo antes de empezar}

La probabilidad puede parecer contra-intuitiva al principio. Por ejemplo, en un salón con solo 23 personas, ¡hay más del 50\% de probabilidad de que dos cumplan años el mismo día! Esto suena increíble, pero las matemáticas lo demuestran. Así que mantén la mente abierta, cuestiona tu intuición, y prepárate para sorprenderte.

Recuerda: la probabilidad no te dice qué va a pasar, sino qué tan probable es que pase. Es como tener un mapa del territorio de la incertidumbre. ¡Vamos a explorarlo juntos!

\newpage

\section{Conceptos Fundamentales}

\subsection{Experimentos Aleatorios y Espacio Muestral}

Empecemos con lo básico. Un \textbf{experimento aleatorio} es cualquier proceso que produce un resultado incierto. Es como lanzar una moneda al aire: sabes que caerá cara o cruz, pero no sabes cuál será hasta que caiga.

\begin{definicion}
Un \textbf{experimento aleatorio} es un proceso que cumple tres condiciones:
\begin{enumerate}
    \item Puede repetirse indefinidamente bajo las mismas condiciones
    \item El resultado no se puede predecir con certeza antes de realizarlo
    \item Se conocen todos los posibles resultados
\end{enumerate}
\end{definicion}

\subsubsection{Ejemplos de experimentos aleatorios}

\begin{itemize}
    \item Lanzar un dado
    \item Sacar una carta de una baraja
    \item Elegir una ficha de dominó
    \item Girar una ruleta
    \item El tiempo que tarda en llegar el bus
    \item El número de llamadas que recibe un call center en una hora
\end{itemize}

\subsubsection{El Espacio Muestral}

El \textbf{espacio muestral} es el conjunto de todos los posibles resultados de un experimento aleatorio. Lo denotamos con la letra griega omega mayúscula: $\Omega$.

\begin{center}
\begin{tikzpicture}[scale=1.2]
    % Título
    \node[font=\large\bfseries, maincolor] at (0, 3.5) {Espacios Muestrales Comunes};

    % Moneda
    \node[draw, circle, minimum size=1.5cm, fill=yellow!20] at (-4, 1.5) {Cara};
    \node[draw, circle, minimum size=1.5cm, fill=yellow!20] at (-2, 1.5) {Cruz};
    \node at (-3, 0.3) {Moneda: $\Omega = \{C, X\}$};

    % Dado
    \draw[thick, fill=red!20] (0, 1) rectangle (1, 2);
    \foreach \x/\y in {0.25/1.25, 0.75/1.25, 0.25/1.75, 0.75/1.75} {
        \filldraw (\x, \y) circle (0.05);
    }
    \node at (1, 0.3) {Dado: $\Omega = \{1,2,3,4,5,6\}$};

    % Ruleta
    \draw[thick, fill=green!20] (4, 1.5) circle (0.8);
    \draw[thick] (4, 1.5) -- (4.57, 2.07);
    \draw[thick] (4, 1.5) -- (4.57, 0.93);
    \draw[thick] (4, 1.5) -- (3.43, 0.93);
    \draw[thick] (4, 1.5) -- (3.43, 2.07);
    \node[font=\tiny] at (4.3, 1.8) {1};
    \node[font=\tiny] at (4.3, 1.2) {2};
    \node[font=\tiny] at (3.7, 1.2) {3};
    \node[font=\tiny] at (3.7, 1.8) {4};
    \node at (5, 0.3) {Ruleta: $\Omega = \{1,2,3,4\}$};
\end{tikzpicture}
\end{center}

\begin{nota}
El espacio muestral puede ser:
\begin{itemize}
    \item \textbf{Finito:} Como en el lanzamiento de un dado ($\Omega = \{1,2,3,4,5,6\}$)
    \item \textbf{Infinito numerable:} Como el número de intentos hasta obtener cara
    \item \textbf{Infinito no numerable:} Como el tiempo exacto de espera del bus
\end{itemize}
\end{nota}

\subsection{Eventos y Tipos de Eventos}

Un \textbf{evento} es cualquier subconjunto del espacio muestral. Es como una pregunta que le hacemos al experimento: "¿Salió un número par?" o "¿La carta es roja?"

\begin{definicion}
Un \textbf{evento} es un subconjunto del espacio muestral $\Omega$. Decimos que un evento $A$ ocurre si el resultado del experimento está en $A$.
\end{definicion}

\subsubsection{Tipos de Eventos}

\begin{center}
	\resizebox{\textwidth}{!}{%
		\begin{tabular}{|l|p{7cm}|p{5cm}|}
			\hline
			\textbf{Tipo de Evento} & \textbf{Descripción} & \textbf{Ejemplo (dado)} \\
			\hline\hline
			Evento simple & Contiene un solo resultado & $A = \{3\}$ (sale el 3) \\
			\hline
			Evento compuesto & Contiene varios resultados & $B = \{2,4,6\}$ (sale par) \\
			\hline
			Evento seguro & Siempre ocurre (es $\Omega$) & $\Omega = \{1,2,3,4,5,6\}$ \\
			\hline
			Evento imposible & Nunca ocurre (es $\emptyset$) & $\emptyset$ (sale el 7) \\
			\hline
			Eventos mutuamente excluyentes & No pueden ocurrir al mismo tiempo & $A = \{1,2\}$, $B = \{5,6\}$ \\
			\hline
			Eventos complementarios & Uno es todo lo que no es el otro & $A = \{1,2,3\}$, $A^c = \{4,5,6\}$ \\
			\hline
		\end{tabular}%
	}
\end{center}


\subsubsection{Operaciones con Eventos}

Los eventos se pueden combinar usando operaciones de conjuntos:

\begin{itemize}
    \item \textbf{Unión} $(A \cup B)$: Ocurre $A$ o $B$ (o ambos)
    \item \textbf{Intersección} $(A \cap B)$: Ocurren $A$ y $B$ simultáneamente
    \item \textbf{Complemento} $(A^c)$: No ocurre $A$
    \item \textbf{Diferencia} $(A - B)$: Ocurre $A$ pero no $B$
\end{itemize}

\subsection{Técnicas de Conteo}

Contar es fundamental en probabilidad. Necesitamos saber cuántos resultados favorables hay y cuántos resultados posibles hay en total. Pero contar uno por uno es tedioso y propenso a errores. Por eso usamos técnicas especiales.

\subsubsection{Principio Fundamental del Conteo}

\begin{definicion}[title=Principio Multiplicativo]
Si una tarea se puede realizar en $n_1$ formas diferentes, y después de realizarla, una segunda tarea se puede realizar en $n_2$ formas diferentes, entonces ambas tareas se pueden realizar en $n_1 \times n_2$ formas diferentes.
\end{definicion}

\textbf{Ejemplo:} Si tienes 3 camisas y 4 pantalones, puedes formar $3 \times 4 = 12$ conjuntos diferentes.

\begin{center}
\begin{tikzpicture}[scale=0.9]
    % Título
    \node[font=\large\bfseries, maincolor] at (2, 5) {Principio Multiplicativo};

    % Camisas
    \node[draw, rectangle, fill=blue!20] at (-0.45, 3) {Camisa 1};
    \node[draw, rectangle, fill=blue!20] at (-0.45, 2) {Camisa 2};
    \node[draw, rectangle, fill=blue!20] at (-0.45, 1) {Camisa 3};

    % Pantalones
    \node[draw, rectangle, fill=green!20] at (4.65, 3.5) {Pantalón 1};
    \node[draw, rectangle, fill=green!20] at (4.65, 2.5) {Pantalón 2};
    \node[draw, rectangle, fill=green!20] at (4.65, 1.5) {Pantalón 3};
    \node[draw, rectangle, fill=green!20] at (4.65, 0.5) {Pantalón 4};

    % Flechas
    \foreach \y in {3, 2, 1} {
        \foreach \yp in {3.5, 2.5, 1.5, 0.5} {
            \draw[-{Latex}] (0.5, \y) -- (3.5, \yp);
        }
    }

    % Resultado
    \node[font=\large] at (2, -0.5) {$3 \times 4 = 12$ combinaciones};
\end{tikzpicture}
\end{center}

\subsubsection{Permutaciones}

Las \textbf{permutaciones} son arreglos ordenados. El orden importa: ABC es diferente de BAC.

\begin{definicion}[title=Permutaciones sin repetición]
El número de formas de ordenar $n$ objetos distintos tomados de $r$ en $r$ es:
\[
P(n,r) = \frac{n!}{(n-r)!}
\]
donde $n! = n \times (n-1) \times (n-2) \times \cdots \times 2 \times 1$
\end{definicion}

\textbf{Casos especiales:}
\begin{itemize}
    \item Permutar todos los $n$ objetos: $P(n,n) = n!$
    \item Con repetición: Si hay $n_1$ objetos del tipo 1, $n_2$ del tipo 2, etc.:
    \[
    P_{\text{rep}} = \frac{n!}{n_1! \times n_2! \times \cdots \times n_k!}
    \]
\end{itemize}

\subsubsection{Combinaciones}

Las \textbf{combinaciones} son selecciones donde el orden NO importa. Elegir ABC es lo mismo que elegir BAC.

\begin{definicion}[title=Combinaciones]
El número de formas de elegir $r$ objetos de un conjunto de $n$ objetos (sin importar el orden) es:
\[
C(n,r) = \binom{n}{r} = \frac{n!}{r!(n-r)!}
\]
\end{definicion}

\subsubsection{Tabla Comparativa: Permutaciones vs Combinaciones}

\begin{center}
\renewcommand{\arraystretch}{1.5}
\begin{tabular}{|p{3.5cm}|p{5.5cm}|p{5.5cm}|}
\hline
\textbf{Característica} & \textbf{Permutaciones} & \textbf{Combinaciones} \\
\hline
\hline
¿El orden importa? & Sí & No \\
\hline
Ejemplo & Formar palabras con letras & Elegir un comité \\
\hline
Fórmula sin repetición & $P(n,r) = \frac{n!}{(n-r)!}$ & $C(n,r) = \frac{n!}{r!(n-r)!}$ \\
\hline
Con 3 letras A,B,C tomadas de 2 en 2 & AB, BA, AC, CA, BC, CB (6 formas) & AB, AC, BC (3 formas) \\
\hline
Pregunta tipo & ¿De cuántas formas puedo ordenar...? & ¿De cuántas formas puedo elegir...? \\
\hline
\end{tabular}
\end{center}

\subsection{Probabilidad Clásica (Regla de Laplace)}

Llegamos al corazón del asunto: calcular probabilidades. La forma más básica es la regla de Laplace.

\begin{definicion}[title=Regla de Laplace]
Si todos los resultados de un experimento son igualmente probables, la probabilidad de un evento $A$ es:
\[
P(A) = \frac{\text{Número de casos favorables}}{\text{Número de casos posibles}} = \frac{|A|}{|\Omega|}
\]
\end{definicion}

\textbf{Propiedades básicas de la probabilidad:}
\begin{enumerate}
    \item $0 \leq P(A) \leq 1$ para cualquier evento $A$
    \item $P(\Omega) = 1$ (el evento seguro tiene probabilidad 1)
    \item $P(\emptyset) = 0$ (el evento imposible tiene probabilidad 0)
    \item $P(A^c) = 1 - P(A)$ (regla del complemento)
    \item Si $A$ y $B$ son mutuamente excluyentes: $P(A \cup B) = P(A) + P(B)$
\end{enumerate}

\subsection{Probabilidad Conjunta, Marginal y Condicional}

Ahora vamos a ver qué pasa cuando tenemos eventos relacionados. Esta es la parte más interesante y útil de la probabilidad.

\subsubsection{Probabilidad Conjunta}

La \textbf{probabilidad conjunta} es la probabilidad de que ocurran dos eventos al mismo tiempo.

\[
P(A \cap B) = P(\text{A y B ocurren})
\]

\subsubsection{Probabilidad Marginal}

La \textbf{probabilidad marginal} es simplemente la probabilidad de un evento sin considerar otros eventos. Es la probabilidad "normal" que hemos estado usando.

\subsubsection{Probabilidad Condicional}

La \textbf{probabilidad condicional} es la probabilidad de que ocurra un evento dado que ya ocurrió otro.

\begin{definicion}[title=Probabilidad Condicional]
La probabilidad de $A$ dado que ocurrió $B$ (con $P(B) > 0$) es:
\[
P(A|B) = \frac{P(A \cap B)}{P(B)}
\]
\end{definicion}

Esta fórmula dice: "De todos los casos donde ocurre $B$, ¿en qué fracción también ocurre $A$?"

\textbf{Regla del producto:}
De la definición anterior, podemos despejar:
\[
P(A \cap B) = P(A|B) \times P(B) = P(B|A) \times P(A)
\]

\subsection{Diagrama de Árbol para Probabilidad}

Los diagramas de árbol son una herramienta visual poderosa para calcular probabilidades, especialmente cuando hay eventos secuenciales.

\begin{definicion}
\textbf{Problema:} En una urna hay \textbf{3 bolas rojas} y \textbf{2 bolas azules} (total: 5 bolas). Se extraen 2 bolas \emph{sin reemplazo} (es decir, no se devuelve la primera bola antes de sacar la segunda). ¿Cuáles son las probabilidades de cada resultado posible?
\end{definicion}

\begin{center}
\begin{tikzpicture}[
    level 1/.style={sibling distance=6cm, level distance=2cm},
    level 2/.style={sibling distance=3cm, level distance=2cm},
    edge from parent/.style={draw, -{Latex}},
    every node/.style={font=\footnotesize}
]
    % Título
    \node[font=\large\bfseries, maincolor] at (0, 1) {Diagrama de Árbol: Sacar 2 bolas sin reemplazo};
    \node[font=\small, maincolor] at (0, 0.4) {Urna: 3 rojas + 2 azules = 5 bolas};

    % Nodo raíz
    \node[circle, draw, fill=gray!20] {Inicio}
        child {
            node[circle, draw, fill=red!20] {Roja}
            edge from parent node[left] {$\frac{3}{5}$}
            child {
                node[circle, draw, fill=red!20] {Roja}
                edge from parent node[above left=5mm] {$\frac{2}{4}$}
            }
            child {
                node[circle, draw, fill=blue!20] {Azul}
                edge from parent node[right] {$\frac{2}{4}$}
            }
        }
        child {
            node[circle, draw, fill=blue!20] {Azul}
            edge from parent node[right] {$\frac{2}{5}$}
            child {
                node[circle, draw, fill=red!20] {Roja}
                edge from parent node[left] {$\frac{3}{4}$}
            }
            child {
                node[circle, draw, fill=blue!20] {Azul}
                edge from parent node[above right=4mm] {$\frac{1}{4}$}
            }
        };

    % Probabilidades finales
    \node at (-4.5, -4.5) {$P(RR) = \frac{3}{5} \times \frac{2}{4} = \frac{6}{20}$};
    \node at (-0.5, -4.5) {$P(RA) = \frac{3}{5} \times \frac{2}{4} = \frac{6}{20}$};
    \node at (3.4, -4.5) {$P(AR) = \frac{2}{5} \times \frac{3}{4} = \frac{6}{20}$};
    \node at (7, -4.5) {$P(AA) = \frac{2}{5} \times \frac{1}{4} = \frac{2}{20}$};
\end{tikzpicture}
\end{center}

\textbf{Reglas para construir un diagrama de árbol:}
\begin{enumerate}
    \item Cada nivel representa una etapa del experimento
    \item Las ramas muestran todos los posibles resultados en cada etapa
    \item En cada rama se escribe la probabilidad condicional correspondiente
    \item La probabilidad de un camino completo es el producto de las probabilidades de sus ramas
    \item La suma de todas las probabilidades finales debe ser 1
\end{enumerate}

\subsection{Aplicación: Combinatoria en el Juego de Dominó}

El dominó es un excelente ejemplo para aplicar todo lo que hemos aprendido. Un juego estándar de dominó tiene 28 fichas, desde el doble blanco (0-0) hasta el doble seis (6-6).

\subsubsection{Estructura del Dominó}

Las fichas de dominó se pueden organizar así:

\begin{center}
\begin{tabular}{|c|c|l|}
\hline
\textbf{Tipo} & \textbf{Cantidad} & \textbf{Fichas} \\
\hline
\hline
Dobles & 7 & 0-0, 1-1, 2-2, 3-3, 4-4, 5-5, 6-6 \\
\hline
Con 0 (sin doble) & 6 & 0-1, 0-2, 0-3, 0-4, 0-5, 0-6 \\
\hline
Con 1 (sin doble) & 5 & 1-2, 1-3, 1-4, 1-5, 1-6 \\
\hline
Con 2 (sin doble) & 4 & 2-3, 2-4, 2-5, 2-6 \\
\hline
Con 3 (sin doble) & 3 & 3-4, 3-5, 3-6 \\
\hline
Con 4 (sin doble) & 2 & 4-5, 4-6 \\
\hline
Con 5 (sin doble) & 1 & 5-6 \\
\hline
\textbf{Total} & \textbf{28} & \\
\hline
\end{tabular}
\end{center}

\subsubsection{¿Por qué hay 28 fichas?}

Podemos calcularlo usando combinaciones. Tenemos 7 números (0 al 6), y queremos formar parejas permitiendo repetición:

\begin{itemize}
    \item Parejas sin repetir: $C(7,2) = \frac{7!}{2!(7-2)!} = \frac{7 \times 6}{2} = 21$
    \item Dobles (parejas con repetición): 7
    \item Total: $21 + 7 = 28$
\end{itemize}

O también: $C(7+1, 2) = C(8,2) = 28$ (combinaciones con repetición)

\subsubsection{Probabilidades en el Dominó}

Veamos algunas probabilidades interesantes:

\begin{enumerate}
    \item \textbf{Probabilidad de sacar un doble:}
    \[
    P(\text{doble}) = \frac{7}{28} = \frac{1}{4} = 0.25
    \]

    \item \textbf{Probabilidad de sacar una ficha con al menos un 6:}

    Fichas con 6: 6-6, 0-6, 1-6, 2-6, 3-6, 4-6, 5-6 (7 fichas)
    \[
    P(\text{al menos un 6}) = \frac{7}{28} = \frac{1}{4} = 0.25
    \]

    \item \textbf{Probabilidad de sacar una ficha cuya suma sea 6:}

    Fichas que suman 6: 0-6, 1-5, 2-4, 3-3 (4 fichas)
    \[
    P(\text{suma = 6}) = \frac{4}{28} = \frac{1}{7} \approx 0.143
    \]
\end{enumerate}

\subsubsection{Estrategias Probabilísticas en el Dominó}

Conocer las probabilidades te puede ayudar a jugar mejor:

\begin{itemize}
    \item \textbf{Contar fichas:} Si sabes qué fichas se han jugado, puedes calcular la probabilidad de que tu oponente tenga cierta ficha
    \item \textbf{Bloquear números:} Los números que aparecen en más fichas (como el 6) son más difíciles de bloquear
    \item \textbf{Guardar dobles:} Los dobles son más difíciles de colocar, así que úsalos estratégicamente
\end{itemize}

\subsection{Fórmulas y Conceptos Clave - Resumen}

\begin{tcolorbox}[enhanced,colback=maincolor!10,colframe=maincolor,title=Tabla de Fórmulas Esenciales]
\renewcommand{\arraystretch}{1.8}
\begin{tabular}{|l|l|}
\hline
\textbf{Concepto} & \textbf{Fórmula} \\
\hline
\hline
Regla de Laplace & $P(A) = \frac{|A|}{|\Omega|}$ \\
\hline
Probabilidad del complemento & $P(A^c) = 1 - P(A)$ \\
\hline
Unión (mutuamente excluyentes) & $P(A \cup B) = P(A) + P(B)$ \\
\hline
Unión (general) & $P(A \cup B) = P(A) + P(B) - P(A \cap B)$ \\
\hline
Probabilidad condicional & $P(A|B) = \frac{P(A \cap B)}{P(B)}$ \\
\hline
Regla del producto & $P(A \cap B) = P(A|B) \times P(B)$ \\
\hline
Permutaciones & $P(n,r) = \frac{n!}{(n-r)!}$ \\
\hline
Combinaciones & $C(n,r) = \binom{n}{r} = \frac{n!}{r!(n-r)!}$ \\
\hline
Principio multiplicativo & $n_1 \times n_2 \times \cdots \times n_k$ \\
\hline
\end{tabular}
\end{tcolorbox}

\newpage

\section{Conclusión}

¡Felicitaciones! Has dado tus primeros pasos en el fascinante mundo de la probabilidad. Ya no eres la misma persona que comenzó a leer esta guía. Ahora tienes herramientas matemáticas poderosas para entender y medir la incertidumbre.

\subsection*{Lo que has aprendido}

Has desarrollado una comprensión sólida de:

\begin{itemize}
    \item Los \textbf{experimentos aleatorios} y cómo modelarlos matemáticamente con espacios muestrales
    \item Los diferentes \textbf{tipos de eventos} y cómo operar con ellos
    \item Las \textbf{técnicas de conteo} que te permiten calcular posibilidades sin enumerar todo
    \item La diferencia crucial entre \textbf{permutaciones y combinaciones}
    \item La \textbf{regla de Laplace} para calcular probabilidades básicas
    \item Las \textbf{probabilidades condicionales} que modelan cómo la información cambia las probabilidades
    \item El uso de \textbf{diagramas de árbol} para visualizar problemas complejos
    \item Aplicaciones prácticas, como el análisis probabilístico del \textbf{dominó}
\end{itemize}

\subsection*{Tabla de Conceptos Clave para Recordar}

\begin{center}
\renewcommand{\arraystretch}{1.5}
\begin{tabular}{|p{4cm}|p{10cm}|}
\hline
\textbf{Concepto} & \textbf{Punto Clave para Recordar} \\
\hline
\hline
Experimento aleatorio & No puedes predecir el resultado, pero conoces todas las posibilidades \\
\hline
Espacio muestral ($\Omega$) & Es el conjunto de TODOS los posibles resultados \\
\hline
Evento & Es un subconjunto del espacio muestral (una pregunta sobre el resultado) \\
\hline
Principio multiplicativo & Si hay $m$ formas de hacer algo Y $n$ formas de hacer otra cosa, hay $m \times n$ formas de hacer ambas \\
\hline
Permutaciones & Úsalas cuando el ORDEN IMPORTA (formar palabras, asignar puestos) \\
\hline
Combinaciones & Úsalas cuando el orden NO importa (formar equipos, elegir elementos) \\
\hline
Probabilidad & Siempre está entre 0 y 1. 0 = imposible, 1 = seguro \\
\hline
Probabilidad condicional & La probabilidad cambia cuando tienes información adicional \\
\hline
Diagrama de árbol & Multiplica a lo largo de las ramas, suma los caminos diferentes \\
\hline
\end{tabular}
\end{center}

\newpage

\subsection*{Consejos para Resolver Problemas de Probabilidad}

\begin{enumerate}
    \item \textbf{Lee con cuidado:} Identifica qué te piden exactamente

    \item \textbf{Define el experimento:} ¿Cuál es el proceso aleatorio?

    \item \textbf{Identifica el espacio muestral:} ¿Cuáles son todos los posibles resultados?

    \item \textbf{Define los eventos:} ¿Qué subconjuntos necesitas considerar?

    \item \textbf{Decide la técnica de conteo:}
    \begin{itemize}
        \item ¿El orden importa? → Permutaciones
        \item ¿El orden no importa? → Combinaciones
        \item ¿Hay etapas sucesivas? → Principio multiplicativo
    \end{itemize}

    \item \textbf{Aplica la fórmula apropiada:} Laplace, condicional, etc.

    \item \textbf{Verifica tu respuesta:} ¿Está entre 0 y 1? ¿Tiene sentido?
\end{enumerate}

\subsection*{Errores Comunes a Evitar}

\begin{nota}[title=Cuidado con estos errores típicos]
\begin{itemize}
    \item Confundir permutaciones con combinaciones
    \item Olvidar que las probabilidades deben sumar 1
    \item No considerar si hay reemplazo o no
    \item Contar el mismo resultado varias veces
    \item Confundir $P(A|B)$ con $P(B|A)$ - ¡no son lo mismo!
    \item Asumir independencia cuando no la hay
\end{itemize}
\end{nota}

\subsection*{Aplicaciones en la Vida Real}

La probabilidad que has aprendido aparece en:

\begin{itemize}
    \item \textbf{Medicina:} Calcular la efectividad de tratamientos y la probabilidad de efectos secundarios
    \item \textbf{Seguros:} Determinar primas basadas en riesgos calculados
    \item \textbf{Tecnología:} Algoritmos de recomendación, corrección de errores en transmisiones
    \item \textbf{Deportes:} Estrategias basadas en estadísticas de jugadores
    \item \textbf{Finanzas:} Evaluación de riesgos en inversiones
    \item \textbf{Videojuegos:} Sistemas de loot, matchmaking, balanceo de personajes
    \item \textbf{Redes sociales:} Predicción de qué contenido te gustará
\end{itemize}

\subsection*{Próximos Pasos}

Si quieres profundizar en probabilidad, los siguientes temas serían:

\begin{enumerate}
    \item \textbf{Distribuciones de probabilidad:} Binomial, normal, Poisson
    \item \textbf{Variables aleatorias:} Discretas y continuas
    \item \textbf{Esperanza matemática y varianza}
    \item \textbf{Teorema de Bayes:} La joya de la probabilidad condicional
    \item \textbf{Cadenas de Markov:} Procesos que dependen del estado anterior
    \item \textbf{Simulación Monte Carlo:} Usar computadoras para estimar probabilidades
\end{enumerate}

\subsection*{Reflexión Final}

La probabilidad es más que matemáticas; es una forma de pensar sobre el mundo. Te enseña que:

\begin{itemize}
    \item La incertidumbre no significa ignorancia total - podemos medirla y trabajar con ella
    \item Las coincidencias sorprendentes a menudo no son tan improbables como parecen
    \item La intuición puede engañarnos - los cálculos nos dan la respuesta correcta
    \item Con suficiente información, podemos tomar mejores decisiones
\end{itemize}

Como dijo el matemático Pierre-Simon Laplace: "La probabilidad es sentido común reducido a cálculo". Ahora tienes las herramientas para convertir tu sentido común en matemáticas precisas.

\vspace{1cm}

\begin{center}
\textit{``El azar favorece a la mente preparada.''} \\
--- Louis Pasteur
\end{center}

\vspace{1cm}

\begin{center}
\Large
\textbf{¡Sigue explorando, sigue aprendiendo, y que la probabilidad esté siempre a tu favor!}
\end{center}

\newpage

% INSERTAR EJEMPLOS RESUELTOS (PARTE 2)
\section{Ejemplos Resueltos}

Ahora vamos a poner en práctica todos los conceptos de probabilidad que hemos aprendido. Cada ejemplo está completamente desarrollado paso a paso para que entiendas el proceso completo.

\begin{ejemplo}{Técnicas de conteo - Principio fundamental del conteo}
María está preparándose para salir y debe elegir su atuendo. En su armario tiene:
\begin{itemize}
    \item 4 blusas (roja, azul, blanca, negra)
    \item 3 pantalones (jeans, negro, beige)
    \item 2 pares de zapatos (deportivos, formales)
\end{itemize}
¿De cuántas maneras diferentes puede María combinar su ropa?

\vspace{0.3cm}
\textbf{Solución:}

\textbf{Paso 1:} Identificar qué tipo de problema es.
Este es un problema de conteo donde debemos aplicar el principio fundamental del conteo (también conocido como principio multiplicativo).

\textbf{Paso 2:} Entender el principio fundamental.
Si una tarea se puede realizar de $n_1$ formas, y para cada una de estas, una segunda tarea se puede realizar de $n_2$ formas, entonces ambas tareas se pueden realizar de $n_1 \times n_2$ formas.

\textbf{Paso 3:} Identificar las decisiones independientes.
María debe tomar 3 decisiones independientes:
\begin{itemize}
    \item Decisión 1: Elegir una blusa (4 opciones)
    \item Decisión 2: Elegir un pantalón (3 opciones)
    \item Decisión 3: Elegir zapatos (2 opciones)
\end{itemize}

\textbf{Paso 4:} Visualizar con un diagrama de árbol parcial.

\begin{center}
\begin{tikzpicture}[scale=0.9,
    level 1/.style={sibling distance=3.5cm},
    level 2/.style={sibling distance=1.5cm},
    level 3/.style={sibling distance=0.8cm},
    every node/.style={circle, draw=maincolor, fill=maincolor!10, minimum size=8mm},
    edge from parent/.style={draw=maincolor!70, -{Latex}, thick}]

\node[rectangle, minimum width=1.5cm] {Inicio}
    child {node[fill=red!20] {Roja}
        child {node[fill=blue!20, scale=0.7] {Jeans}
            child {node[fill=gray!20, scale=0.6] {Dep}}
            child {node[fill=gray!20, scale=0.6] {For}}
        }
        child {node[fill=blue!20, scale=0.7] {Negro}
            child {node[fill=gray!20, scale=0.6] {Dep}}
            child {node[fill=gray!20, scale=0.6] {For}}
        }
        child {node[fill=blue!20, scale=0.7] {Beige}
            child {node[fill=gray!20, scale=0.6] {Dep}}
            child {node[fill=gray!20, scale=0.6] {For}}
        }
    }
    child {node[fill=blue!30] {Azul}
        child[dashed] {node[fill=blue!20, scale=0.7] {...}}
    }
    child {node[fill=white] {Blanca}
        child[dashed] {node[fill=blue!20, scale=0.7] {...}}
    }
    child {node[fill=black!30] {Negra}
        child[dashed] {node[fill=blue!20, scale=0.7] {...}}
    };

\node[text width=4cm, draw=none, fill=none, right] at (5,-2) {\small\textit{El árbol completo tendría 24 ramas finales}};
\end{tikzpicture}
\end{center}

\textbf{Paso 5:} Aplicar el principio multiplicativo.
\[
\text{Total de combinaciones} = 4 \times 3 \times 2 = 24
\]

\textbf{Paso 6:} Verificación alternativa - Enumerar algunas combinaciones.
\begin{align*}
&\text{(Roja, Jeans, Deportivos), (Roja, Jeans, Formales),} \\
&\text{(Roja, Negro, Deportivos), ..., (Negra, Beige, Formales)}
\end{align*}

\textbf{Paso 7:} Interpretación del resultado.
María tiene 24 maneras diferentes de combinar su ropa. Esto significa que podría usar un atuendo diferente durante 24 días sin repetir ninguna combinación.

\textbf{Paso 8:} Generalización del principio.
Si María comprara una blusa más, tendría: $5 \times 3 \times 2 = 30$ combinaciones.
Si comprara un pantalón más, tendría: $4 \times 4 \times 2 = 32$ combinaciones.

\textbf{Respuesta final:} $\boxed{\text{María puede combinar su ropa de 24 maneras diferentes}}$
\end{ejemplo}

\begin{ejemplo}{Permutaciones - Ordenamiento de objetos}
En una carrera escolar participan 8 estudiantes. ¿De cuántas maneras diferentes pueden quedar los primeros 3 lugares (oro, plata, bronce)?

\vspace{0.3cm}
\textbf{Solución:}

\textbf{Paso 1:} Identificar el tipo de problema.
Este es un problema de permutaciones porque:
\begin{itemize}
    \item El orden importa (oro $\neq$ plata $\neq$ bronce)
    \item No hay repetición (un estudiante no puede ocupar dos lugares)
\end{itemize}

\textbf{Paso 2:} Analizar las decisiones.
\begin{itemize}
    \item Para el oro: 8 opciones (cualquier estudiante)
    \item Para la plata: 7 opciones (todos menos el que ganó oro)
    \item Para el bronce: 6 opciones (todos menos los dos anteriores)
\end{itemize}

\textbf{Paso 3:} Calcular usando el principio multiplicativo.
\[
\text{Maneras} = 8 \times 7 \times 6 = 336
\]

\textbf{Paso 4:} Verificar usando la fórmula de permutaciones.
La fórmula para permutaciones de $n$ elementos tomados de $r$ en $r$ es:
\[
P(n,r) = \frac{n!}{(n-r)!}
\]

En nuestro caso: $n = 8$ estudiantes, $r = 3$ lugares.
\[
P(8,3) = \frac{8!}{(8-3)!} = \frac{8!}{5!} = \frac{8 \times 7 \times 6 \times 5!}{5!} = 8 \times 7 \times 6 = 336
\]

\textbf{Paso 5:} Interpretación práctica.
Si llamamos a los estudiantes A, B, C, D, E, F, G, H, algunas posibles configuraciones son:
\begin{itemize}
    \item Oro: A, Plata: B, Bronce: C
    \item Oro: A, Plata: C, Bronce: B (diferente a la anterior)
    \item Oro: B, Plata: A, Bronce: C (también diferente)
\end{itemize}

\textbf{Paso 6:} Comparación con combinaciones.
Si solo nos interesara quiénes son los 3 primeros sin importar el orden, sería:
\[
C(8,3) = \frac{8!}{3!(8-3)!} = \frac{8!}{3! \cdot 5!} = \frac{336}{6} = 56
\]

\textbf{Verificación:} Las 336 permutaciones = 56 grupos × 6 ordenamientos por grupo.

\textbf{Respuesta final:} $\boxed{\text{Los primeros 3 lugares pueden quedar de 336 maneras diferentes}}$
\end{ejemplo}

\begin{ejemplo}{Combinaciones - Selección de comité}
De un grupo de 12 personas (7 mujeres y 5 hombres), se debe formar un comité de 4 personas.
a) ¿De cuántas maneras se puede formar el comité?
b) ¿De cuántas maneras si debe haber exactamente 2 mujeres y 2 hombres?

\vspace{0.3cm}
\textbf{Solución:}

\textbf{Parte a) Comité sin restricciones}

\textbf{Paso 1:} Identificar que es un problema de combinaciones.
El orden no importa: el comité \{Ana, Luis, María, Pedro\} es el mismo que \{Pedro, Ana, María, Luis\}.

\textbf{Paso 2:} Aplicar la fórmula de combinaciones.
\[
C(n,r) = \binom{n}{r} = \frac{n!}{r!(n-r)!}
\]

Donde $n = 12$ personas totales, $r = 4$ personas a elegir.

\textbf{Paso 3:} Calcular.
\[
C(12,4) = \frac{12!}{4!(12-4)!} = \frac{12!}{4! \cdot 8!}
\]

\textbf{Paso 4:} Simplificar.
\[
C(12,4) = \frac{12 \times 11 \times 10 \times 9 \times 8!}{4! \cdot 8!} = \frac{12 \times 11 \times 10 \times 9}{4 \times 3 \times 2 \times 1}
\]

\textbf{Paso 5:} Calcular el resultado.
\[
C(12,4) = \frac{11880}{24} = 495
\]

\textbf{Parte b) Comité con 2 mujeres y 2 hombres}

\textbf{Paso 1:} Descomponer el problema.
Debemos elegir:
\begin{itemize}
    \item 2 mujeres de 7 disponibles
    \item 2 hombres de 5 disponibles
\end{itemize}

\textbf{Paso 2:} Calcular las combinaciones por separado.
\[
C(7,2) = \frac{7!}{2! \cdot 5!} = \frac{7 \times 6}{2 \times 1} = 21
\]
\[
C(5,2) = \frac{5!}{2! \cdot 3!} = \frac{5 \times 4}{2 \times 1} = 10
\]

\textbf{Paso 3:} Aplicar el principio multiplicativo.
\[
\text{Total} = C(7,2) \times C(5,2) = 21 \times 10 = 210
\]

\textbf{Paso 4:} Comparación con permutaciones.
Si el orden importara (presidente, vicepresidente, secretario, tesorero):
\[
P(12,4) = \frac{12!}{8!} = 12 \times 11 \times 10 \times 9 = 11,880
\]
Que es exactamente $495 \times 24 = 11,880$ (24 formas de ordenar 4 personas).

\textbf{Respuestas finales:}
\[
\boxed{
\begin{aligned}
\text{a) } &\text{495 maneras de formar el comité} \\
\text{b) } &\text{210 maneras con 2 mujeres y 2 hombres}
\end{aligned}
}
\]
\end{ejemplo}

\begin{ejemplo}{Probabilidad clásica - Regla de Laplace con dados}
Se lanzan dos dados justos. Calcula:
a) La probabilidad de obtener suma 7
b) La probabilidad de obtener al menos un 6
c) La probabilidad de obtener suma par

\vspace{0.3cm}
\textbf{Solución:}

\textbf{Paso 1:} Determinar el espacio muestral.
Al lanzar dos dados, cada uno puede mostrar 1, 2, 3, 4, 5 o 6.
Total de resultados posibles = $6 \times 6 = 36$

\textbf{Paso 2:} Visualizar el espacio muestral.

\begin{center}
\begin{tikzpicture}[scale=0.85]
    % Cuadrícula
    \draw[gray!30] (0,0) grid (6,6);

    % Ejes
    \draw[-{Latex}, thick] (-0.5,0) -- (6.5,0) node[right] {Dado 1};
    \draw[-{Latex}, thick] (0,-0.5) -- (0,6.5) node[above] {Dado 2};

    % Etiquetas
    \foreach \x in {1,...,6}
        \node at (\x-0.5,-0.3) {\x};
    \foreach \y in {1,...,6}
        \node at (-0.3,\y-0.5) {\y};

    % Puntos que suman 7 (diagonal)
    \foreach \x/\y in {1/6,2/5,3/4,4/3,5/2,6/1}
        \filldraw[red] (\x-0.5,\y-0.5) circle (0.15);

    % Leyenda
    \node[red] at (3,-1.5) {Puntos rojos: suma = 7};
\end{tikzpicture}
\end{center}

\textbf{Parte a) Probabilidad de suma 7}

\textbf{Paso 3:} Identificar casos favorables.
Los pares que suman 7 son: (1,6), (2,5), (3,4), (4,3), (5,2), (6,1)
Casos favorables = 6

\textbf{Paso 4:} Aplicar la regla de Laplace.
\[
P(\text{suma} = 7) = \frac{\text{casos favorables}}{\text{casos totales}} = \frac{6}{36} = \frac{1}{6}
\]

\textbf{Parte b) Probabilidad de al menos un 6}

\textbf{Paso 5:} Usar el complemento (más fácil).
Es más fácil calcular P(ningún 6) y restar de 1.

Casos sin ningún 6: cada dado muestra 1, 2, 3, 4 o 5.
Total sin ningún 6 = $5 \times 5 = 25$

\[
P(\text{ningún 6}) = \frac{25}{36}
\]
\[
P(\text{al menos un 6}) = 1 - P(\text{ningún 6}) = 1 - \frac{25}{36} = \frac{11}{36}
\]

\textbf{Parte c) Probabilidad de suma par}

\textbf{Paso 6:} Analizar cuándo la suma es par.
La suma es par cuando:
\begin{itemize}
    \item Ambos dados muestran par (par + par = par)
    \item Ambos dados muestran impar (impar + impar = par)
\end{itemize}

Dados pares: 2, 4, 6 (3 opciones)
Dados impares: 1, 3, 5 (3 opciones)

Casos favorables = $(3 \times 3) + (3 \times 3) = 9 + 9 = 18$

\[
P(\text{suma par}) = \frac{18}{36} = \frac{1}{2}
\]

\textbf{Paso 7:} Crear gráfica de probabilidades de todas las sumas.

\begin{center}
\begin{tikzpicture}[scale=0.9]
    \begin{axis}[
        ybar,
        bar width=0.5cm,
        xlabel={Suma de los dados},
        ylabel={Probabilidad},
        ymin=0, ymax=0.2,
        xtick={2,3,4,5,6,7,8,9,10,11,12},
        ytick={0, 0.05, 0.1, 0.15, 0.2},
        yticklabel={\pgfmathprintnumber{\tick}},
        grid=major,
        grid style={dashed, gray!30},
        axis lines=left,
        width=0.95\textwidth,
        height=6cm,
        nodes near coords,
        nodes near coords align={vertical},
        every node near coord/.append style={font=\scriptsize}
    ]
    \addplot[fill=maincolor!70] coordinates {
        (2, 0.0278) (3, 0.0556) (4, 0.0833) (5, 0.1111)
        (6, 0.1389) (7, 0.1667) (8, 0.1389) (9, 0.1111)
        (10, 0.0833) (11, 0.0556) (12, 0.0278)
    };
    \end{axis}
\end{tikzpicture}
\end{center}

\textbf{Respuestas finales:}
\[
\boxed{
\begin{aligned}
\text{a) } P(\text{suma} = 7) &= \frac{1}{6} \approx 0.167 \\
\text{b) } P(\text{al menos un 6}) &= \frac{11}{36} \approx 0.306 \\
\text{c) } P(\text{suma par}) &= \frac{1}{2} = 0.5
\end{aligned}
}
\]
\end{ejemplo}

\begin{ejemplo}{Probabilidad condicional - Problema de urnas}
Una urna contiene 5 bolas rojas y 3 bolas azules. Se extraen dos bolas sin reemplazo.
a) ¿Cuál es la probabilidad de que ambas sean rojas?
b) Si la primera bola extraída fue roja, ¿cuál es la probabilidad de que la segunda también sea roja?
c) Verifica que $P(A \cap B) = P(A) \cdot P(B|A)$

\vspace{0.3cm}
\textbf{Solución:}

\textbf{Paso 1:} Definir los eventos.
\begin{itemize}
    \item $A$: La primera bola es roja
    \item $B$: La segunda bola es roja
    \item $A \cap B$: Ambas bolas son rojas
\end{itemize}

\textbf{Paso 2:} Construir el diagrama de árbol completo.

\begin{center}
\begin{tikzpicture}[scale=0.95,
    grow=right,
    level 1/.style={sibling distance=4cm, level distance=4cm},
    level 2/.style={sibling distance=2cm, level distance=4cm},
    every node/.style={circle, draw, minimum size=8mm},
    edge from parent/.style={draw, -{Latex}, thick}]

\node[rectangle, fill=gray!20] {Inicio}
    child {node[fill=red!30] {R}
        child {node[fill=red!30] {R}
            edge from parent node[above, draw=none] {$\frac{4}{7}$}
        }
        child {node[fill=blue!30] {A}
            edge from parent node[below, draw=none] {$\frac{3}{7}$}
        }
        edge from parent node[above, draw=none] {$\frac{5}{8}$}
    }
    child {node[fill=blue!30] {A}
        child {node[fill=red!30] {R}
            edge from parent node[above, draw=none] {$\frac{5}{7}$}
        }
        child {node[fill=blue!30] {A}
            edge from parent node[below, draw=none] {$\frac{2}{7}$}
        }
        edge from parent node[below, draw=none] {$\frac{3}{8}$}
    };

% Etiquetas de resultados
\node[draw=none] at (9.5, 2) {$P(RR) = \frac{5}{8} \times \frac{4}{7} = \frac{20}{56} = \frac{5}{14}$};
\node[draw=none] at (9.5, 0.7) {$P(RA) = \frac{5}{8} \times \frac{3}{7} = \frac{15}{56}$};
\node[draw=none] at (9.5, -0.7) {$P(AR) = \frac{3}{8} \times \frac{5}{7} = \frac{15}{56}$};
\node[draw=none] at (9.5, -2) {$P(AA) = \frac{3}{8} \times \frac{2}{7} = \frac{6}{56} = \frac{3}{28}$};
\end{tikzpicture}
\end{center}

\textbf{Parte a) Probabilidad de que ambas sean rojas}

\textbf{Paso 3:} Calcular $P(A \cap B)$ usando el diagrama.
Del diagrama de árbol:
\[
P(\text{ambas rojas}) = P(A \cap B) = \frac{5}{8} \times \frac{4}{7} = \frac{20}{56} = \frac{5}{14}
\]

\textbf{Parte b) Probabilidad condicional}

\textbf{Paso 4:} Aplicar la definición de probabilidad condicional.
\[
P(B|A) = P(\text{segunda roja | primera roja})
\]

Si la primera fue roja, quedan:
\begin{itemize}
    \item 4 bolas rojas
    \item 3 bolas azules
    \item Total: 7 bolas
\end{itemize}

Por lo tanto:
\[
P(B|A) = \frac{4}{7}
\]

\textbf{Parte c) Verificación de la fórmula}

\textbf{Paso 5:} Verificar que $P(A \cap B) = P(A) \cdot P(B|A)$

Ya calculamos:
\begin{itemize}
    \item $P(A) = \frac{5}{8}$ (probabilidad de que la primera sea roja)
    \item $P(B|A) = \frac{4}{7}$ (probabilidad de que la segunda sea roja dado que la primera fue roja)
    \item $P(A \cap B) = \frac{5}{14}$ (probabilidad de que ambas sean rojas)
\end{itemize}

Verificación:
\[
P(A) \cdot P(B|A) = \frac{5}{8} \times \frac{4}{7} = \frac{20}{56} = \frac{5}{14} = P(A \cap B) \quad \checkmark
\]

\textbf{Paso 6:} Calcular también la probabilidad marginal de B.
\[
P(B) = P(B|A) \cdot P(A) + P(B|A^c) \cdot P(A^c)
\]
\[
P(B) = \frac{4}{7} \cdot \frac{5}{8} + \frac{5}{7} \cdot \frac{3}{8} = \frac{20}{56} + \frac{15}{56} = \frac{35}{56} = \frac{5}{8}
\]

Interesante: $P(B) = P(A) = \frac{5}{8}$ (simétrico porque hay 5 rojas de 8 totales).

\textbf{Paso 7:} Tabla de probabilidad conjunta.

\begin{center}
\begin{tabular}{|c|c|c|c|}
\hline
& \textbf{Segunda R} & \textbf{Segunda A} & \textbf{Total} \\
\hline
\textbf{Primera R} & $\frac{5}{14}$ & $\frac{15}{56}$ & $\frac{5}{8}$ \\
\hline
\textbf{Primera A} & $\frac{15}{56}$ & $\frac{3}{28}$ & $\frac{3}{8}$ \\
\hline
\textbf{Total} & $\frac{5}{8}$ & $\frac{3}{8}$ & $1$ \\
\hline
\end{tabular}
\end{center}

\textbf{Respuestas finales:}
\[
\boxed{
\begin{aligned}
\text{a) } P(\text{ambas rojas}) &= \frac{5}{14} \approx 0.357 \\
\text{b) } P(\text{segunda roja | primera roja}) &= \frac{4}{7} \approx 0.571 \\
\text{c) } \text{Verificado: } P(A \cap B) &= P(A) \cdot P(B|A) = \frac{5}{14}
\end{aligned}
}
\]
\end{ejemplo}

\newpage

\section{Ejercicios Inversos}

Los ejercicios inversos te desafían a aplicar los conceptos de probabilidad de manera creativa y en contextos más complejos. Intenta resolverlos antes de ver las soluciones.

\begin{ejercicio}{Probabilidad en el dominó}
En el juego de dominó tradicional hay 28 fichas diferentes (desde el doble blanco [0|0] hasta el doble seis [6|6]). Si se seleccionan al azar 3 fichas del conjunto completo:

a) ¿Cuál es la probabilidad de que las 3 fichas sean dobles (como [1|1], [3|3], etc.)?

b) ¿Cuál es la probabilidad de que al menos una ficha contenga el número 6?

c) Si ya sacaste una ficha doble, ¿cuál es la probabilidad de que la siguiente ficha también sea doble?
\end{ejercicio}

\begin{ejercicio}{Pronóstico meteorológico y eventos dependientes}
El servicio meteorológico ha observado los siguientes patrones en una ciudad tropical:
\begin{itemize}
    \item Si llueve un día, la probabilidad de que llueva al día siguiente es 0.7
    \item Si no llueve un día, la probabilidad de que llueva al día siguiente es 0.3
    \item En general, llueve el 40\% de los días del año
\end{itemize}

Si hoy está lloviendo:
a) ¿Cuál es la probabilidad de que llueva los próximos 2 días consecutivos?
b) ¿Cuál es la probabilidad de que no llueva ninguno de los próximos 2 días?
c) Construye un diagrama de árbol para visualizar todos los escenarios posibles de los próximos 2 días.
\end{ejercicio}

\begin{ejercicio}{Diagnóstico médico y teorema de Bayes}
Una prueba médica para detectar una enfermedad rara tiene las siguientes características:
\begin{itemize}
    \item La enfermedad afecta al 1\% de la población
    \item Si una persona tiene la enfermedad, la prueba da positivo en el 95\% de los casos (sensibilidad)
    \item Si una persona no tiene la enfermedad, la prueba da negativo en el 90\% de los casos (especificidad)
\end{itemize}

Si una persona seleccionada al azar da positivo en la prueba:
a) ¿Cuál es la probabilidad de que realmente tenga la enfermedad?
b) ¿Cuál es la probabilidad de que sea un falso positivo?
c) Explica por qué el resultado puede parecer contraintuitivo.
\end{ejercicio}

\newpage

\section{Soluciones de Ejercicios Inversos}

\begin{solucion}
\textbf{Ejercicio Inverso 1: Probabilidad en el dominó}

\textbf{Información inicial:}
\begin{itemize}
    \item Total de fichas de dominó: 28
    \item Fichas dobles: [0|0], [1|1], [2|2], [3|3], [4|4], [5|5], [6|6] = 7 fichas
    \item Fichas con el número 6: [6|0], [6|1], [6|2], [6|3], [6|4], [6|5], [6|6] = 7 fichas
\end{itemize}

\textbf{Parte a) Probabilidad de que las 3 fichas sean dobles}

\textbf{Paso 1:} Calcular el número total de formas de elegir 3 fichas de 28.
\[
C(28, 3) = \frac{28!}{3!(28-3)!} = \frac{28 \times 27 \times 26}{3 \times 2 \times 1} = \frac{19656}{6} = 3276
\]

\textbf{Paso 2:} Calcular el número de formas de elegir 3 fichas dobles de las 7 disponibles.
\[
C(7, 3) = \frac{7!}{3!(7-3)!} = \frac{7 \times 6 \times 5}{3 \times 2 \times 1} = \frac{210}{6} = 35
\]

\textbf{Paso 3:} Calcular la probabilidad.
\[
P(\text{3 dobles}) = \frac{C(7,3)}{C(28,3)} = \frac{35}{3276} = \frac{35}{3276} \approx 0.0107
\]

\textbf{Parte b) Probabilidad de al menos una ficha con el número 6}

\textbf{Paso 1:} Usar el complemento (más eficiente).
\[
P(\text{al menos un 6}) = 1 - P(\text{ningún 6})
\]

\textbf{Paso 2:} Calcular fichas sin el número 6.
Total de fichas sin 6: $28 - 7 = 21$ fichas

\textbf{Paso 3:} Calcular formas de elegir 3 fichas sin ningún 6.
\[
C(21, 3) = \frac{21 \times 20 \times 19}{3 \times 2 \times 1} = \frac{7980}{6} = 1330
\]

\textbf{Paso 4:} Calcular la probabilidad.
\[
P(\text{ningún 6}) = \frac{1330}{3276} = \frac{665}{1638}
\]
\[
P(\text{al menos un 6}) = 1 - \frac{665}{1638} = \frac{973}{1638} \approx 0.594
\]

\textbf{Parte c) Probabilidad condicional de sacar otro doble}

\textbf{Paso 1:} Establecer la situación después de sacar un doble.
\begin{itemize}
    \item Fichas restantes: 27
    \item Dobles restantes: 6
\end{itemize}

\textbf{Paso 2:} Calcular la probabilidad condicional.
\[
P(\text{segunda es doble | primera es doble}) = \frac{6}{27} = \frac{2}{9} \approx 0.222
\]

\textbf{Respuestas finales:}
\[
\boxed{
\begin{aligned}
\text{a) } P(\text{3 dobles}) &= \frac{35}{3276} \approx 0.0107 \text{ o } 1.07\% \\
\text{b) } P(\text{al menos un 6}) &= \frac{973}{1638} \approx 0.594 \text{ o } 59.4\% \\
\text{c) } P(\text{2ª doble | 1ª doble}) &= \frac{2}{9} \approx 0.222 \text{ o } 22.2\%
\end{aligned}
}
\]
\end{solucion}

\begin{solucion}
\textbf{Ejercicio Inverso 2: Pronóstico meteorológico}

\textbf{Datos dados:}
\begin{itemize}
    \item $P(\text{lluvia mañana | llueve hoy}) = 0.7$
    \item $P(\text{lluvia mañana | no llueve hoy}) = 0.3$
    \item Condición inicial: Hoy está lloviendo
\end{itemize}

\textbf{Parte c) Diagrama de árbol (lo hacemos primero para visualizar)}

\begin{center}
\begin{tikzpicture}[scale=0.9,
    grow=right,
    level 1/.style={sibling distance=3.5cm, level distance=4.5cm},
    level 2/.style={sibling distance=1.8cm, level distance=4.5cm},
    every node/.style={rectangle, draw, minimum height=8mm, minimum width=15mm},
    edge from parent/.style={draw, -{Latex}, thick}]

\node[fill=blue!30] {Hoy: Llueve}
    child {node[fill=blue!20] {Día 1: Llueve}
        child {node[fill=blue!10] {Día 2: Llueve}
            edge from parent node[above, draw=none, font=\small] {0.7}
        }
        child {node[fill=yellow!20] {Día 2: No llueve}
            edge from parent node[below, draw=none, font=\small] {0.3}
        }
        edge from parent node[above, draw=none, font=\small] {0.7}
    }
    child {node[fill=yellow!30] {Día 1: No llueve}
        child {node[fill=blue!10] {Día 2: Llueve}
            edge from parent node[above, draw=none, font=\small] {0.3}
        }
        child {node[fill=yellow!20] {Día 2: No llueve}
            edge from parent node[below, draw=none, font=\small] {0.7}
        }
        edge from parent node[below, draw=none, font=\small] {0.3}
    };

% Probabilidades finales
\node[draw=none, font=\small] at (11, 2.5) {$P = 0.7 \times 0.7 = 0.49$};
\node[draw=none, font=\small] at (11, 0.85) {$P = 0.7 \times 0.3 = 0.21$};
\node[draw=none, font=\small] at (11, -0.85) {$P = 0.3 \times 0.3 = 0.09$};
\node[draw=none, font=\small] at (11, -2.5) {$P = 0.3 \times 0.7 = 0.21$};
\end{tikzpicture}
\end{center}

\textbf{Parte a) Probabilidad de lluvia los próximos 2 días consecutivos}

\textbf{Paso 1:} Identificar la ruta en el diagrama.
La ruta es: Hoy (Llueve) → Día 1 (Llueve) → Día 2 (Llueve)

\textbf{Paso 2:} Calcular la probabilidad multiplicando las probabilidades en la ruta.
\[
P(\text{lluvia día 1 y día 2}) = P(\text{LL}) = 0.7 \times 0.7 = 0.49
\]

\textbf{Parte b) Probabilidad de que no llueva ninguno de los próximos 2 días}

\textbf{Paso 1:} Identificar la ruta en el diagrama.
La ruta es: Hoy (Llueve) → Día 1 (No llueve) → Día 2 (No llueve)

\textbf{Paso 2:} Calcular la probabilidad.
\[
P(\text{no lluvia día 1 ni día 2}) = P(\text{NN}) = 0.3 \times 0.7 = 0.21
\]

\textbf{Verificación:} Todas las probabilidades deben sumar 1.
\[
0.49 + 0.21 + 0.09 + 0.21 = 1.00 \quad \checkmark
\]

\textbf{Análisis adicional:}
\begin{itemize}
    \item Probabilidad de exactamente 1 día de lluvia: $0.21 + 0.09 = 0.30$
    \item Probabilidad de al menos 1 día de lluvia: $1 - 0.21 = 0.79$
    \item El patrón más probable es que llueva ambos días (49\%)
\end{itemize}

\textbf{Respuestas finales:}
\[
\boxed{
\begin{aligned}
\text{a) } P(\text{lluvia 2 días consecutivos}) &= 0.49 \text{ o } 49\% \\
\text{b) } P(\text{no lluvia ningún día}) &= 0.21 \text{ o } 21\% \\
\text{c) } &\text{Diagrama de árbol mostrado arriba}
\end{aligned}
}
\]
\end{solucion}

\begin{solucion}
\textbf{Ejercicio Inverso 3: Diagnóstico médico y teorema de Bayes}

\textbf{Datos dados:}
\begin{itemize}
    \item $P(\text{Enfermo}) = 0.01$ (prevalencia)
    \item $P(\text{Positivo | Enfermo}) = 0.95$ (sensibilidad)
    \item $P(\text{Negativo | Sano}) = 0.90$ (especificidad)
    \item Por lo tanto: $P(\text{Positivo | Sano}) = 0.10$ (falso positivo)
\end{itemize}

\textbf{Paso 1:} Construir un diagrama de árbol completo.

\begin{center}
\begin{tikzpicture}[scale=0.95,
    grow=right,
    level 1/.style={sibling distance=4cm, level distance=4cm},
    level 2/.style={sibling distance=2cm, level distance=4.5cm},
    every node/.style={rectangle, draw, minimum height=8mm},
    edge from parent/.style={draw, -{Latex}, thick}]

\node[fill=gray!20] {Población}
    child {node[fill=red!30] {Enfermo}
        child {node[fill=red!20] {Test +}
            edge from parent node[above, draw=none, font=\small] {0.95}
        }
        child {node[fill=green!20] {Test -}
            edge from parent node[below, draw=none, font=\small] {0.05}
        }
        edge from parent node[above, draw=none] {0.01}
    }
    child {node[fill=green!30] {Sano}
        child {node[fill=red!20] {Test +}
            edge from parent node[above, draw=none, font=\small] {0.10}
        }
        child {node[fill=green!20] {Test -}
            edge from parent node[below, draw=none, font=\small] {0.90}
        }
        edge from parent node[below, draw=none] {0.99}
    };

% Probabilidades conjuntas
\node[draw=none, font=\small] at (10.5, 2.5) {$P = 0.01 \times 0.95 = 0.0095$};
\node[draw=none, font=\small] at (10.5, 0.8) {$P = 0.01 \times 0.05 = 0.0005$};
\node[draw=none, font=\small] at (10.5, -0.8) {$P = 0.99 \times 0.10 = 0.099$};
\node[draw=none, font=\small] at (10.5, -2.5) {$P = 0.99 \times 0.90 = 0.891$};
\end{tikzpicture}
\end{center}

\textbf{Parte a) Probabilidad de tener la enfermedad dado test positivo}

\textbf{Paso 2:} Aplicar el teorema de Bayes.
\[
P(\text{Enfermo | Test+}) = \frac{P(\text{Test+ | Enfermo}) \cdot P(\text{Enfermo})}{P(\text{Test+})}
\]

\textbf{Paso 3:} Calcular $P(\text{Test+})$ (probabilidad total).
\begin{align*}
P(\text{Test+}) &= P(\text{Test+ | Enfermo}) \cdot P(\text{Enfermo}) + P(\text{Test+ | Sano}) \cdot P(\text{Sano}) \\
&= 0.95 \times 0.01 + 0.10 \times 0.99 \\
&= 0.0095 + 0.099 \\
&= 0.1085
\end{align*}

\textbf{Paso 4:} Calcular la probabilidad condicional.
\[
P(\text{Enfermo | Test+}) = \frac{0.0095}{0.1085} = \frac{95}{1085} = \frac{19}{217} \approx 0.0876
\]

\textbf{Parte b) Probabilidad de falso positivo}

\textbf{Paso 5:} Un falso positivo es estar sano pero dar positivo.
\[
P(\text{Sano | Test+}) = 1 - P(\text{Enfermo | Test+}) = 1 - 0.0876 = 0.9124
\]

\textbf{Parte c) Explicación del resultado contraintuitivo}

\textbf{Paso 6:} Análisis del resultado.

El resultado parece contraintuitivo porque aunque la prueba tiene 95\% de sensibilidad, cuando alguien da positivo, solo hay un 8.76\% de probabilidad de que realmente esté enfermo.

\textbf{Explicación:}
\begin{itemize}
    \item La enfermedad es muy rara (1\% de la población)
    \item De 1000 personas: 10 enfermas, 990 sanas
    \item De los 10 enfermos: 9.5 darán positivo (verdaderos positivos)
    \item De los 990 sanos: 99 darán positivo (falsos positivos)
    \item Total de positivos: 108.5
    \item Proporción de enfermos entre los positivos: $\frac{9.5}{108.5} \approx 8.76\%$
\end{itemize}

Este es un ejemplo clásico de por qué las pruebas de detección masiva para enfermedades raras pueden generar muchos falsos positivos, causando ansiedad innecesaria. Por eso, estas pruebas generalmente se reservan para poblaciones de alto riesgo donde la prevalencia es mayor.

\textbf{Respuestas finales:}
\[
\boxed{
\begin{aligned}
\text{a) } P(\text{Enfermo | Test+}) &= \frac{19}{217} \approx 0.0876 \text{ o } 8.76\% \\
\text{b) } P(\text{Falso positivo | Test+}) &= \frac{198}{217} \approx 0.9124 \text{ o } 91.24\% \\
\text{c) } &\text{La baja prevalencia hace que los falsos positivos} \\
&\text{superen ampliamente a los verdaderos positivos}
\end{aligned}
}
\]
\end{solucion}

\newpage

% INSERTAR EJERCICIOS PROPUESTOS Y SOLUCIONES (PARTE 3)
\section{Ejercicios Propuestos}

Ahora es tu turno de poner en práctica todo lo que has aprendido sobre probabilidad y técnicas de conteo. Los ejercicios están organizados por nivel de dificultad. ¡Intenta resolverlos antes de ver las soluciones!

\begin{ejercicio}{Principio Fundamental de Conteo}
En una cafetería escolar, el menú del día ofrece:
\begin{itemize}
    \item 3 opciones de entrada (ensalada, sopa, empanada)
    \item 4 opciones de plato principal (pollo, pescado, pasta, hamburguesa)
    \item 2 opciones de bebida (jugo, gaseosa)
\end{itemize}

\begin{enumerate}[label=\alph*)]
    \item ¿De cuántas maneras diferentes puede un estudiante armar su almuerzo completo?
    \item Si además se agregan 3 opciones de postre, ¿cuántas combinaciones posibles hay ahora?
\end{enumerate}
\end{ejercicio}

\begin{ejercicio}{Permutaciones Simples}
El comité estudiantil necesita organizar sus directivos.

\begin{enumerate}[label=\alph*)]
    \item Si hay 5 candidatos para presidente, vicepresidente y secretario, ¿de cuántas maneras diferentes se pueden asignar estos tres cargos?
    \item En un torneo de ajedrez con 8 participantes, ¿de cuántas formas pueden quedar los primeros 3 lugares (oro, plata, bronce)?
\end{enumerate}
\end{ejercicio}

\begin{ejercicio}{Combinaciones}
En tu clase de matemáticas hay 25 estudiantes.

\begin{enumerate}[label=\alph*)]
    \item El profesor quiere formar un grupo de estudio de 4 estudiantes. ¿Cuántos grupos diferentes puede formar?
    \item Para un proyecto especial, se necesitan elegir 3 estudiantes que representen al salón en una feria de ciencias. ¿De cuántas formas se puede hacer esta selección?
\end{enumerate}
\end{ejercicio}

\begin{ejercicio}{Probabilidad con Dados y Monedas}
Vamos a experimentar con objetos clásicos de la probabilidad.

\begin{enumerate}[label=\alph*)]
    \item Se lanzan dos dados justos simultáneamente. ¿Cuál es la probabilidad de que la suma de los puntos sea exactamente 7?
    \item Se lanzan tres monedas al aire. ¿Cuál es la probabilidad de obtener exactamente dos caras?
\end{enumerate}
\end{ejercicio}

\begin{ejercicio}{Probabilidad con Cartas}
De una baraja española de 40 cartas (4 palos: oros, copas, espadas y bastos, cada uno con cartas del 1 al 7, sota, caballo y rey).

\begin{enumerate}[label=\alph*)]
    \item Se extrae una carta al azar. ¿Cuál es la probabilidad de que sea una figura (sota, caballo o rey)?
    \item Se extraen dos cartas sin reposición. ¿Cuál es la probabilidad de que ambas sean del mismo palo?
\end{enumerate}
\end{ejercicio}

\begin{ejercicio}{Probabilidad Condicional}
En un colegio, el 60\% de los estudiantes practica algún deporte, el 40\% participa en actividades artísticas, y el 25\% participa en ambas actividades.

\begin{enumerate}[label=\alph*)]
    \item Si se elige un estudiante al azar y sabemos que practica deporte, ¿cuál es la probabilidad de que también participe en actividades artísticas?
    \item Si un estudiante participa en actividades artísticas, ¿cuál es la probabilidad de que también practique algún deporte?
\end{enumerate}
\end{ejercicio}

\begin{ejercicio}{Combinatoria en Dominó}
Un juego de dominó tradicional tiene 28 fichas, cada una con dos números del 0 al 6 (las mulas o dobles tienen el mismo número en ambos lados).

\begin{enumerate}[label=\alph*)]
    \item Si sacas una ficha al azar, ¿cuál es la probabilidad de que sea una mula (doble)?
    \item Si sacas dos fichas al azar sin reposición, ¿cuál es la probabilidad de que ambas tengan al menos un 6?
\end{enumerate}
\end{ejercicio}

\begin{ejercicio}{Control de Calidad}
Una fábrica produce bombillos LED. De cada lote de 100 bombillos, en promedio 3 salen defectuosos. Un inspector toma una muestra aleatoria de 5 bombillos de un lote.

\begin{enumerate}[label=\alph*)]
    \item ¿Cuál es la probabilidad de que todos los bombillos de la muestra estén buenos (sin defectos)?
    \item ¿Cuál es la probabilidad de encontrar exactamente 1 bombillo defectuoso en la muestra?
\end{enumerate}
\end{ejercicio}

\newpage

\section{Soluciones Detalladas}

\begin{solucion}{Solución Ejercicio 1 - Principio Fundamental de Conteo}
\textbf{Parte a):} Almuerzo con entrada, plato principal y bebida

Aplicamos el principio fundamental de conteo. Para cada elección de entrada, tenemos todas las opciones de plato principal, y para cada combinación de estos, todas las opciones de bebida.

\begin{align*}
\text{Total de maneras} &= \text{(Entradas)} \times \text{(Platos principales)} \times \text{(Bebidas)} \\
&= 3 \times 4 \times 2 \\
&= 24
\end{align*}

\textbf{Respuesta:} \boxed{24 \text{ maneras diferentes}}

\textbf{Parte b):} Agregando postres al menú

Ahora tenemos 4 decisiones que tomar:

\begin{align*}
\text{Total} &= \text{(Entradas)} \times \text{(Platos)} \times \text{(Bebidas)} \times \text{(Postres)} \\
&= 3 \times 4 \times 2 \times 3 \\
&= 72
\end{align*}

\textbf{Respuesta:} \boxed{72 \text{ combinaciones posibles}}
\end{solucion}

\begin{solucion}{Solución Ejercicio 2 - Permutaciones Simples}
\textbf{Parte a):} Asignar presidente, vicepresidente y secretario

Necesitamos elegir 3 personas de 5 candidatos, donde el orden importa (cada cargo es diferente).

Usamos la fórmula de permutación: $P(n,r) = \frac{n!}{(n-r)!}$

\begin{align*}
P(5,3) &= \frac{5!}{(5-3)!} = \frac{5!}{2!} \\
&= \frac{5 \times 4 \times 3 \times 2 \times 1}{2 \times 1} \\
&= \frac{120}{2} = 60
\end{align*}

\textbf{Respuesta:} \boxed{60 \text{ maneras diferentes}}

\textbf{Parte b):} Primeros 3 lugares en torneo de ajedrez

De 8 participantes, elegimos 3 para oro, plata y bronce. El orden importa.

\begin{align*}
P(8,3) &= \frac{8!}{(8-3)!} = \frac{8!}{5!} \\
&= 8 \times 7 \times 6 \\
&= 336
\end{align*}

\textbf{Respuesta:} \boxed{336 \text{ formas diferentes}}
\end{solucion}

\begin{solucion}{Solución Ejercicio 3 - Combinaciones}
\textbf{Parte a):} Formar grupo de 4 estudiantes de 25

Aquí el orden NO importa (un grupo es el mismo sin importar el orden).

Usamos la fórmula de combinación: $C(n,r) = \binom{n}{r} = \frac{n!}{r!(n-r)!}$

\begin{align*}
C(25,4) &= \binom{25}{4} = \frac{25!}{4!(25-4)!} \\
&= \frac{25!}{4! \times 21!} \\
&= \frac{25 \times 24 \times 23 \times 22}{4 \times 3 \times 2 \times 1} \\
&= \frac{303,600}{24} = 12,650
\end{align*}

\textbf{Respuesta:} \boxed{12,650 \text{ grupos diferentes}}

\textbf{Parte b):} Elegir 3 representantes de 25 estudiantes

\begin{align*}
C(25,3) &= \binom{25}{3} = \frac{25!}{3!(25-3)!} \\
&= \frac{25 \times 24 \times 23}{3 \times 2 \times 1} \\
&= \frac{13,800}{6} = 2,300
\end{align*}

\textbf{Respuesta:} \boxed{2,300 \text{ formas de selección}}
\end{solucion}

\begin{solucion}{Solución Ejercicio 4 - Probabilidad con Dados y Monedas}
\textbf{Parte a):} Suma de dos dados igual a 7

Espacio muestral: Al lanzar dos dados, hay $6 \times 6 = 36$ resultados posibles.

Casos favorables donde la suma es 7:
\begin{itemize}
    \item $(1,6)$: $1 + 6 = 7$
    \item $(2,5)$: $2 + 5 = 7$
    \item $(3,4)$: $3 + 4 = 7$
    \item $(4,3)$: $4 + 3 = 7$
    \item $(5,2)$: $5 + 2 = 7$
    \item $(6,1)$: $6 + 1 = 7$
\end{itemize}

Total de casos favorables: 6

\[P(\text{suma} = 7) = \frac{\text{casos favorables}}{\text{casos totales}} = \frac{6}{36} = \frac{1}{6}\]

\textbf{Respuesta:} \boxed{P = \frac{1}{6} \approx 0.167 \text{ o } 16.7\%}

\textbf{Parte b):} Exactamente dos caras en tres monedas

Espacio muestral: $2^3 = 8$ resultados posibles

Casos favorables: CCS, CSC, SCC = 3 casos

\[P(\text{exactamente 2 caras}) = \frac{3}{8}\]

\textbf{Respuesta:} \boxed{P = \frac{3}{8} = 0.375 \text{ o } 37.5\%}
\end{solucion}

\begin{solucion}{Solución Ejercicio 5 - Probabilidad con Cartas}
\textbf{Parte a):} Probabilidad de sacar una figura

En una baraja española de 40 cartas:
\begin{itemize}
    \item Figuras por palo: sota, caballo, rey (3 figuras)
    \item Número de palos: 4
    \item Total de figuras: $3 \times 4 = 12$
\end{itemize}

\[P(\text{figura}) = \frac{12}{40} = \frac{3}{10}\]

\textbf{Respuesta:} \boxed{P = \frac{3}{10} = 0.3 \text{ o } 30\%}

\textbf{Parte b):} Dos cartas del mismo palo (sin reposición)

\textbf{Método - Probabilidad condicional:}
\begin{itemize}
    \item Primera carta: cualquier carta (probabilidad = 1)
    \item Segunda carta: debe ser del mismo palo
    \item Quedan 9 cartas del mismo palo de 39 totales
\end{itemize}

\[P = 1 \times \frac{9}{39} = \frac{9}{39} = \frac{3}{13}\]

\textbf{Respuesta:} \boxed{P = \frac{3}{13} \approx 0.231 \text{ o } 23.1\%}
\end{solucion}

\begin{solucion}{Solución Ejercicio 6 - Probabilidad Condicional}
Datos:
\begin{itemize}
    \item $P(D) = 0.60$ (practica deporte)
    \item $P(A) = 0.40$ (actividades artísticas)
    \item $P(D \cap A) = 0.25$ (ambas)
\end{itemize}

\textbf{Parte a):} $P(A|D)$ = Probabilidad de arte dado que practica deporte

Fórmula de probabilidad condicional:
\[P(A|D) = \frac{P(A \cap D)}{P(D)} = \frac{0.25}{0.60} = \frac{25}{60} = \frac{5}{12}\]

\textbf{Respuesta:} \boxed{P(A|D) = \frac{5}{12} \approx 0.417 \text{ o } 41.7\%}

\textbf{Parte b):} $P(D|A)$ = Probabilidad de deporte dado que hace arte

\[P(D|A) = \frac{P(D \cap A)}{P(A)} = \frac{0.25}{0.40} = \frac{25}{40} = \frac{5}{8}\]

\textbf{Respuesta:} \boxed{P(D|A) = \frac{5}{8} = 0.625 \text{ o } 62.5\%}
\end{solucion}

\begin{solucion}{Solución Ejercicio 7 - Combinatoria en Dominó}
\textbf{Información sobre el dominó:}
\begin{itemize}
    \item Total de fichas: 28
    \item Mulas (dobles): 7 fichas
    \item Fichas con al menos un 6: 7 fichas
\end{itemize}

\textbf{Parte a):} Probabilidad de sacar una mula

\[P(\text{mula}) = \frac{7}{28} = \frac{1}{4}\]

\textbf{Respuesta:} \boxed{P = \frac{1}{4} = 0.25 \text{ o } 25\%}

\textbf{Parte b):} Dos fichas con al menos un 6

Casos totales al sacar 2 fichas: $\binom{28}{2} = \frac{28 \times 27}{2} = 378$

Casos favorables (ambas con al menos un 6): $\binom{7}{2} = \frac{7 \times 6}{2} = 21$

\[P(\text{ambas con 6}) = \frac{21}{378} = \frac{1}{18}\]

\textbf{Respuesta:} \boxed{P = \frac{1}{18} \approx 0.056 \text{ o } 5.6\%}
\end{solucion}

\begin{solucion}{Solución Ejercicio 8 - Control de Calidad}
Datos:
\begin{itemize}
    \item Lote: 100 bombillos
    \item Defectuosos: 3 bombillos
    \item Buenos: 97 bombillos
    \item Muestra: 5 bombillos
\end{itemize}

\textbf{Parte a):} Todos los bombillos buenos en la muestra

\textbf{Método (más simple):}
\begin{align*}
P &= \frac{97}{100} \times \frac{96}{99} \times \frac{95}{98} \times \frac{94}{97} \times \frac{93}{96} \\
&\approx 0.8560
\end{align*}

\textbf{Respuesta:} \boxed{P \approx 0.856 \text{ o } 85.6\%}

\textbf{Parte b):} Exactamente 1 defectuoso en la muestra

Necesitamos elegir:
\begin{itemize}
    \item 1 defectuoso de 3: $\binom{3}{1} = 3$
    \item 4 buenos de 97: $\binom{97}{4} = 3,764,376$
\end{itemize}

Casos totales: $\binom{100}{5} = 75,287,520$

Casos favorables: $3 \times 3,764,376 = 11,293,128$

\[P(\text{exactamente 1 defectuoso}) = \frac{11,293,128}{75,287,520} \approx 0.1500\]

\textbf{Respuesta:} \boxed{P \approx 0.150 \text{ o } 15.0\%}
\end{solucion}

\end{document}
