% !TEX program = lualatex
% PARTE 3 de 3: Ejercicios Propuestos y Soluciones - Probabilidad
\documentclass[12pt,a4paper,twoside]{article}
\usepackage{fontspec}
\usepackage[spanish,es-nodecimaldot]{babel}
\usepackage{amsmath,amssymb}
\usepackage[margin=2.5cm]{geometry}
\usepackage{xcolor}
\usepackage{tikz,pgfplots}
\usetikzlibrary{calc,arrows.meta,babel,trees,shapes}
\usepackage{multicol}
\usepackage{enumitem}
\pgfplotsset{compat=1.18}
\definecolor{maincolor}{RGB}{26,35,126}
\definecolor{accentcolor}{RGB}{255,87,34}

% Configuración de títulos y formato
\usepackage{titlesec}
\titleformat{\section}{\Large\bfseries\color{maincolor}}{\thesection}{1em}{}
\titleformat{\subsection}{\large\bfseries\color{accentcolor}}{\thesubsection}{1em}{}

% Configuración de cajas para ejemplos
\usepackage{tcolorbox}
\tcbuselibrary{skins,breakable}

\usepackage{fancyhdr}

\pagestyle{fancy}
\fancyhf{}
\fancyhead[LE]{\small\textcolor{maincolor}{\thepage \quad Probabilidad y Técnicas de Conteo}}
\fancyhead[RO]{\small\textcolor{maincolor}{Probabilidad y Técnicas de Conteo \quad \thepage}}
\fancyhead[LO]{\small\textcolor{maincolor}{Grado 10 - Estadística}}
\fancyhead[RE]{\small\textcolor{maincolor}{Prof. Toribio De J Arrieta F}}
\fancyfoot[C]{}
\renewcommand{\headrulewidth}{0.5pt}
\renewcommand{\footrulewidth}{0pt}
\setlength{\headheight}{14pt}

\newtcolorbox{ejercicio}[1][]{
  enhanced,
  breakable,
  colback=accentcolor!5,
  colframe=accentcolor,
  fonttitle=\bfseries,
  title=#1
}

\newtcolorbox{solucion}[1][]{
  enhanced,
  breakable,
  colback=green!5,
  colframe=green!60!black,
  fonttitle=\bfseries,
  title=#1
}

\newtcolorbox{nota}[1][]{
  enhanced,
  colback=yellow!10,
  colframe=orange!80!black,
  fonttitle=\bfseries,
  title=Nota Importante,
  #1
}

\begin{document}

\section{Ejercicios Propuestos}

Ahora es tu turno de poner en práctica todo lo que has aprendido sobre probabilidad y técnicas de conteo. Los ejercicios están organizados por nivel de dificultad. ¡Intenta resolverlos antes de ver las soluciones!

\begin{ejercicio}{Principio Fundamental de Conteo}
En una cafetería escolar, el menú del día ofrece:
\begin{itemize}
    \item 3 opciones de entrada (ensalada, sopa, empanada)
    \item 4 opciones de plato principal (pollo, pescado, pasta, hamburguesa)
    \item 2 opciones de bebida (jugo, gaseosa)
\end{itemize}

\begin{enumerate}[label=\alph*)]
    \item ¿De cuántas maneras diferentes puede un estudiante armar su almuerzo completo?
    \item Si además se agregan 3 opciones de postre, ¿cuántas combinaciones posibles hay ahora?
\end{enumerate}
\end{ejercicio}

\begin{ejercicio}{Permutaciones Simples}
El comité estudiantil necesita organizar sus directivos.

\begin{enumerate}[label=\alph*)]
    \item Si hay 5 candidatos para presidente, vicepresidente y secretario, ¿de cuántas maneras diferentes se pueden asignar estos tres cargos?
    \item En un torneo de ajedrez con 8 participantes, ¿de cuántas formas pueden quedar los primeros 3 lugares (oro, plata, bronce)?
\end{enumerate}
\end{ejercicio}

\begin{ejercicio}{Combinaciones}
En tu clase de matemáticas hay 25 estudiantes.

\begin{enumerate}[label=\alph*)]
    \item El profesor quiere formar un grupo de estudio de 4 estudiantes. ¿Cuántos grupos diferentes puede formar?
    \item Para un proyecto especial, se necesitan elegir 3 estudiantes que representen al salón en una feria de ciencias. ¿De cuántas formas se puede hacer esta selección?
\end{enumerate}
\end{ejercicio}

\begin{ejercicio}{Probabilidad con Dados y Monedas}
Vamos a experimentar con objetos clásicos de la probabilidad.

\begin{enumerate}[label=\alph*)]
    \item Se lanzan dos dados justos simultáneamente. ¿Cuál es la probabilidad de que la suma de los puntos sea exactamente 7?
    \item Se lanzan tres monedas al aire. ¿Cuál es la probabilidad de obtener exactamente dos caras?
\end{enumerate}
\end{ejercicio}

\begin{ejercicio}{Probabilidad con Cartas}
De una baraja española de 40 cartas (4 palos: oros, copas, espadas y bastos, cada uno con cartas del 1 al 7, sota, caballo y rey).

\begin{enumerate}[label=\alph*)]
    \item Se extrae una carta al azar. ¿Cuál es la probabilidad de que sea una figura (sota, caballo o rey)?
    \item Se extraen dos cartas sin reposición. ¿Cuál es la probabilidad de que ambas sean del mismo palo?
\end{enumerate}
\end{ejercicio}

\begin{ejercicio}{Probabilidad Condicional}
En un colegio, el 60\% de los estudiantes practica algún deporte, el 40\% participa en actividades artísticas, y el 25\% participa en ambas actividades.

\begin{enumerate}[label=\alph*)]
    \item Si se elige un estudiante al azar y sabemos que practica deporte, ¿cuál es la probabilidad de que también participe en actividades artísticas?
    \item Si un estudiante participa en actividades artísticas, ¿cuál es la probabilidad de que también practique algún deporte?
\end{enumerate}
\end{ejercicio}

\begin{ejercicio}{Combinatoria en Dominó}
Un juego de dominó tradicional tiene 28 fichas, cada una con dos números del 0 al 6 (las mulas o dobles tienen el mismo número en ambos lados).

\begin{enumerate}[label=\alph*)]
    \item Si sacas una ficha al azar, ¿cuál es la probabilidad de que sea una mula (doble)?
    \item Si sacas dos fichas al azar sin reposición, ¿cuál es la probabilidad de que ambas tengan al menos un 6?
\end{enumerate}
\end{ejercicio}

\begin{ejercicio}{Control de Calidad}
Una fábrica produce bombillos LED. De cada lote de 100 bombillos, en promedio 3 salen defectuosos. Un inspector toma una muestra aleatoria de 5 bombillos de un lote.

\begin{enumerate}[label=\alph*)]
    \item ¿Cuál es la probabilidad de que todos los bombillos de la muestra estén buenos (sin defectos)?
    \item ¿Cuál es la probabilidad de encontrar exactamente 1 bombillo defectuoso en la muestra?
\end{enumerate}
\end{ejercicio}

\newpage

\section{Soluciones Detalladas}

\begin{solucion}{Solución Ejercicio 1 - Principio Fundamental de Conteo}
\textbf{Parte a):} Almuerzo con entrada, plato principal y bebida

Aplicamos el principio fundamental de conteo. Para cada elección de entrada, tenemos todas las opciones de plato principal, y para cada combinación de estos, todas las opciones de bebida.

\begin{align*}
\text{Total de maneras} &= \text{(Entradas)} \times \text{(Platos principales)} \times \text{(Bebidas)} \\
&= 3 \times 4 \times 2 \\
&= 24
\end{align*}

\textbf{Verificación con diagrama de árbol (parcial):}

\begin{center}
\begin{tikzpicture}[scale=0.9,
    level 1/.style={sibling distance=4cm},
    level 2/.style={sibling distance=1.5cm},
    level 3/.style={sibling distance=0.8cm}]
  \node {Inicio}
    child {node {Ensalada}
      child {node {Pollo}
        child {node {Jugo}}
        child {node {Gaseosa}}
      }
      child {node {Pescado}
        child {node {J}}
        child {node {G}}
      }
      child {node {...}}
    }
    child {node {Sopa}
      child {node {...}}
    }
    child {node {Empanada}
      child {node {...}}
    };
\end{tikzpicture}
\end{center}

\textbf{Respuesta:} \boxed{24 \text{ maneras diferentes}}

\textbf{Parte b):} Agregando postres al menú

Ahora tenemos 4 decisiones que tomar:

\begin{align*}
\text{Total} &= \text{(Entradas)} \times \text{(Platos)} \times \text{(Bebidas)} \times \text{(Postres)} \\
&= 3 \times 4 \times 2 \times 3 \\
&= 72
\end{align*}

\textbf{Respuesta:} \boxed{72 \text{ combinaciones posibles}}

\textbf{Explicación:} El principio fundamental dice que si tenemos $n$ decisiones independientes con $a_1, a_2, ..., a_n$ opciones respectivamente, el total de formas es el producto $a_1 \times a_2 \times ... \times a_n$.
\end{solucion}

\begin{solucion}{Solución Ejercicio 2 - Permutaciones Simples}
\textbf{Parte a):} Asignar presidente, vicepresidente y secretario

Necesitamos elegir 3 personas de 5 candidatos, donde el orden importa (cada cargo es diferente).

Usamos la fórmula de permutación: $P(n,r) = \frac{n!}{(n-r)!}$

\begin{align*}
P(5,3) &= \frac{5!}{(5-3)!} = \frac{5!}{2!} \\
&= \frac{5 \times 4 \times 3 \times 2 \times 1}{2 \times 1} \\
&= \frac{120}{2} = 60
\end{align*}

\textbf{Forma alternativa de pensar:}
\begin{itemize}
    \item Para presidente: 5 opciones
    \item Para vicepresidente: 4 opciones (ya elegimos uno)
    \item Para secretario: 3 opciones (ya elegimos dos)
    \item Total: $5 \times 4 \times 3 = 60$
\end{itemize}

\textbf{Respuesta:} \boxed{60 \text{ maneras diferentes}}

\textbf{Parte b):} Primeros 3 lugares en torneo de ajedrez

De 8 participantes, elegimos 3 para oro, plata y bronce. El orden importa.

\begin{align*}
P(8,3) &= \frac{8!}{(8-3)!} = \frac{8!}{5!} \\
&= 8 \times 7 \times 6 \\
&= 336
\end{align*}

\textbf{Respuesta:} \boxed{336 \text{ formas diferentes}}
\end{solucion}

\begin{solucion}{Solución Ejercicio 3 - Combinaciones}
\textbf{Parte a):} Formar grupo de 4 estudiantes de 25

Aquí el orden NO importa (un grupo es el mismo sin importar el orden).

Usamos la fórmula de combinación: $C(n,r) = \binom{n}{r} = \frac{n!}{r!(n-r)!}$

\begin{align*}
C(25,4) &= \binom{25}{4} = \frac{25!}{4!(25-4)!} \\
&= \frac{25!}{4! \times 21!} \\
&= \frac{25 \times 24 \times 23 \times 22}{4 \times 3 \times 2 \times 1} \\
&= \frac{303,600}{24} = 12,650
\end{align*}

\textbf{Respuesta:} \boxed{12,650 \text{ grupos diferentes}}

\textbf{Parte b):} Elegir 3 representantes de 25 estudiantes

\begin{align*}
C(25,3) &= \binom{25}{3} = \frac{25!}{3!(25-3)!} \\
&= \frac{25 \times 24 \times 23}{3 \times 2 \times 1} \\
&= \frac{13,800}{6} = 2,300
\end{align*}

\textbf{Respuesta:} \boxed{2,300 \text{ formas de selección}}
\end{solucion}

\begin{solucion}{Solución Ejercicio 4 - Probabilidad con Dados y Monedas}
\textbf{Parte a):} Suma de dos dados igual a 7

Espacio muestral: Al lanzar dos dados, hay $6 \times 6 = 36$ resultados posibles.

Casos favorables donde la suma es 7:
\begin{itemize}
    \item $(1,6)$: $1 + 6 = 7$
    \item $(2,5)$: $2 + 5 = 7$
    \item $(3,4)$: $3 + 4 = 7$
    \item $(4,3)$: $4 + 3 = 7$
    \item $(5,2)$: $5 + 2 = 7$
    \item $(6,1)$: $6 + 1 = 7$
\end{itemize}

Total de casos favorables: 6

\[P(\text{suma} = 7) = \frac{\text{casos favorables}}{\text{casos totales}} = \frac{6}{36} = \frac{1}{6}\]

\textbf{Respuesta:} \boxed{P = \frac{1}{6} \approx 0.167 \text{ o } 16.7\%}

\textbf{Parte b):} Exactamente dos caras en tres monedas

Espacio muestral: $2^3 = 8$ resultados posibles

Listemos todos los resultados (C = cara, S = sello):
\begin{itemize}
    \item CCC
    \item CCS ← exactamente 2 caras
    \item CSC ← exactamente 2 caras
    \item CSS
    \item SCC ← exactamente 2 caras
    \item SCS
    \item SSC
    \item SSS
\end{itemize}

Casos favorables: 3 (CCS, CSC, SCC)

\[P(\text{exactamente 2 caras}) = \frac{3}{8}\]

\textbf{Forma alternativa usando combinaciones:}
Elegir 2 posiciones de 3 para las caras: $\binom{3}{2} = 3$

\[P = \frac{\binom{3}{2}}{2^3} = \frac{3}{8}\]

\textbf{Respuesta:} \boxed{P = \frac{3}{8} = 0.375 \text{ o } 37.5\%}
\end{solucion}

\begin{solucion}{Solución Ejercicio 5 - Probabilidad con Cartas}
\textbf{Parte a):} Probabilidad de sacar una figura

En una baraja española de 40 cartas:
\begin{itemize}
    \item Figuras por palo: sota, caballo, rey (3 figuras)
    \item Número de palos: 4
    \item Total de figuras: $3 \times 4 = 12$
\end{itemize}

\[P(\text{figura}) = \frac{12}{40} = \frac{3}{10}\]

\textbf{Respuesta:} \boxed{P = \frac{3}{10} = 0.3 \text{ o } 30\%}

\textbf{Parte b):} Dos cartas del mismo palo (sin reposición)

\textbf{Método 1 - Casos favorables sobre casos totales:}

Casos totales al sacar 2 cartas de 40: $\binom{40}{2} = \frac{40 \times 39}{2} = 780$

Casos favorables (mismo palo):
\begin{itemize}
    \item Por cada palo: $\binom{10}{2} = \frac{10 \times 9}{2} = 45$ formas
    \item Total (4 palos): $4 \times 45 = 180$
\end{itemize}

\[P(\text{mismo palo}) = \frac{180}{780} = \frac{3}{13}\]

\textbf{Método 2 - Probabilidad condicional:}
\begin{itemize}
    \item Primera carta: cualquier carta (probabilidad = 1)
    \item Segunda carta: debe ser del mismo palo
    \item Quedan 9 cartas del mismo palo de 39 totales
\end{itemize}

\[P = 1 \times \frac{9}{39} = \frac{9}{39} = \frac{3}{13}\]

\textbf{Respuesta:} \boxed{P = \frac{3}{13} \approx 0.231 \text{ o } 23.1\%}
\end{solucion}

\begin{solucion}{Solución Ejercicio 6 - Probabilidad Condicional}
Datos:
\begin{itemize}
    \item $P(D) = 0.60$ (practica deporte)
    \item $P(A) = 0.40$ (actividades artísticas)
    \item $P(D \cap A) = 0.25$ (ambas)
\end{itemize}

\textbf{Diagrama de Venn:}

\begin{center}
\begin{tikzpicture}[scale=0.85]
    % Rectángulo universal
    \draw[thick] (-3,-2) rectangle (3,2);
    \node at (2.5,1.5) {$U$};

    % Círculos
    \draw[thick,fill=blue!20] (-0.8,0) circle (1.5cm);
    \draw[thick,fill=red!20] (0.8,0) circle (1.5cm);

    % Etiquetas
    \node at (-1.5,0) {$0.35$};
    \node at (0,0) {$0.25$};
    \node at (1.5,0) {$0.15$};
    \node at (-1.5,1.8) {Deporte};
    \node at (1.5,1.8) {Arte};
    \node at (0,-2.5) {Ninguna: $0.25$};
\end{tikzpicture}
\end{center}

\textbf{Parte a):} $P(A|D)$ = Probabilidad de arte dado que practica deporte

Fórmula de probabilidad condicional:
\[P(A|D) = \frac{P(A \cap D)}{P(D)} = \frac{0.25}{0.60} = \frac{25}{60} = \frac{5}{12}\]

\textbf{Respuesta:} \boxed{P(A|D) = \frac{5}{12} \approx 0.417 \text{ o } 41.7\%}

\textbf{Parte b):} $P(D|A)$ = Probabilidad de deporte dado que hace arte

\[P(D|A) = \frac{P(D \cap A)}{P(A)} = \frac{0.25}{0.40} = \frac{25}{40} = \frac{5}{8}\]

\textbf{Respuesta:} \boxed{P(D|A) = \frac{5}{8} = 0.625 \text{ o } 62.5\%}

\textbf{Interpretación:} Es más probable que un estudiante artístico practique deporte (62.5\%) que un deportista haga arte (41.7\%).
\end{solucion}

\begin{solucion}{Solución Ejercicio 7 - Combinatoria en Dominó}
\textbf{Información sobre el dominó:}
\begin{itemize}
    \item Total de fichas: 28
    \item Números en cada ficha: del 0 al 6 (7 números)
    \item Mulas (dobles): 0-0, 1-1, 2-2, 3-3, 4-4, 5-5, 6-6 (7 fichas)
    \item Fichas con al menos un 6: todas las que tienen 6
\end{itemize}

\textbf{Parte a):} Probabilidad de sacar una mula

Casos favorables: 7 mulas
Casos totales: 28 fichas

\[P(\text{mula}) = \frac{7}{28} = \frac{1}{4}\]

\textbf{Respuesta:} \boxed{P = \frac{1}{4} = 0.25 \text{ o } 25\%}

\textbf{Parte b):} Dos fichas con al menos un 6

Primero, contemos las fichas con al menos un 6:
\begin{itemize}
    \item Mula 6-6: 1 ficha
    \item Fichas 6-x (donde x ≠ 6): 6 fichas (6-0, 6-1, 6-2, 6-3, 6-4, 6-5)
    \item Total con al menos un 6: 7 fichas
\end{itemize}

Casos totales al sacar 2 fichas: $\binom{28}{2} = \frac{28 \times 27}{2} = 378$

Casos favorables (ambas con al menos un 6): $\binom{7}{2} = \frac{7 \times 6}{2} = 21$

\[P(\text{ambas con 6}) = \frac{21}{378} = \frac{1}{18}\]

\textbf{Respuesta:} \boxed{P = \frac{1}{18} \approx 0.056 \text{ o } 5.6\%}

\textbf{Verificación con árbol de probabilidad:}

\begin{center}
\begin{tikzpicture}[scale=0.95]
    \node {Inicio}
    child {node {1ra con 6} edge from parent node[left] {$\frac{7}{28}$}
        child {node {2da con 6} edge from parent node[left] {$\frac{6}{27}$}}
    };
    \node at (2,-2.5) {$P = \frac{7}{28} \times \frac{6}{27} = \frac{42}{756} = \frac{1}{18}$};
\end{tikzpicture}
\end{center}
\end{solucion}

\begin{solucion}{Solución Ejercicio 8 - Control de Calidad}
Datos:
\begin{itemize}
    \item Lote: 100 bombillos
    \item Defectuosos: 3 bombillos
    \item Buenos: 97 bombillos
    \item Muestra: 5 bombillos
\end{itemize}

\textbf{Parte a):} Todos los bombillos buenos en la muestra

Usamos combinaciones sin reposición:

Casos totales: $\binom{100}{5} = \frac{100!}{5! \times 95!} = 75,287,520$

Casos favorables (5 buenos de 97): $\binom{97}{5} = \frac{97!}{5! \times 92!} = 64,446,024$

\[P(\text{todos buenos}) = \frac{\binom{97}{5}}{\binom{100}{5}} = \frac{64,446,024}{75,287,520} \approx 0.8560\]

\textbf{Método alternativo (más simple):}
\begin{align*}
P &= \frac{97}{100} \times \frac{96}{99} \times \frac{95}{98} \times \frac{94}{97} \times \frac{93}{96} \\
&= \frac{97 \times 96 \times 95 \times 94 \times 93}{100 \times 99 \times 98 \times 97 \times 96} \\
&= \frac{95 \times 94 \times 93}{100 \times 99 \times 98} \\
&\approx 0.8560
\end{align*}

\textbf{Respuesta:} \boxed{P \approx 0.856 \text{ o } 85.6\%}

\textbf{Parte b):} Exactamente 1 defectuoso en la muestra

Necesitamos elegir:
\begin{itemize}
    \item 1 defectuoso de 3: $\binom{3}{1} = 3$
    \item 4 buenos de 97: $\binom{97}{4} = 3,764,376$
\end{itemize}

Casos favorables: $\binom{3}{1} \times \binom{97}{4} = 3 \times 3,764,376 = 11,293,128$

\[P(\text{exactamente 1 defectuoso}) = \frac{11,293,128}{75,287,520} \approx 0.1500\]

\textbf{Respuesta:} \boxed{P \approx 0.150 \text{ o } 15.0\%}

\textbf{Interpretación práctica:}
\begin{itemize}
    \item Hay un 85.6\% de probabilidad de no encontrar defectos en la muestra
    \item Hay un 15\% de probabilidad de encontrar exactamente 1 defecto
    \item La probabilidad de 2 o más defectos es muy baja (menos del 0.5\%)
\end{itemize}

Esto sugiere que el proceso de producción está funcionando bien, con una tasa de defectos del 3\% que es aceptable para muchas industrias.
\end{solucion}

\end{document}