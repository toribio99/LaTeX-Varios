% !TEX program = lualatex
\documentclass[12pt,a4paper,twoside]{article}
\usepackage{fontspec}
\usepackage[spanish,es-nodecimaldot]{babel}
\usepackage{amsmath,amssymb}
\usepackage[margin=2.5cm]{geometry}
\usepackage{xcolor}
\usepackage{tikz,pgfplots}
\usetikzlibrary{calc,arrows.meta,babel}
\usepackage{multicol}
\usepackage{enumitem}
\pgfplotsset{compat=1.18}
\definecolor{maincolor}{RGB}{26,35,126}
\definecolor{accentcolor}{RGB}{255,87,34}

% Configuración de títulos y formato
\usepackage{titlesec}
\titleformat{\section}{\Large\bfseries\color{maincolor}}{\thesection}{1em}{}
\titleformat{\subsection}{\large\bfseries\color{accentcolor}}{\thesubsection}{1em}{}

% Configuración de cajas para ejemplos
\usepackage{tcolorbox}
\tcbuselibrary{skins,breakable}

\usepackage{fancyhdr}

\pagestyle{fancy}
\fancyhf{}
\fancyhead[LE]{\small\textcolor{maincolor}{\thepage \quad Gráficas de Funciones Trigonométricas}}
\fancyhead[RO]{\small\textcolor{maincolor}{Gráficas de Funciones Trigonométricas \quad \thepage}}
\fancyhead[LO]{\small\textcolor{maincolor}{Grado 10 - Trigonometría}}
\fancyhead[RE]{\small\textcolor{maincolor}{Prof. Toribio De J Arrieta F}}
\fancyfoot[C]{}
\renewcommand{\headrulewidth}{0.5pt}
\renewcommand{\footrulewidth}{0pt}
\setlength{\headheight}{14pt}

\newtcolorbox{ejemplo}[1][]{
  enhanced,
  breakable,
  colback=maincolor!5,
  colframe=maincolor,
  fonttitle=\bfseries,
  title=Ejemplo Resuelto,
  #1
}

\newtcolorbox{ejercicio}[1][]{
  enhanced,
  breakable,
  colback=accentcolor!5,
  colframe=accentcolor,
  fonttitle=\bfseries,
  title=Ejercicio,
  #1
}

\newtcolorbox{solucion}[1][]{
  enhanced,
  breakable,
  colback=green!5,
  colframe=green!60!black,
  fonttitle=\bfseries,
  title=Solución,
  #1
}

\newtcolorbox{nota}[1][]{
  enhanced,
  colback=yellow!10,
  colframe=orange!80!black,
  fonttitle=\bfseries,
  title=Nota Importante,
  #1
}

% Título
\title{\textbf{\Huge Graficas De Las Funciones Trigonometricas}\\[0.5cm]
\Large Guía de Trigonometría}
\author{Prof. Toribio De J Arrieta F\\
\textit{La Pruebita}\\
Grado 10}
\date{\today}

\begin{document}

\maketitle

\tableofcontents
\newpage

\section{Introducción}

¡Bienvenidos al emocionante mundo de las gráficas de las funciones trigonométricas! Si alguna vez te has preguntado cómo se ven las ondas del mar, cómo se representan las señales de radio, o cómo vibra una cuerda de guitarra, estás a punto de descubrirlo. Las gráficas de las funciones trigonométricas son la clave para visualizar y entender todos estos fenómenos ondulatorios que nos rodean.

En el tema anterior estudiamos las funciones trigonométricas desde el punto de vista del círculo unitario: definimos el seno, coseno, tangente y sus recíprocas, y aprendimos a calcular sus valores para diferentes ángulos. Ahora vamos a dar un paso más: vamos a \textbf{graficar} estas funciones, es decir, vamos a dibujar sus comportamientos en el plano cartesiano. Esto nos permitirá ver patrones, identificar propiedades visuales, y aplicar estas funciones a problemas del mundo real.

\subsection*{¿Por qué son importantes las gráficas trigonométricas?}

Las gráficas de las funciones trigonométricas aparecen constantemente en la ciencia y la tecnología:

\begin{itemize}
    \item \textbf{Ondas sonoras:} Cuando hablas, tu voz genera ondas que tienen forma de función seno. Los ingenieros de audio usan estas gráficas para procesar y mejorar el sonido.
    \item \textbf{Señales eléctricas:} La corriente alterna (AC) que llega a tu casa tiene forma senoidal. Los electricistas deben entender estas gráficas para diseñar circuitos seguros y eficientes.
    \item \textbf{Movimiento armónico simple:} Un péndulo oscilando, un resorte vibrando, o las ruedas de un carro girando, todos describen movimientos que se grafican con funciones trigonométricas.
    \item \textbf{Mareas:} El nivel del mar sube y baja siguiendo un patrón cosenoidal. Los navegantes usan estas gráficas para predecir las mareas.
    \item \textbf{Análisis de fenómenos periódicos:} Cualquier cosa que se repite en el tiempo (como el día y la noche, las estaciones del año, o los latidos del corazón) puede modelarse con funciones trigonométricas.
    \item \textbf{Ingeniería de señales:} En telecomunicaciones, radio, televisión, WiFi, y celulares, todas las señales son combinaciones de ondas senoidales. Sin entender estas gráficas, no existiría la tecnología moderna.
\end{itemize}

\subsection*{¿Qué vamos a aprender?}

En esta guía vamos a explorar las gráficas de las seis funciones trigonométricas:

\begin{enumerate}
    \item \textbf{Función seno} ($y = \sin(x)$): La onda más básica y fundamental
    \item \textbf{Función coseno} ($y = \cos(x)$): El hermano gemelo del seno, desplazado en fase
    \item \textbf{Función tangente} ($y = \tan(x)$): Con sus características asíntotas verticales
    \item \textbf{Función cotangente} ($y = \cot(x)$): La tangente invertida
    \item \textbf{Función secante} ($y = \sec(x)$): Con sus curvas en forma de U
    \item \textbf{Función cosecante} ($y = \csc(x)$): El recíproco del seno
\end{enumerate}

Para cada función vamos a estudiar:
\begin{itemize}
    \item Su forma básica y características principales
    \item Dominio y rango
    \item Período (cada cuánto se repite)
    \item Amplitud (qué tan alto/bajo llega)
    \item Asíntotas (si existen)
    \item Simetrías y propiedades especiales
    \item Transformaciones (desplazamientos, estiramientos, reflexiones)
\end{itemize}

Prepárate para ver las funciones trigonométricas desde una nueva perspectiva. No solo son números y fórmulas, ¡son ondas, oscilaciones, y patrones visuales que describen el comportamiento del universo!

\newpage

\section{Conceptos Fundamentales}

Antes de graficar las funciones trigonométricas, necesitamos recordar algunos conceptos clave y establecer un vocabulario común. Esta sección te dará las herramientas necesarias para entender y analizar las gráficas que estudiaremos.

\subsection{Repaso: Las Seis Funciones Trigonométricas}

Recordemos las definiciones de las seis funciones trigonométricas en el círculo unitario. Para un ángulo $\theta$, si el punto correspondiente en el círculo unitario es $(x, y)$, entonces:

\begin{center}
\renewcommand{\arraystretch}{2}
\begin{tabular}{|c|c|c|}
\hline
\textbf{Función} & \textbf{Definición} & \textbf{Valor en círculo unitario} \\
\hline
Seno & $\sin\theta = \dfrac{\text{cateto opuesto}}{\text{hipotenusa}}$ & $\sin\theta = y$ \\
\hline
Coseno & $\cos\theta = \dfrac{\text{cateto adyacente}}{\text{hipotenusa}}$ & $\cos\theta = x$ \\
\hline
Tangente & $\tan\theta = \dfrac{\text{cateto opuesto}}{\text{cateto adyacente}}$ & $\tan\theta = \dfrac{y}{x} = \dfrac{\sin\theta}{\cos\theta}$ \\
\hline
Cosecante & $\csc\theta = \dfrac{\text{hipotenusa}}{\text{cateto opuesto}}$ & $\csc\theta = \dfrac{1}{y} = \dfrac{1}{\sin\theta}$ \\
\hline
Secante & $\sec\theta = \dfrac{\text{hipotenusa}}{\text{cateto adyacente}}$ & $\sec\theta = \dfrac{1}{x} = \dfrac{1}{\cos\theta}$ \\
\hline
Cotangente & $\cot\theta = \dfrac{\text{cateto adyacente}}{\text{cateto opuesto}}$ & $\cot\theta = \dfrac{x}{y} = \dfrac{\cos\theta}{\sin\theta}$ \\
\hline
\end{tabular}
\end{center}

\begin{nota}
Cuando graficamos funciones trigonométricas, usualmente escribimos la variable independiente como $x$ (en lugar de $\theta$) y la variable dependiente como $y$. Así, en lugar de escribir $\sin\theta$, escribimos $y = \sin(x)$.
\end{nota}

\subsection{Ángulos en Radianes}

Aunque hasta ahora hemos trabajado principalmente con grados, cuando graficamos funciones trigonométricas es más conveniente usar \textbf{radianes}. Un radián es la medida del ángulo que subtiende un arco de longitud igual al radio del círculo.

\textbf{Conversión entre grados y radianes:}
\[
\boxed{180° = \pi \text{ radianes}}
\]

De aquí se obtiene:
\[
1° = \frac{\pi}{180} \text{ rad} \quad \text{y} \quad 1 \text{ rad} = \frac{180°}{\pi}
\]

\textbf{Ángulos importantes en radianes:}

\begin{center}
\begin{tabular}{|c|c||c|c|}
\hline
\textbf{Grados} & \textbf{Radianes} & \textbf{Grados} & \textbf{Radianes} \\
\hline
$0°$ & $0$ & $180°$ & $\pi$ \\
\hline
$30°$ & $\dfrac{\pi}{6}$ & $210°$ & $\dfrac{7\pi}{6}$ \\
\hline
$45°$ & $\dfrac{\pi}{4}$ & $225°$ & $\dfrac{5\pi}{4}$ \\
\hline
$60°$ & $\dfrac{\pi}{3}$ & $240°$ & $\dfrac{4\pi}{3}$ \\
\hline
$90°$ & $\dfrac{\pi}{2}$ & $270°$ & $\dfrac{3\pi}{2}$ \\
\hline
$120°$ & $\dfrac{2\pi}{3}$ & $300°$ & $\dfrac{5\pi}{3}$ \\
\hline
$135°$ & $\dfrac{3\pi}{4}$ & $315°$ & $\dfrac{7\pi}{4}$ \\
\hline
$150°$ & $\dfrac{5\pi}{6}$ & $360°$ & $2\pi$ \\
\hline
\end{tabular}
\end{center}

\begin{nota}
De ahora en adelante, cuando grafiquemos funciones trigonométricas, usaremos radianes en el eje horizontal (eje $x$).
\end{nota}

\subsection{Periodicidad}

Una función es \textbf{periódica} si sus valores se repiten a intervalos regulares. El intervalo más pequeño después del cual la función se repite se llama \textbf{período}.

\textbf{Definición formal:} Una función $f$ es periódica con período $P$ si:
\[
f(x + P) = f(x) \quad \text{para todo } x
\]

Las funciones trigonométricas son todas periódicas:

\begin{center}
\renewcommand{\arraystretch}{1.8}
\begin{tabular}{|c|c|}
\hline
\textbf{Función} & \textbf{Período} \\
\hline
$y = \sin(x)$ & $2\pi$ \\
\hline
$y = \cos(x)$ & $2\pi$ \\
\hline
$y = \tan(x)$ & $\pi$ \\
\hline
$y = \cot(x)$ & $\pi$ \\
\hline
$y = \sec(x)$ & $2\pi$ \\
\hline
$y = \csc(x)$ & $2\pi$ \\
\hline
\end{tabular}
\end{center}

\textbf{Interpretación:} Por ejemplo, $\sin(x)$ tiene período $2\pi$ significa que:
\[
\sin(x + 2\pi) = \sin(x), \quad \sin(x + 4\pi) = \sin(x), \quad \sin(x + 6\pi) = \sin(x), \ldots
\]

En otras palabras, la gráfica se repite cada $2\pi$ unidades.

\subsection{Amplitud}

La \textbf{amplitud} de una función periódica es la mitad de la distancia entre su valor máximo y su valor mínimo. Es una medida de ``qué tan alto'' oscila la función.

Para las funciones seno y coseno básicas:
\[
\text{Amplitud} = \frac{\text{máximo} - \text{mínimo}}{2}
\]

\textbf{Ejemplos:}
\begin{itemize}
    \item $y = \sin(x)$ tiene máximo $1$ y mínimo $-1$, entonces: Amplitud $= \dfrac{1-(-1)}{2} = 1$
    \item $y = \cos(x)$ tiene máximo $1$ y mínimo $-1$, entonces: Amplitud $= 1$
    \item $y = 3\sin(x)$ tiene máximo $3$ y mínimo $-3$, entonces: Amplitud $= 3$
\end{itemize}

\begin{nota}
Las funciones tangente, cotangente, secante y cosecante no tienen amplitud definida porque sus valores se extienden desde $-\infty$ hasta $+\infty$.
\end{nota}

\subsection{Dominio y Rango}

El \textbf{dominio} de una función es el conjunto de todos los valores de $x$ para los cuales la función está definida.

El \textbf{rango} es el conjunto de todos los valores posibles de $y$ que la función puede tomar.

\begin{center}
\renewcommand{\arraystretch}{1.8}
\begin{tabular}{|c|c|c|}
\hline
\textbf{Función} & \textbf{Dominio} & \textbf{Rango} \\
\hline
$y = \sin(x)$ & $\mathbb{R}$ (todos los reales) & $[-1, 1]$ \\
\hline
$y = \cos(x)$ & $\mathbb{R}$ & $[-1, 1]$ \\
\hline
$y = \tan(x)$ & $x \neq \dfrac{\pi}{2} + n\pi, \, n \in \mathbb{Z}$ & $\mathbb{R}$ \\
\hline
$y = \cot(x)$ & $x \neq n\pi, \, n \in \mathbb{Z}$ & $\mathbb{R}$ \\
\hline
$y = \sec(x)$ & $x \neq \dfrac{\pi}{2} + n\pi, \, n \in \mathbb{Z}$ & $(-\infty, -1] \cup [1, \infty)$ \\
\hline
$y = \csc(x)$ & $x \neq n\pi, \, n \in \mathbb{Z}$ & $(-\infty, -1] \cup [1, \infty)$ \\
\hline
\end{tabular}
\end{center}

\subsection{Asíntotas}

Una \textbf{asíntota vertical} es una recta vertical $x = a$ donde la función crece o decrece sin límite (tiende a $\pm\infty$) cuando $x$ se acerca a $a$.

Las funciones tangente, cotangente, secante y cosecante tienen asíntotas verticales:

\begin{itemize}
    \item $y = \tan(x)$ tiene asíntotas en $x = \dfrac{\pi}{2} + n\pi$ para $n \in \mathbb{Z}$
    \item $y = \cot(x)$ tiene asíntotas en $x = n\pi$ para $n \in \mathbb{Z}$
    \item $y = \sec(x)$ tiene asíntotas en $x = \dfrac{\pi}{2} + n\pi$ para $n \in \mathbb{Z}$
    \item $y = \csc(x)$ tiene asíntotas en $x = n\pi$ para $n \in \mathbb{Z}$
\end{itemize}

\textbf{¿Por qué existen estas asíntotas?}

\begin{itemize}
    \item $\tan(x) = \dfrac{\sin(x)}{\cos(x)}$ tiene asíntotas donde $\cos(x) = 0$ (división por cero)
    \item $\cot(x) = \dfrac{\cos(x)}{\sin(x)}$ tiene asíntotas donde $\sin(x) = 0$
    \item $\sec(x) = \dfrac{1}{\cos(x)}$ tiene asíntotas donde $\cos(x) = 0$
    \item $\csc(x) = \dfrac{1}{\sin(x)}$ tiene asíntotas donde $\sin(x) = 0$
\end{itemize}

\subsection{Gráfica de la Función Seno: y = sen(x)}

La función seno es la función trigonométrica más fundamental. Su gráfica tiene forma de onda y se repite infinitamente en ambas direcciones.

\textbf{Propiedades principales:}
\begin{itemize}
    \item \textbf{Dominio:} $\mathbb{R}$ (todos los números reales)
    \item \textbf{Rango:} $[-1, 1]$
    \item \textbf{Período:} $2\pi$ (la onda se repite cada $2\pi$ radianes)
    \item \textbf{Amplitud:} $1$
    \item \textbf{Ceros:} $x = n\pi$ para $n \in \mathbb{Z}$ (es decir, $x = 0, \pm\pi, \pm2\pi, \pm3\pi, \ldots$)
    \item \textbf{Máximos:} $y = 1$ cuando $x = \dfrac{\pi}{2} + 2n\pi$
    \item \textbf{Mínimos:} $y = -1$ cuando $x = \dfrac{3\pi}{2} + 2n\pi$
    \item \textbf{Función impar:} $\sin(-x) = -\sin(x)$ (simétrica respecto al origen)
\end{itemize}

\textbf{Valores importantes:}

\begin{center}
\begin{tabular}{|c|c|c|c|c|c|c|c|c|c|}
\hline
$x$ & $0$ & $\dfrac{\pi}{6}$ & $\dfrac{\pi}{4}$ & $\dfrac{\pi}{3}$ & $\dfrac{\pi}{2}$ & $\pi$ & $\dfrac{3\pi}{2}$ & $2\pi$ \\
\hline
$\sin(x)$ & $0$ & $\dfrac{1}{2}$ & $\dfrac{\sqrt{2}}{2}$ & $\dfrac{\sqrt{3}}{2}$ & $1$ & $0$ & $-1$ & $0$ \\
\hline
\end{tabular}
\end{center}

\textbf{Gráfica de $y = \sin(x)$:}

\begin{center}
\begin{tikzpicture}
\begin{axis}[
    width=14cm,
    height=6cm,
    axis lines=middle,
    xlabel={$x$},
    ylabel={$y$},
    xlabel style={at={(axis description cs:1,0.5)},anchor=north},
    ylabel style={at={(axis description cs:0.5,1)},anchor=south},
    xmin=-6.5, xmax=6.5,
    ymin=-1.5, ymax=1.5,
    xtick={-6.28318, -4.7124, -3.14159, -1.5708, 0, 1.5708, 3.14159, 4.7124, 6.28318},
    xticklabels={$-2\pi$, $-\frac{3\pi}{2}$, $-\pi$, $-\frac{\pi}{2}$, $0$, $\frac{\pi}{2}$, $\pi$, $\frac{3\pi}{2}$, $2\pi$},
    ytick={-1, -0.5, 0, 0.5, 1},
    grid=major,
    grid style={dashed, gray!30},
    samples=200,
    domain=-6.5:6.5,
]
\addplot[maincolor, very thick] {sin(deg(x))};
\addplot[accentcolor, only marks, mark=*, mark size=2pt] coordinates {
    (0,0) (1.5708,1) (3.14159,0) (4.7124,-1) (6.28318,0)
    (-1.5708,-1) (-3.14159,0) (-4.7124,1) (-6.28318,0)
};
\end{axis}
\end{tikzpicture}
\end{center}

\textbf{Observaciones importantes:}
\begin{enumerate}
    \item La función oscila suavemente entre $-1$ y $1$
    \item Cruza el eje $x$ en múltiplos de $\pi$
    \item Alcanza su máximo ($1$) en $\dfrac{\pi}{2}, \dfrac{5\pi}{2}, \dfrac{9\pi}{2}, \ldots$
    \item Alcanza su mínimo ($-1$) en $\dfrac{3\pi}{2}, \dfrac{7\pi}{2}, \dfrac{11\pi}{2}, \ldots$
    \item La gráfica se repite exactamente cada $2\pi$ unidades
\end{enumerate}

\subsection{Gráfica de la Función Coseno: y = cos(x)}

La función coseno es muy similar al seno, pero está desplazada horizontalmente. De hecho, $\cos(x) = \sin(x + \frac{\pi}{2})$.

\textbf{Propiedades principales:}
\begin{itemize}
    \item \textbf{Dominio:} $\mathbb{R}$
    \item \textbf{Rango:} $[-1, 1]$
    \item \textbf{Período:} $2\pi$
    \item \textbf{Amplitud:} $1$
    \item \textbf{Ceros:} $x = \dfrac{\pi}{2} + n\pi$ para $n \in \mathbb{Z}$
    \item \textbf{Máximos:} $y = 1$ cuando $x = 2n\pi$
    \item \textbf{Mínimos:} $y = -1$ cuando $x = \pi + 2n\pi$
    \item \textbf{Función par:} $\cos(-x) = \cos(x)$ (simétrica respecto al eje $y$)
\end{itemize}

\textbf{Valores importantes:}

\begin{center}
\begin{tabular}{|c|c|c|c|c|c|c|c|c|c|}
\hline
$x$ & $0$ & $\dfrac{\pi}{6}$ & $\dfrac{\pi}{4}$ & $\dfrac{\pi}{3}$ & $\dfrac{\pi}{2}$ & $\pi$ & $\dfrac{3\pi}{2}$ & $2\pi$ \\
\hline
$\cos(x)$ & $1$ & $\dfrac{\sqrt{3}}{2}$ & $\dfrac{\sqrt{2}}{2}$ & $\dfrac{1}{2}$ & $0$ & $-1$ & $0$ & $1$ \\
\hline
\end{tabular}
\end{center}

\textbf{Gráfica de $y = \cos(x)$:}

\begin{center}
\begin{tikzpicture}
\begin{axis}[
    width=14cm,
    height=6cm,
    axis lines=middle,
    xlabel={$x$},
    ylabel={$y$},
    xlabel style={at={(axis description cs:1,0.5)},anchor=north},
    ylabel style={at={(axis description cs:0.5,1)},anchor=south},
    xmin=-6.5, xmax=6.5,
    ymin=-1.5, ymax=1.5,
    xtick={-6.28318, -4.7124, -3.14159, -1.5708, 0, 1.5708, 3.14159, 4.7124, 6.28318},
    xticklabels={$-2\pi$, $-\frac{3\pi}{2}$, $-\pi$, $-\frac{\pi}{2}$, $0$, $\frac{\pi}{2}$, $\pi$, $\frac{3\pi}{2}$, $2\pi$},
    ytick={-1, -0.5, 0, 0.5, 1},
    grid=major,
    grid style={dashed, gray!30},
    samples=200,
    domain=-6.5:6.5,
]
\addplot[maincolor, very thick] {cos(deg(x))};
\addplot[accentcolor, only marks, mark=*, mark size=2pt] coordinates {
    (0,1) (1.5708,0) (3.14159,-1) (4.7124,0) (6.28318,1)
    (-1.5708,0) (-3.14159,-1) (-4.7124,0) (-6.28318,1)
};
\end{axis}
\end{tikzpicture}
\end{center}

\textbf{Comparación entre seno y coseno:}
\begin{itemize}
    \item Ambas tienen la misma forma de onda
    \item El coseno está desplazado $\dfrac{\pi}{2}$ radianes a la izquierda del seno
    \item El coseno comienza en su máximo ($1$) cuando $x = 0$
    \item El seno comienza en cero cuando $x = 0$
    \item Relación: $\cos(x) = \sin\left(x + \dfrac{\pi}{2}\right)$ y $\sin(x) = \cos\left(x - \dfrac{\pi}{2}\right)$
\end{itemize}

\subsection{Gráfica de la Función Tangente: y = tan(x)}

La función tangente tiene un comportamiento muy diferente al seno y coseno. En lugar de oscilar entre valores fijos, la tangente crece sin límite cerca de sus asíntotas.

\textbf{Propiedades principales:}
\begin{itemize}
    \item \textbf{Dominio:} $\mathbb{R} - \left\{\dfrac{\pi}{2} + n\pi : n \in \mathbb{Z}\right\}$ (todos los reales excepto $\pm\dfrac{\pi}{2}, \pm\dfrac{3\pi}{2}, \ldots$)
    \item \textbf{Rango:} $\mathbb{R}$ (todos los números reales)
    \item \textbf{Período:} $\pi$ (más corto que seno y coseno)
    \item \textbf{Amplitud:} No definida (la función crece sin límite)
    \item \textbf{Ceros:} $x = n\pi$ para $n \in \mathbb{Z}$
    \item \textbf{Asíntotas verticales:} $x = \dfrac{\pi}{2} + n\pi$ para $n \in \mathbb{Z}$
    \item \textbf{Función impar:} $\tan(-x) = -\tan(x)$
\end{itemize}

\textbf{Valores importantes:}

\begin{center}
\begin{tabular}{|c|c|c|c|c|c|c|c|}
\hline
$x$ & $0$ & $\dfrac{\pi}{6}$ & $\dfrac{\pi}{4}$ & $\dfrac{\pi}{3}$ & $\pi$ & $-\dfrac{\pi}{4}$ & $-\dfrac{\pi}{3}$ \\
\hline
$\tan(x)$ & $0$ & $\dfrac{\sqrt{3}}{3}$ & $1$ & $\sqrt{3}$ & $0$ & $-1$ & $-\sqrt{3}$ \\
\hline
\end{tabular}
\end{center}

\textbf{Gráfica de $y = \tan(x)$:}

\begin{center}
\begin{tikzpicture}
\begin{axis}[
    width=14cm,
    height=8cm,
    axis lines=middle,
    xlabel={$x$},
    ylabel={$y$},
    xlabel style={at={(axis description cs:1,0.5)},anchor=north},
    ylabel style={at={(axis description cs:0.5,1)},anchor=south},
    xmin=-5, xmax=5,
    ymin=-5, ymax=5,
    xtick={-4.7124, -3.14159, -1.5708, 0, 1.5708, 3.14159, 4.7124},
    xticklabels={$-\frac{3\pi}{2}$, $-\pi$, $-\frac{\pi}{2}$, $0$, $\frac{\pi}{2}$, $\pi$, $\frac{3\pi}{2}$},
    ytick={-4, -2, 0, 2, 4},
    grid=major,
    grid style={dashed, gray!30},
    samples=200,
    restrict y to domain=-10:10,
]
\addplot[maincolor, very thick, domain=-1.4:1.4] {tan(deg(x))};
\addplot[maincolor, very thick, domain=1.75:4.55] {tan(deg(x))};
\addplot[maincolor, very thick, domain=-4.55:-1.75] {tan(deg(x))};
% Asíntotas
\addplot[red, dashed, thick] coordinates {(1.5708,-10) (1.5708,10)};
\addplot[red, dashed, thick] coordinates {(-1.5708,-10) (-1.5708,10)};
\addplot[red, dashed, thick] coordinates {(4.7124,-10) (4.7124,10)};
\addplot[red, dashed, thick] coordinates {(-4.7124,-10) (-4.7124,10)};
\end{axis}
\end{tikzpicture}
\end{center}

\textbf{Observaciones importantes:}
\begin{enumerate}
    \item La función tiene asíntotas verticales en $x = \pm\dfrac{\pi}{2}, \pm\dfrac{3\pi}{2}, \ldots$
    \item Entre cada par de asíntotas, la función crece continuamente de $-\infty$ a $+\infty$
    \item El período es $\pi$ (la mitad del período de seno y coseno)
    \item Cruza el eje $x$ en los mismos puntos que el seno: $x = 0, \pm\pi, \pm2\pi, \ldots$
    \item Es creciente en todo su dominio (dentro de cada intervalo entre asíntotas)
\end{enumerate}

\subsection{Gráfica de la Función Cotangente: y = cot(x)}

La cotangente es el recíproco de la tangente: $\cot(x) = \dfrac{1}{\tan(x)} = \dfrac{\cos(x)}{\sin(x)}$.

\textbf{Propiedades principales:}
\begin{itemize}
    \item \textbf{Dominio:} $\mathbb{R} - \{n\pi : n \in \mathbb{Z}\}$ (todos los reales excepto $0, \pm\pi, \pm2\pi, \ldots$)
    \item \textbf{Rango:} $\mathbb{R}$
    \item \textbf{Período:} $\pi$
    \item \textbf{Amplitud:} No definida
    \item \textbf{Ceros:} $x = \dfrac{\pi}{2} + n\pi$ para $n \in \mathbb{Z}$
    \item \textbf{Asíntotas verticales:} $x = n\pi$ para $n \in \mathbb{Z}$
    \item \textbf{Función impar:} $\cot(-x) = -\cot(x)$
\end{itemize}

\textbf{Gráfica de $y = \cot(x)$:}

\begin{center}
\begin{tikzpicture}
\begin{axis}[
    width=14cm,
    height=8cm,
    axis lines=middle,
    xlabel={$x$},
    ylabel={$y$},
    xlabel style={at={(axis description cs:1,0.5)},anchor=north},
    ylabel style={at={(axis description cs:0.5,1)},anchor=south},
    xmin=-5, xmax=5,
    ymin=-5, ymax=5,
    xtick={-3.14159, -1.5708, 0, 1.5708, 3.14159},
    xticklabels={$-\pi$, $-\frac{\pi}{2}$, $0$, $\frac{\pi}{2}$, $\pi$},
    ytick={-4, -2, 0, 2, 4},
    grid=major,
    grid style={dashed, gray!30},
    samples=200,
    restrict y to domain=-10:10,
]
\addplot[maincolor, very thick, domain=0.2:3.04] {cot(deg(x))};
\addplot[maincolor, very thick, domain=3.24:6.18] {cot(deg(x))};
\addplot[maincolor, very thick, domain=-3.04:-0.2] {cot(deg(x))};
% Asíntotas
\addplot[red, dashed, thick] coordinates {(0,-10) (0,10)};
\addplot[red, dashed, thick] coordinates {(3.14159,-10) (3.14159,10)};
\addplot[red, dashed, thick] coordinates {(-3.14159,-10) (-3.14159,10)};
\end{axis}
\end{tikzpicture}
\end{center}

\textbf{Diferencias entre tangente y cotangente:}
\begin{itemize}
    \item La tangente es creciente, la cotangente es decreciente
    \item Sus asíntotas están en lugares diferentes
    \item La tangente pasa por el origen, la cotangente tiene una asíntota en el origen
\end{itemize}

\subsection{Gráfica de la Función Secante: y = sec(x)}

La secante es el recíproco del coseno: $\sec(x) = \dfrac{1}{\cos(x)}$.

\textbf{Propiedades principales:}
\begin{itemize}
    \item \textbf{Dominio:} $\mathbb{R} - \left\{\dfrac{\pi}{2} + n\pi : n \in \mathbb{Z}\right\}$
    \item \textbf{Rango:} $(-\infty, -1] \cup [1, \infty)$ (nunca está entre $-1$ y $1$)
    \item \textbf{Período:} $2\pi$
    \item \textbf{Amplitud:} No definida
    \item \textbf{Asíntotas verticales:} $x = \dfrac{\pi}{2} + n\pi$ para $n \in \mathbb{Z}$
    \item \textbf{Función par:} $\sec(-x) = \sec(x)$
\end{itemize}

\textbf{Gráfica de $y = \sec(x)$ junto con $y = \cos(x)$:}

\begin{center}
\begin{tikzpicture}
\begin{axis}[
    width=14cm,
    height=8cm,
    axis lines=middle,
    xlabel={$x$},
    ylabel={$y$},
    xlabel style={at={(axis description cs:1,0.5)},anchor=north},
    ylabel style={at={(axis description cs:0.5,1)},anchor=south},
    xmin=-6.5, xmax=6.5,
    ymin=-4, ymax=4,
    xtick={-6.28318, -4.7124, -3.14159, -1.5708, 0, 1.5708, 3.14159, 4.7124, 6.28318},
    xticklabels={$-2\pi$, $-\frac{3\pi}{2}$, $-\pi$, $-\frac{\pi}{2}$, $0$, $\frac{\pi}{2}$, $\pi$, $\frac{3\pi}{2}$, $2\pi$},
    ytick={-3, -2, -1, 0, 1, 2, 3},
    grid=major,
    grid style={dashed, gray!30},
    samples=200,
    restrict y to domain=-10:10,
    legend pos=north east,
]
% Coseno (para referencia)
\addplot[blue!50, dashed, thick] {cos(deg(x))};
% Secante
\addplot[maincolor, very thick, domain=-1.4:1.4] {1/cos(deg(x))};
\addplot[maincolor, very thick, domain=1.75:4.55] {1/cos(deg(x))};
\addplot[maincolor, very thick, domain=4.85:7.8] {1/cos(deg(x))};
\addplot[maincolor, very thick, domain=-4.55:-1.75] {1/cos(deg(x))};
\addplot[maincolor, very thick, domain=-7.8:-4.85] {1/cos(deg(x))};
% Asíntotas
\addplot[red, dashed, thick] coordinates {(1.5708,-10) (1.5708,10)};
\addplot[red, dashed, thick] coordinates {(-1.5708,-10) (-1.5708,10)};
\addplot[red, dashed, thick] coordinates {(4.7124,-10) (4.7124,10)};
\addplot[red, dashed, thick] coordinates {(-4.7124,-10) (-4.7124,10)};
\legend{$y=\cos(x)$, $y=\sec(x)$}
\end{axis}
\end{tikzpicture}
\end{center}

\textbf{Observaciones importantes:}
\begin{enumerate}
    \item La secante tiene forma de ``U'' invertida y normal, alternadas
    \item Tiene asíntotas exactamente donde el coseno es cero
    \item Cuando $|\cos(x)|$ es pequeño, $|\sec(x)|$ es grande
    \item Los valores mínimos locales de secante son $1$ y los máximos locales son $-1$
    \item La secante nunca toma valores entre $-1$ y $1$
\end{enumerate}

\subsection{Gráfica de la Función Cosecante: y = csc(x)}

La cosecante es el recíproco del seno: $\csc(x) = \dfrac{1}{\sin(x)}$.

\textbf{Propiedades principales:}
\begin{itemize}
    \item \textbf{Dominio:} $\mathbb{R} - \{n\pi : n \in \mathbb{Z}\}$
    \item \textbf{Rango:} $(-\infty, -1] \cup [1, \infty)$
    \item \textbf{Período:} $2\pi$
    \item \textbf{Amplitud:} No definida
    \item \textbf{Asíntotas verticales:} $x = n\pi$ para $n \in \mathbb{Z}$
    \item \textbf{Función impar:} $\csc(-x) = -\csc(x)$
\end{itemize}

\textbf{Gráfica de $y = \csc(x)$ junto con $y = \sin(x)$:}

\begin{center}
\begin{tikzpicture}
\begin{axis}[
    width=14cm,
    height=8cm,
    axis lines=middle,
    xlabel={$x$},
    ylabel={$y$},
    xlabel style={at={(axis description cs:1,0.5)},anchor=north},
    ylabel style={at={(axis description cs:0.5,1)},anchor=south},
    xmin=-6.5, xmax=6.5,
    ymin=-4, ymax=4,
    xtick={-6.28318, -4.7124, -3.14159, -1.5708, 0, 1.5708, 3.14159, 4.7124, 6.28318},
    xticklabels={$-2\pi$, $-\frac{3\pi}{2}$, $-\pi$, $-\frac{\pi}{2}$, $0$, $\frac{\pi}{2}$, $\pi$, $\frac{3\pi}{2}$, $2\pi$},
    ytick={-3, -2, -1, 0, 1, 2, 3},
    grid=major,
    grid style={dashed, gray!30},
    samples=200,
    restrict y to domain=-10:10,
    legend pos=north east,
]
% Seno (para referencia)
\addplot[blue!50, dashed, thick] {sin(deg(x))};
% Cosecante
\addplot[maincolor, very thick, domain=0.3:2.84] {1/sin(deg(x))};
\addplot[maincolor, very thick, domain=3.44:6.0] {1/sin(deg(x))};
\addplot[maincolor, very thick, domain=-2.84:-0.3] {1/sin(deg(x))};
\addplot[maincolor, very thick, domain=-6.0:-3.44] {1/sin(deg(x))};
% Asíntotas
\addplot[red, dashed, thick] coordinates {(0,-10) (0,10)};
\addplot[red, dashed, thick] coordinates {(3.14159,-10) (3.14159,10)};
\addplot[red, dashed, thick] coordinates {(-3.14159,-10) (-3.14159,10)};
\addplot[red, dashed, thick] coordinates {(6.28318,-10) (6.28318,10)};
\addplot[red, dashed, thick] coordinates {(-6.28318,-10) (-6.28318,10)};
\legend{$y=\sin(x)$, $y=\csc(x)$}
\end{axis}
\end{tikzpicture}
\end{center}

\textbf{Observaciones importantes:}
\begin{enumerate}
    \item La cosecante tiene forma similar a la secante, pero desplazada
    \item Tiene asíntotas exactamente donde el seno es cero
    \item Los puntos donde $|\sin(x)| = 1$ corresponden a puntos donde $|\csc(x)| = 1$
    \item Como el seno, la cosecante es una función impar
\end{enumerate}

\newpage

%INSERTAR_EJEMPLOS_AQUI%

\newpage

%INSERTAR_EJERCICIOS_PROPUESTOS_AQUI%

\newpage

%INSERTAR_SOLUCIONES_AQUI%

\newpage

%INSERTAR_EJERCICIOS_INVERSOS_AQUI%

\newpage

%INSERTAR_SOLUCIONES_INVERSOS_AQUI%

\newpage

\section{Conclusión}

¡Felicitaciones! Has completado esta primera parte de la guía sobre gráficas de funciones trigonométricas. Ahora tienes las herramientas fundamentales para:

\begin{itemize}
    \item Visualizar y entender las seis funciones trigonométricas como gráficas en el plano cartesiano
    \item Identificar características clave: período, amplitud, dominio, rango, asíntotas
    \item Reconocer patrones periódicos y ondulatorios
    \item Relacionar las definiciones del círculo unitario con las representaciones gráficas
    \item Distinguir entre las funciones básicas (seno, coseno) y sus recíprocas (secante, cosecante)
    \item Comprender el comportamiento de tangente y cotangente con sus asíntotas
\end{itemize}

\subsection*{Resumen de Conceptos Clave}

\begin{tcolorbox}[enhanced,colback=maincolor!10,colframe=maincolor,title=Características de las Funciones Trigonométricas]

\textbf{Seno y Coseno:}
\begin{itemize}
    \item Rango: $[-1, 1]$
    \item Período: $2\pi$
    \item Sin asíntotas
    \item Formas de onda suaves
\end{itemize}

\textbf{Tangente y Cotangente:}
\begin{itemize}
    \item Rango: $\mathbb{R}$ (todos los reales)
    \item Período: $\pi$
    \item Con asíntotas verticales
    \item Crecen/decrecen sin límite
\end{itemize}

\textbf{Secante y Cosecante:}
\begin{itemize}
    \item Rango: $(-\infty, -1] \cup [1, \infty)$
    \item Período: $2\pi$
    \item Con asíntotas verticales
    \item Formas de ``U'' invertidas
\end{itemize}

\textbf{Periodicidad:}
\[
\sin(x + 2\pi) = \sin(x), \quad \cos(x + 2\pi) = \cos(x), \quad \tan(x + \pi) = \tan(x)
\]

\textbf{Simetría:}
\[
\text{Pares: } \cos(-x) = \cos(x), \, \sec(-x) = \sec(x)
\]
\[
\text{Impares: } \sin(-x) = -\sin(x), \, \tan(-x) = -\tan(x), \, \csc(-x) = -\csc(x), \, \cot(-x) = -\cot(x)
\]
\end{tcolorbox}

\subsection*{Fórmulas Clave}

\begin{tcolorbox}[enhanced,colback=accentcolor!10,colframe=accentcolor,title=Relaciones Entre Funciones]
\textbf{Definiciones recíprocas:}
\[
\csc(x) = \frac{1}{\sin(x)}, \quad \sec(x) = \frac{1}{\cos(x)}, \quad \cot(x) = \frac{1}{\tan(x)}
\]

\textbf{Definiciones cociente:}
\[
\tan(x) = \frac{\sin(x)}{\cos(x)}, \quad \cot(x) = \frac{\cos(x)}{\sin(x)}
\]

\textbf{Identidad pitagórica:}
\[
\sin^2(x) + \cos^2(x) = 1
\]

\textbf{Conversión radianes-grados:}
\[
180° = \pi \text{ rad}, \quad 1° = \frac{\pi}{180} \text{ rad}, \quad 1 \text{ rad} = \frac{180°}{\pi}
\]
\end{tcolorbox}

\subsection*{Aplicaciones Prácticas Recordatorio}

Las gráficas que has estudiado son fundamentales para:

\begin{itemize}
    \item \textbf{Ondas sonoras:} La voz humana, instrumentos musicales, y audio digital se modelan con funciones seno
    \item \textbf{Señales eléctricas:} La corriente AC en tu casa sigue un patrón senoidal
    \item \textbf{Movimiento armónico simple:} Péndulos, resortes, y vibraciones mecánicas
    \item \textbf{Mareas:} El nivel del mar varía siguiendo funciones coseno
    \item \textbf{Análisis de fenómenos periódicos:} Ciclos naturales, patrones climáticos
    \item \textbf{Ingeniería de señales:} Telecomunicaciones, WiFi, radio, procesamiento de señales
\end{itemize}

\subsection*{Próximos Pasos}

En las siguientes partes de esta guía estudiaremos:

\begin{enumerate}
    \item \textbf{Transformaciones de funciones trigonométricas:} Desplazamientos horizontales y verticales, cambios de amplitud y período
    \item \textbf{Ejemplos resueltos:} Problemas paso a paso que integran todos los conceptos
    \item \textbf{Ejercicios propuestos:} Práctica guiada con soluciones detalladas
    \item \textbf{Aplicaciones avanzadas:} Modelado de fenómenos reales usando funciones trigonométricas
\end{enumerate}

\subsection*{Consejos para el Éxito}

\begin{enumerate}
    \item \textbf{Practica dibujar las gráficas:} La mejor manera de entender las funciones trigonométricas es dibujarlas a mano varias veces
    \item \textbf{Memoriza los valores clave:} Conocer $\sin(\frac{\pi}{6})$, $\cos(\frac{\pi}{4})$, etc., te ahorrará mucho tiempo
    \item \textbf{Identifica patrones:} Las funciones trigonométricas tienen simetrías y periodicidades que puedes aprovechar
    \item \textbf{Relaciona con el círculo unitario:} Siempre que tengas dudas, vuelve a la definición en el círculo unitario
    \item \textbf{Usa la tecnología:} Graficadores como Desmos o GeoGebra te ayudarán a visualizar conceptos difíciles
    \item \textbf{Conecta con el mundo real:} Busca ejemplos de ondas y oscilaciones a tu alrededor
\end{enumerate}

\vspace{1cm}

\begin{center}
\textit{``Las ondas son el lenguaje del universo.''} \\
\textit{``Aprender a leerlas es aprender a entender la naturaleza.''}
\end{center}

\vspace{0.5cm}

¡Sigue adelante con las siguientes partes de esta guía. El viaje apenas comienza!

\end{document}
