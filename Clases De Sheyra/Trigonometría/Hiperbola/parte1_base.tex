% !TEX program = lualatex
\documentclass[12pt,a4paper,twoside]{article}
\usepackage{fontspec}
\usepackage[spanish,es-nodecimaldot]{babel}
\usepackage{amsmath,amssymb}
\usepackage[margin=2.5cm]{geometry}
\usepackage[table]{xcolor}
\usepackage{tikz,pgfplots}
\usetikzlibrary{calc,arrows.meta,babel}
\usepackage{multicol}
\usepackage{enumitem}
\pgfplotsset{compat=1.18}
\definecolor{maincolor}{RGB}{26,35,126}
\definecolor{accentcolor}{RGB}{255,87,34}

% Configuración de títulos y formato
\usepackage{titlesec}
\titleformat{\section}{\Large\bfseries\color{maincolor}}{\thesection}{1em}{}
\titleformat{\subsection}{\large\bfseries\color{accentcolor}}{\thesubsection}{1em}{}

% Configuración de cajas para ejemplos
\usepackage{tcolorbox}
\tcbuselibrary{skins,breakable}

\usepackage{fancyhdr}

\pagestyle{fancy}
\fancyhf{}
\fancyhead[LE]{\small\textcolor{maincolor}{\thepage \quad La Hipérbola}}
\fancyhead[RO]{\small\textcolor{maincolor}{La Hipérbola \quad \thepage}}
\fancyhead[LO]{\small\textcolor{maincolor}{Grado 10 - Trigonometría}}
\fancyhead[RE]{\small\textcolor{maincolor}{Prof. Toribio De J Arrieta F}}
\fancyfoot[C]{}
\renewcommand{\headrulewidth}{0.5pt}
\renewcommand{\footrulewidth}{0pt}
\setlength{\headheight}{14pt}

\newtcolorbox{definicion}[1][]{
  enhanced,
  breakable,
  colback=maincolor!5,
  colframe=maincolor,
  fonttitle=\bfseries,
  title=Definición,
  #1
}

\newtcolorbox{ejemplo}[1][]{
  enhanced,
  breakable,
  colback=maincolor!5,
  colframe=maincolor,
  fonttitle=\bfseries,
  title=Ejemplo Resuelto,
  #1
}

\newtcolorbox{ejercicio}[1][]{
  enhanced,
  breakable,
  colback=accentcolor!5,
  colframe=accentcolor,
  fonttitle=\bfseries,
  title=Ejercicio,
  #1
}

\newtcolorbox{solucion}[1][]{
  enhanced,
  breakable,
  colback=green!5,
  colframe=green!60!black,
  fonttitle=\bfseries,
  title=Solución,
  #1
}

\newtcolorbox{nota}[1][]{
  enhanced,
  colback=yellow!10,
  colframe=orange!80!black,
  fonttitle=\bfseries,
  title=Nota Importante,
  #1
}

% Título
\title{\textbf{\Huge GEOMETRIA ANALITICA}\\[0.5cm]
\Large La Hipérbola}
\author{Prof: Toribio De J Arrieta F\\
\textit{La Pruebita}\\
Grado 10}
\date{\today}

\begin{document}

\maketitle

\tableofcontents
\newpage

\section{Introducción}

¡Bienvenidos al fascinante mundo de la hipérbola! Si alguna vez te has preguntado cómo funcionan los sistemas de GPS, cómo los astrónomos rastrean cometas, o cómo los radares detectan objetos, la respuesta está en las hipérbolas.

La hipérbola es una de las curvas más interesantes en geometría analítica. Aunque su nombre puede sonar complicado, verás que es una curva con propiedades geométricas bellísimas y aplicaciones super prácticas en el mundo real.

\subsection*{¿Por qué son importantes las hipérbolas?}

Las hipérbolas aparecen en muchos lugares inesperados:

\begin{itemize}[leftmargin=1.5cm]
    \item \textbf{Navegación GPS:} Los sistemas de posicionamiento global usan diferencias de tiempo en señales que forman hipérbolas para determinar tu ubicación exacta.
    \item \textbf{Astronomía:} Las trayectorias de algunos cometas y objetos espaciales que pasan cerca del Sol describen hipérbolas.
    \item \textbf{Sistemas de radar:} Los radares utilizan la diferencia de tiempos de llegada de señales, que geometricamente forman hipérbolas, para localizar objetos.
    \item \textbf{Arquitectura:} Algunas torres de enfriamiento de plantas nucleares tienen forma hiperbólica por su resistencia estructural.
\end{itemize}

\subsection*{¿Qué aprenderás en esta guía?}

En esta guía vas a descubrir:

\begin{enumerate}[leftmargin=1.5cm]
    \item Qué es una hipérbola y cómo se construye
    \item Los elementos fundamentales: centro, focos, vértices, asíntotas
    \item Las ecuaciones canónicas con centro en (0,0) y en (h,k)
    \item Cómo trabajar con la ecuación general de segundo grado
    \item Resolver problemas prácticos con hipérbolas
\end{enumerate}

Vamos a usar un lenguaje sencillo, muchos ejemplos resueltos paso a paso, y gráficas que te ayudarán a visualizar cada concepto. ¡Prepárate para dominar las hipérbolas!

\newpage

\section{Conceptos Fundamentales}

\subsection{Definición Geométrica de la Hipérbola}

\begin{definicion}[title={La Hipérbola}]
Una \textbf{hipérbola} es el lugar geométrico de todos los puntos P en el plano tales que la \textbf{diferencia} de las distancias desde P hasta dos puntos fijos llamados focos es constante.

Matemáticamente: Si $F_1$ y $F_2$ son los focos, y P es cualquier punto de la hipérbola, entonces:
\[
|d(P,F_1) - d(P,F_2)| = 2a
\]
donde $2a$ es una constante positiva.
\end{definicion}

\begin{nota}
La diferencia clave con la elipse: en la elipse es la \textbf{suma} de distancias, en la hipérbola es la \textbf{diferencia}.
\end{nota}

\subsection{Elementos de la Hipérbola}

Toda hipérbola tiene los siguientes elementos fundamentales:

\begin{itemize}[leftmargin=2cm]
    \item \textbf{Centro (C):} Punto medio entre los dos focos.
    \item \textbf{Focos ($F_1$ y $F_2$):} Los dos puntos fijos que definen la hipérbola.
    \item \textbf{Vértices ($V_1$ y $V_2$):} Puntos donde la hipérbola interseca el eje transverso.
    \item \textbf{Eje transverso:} Segmento de línea que une los vértices, con longitud $2a$.
    \item \textbf{Eje conjugado:} Segmento perpendicular al eje transverso en el centro, con longitud $2b$.
    \item \textbf{Asíntotas:} Dos rectas que la hipérbola se acerca infinitamente pero nunca toca.
    \item \textbf{Distancia focal:} Distancia desde el centro hasta cada foco, denotada por $c$.
\end{itemize}

\subsection{Relación Fundamental}

En toda hipérbola se cumple una relación pitagórica fundamental:

\begin{nota}[title={Relación Fundamental de la Hipérbola}]
\[
c^2 = a^2 + b^2
\]
donde:
\begin{itemize}
    \item $a$ = semidistancia del eje transverso
    \item $b$ = semidistancia del eje conjugado
    \item $c$ = distancia del centro a cada foco
\end{itemize}

Nota: En la hipérbola $c > a$ (a diferencia de la elipse donde $c < a$).
\end{nota}

\subsection{Ecuación Canónica con Centro en el Origen}

Cuando el centro de la hipérbola está en el origen $(0,0)$, existen dos formas canónicas:

\subsubsection*{Hipérbola Horizontal (eje transverso sobre el eje x)}

\[
\frac{x^2}{a^2} - \frac{y^2}{b^2} = 1
\]

Características:
\begin{itemize}
    \item Centro: $C(0,0)$
    \item Vértices: $V_1(-a,0)$ y $V_2(a,0)$
    \item Focos: $F_1(-c,0)$ y $F_2(c,0)$ donde $c^2 = a^2 + b^2$
    \item Asíntotas: $y = \pm \frac{b}{a}x$
\end{itemize}

\subsubsection*{Hipérbola Vertical (eje transverso sobre el eje y)}

\[
\frac{y^2}{a^2} - \frac{x^2}{b^2} = 1
\]

Características:
\begin{itemize}
    \item Centro: $C(0,0)$
    \item Vértices: $V_1(0,-a)$ y $V_2(0,a)$
    \item Focos: $F_1(0,-c)$ y $F_2(0,c)$ donde $c^2 = a^2 + b^2$
    \item Asíntotas: $y = \pm \frac{a}{b}x$
\end{itemize}

\begin{nota}
Para identificar rápidamente:
\begin{itemize}
    \item Si el término positivo es $x^2$, la hipérbola es \textbf{horizontal}.
    \item Si el término positivo es $y^2$, la hipérbola es \textbf{vertical}.
\end{itemize}
\end{nota}

\newpage

\subsection{Gráfica Ilustrativa}

\begin{center}
\begin{tikzpicture}
\begin{axis}[
    width=0.9\textwidth,
    height=0.7\textwidth,
    axis equal image,
    axis lines=middle,
    xlabel={$x$},
    ylabel={$y$},
    xmin=-8, xmax=8,
    ymin=-6, ymax=6,
    xtick={-6,-4,-2,0,2,4,6},
    ytick={-4,-2,0,2,4},
    grid=major,
    grid style={dashed,gray!30},
    legend pos=north east,
]

% Hipérbola (a=3, b=2, horizontal)
\addplot[maincolor, very thick, smooth, samples=200, domain=3:7]
    ({x}, {2*sqrt((x^2/9)-1)});
\addplot[maincolor, very thick, smooth, samples=200, domain=3:7]
    ({x}, {-2*sqrt((x^2/9)-1)});
\addplot[maincolor, very thick, smooth, samples=200, domain=-7:-3]
    ({x}, {2*sqrt((x^2/9)-1)});
\addplot[maincolor, very thick, smooth, samples=200, domain=-7:-3]
    ({x}, {-2*sqrt((x^2/9)-1)});

% Asíntotas
\addplot[red, dashed, thick, domain=-7:7] {(2/3)*x};
\addplot[red, dashed, thick, domain=-7:7] {-(2/3)*x};

% Focos (c^2 = 9 + 4 = 13, c ≈ 3.6)
\addplot[mark=*, mark size=3pt, accentcolor] coordinates {(-3.606,0) (3.606,0)};
\node at (axis cs:-3.606,-0.7) {\small $F_1$};
\node at (axis cs:3.606,-0.7) {\small $F_2$};

% Vértices
\addplot[mark=*, mark size=3pt, maincolor] coordinates {(-3,0) (3,0)};
\node at (axis cs:-3,-0.7) {\small $V_1$};
\node at (axis cs:3,-0.7) {\small $V_2$};

% Centro
\addplot[mark=*, mark size=2pt, black] coordinates {(0,0)};
\node at (axis cs:0.3,-0.5) {\small $C$};

\legend{Hipérbola, Asíntotas}

\end{axis}
\end{tikzpicture}
\end{center}

\subsection{Ecuación Canónica con Centro en (h,k)}

Cuando el centro de la hipérbola está en un punto $(h,k)$ cualquiera:

\subsubsection*{Hipérbola Horizontal}

\[
\frac{(x-h)^2}{a^2} - \frac{(y-k)^2}{b^2} = 1
\]

Características:
\begin{itemize}
    \item Centro: $C(h,k)$
    \item Vértices: $V_1(h-a,k)$ y $V_2(h+a,k)$
    \item Focos: $F_1(h-c,k)$ y $F_2(h+c,k)$
    \item Asíntotas: $y - k = \pm \frac{b}{a}(x - h)$
\end{itemize}

\subsubsection*{Hipérbola Vertical}

\[
\frac{(y-k)^2}{a^2} - \frac{(x-h)^2}{b^2} = 1
\]

Características:
\begin{itemize}
    \item Centro: $C(h,k)$
    \item Vértices: $V_1(h,k-a)$ y $V_2(h,k+a)$
    \item Focos: $F_1(h,k-c)$ y $F_2(h,k+c)$
    \item Asíntotas: $y - k = \pm \frac{a}{b}(x - h)$
\end{itemize}

\subsection{Excentricidad}

La excentricidad mide qué tan "abierta" es la hipérbola:

\[
e = \frac{c}{a}
\]

En una hipérbola siempre: $e > 1$

\begin{itemize}
    \item Si $e$ está cerca de 1, la hipérbola es más "cerrada".
    \item Si $e$ es muy grande, la hipérbola es muy "abierta".
\end{itemize}

\subsection{Ecuación General de Segundo Grado}

La forma general de una hipérbola es:

\[
Ax^2 + Cy^2 + Dx + Ey + F = 0
\]

donde $A$ y $C$ tienen \textbf{signos opuestos} (esto distingue a la hipérbola de la elipse).

Para convertir de forma general a canónica, se usa \textbf{completación de cuadrados}.

\newpage

\section{Tabla de Referencia Rápida}

\begin{center}
\begin{tabular}{|p{4cm}|p{5cm}|p{5cm}|}
\hline
\rowcolor{maincolor!20}
\textbf{Elemento} & \textbf{Hipérbola Horizontal} & \textbf{Hipérbola Vertical} \\
\hline
Ecuación canónica & $\displaystyle\frac{(x-h)^2}{a^2} - \frac{(y-k)^2}{b^2} = 1$ & $\displaystyle\frac{(y-k)^2}{a^2} - \frac{(x-h)^2}{b^2} = 1$ \\
\hline
Centro & $(h,k)$ & $(h,k)$ \\
\hline
Vértices & $(h \pm a, k)$ & $(h, k \pm a)$ \\
\hline
Focos & $(h \pm c, k)$ & $(h, k \pm c)$ \\
\hline
Relación & $c^2 = a^2 + b^2$ & $c^2 = a^2 + b^2$ \\
\hline
Asíntotas & $y - k = \pm \frac{b}{a}(x-h)$ & $y - k = \pm \frac{a}{b}(x-h)$ \\
\hline
Excentricidad & $e = \frac{c}{a}$ donde $e > 1$ & $e = \frac{c}{a}$ donde $e > 1$ \\
\hline
\end{tabular}
\end{center}

%INSERTAR_EJEMPLOS_AQUI%

%INSERTAR_EJERCICIOS_AQUI%

\newpage

\section{Conclusión}

¡Felicitaciones! Has completado el estudio de la hipérbola, una de las curvas más fascinantes de la geometría analítica.

\subsection*{Resumen de Conceptos Clave}

Recuerda siempre:

\begin{enumerate}[leftmargin=2cm]
    \item La hipérbola se define por la \textbf{diferencia constante} de distancias a dos focos.
    \item La relación fundamental es: $c^2 = a^2 + b^2$ (donde $c > a$).
    \item El término positivo en la ecuación indica la orientación (horizontal o vertical).
    \item Las asíntotas son guías visuales que la hipérbola nunca toca.
    \item La excentricidad $e = c/a$ siempre es mayor que 1.
\end{enumerate}

\subsection*{Fórmulas Importantes}

\begin{tcolorbox}[colback=maincolor!10, colframe=maincolor, title=Fórmulas Clave]
\begin{itemize}
    \item Relación fundamental: $c^2 = a^2 + b^2$
    \item Excentricidad: $e = \dfrac{c}{a}$ donde $e > 1$
    \item Ecuación canónica horizontal: $\dfrac{(x-h)^2}{a^2} - \dfrac{(y-k)^2}{b^2} = 1$
    \item Ecuación canónica vertical: $\dfrac{(y-k)^2}{a^2} - \dfrac{(x-h)^2}{b^2} = 1$
    \item Asíntotas (horizontal): $y - k = \pm \dfrac{b}{a}(x - h)$
    \item Asíntotas (vertical): $y - k = \pm \dfrac{a}{b}(x - h)$
\end{itemize}
\end{tcolorbox}

\subsection*{Consejos para el Éxito}

\begin{itemize}[leftmargin=2cm]
    \item Siempre identifica primero qué término es positivo para saber la orientación.
    \item Dibuja un rectángulo auxiliar de dimensiones $2a \times 2b$ para trazar las asíntotas.
    \item Verifica que $c^2 = a^2 + b^2$ después de encontrar $a$, $b$ y $c$.
    \item Practica la completación de cuadrados, es esencial para trabajar con la forma general.
    \item Visualiza siempre con gráficas, te ayudará a entender mejor cada problema.
\end{itemize}

\subsection*{Aplicaciones en el Mundo Real}

La hipérbola no es solo teoría matemática, es una herramienta práctica que:

\begin{itemize}[leftmargin=2cm]
    \item Permite a los sistemas GPS determinar posiciones con precisión
    \item Ayuda a los astrónomos a predecir trayectorias de cometas
    \item Se usa en sistemas de radar para localizar objetos
    \item Aparece en estructuras arquitectónicas eficientes
\end{itemize}

\subsection*{Sigue Practicando}

La matemática se aprende haciendo. Resuelve todos los ejercicios propuestos, verifica tus soluciones, y no temas cometer errores, ¡son parte del aprendizaje!

¡Éxito en tu camino matemático!

\end{document}
