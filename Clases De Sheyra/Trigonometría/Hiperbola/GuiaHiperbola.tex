% !TEX program = lualatex
\documentclass[12pt,a4paper,twoside]{article}
\usepackage{fontspec}
\usepackage[spanish,es-nodecimaldot]{babel}
\usepackage{amsmath,amssymb}
\usepackage[margin=2.5cm]{geometry}
\usepackage{xcolor}
\usepackage{tikz,pgfplots}
\usetikzlibrary{calc,arrows.meta,babel}
\usepackage{multicol}
\usepackage{enumitem}
\pgfplotsset{compat=1.18}
\definecolor{maincolor}{RGB}{26,35,126}
\definecolor{accentcolor}{RGB}{255,87,34}

% Configuración de títulos y formato
\usepackage{titlesec}
\titleformat{\section}{\Large\bfseries\color{maincolor}}{\thesection}{1em}{}
\titleformat{\subsection}{\large\bfseries\color{accentcolor}}{\thesubsection}{1em}{}

% Configuración de cajas para ejemplos
\usepackage{tcolorbox}
\tcbuselibrary{skins,breakable}

\usepackage{fancyhdr}

\pagestyle{fancy}
\fancyhf{}
\fancyhead[LE]{\small\textcolor{maincolor}{\thepage \quad La Hipérbola}}
\fancyhead[RO]{\small\textcolor{maincolor}{La Hipérbola \quad \thepage}}
\fancyhead[LO]{\small\textcolor{maincolor}{Grado 10 - Trigonometría}}
\fancyhead[RE]{\small\textcolor{maincolor}{Prof. Toribio De J Arrieta F}}
\fancyfoot[C]{}
\renewcommand{\headrulewidth}{0.5pt}
\renewcommand{\footrulewidth}{0pt}
\setlength{\headheight}{14pt}

\newtcolorbox{definicion}[1][]{
  enhanced,
  breakable,
  colback=maincolor!5,
  colframe=maincolor,
  fonttitle=\bfseries,
  title=Definición,
  #1
}

\newtcolorbox{ejemplo}[1][]{
  enhanced,
  breakable,
  colback=green!5,
  colframe=green!60!black,
  fonttitle=\bfseries,
  title=Ejemplo Resuelto,
  #1
}

\newtcolorbox{ejercicio}[1][]{
  enhanced,
  breakable,
  colback=accentcolor!5,
  colframe=accentcolor,
  fonttitle=\bfseries,
  title=Ejercicio,
  #1
}

\newtcolorbox{solucion}[1][]{
  enhanced,
  breakable,
  colback=green!5,
  colframe=green!60!black,
  fonttitle=\bfseries,
  title=Solución,
  #1
}

\newtcolorbox{nota}[1][]{
  enhanced,
  breakable,
  colback=yellow!10,
  colframe=orange!80!black,
  fonttitle=\bfseries,
  title=Nota Importante,
  #1
}

% Título
\title{\textbf{\Huge GEOMETRÍA ANALÍTICA}\\[0.5cm]
\Large La Hipérbola}
\author{Prof. Toribio De J Arrieta F\\
\textit{La Pruebita}\\
Grado 10}
\date{\today}

\begin{document}

\maketitle

\tableofcontents
\newpage

\section{Introducción}

¡Bienvenidos al fascinante mundo de las hipérbolas! Si alguna vez te has preguntado cómo funciona un sistema de navegación GPS, cómo los astrónomos calculan las trayectorias de los cometas, o cómo se diseñan los sistemas de radar, la respuesta está en las hipérbolas.

La hipérbola es una de las cuatro cónicas (junto con la circunferencia, la elipse y la parábola) y tiene propiedades únicas que la hacen indispensable en ciencia y tecnología. A diferencia de la elipse, donde sumamos distancias, en la hipérbola restamos distancias, lo que le da características muy especiales.

\subsection*{¿Qué es una hipérbola?}

Una hipérbola es el lugar geométrico de todos los puntos en el plano cuya diferencia de distancias a dos puntos fijos (llamados focos) es constante. Imagínate que tienes dos estaciones de radio en diferentes posiciones: un receptor puede determinar su posición midiendo la diferencia en el tiempo que tardan las señales en llegar desde cada estación. ¡Eso traza una hipérbola!

\subsection*{¿Por qué son importantes?}

Las hipérbolas aparecen en muchas aplicaciones prácticas:

\begin{itemize}
\item \textbf{Navegación GPS:} Los sistemas de posicionamiento global usan diferencias de tiempo entre señales de satélites, creando hipérbolas que se intersectan para determinar tu ubicación exacta.

\item \textbf{Astronomía:} Las órbitas de algunos cometas que pasan cerca del Sol y nunca regresan siguen trayectorias hiperbólicas. También las trayectorias de objetos expulsados de sistemas estelares.

\item \textbf{Sistemas de radar:} Los radares militares y de aviación usan propiedades de las hipérbolas para triangular posiciones de aviones y objetos.

\item \textbf{Diseño de telescopios:} Los espejos hiperbólicos se usan junto con espejos parabólicos en telescopios reflectores de alto rendimiento, como el telescopio Hubble.

\item \textbf{Arquitectura:} Las torres de enfriamiento de las plantas nucleares tienen forma hiperbólica porque esta geometría proporciona máxima resistencia con mínimo material.

\item \textbf{Física de partículas:} Las partículas cargadas que pasan cerca de núcleos atómicos siguen trayectorias hiperbólicas debido a la repulsión electrostática.
\end{itemize}

En esta guía, vamos a explorar la hipérbola desde su definición geométrica hasta sus ecuaciones algebraicas, y veremos cómo aplicar estos conceptos para resolver problemas del mundo real. ¡Prepárate para descubrir una de las curvas más interesantes de la geometría analítica!

\newpage

\section{Conceptos Fundamentales}

\subsection{Definición geométrica de la hipérbola}

\begin{definicion}
Una \textbf{hipérbola} es el lugar geométrico de todos los puntos $P$ en el plano tales que la \textcolor{red}{diferencia} de sus distancias a dos puntos fijos $F_1$ y $F_2$ (llamados focos) es constante.

Matemáticamente:
\[
|d(P,F_1) - d(P,F_2)| = 2a
\]
donde $2a$ es una constante positiva y $a$ es la distancia del centro a cada vértice.
\end{definicion}

\subsection{Elementos de la hipérbola}

\begin{multicols}{2}
\begin{itemize}
\item \textbf{Centro $(C)$:} Punto medio entre los focos
\item \textbf{Focos $(F_1, F_2)$:} Puntos fijos de referencia
\item \textbf{Vértices $(V_1, V_2)$:} Puntos de intersección con el eje focal
\item \textbf{Eje focal:} Recta que pasa por los focos
\item \textbf{Eje conjugado:} Recta perpendicular al eje focal que pasa por el centro
\item \textbf{Distancia focal $(2c)$:} Distancia entre los focos
\item \textbf{Eje transversal $(2a)$:} Distancia entre vértices
\item \textbf{Eje conjugado $(2b)$:} Longitud perpendicular al eje focal
\item \textbf{Asíntotas:} Rectas a las que se aproxima la hipérbola en el infinito
\item \textbf{Excentricidad $(e)$:} Medida de la apertura de la hipérbola, $e = \frac{c}{a}$ donde $e > 1$
\end{itemize}
\end{multicols}

\begin{center}
\begin{tikzpicture}
\begin{axis}[
    width=0.90\textwidth, height=0.60\textwidth,
    axis equal image,
    axis lines=middle,
    xlabel={$x$}, ylabel={$y$},
    xmin=-8, xmax=8,
    ymin=-6, ymax=6,
    xtick={-8,-6,...,8},
    ytick={-6,-4,...,6},
    grid=both,
    grid style={line width=.1pt, draw=gray!30},
    axis line style={-{Latex},thick},
    tick label style={font=\small},
    samples=100,
]

% Hipérbola: x²/16 - y²/9 = 1 (a=4, b=3, c=5)
\addplot[red, very thick, domain=4:7.5, samples=50] {3*sqrt(x^2/16 - 1)};
\addplot[red, very thick, domain=4:7.5, samples=50] {-3*sqrt(x^2/16 - 1)};
\addplot[red, very thick, domain=-7.5:-4, samples=50] {3*sqrt(x^2/16 - 1)};
\addplot[red, very thick, domain=-7.5:-4, samples=50] {-3*sqrt(x^2/16 - 1)};

% Asíntotas: y = ± (b/a)x = ± (3/4)x
\addplot[blue, dashed, domain=-8:8] {0.75*x};
\addplot[blue, dashed, domain=-8:8] {-0.75*x};

% Centro
\node[circle, fill=black, inner sep=2pt] at (0,0) {};
\node[below left] at (0,0) {$C(0,0)$};

% Focos (±c, 0) = (±5, 0)
\node[circle, fill=blue, inner sep=2.5pt] at (5,0) {};
\node[below right] at (5,0) {$F_1(5,0)$};
\node[circle, fill=blue, inner sep=2.5pt] at (-5,0) {};
\node[below left] at (-5,0) {$F_2(-5,0)$};

% Vértices (±a, 0) = (±4, 0)
\node[circle, fill=green!60!black, inner sep=2.5pt] at (4,0) {};
\node[above right] at (4,0) {$V_1(4,0)$};
\node[circle, fill=green!60!black, inner sep=2.5pt] at (-4,0) {};
\node[above left] at (-4,0) {$V_2(-4,0)$};

% Puntos auxiliares para b
\node[circle, fill=orange, inner sep=2pt] at (0,3) {};
\node[above right] at (0,3) {$(0,3)$};
\node[circle, fill=orange, inner sep=2pt] at (0,-3) {};
\node[below right] at (0,-3) {$(0,-3)$};

% Distancias
\draw[<->, green!60!black, very thick] (-4,-5) -- (4,-5) node[midway, below] {$2a=8$};
\draw[<->, orange, thick] (-6.5,3) -- (-6.5,-3) node[midway, left] {$2b=6$};
\draw[<->, blue, thick] (-5,-5.5) -- (5,-5.5) node[midway, below] {$2c=10$};

\end{axis}
\end{tikzpicture}
\end{center}

\subsection{Relación fundamental de la hipérbola}

En toda hipérbola se cumple una relación fundamental entre $a$, $b$ y $c$:

\begin{nota}
\textbf{Relación fundamental:}
\[
c^2 = a^2 + b^2
\]
donde:
\begin{itemize}
\item $c$ = distancia del centro a cada foco
\item $a$ = distancia del centro a cada vértice
\item $b$ = longitud del semieje conjugado
\item Siempre $c > a$ (los focos están fuera de los vértices)
\end{itemize}

Esta es la característica que distingue a la hipérbola de la elipse. En la elipse, $a^2 = b^2 + c^2$ (donde $a > c$), mientras que en la hipérbola los términos se suman.
\end{nota}

\newpage

\subsection{Ecuación canónica con centro en el origen}

Existen dos formas de la ecuación canónica de la hipérbola con centro en $(0,0)$, dependiendo de la orientación del eje focal:

\subsubsection{Hipérbola con eje focal horizontal}

\begin{definicion}[title=Hipérbola horizontal]
\[
\frac{x^2}{a^2} - \frac{y^2}{b^2} = 1
\]

Características:
\begin{itemize}
\item Centro: $C(0,0)$
\item Vértices: $V_1(a,0)$ y $V_2(-a,0)$
\item Focos: $F_1(c,0)$ y $F_2(-c,0)$ donde $c^2 = a^2 + b^2$
\item Asíntotas: $y = \pm \frac{b}{a}x$
\item Eje focal: eje $x$
\item Excentricidad: $e = \frac{c}{a} > 1$
\end{itemize}
\end{definicion}

\subsubsection{Hipérbola con eje focal vertical}

\begin{definicion}[title=Hipérbola vertical]
\[
\frac{y^2}{a^2} - \frac{x^2}{b^2} = 1
\]

Características:
\begin{itemize}
\item Centro: $C(0,0)$
\item Vértices: $V_1(0,a)$ y $V_2(0,-a)$
\item Focos: $F_1(0,c)$ y $F_2(0,-c)$ donde $c^2 = a^2 + b^2$
\item Asíntotas: $y = \pm \frac{a}{b}x$
\item Eje focal: eje $y$
\item Excentricidad: $e = \frac{c}{a} > 1$
\end{itemize}
\end{definicion}

\begin{nota}
¿Cómo identificar la orientación de una hipérbola?

En la forma canónica $\frac{\text{término positivo}}{\text{denominador}_1} - \frac{\text{término negativo}}{\text{denominador}_2} = 1$:

\begin{itemize}
\item Si el término positivo es $x^2$, la hipérbola es horizontal
\item Si el término positivo es $y^2$, la hipérbola es vertical
\item El denominador del término positivo siempre es $a^2$
\item El denominador del término negativo siempre es $b^2$
\end{itemize}
\end{nota}

\subsection{Ecuación canónica con centro en $(h,k)$}

Cuando el centro de la hipérbola está en el punto $(h,k)$ en lugar del origen, la ecuación se traslada:

\subsubsection{Hipérbola horizontal con centro en $(h,k)$}

\begin{definicion}[title=Hipérbola horizontal trasladada]
\[
\frac{(x-h)^2}{a^2} - \frac{(y-k)^2}{b^2} = 1
\]

Características:
\begin{itemize}
\item Centro: $C(h,k)$
\item Vértices: $V_1(h+a,k)$ y $V_2(h-a,k)$
\item Focos: $F_1(h+c,k)$ y $F_2(h-c,k)$ donde $c^2 = a^2 + b^2$
\item Asíntotas: $y - k = \pm \frac{b}{a}(x - h)$
\item Eje focal: $y = k$ (paralelo al eje $x$)
\item Excentricidad: $e = \frac{c}{a} > 1$
\end{itemize}
\end{definicion}

\subsubsection{Hipérbola vertical con centro en $(h,k)$}

\begin{definicion}[title=Hipérbola vertical trasladada]
\[
\frac{(y-k)^2}{a^2} - \frac{(x-h)^2}{b^2} = 1
\]

Características:
\begin{itemize}
\item Centro: $C(h,k)$
\item Vértices: $V_1(h,k+a)$ y $V_2(h,k-a)$
\item Focos: $F_1(h,k+c)$ y $F_2(h,k-c)$ donde $c^2 = a^2 + b^2$
\item Asíntotas: $y - k = \pm \frac{a}{b}(x - h)$
\item Eje focal: $x = h$ (paralelo al eje $y$)
\item Excentricidad: $e = \frac{c}{a} > 1$
\end{itemize}
\end{definicion}

\subsection{Ecuación general de la hipérbola}

La forma general de la ecuación de una hipérbola es:

\[
Ax^2 + Cy^2 + Dx + Ey + F = 0
\]

donde $A$ y $C$ tienen signos opuestos (uno positivo y otro negativo).

Para convertir de la forma general a la forma canónica, usamos el método de \textbf{completación de cuadrados}.

\subsection{Excentricidad de la hipérbola}

La excentricidad $e$ de una hipérbola mide qué tan ``abierta'' es la curva:

\begin{definicion}[title=Excentricidad]
\[
e = \frac{c}{a}
\]

Para toda hipérbola: $e > 1$

\begin{itemize}
\item Si $e$ está cerca de $1$: la hipérbola es más cerrada (los focos están cerca de los vértices)
\item Si $e$ es muy grande: la hipérbola es muy abierta (los focos están muy lejos de los vértices)
\item A mayor excentricidad, las ramas de la hipérbola se abren más
\end{itemize}
\end{definicion}

\subsection{Asíntotas de la hipérbola}

Las asíntotas son rectas a las que la hipérbola se aproxima indefinidamente cuando los valores de $x$ o $y$ tienden al infinito, pero nunca las toca.

\begin{definicion}[title=Asíntotas]
Para una hipérbola con centro en $(h,k)$:

\textbf{Hipérbola horizontal:} $\frac{(x-h)^2}{a^2} - \frac{(y-k)^2}{b^2} = 1$
\[
y - k = \pm \frac{b}{a}(x - h)
\]

\textbf{Hipérbola vertical:} $\frac{(y-k)^2}{a^2} - \frac{(x-h)^2}{b^2} = 1$
\[
y - k = \pm \frac{a}{b}(x - h)
\]

Las asíntotas pasan por el centro de la hipérbola y forman un rectángulo imaginario de lados $2a$ y $2b$.
\end{definicion}

\newpage

\section{Ejemplos Resueltos}

\begin{ejemplo}[title=Hipérbola con centro en el origen - Navegación GPS]
Un sistema de navegación GPS simplificado utiliza dos estaciones terrestres ubicadas en los puntos $F_1(5,0)$ y $F_2(-5,0)$. Un receptor detecta que la diferencia de las distancias a las dos estaciones es $8$ unidades. Determine:
\begin{enumerate}[label=\alph*)]
\item La ecuación de la hipérbola que describe las posibles posiciones del receptor
\item Los vértices de la hipérbola
\item La excentricidad del sistema
\item Las ecuaciones de las asíntotas
\item Graficar la hipérbola con todos sus elementos
\end{enumerate}
\end{ejemplo}

\begin{solucion}
\textbf{Datos:}
\begin{itemize}
\item Focos: $F_1(5,0)$ y $F_2(-5,0)$, entonces $c = 5$
\item Diferencia de distancias: $2a = 8$, entonces $a = 4$
\item Centro en el origen: $C(0,0)$
\item Eje focal horizontal (focos sobre el eje $x$)
\end{itemize}

\textbf{Paso 1:} Encontrar $b$ usando la relación fundamental

Ya que $c^2 = a^2 + b^2$:
\[
25 = 16 + b^2
\]
\[
b^2 = 9 \quad \Rightarrow \quad b = 3
\]

\textbf{Paso 2:} Escribir la ecuación canónica

Como la hipérbola es horizontal (focos sobre el eje $x$) con centro en el origen:
\[
\boxed{\frac{x^2}{16} - \frac{y^2}{9} = 1}
\]

\textbf{Paso 3:} Determinar los vértices

Para una hipérbola horizontal: $V_1(a,0)$ y $V_2(-a,0)$
\[
\boxed{V_1(4,0) \text{ y } V_2(-4,0)}
\]

\textbf{Paso 4:} Calcular la excentricidad

\[
e = \frac{c}{a} = \frac{5}{4} = 1.25
\]
\[
\boxed{e = 1.25}
\]

Como $e > 1$, confirma que es una hipérbola.

\textbf{Paso 5:} Encontrar las asíntotas

Para una hipérbola horizontal con centro en el origen:
\[
y = \pm \frac{b}{a}x = \pm \frac{3}{4}x
\]
\[
\boxed{y = \frac{3}{4}x \quad \text{y} \quad y = -\frac{3}{4}x}
\]

\textbf{Paso 6:} Gráfica

\begin{center}
\begin{tikzpicture}
\begin{axis}[
    width=0.90\textwidth, height=0.60\textwidth,
    axis equal image,
    axis lines=middle,
    xlabel={$x$}, ylabel={$y$},
    xmin=-8, xmax=8,
    ymin=-6, ymax=6,
    xtick={-8,-6,...,8},
    ytick={-6,-4,...,6},
    grid=both,
    grid style={line width=.1pt, draw=gray!30},
    axis line style={-{Latex},thick},
    tick label style={font=\small},
    samples=100,
]

% Hipérbola
\addplot[red, very thick, domain=4:7.5] {3*sqrt(x^2/16 - 1)};
\addplot[red, very thick, domain=4:7.5] {-3*sqrt(x^2/16 - 1)};
\addplot[red, very thick, domain=-7.5:-4] {3*sqrt(x^2/16 - 1)};
\addplot[red, very thick, domain=-7.5:-4] {-3*sqrt(x^2/16 - 1)};

% Asíntotas
\addplot[blue, dashed, thick, domain=-8:8] {0.75*x};
\addplot[blue, dashed, thick, domain=-8:8] {-0.75*x};

% Centro
\node[circle, fill=black, inner sep=2pt] at (0,0) {};
\node[below left] at (0,0) {$C$};

% Focos
\node[circle, fill=blue, inner sep=2.5pt] at (5,0) {};
\node[below] at (5,-0.3) {$F_1(5,0)$};
\node[circle, fill=blue, inner sep=2.5pt] at (-5,0) {};
\node[below] at (-5,-0.3) {$F_2(-5,0)$};

% Vértices
\node[circle, fill=green!60!black, inner sep=2.5pt] at (4,0) {};
\node[above] at (4,0.3) {$V_1(4,0)$};
\node[circle, fill=green!60!black, inner sep=2.5pt] at (-4,0) {};
\node[above] at (-4,0.3) {$V_2(-4,0)$};

\end{axis}
\end{tikzpicture}
\end{center}

\textbf{Verificación:} Tomemos un punto de la hipérbola, por ejemplo $(8,3\sqrt{3})$:
\[
\frac{64}{16} - \frac{27}{9} = 4 - 3 = 1 \quad \checkmark
\]
\end{solucion}

\begin{ejemplo}[title=Hipérbola con centro trasladado - Sistema de radar]
Una estación de radar detecta un objeto cuya trayectoria sigue la ecuación:
\[
\frac{(x-3)^2}{25} - \frac{(y+2)^2}{16} = 1
\]
Determine:
\begin{enumerate}[label=\alph*)]
\item El centro de la hipérbola
\item Los vértices y focos
\item La excentricidad
\item Las ecuaciones de las asíntotas
\item Graficar la hipérbola
\end{enumerate}
\end{ejemplo}

\begin{solucion}
\textbf{Paso 1:} Identificar los elementos de la ecuación

De la forma $\frac{(x-h)^2}{a^2} - \frac{(y-k)^2}{b^2} = 1$:
\begin{itemize}
\item $h = 3$, $k = -2$
\item $a^2 = 25$, entonces $a = 5$
\item $b^2 = 16$, entonces $b = 4$
\item Hipérbola horizontal (término $x^2$ es positivo)
\end{itemize}

\textbf{Paso 2:} Centro

\[
\boxed{C(3,-2)}
\]

\textbf{Paso 3:} Calcular $c$

\[
c^2 = a^2 + b^2 = 25 + 16 = 41
\]
\[
c = \sqrt{41} \approx 6.40
\]

\textbf{Paso 4:} Vértices

Para hipérbola horizontal: $V_1(h+a, k)$ y $V_2(h-a, k)$
\[
V_1(3+5, -2) = V_1(8,-2)
\]
\[
V_2(3-5, -2) = V_2(-2,-2)
\]
\[
\boxed{V_1(8,-2) \text{ y } V_2(-2,-2)}
\]

\textbf{Paso 5:} Focos

Para hipérbola horizontal: $F_1(h+c, k)$ y $F_2(h-c, k)$
\[
F_1(3+\sqrt{41}, -2) \approx F_1(9.40,-2)
\]
\[
F_2(3-\sqrt{41}, -2) \approx F_2(-3.40,-2)
\]
\[
\boxed{F_1(3+\sqrt{41},-2) \text{ y } F_2(3-\sqrt{41},-2)}
\]

\textbf{Paso 6:} Excentricidad

\[
e = \frac{c}{a} = \frac{\sqrt{41}}{5} \approx \frac{6.40}{5} = 1.28
\]
\[
\boxed{e = \frac{\sqrt{41}}{5} \approx 1.28}
\]

\textbf{Paso 7:} Asíntotas

Para hipérbola horizontal con centro en $(h,k)$:
\[
y - k = \pm \frac{b}{a}(x - h)
\]
\[
y - (-2) = \pm \frac{4}{5}(x - 3)
\]
\[
y + 2 = \pm \frac{4}{5}(x - 3)
\]
\[
\boxed{y = \frac{4}{5}x - \frac{22}{5} \quad \text{y} \quad y = -\frac{4}{5}x + \frac{2}{5}}
\]

\textbf{Paso 8:} Gráfica

\begin{center}
\begin{tikzpicture}
\begin{axis}[
    width=0.90\textwidth, height=0.60\textwidth,
    axis equal image,
    axis lines=middle,
    xlabel={$x$}, ylabel={$y$},
    xmin=-6, xmax=12,
    ymin=-8, ymax=4,
    xtick={-6,-4,...,12},
    ytick={-8,-6,...,4},
    grid=both,
    grid style={line width=.1pt, draw=gray!30},
    axis line style={-{Latex},thick},
    tick label style={font=\small},
    samples=100,
]

% Hipérbola trasladada
\addplot[red, very thick, domain=8:11.5] {-2 + 4*sqrt((x-3)^2/25 - 1)};
\addplot[red, very thick, domain=8:11.5] {-2 - 4*sqrt((x-3)^2/25 - 1)};
\addplot[red, very thick, domain=-5:-2] {-2 + 4*sqrt((x-3)^2/25 - 1)};
\addplot[red, very thick, domain=-5:-2] {-2 - 4*sqrt((x-3)^2/25 - 1)};

% Asíntotas
\addplot[blue, dashed, thick, domain=-6:12] {0.8*x - 4.4};
\addplot[blue, dashed, thick, domain=-6:12] {-0.8*x + 0.4};

% Centro
\node[circle, fill=black, inner sep=2.5pt] at (3,-2) {};
\node[below left] at (3,-2) {$C(3,-2)$};

% Vértices
\node[circle, fill=green!60!black, inner sep=2.5pt] at (8,-2) {};
\node[above right] at (8,-2) {$V_1(8,-2)$};
\node[circle, fill=green!60!black, inner sep=2.5pt] at (-2,-2) {};
\node[above left] at (-2,-2) {$V_2(-2,-2)$};

% Focos
\node[circle, fill=blue, inner sep=2.5pt] at (9.4,-2) {};
\node[below right] at (9.4,-2) {$F_1$};
\node[circle, fill=blue, inner sep=2.5pt] at (-3.4,-2) {};
\node[below left] at (-3.4,-2) {$F_2$};

\end{axis}
\end{tikzpicture}
\end{center}

\textbf{Verificación:} El centro está en $(3,-2)$, los vértices están separados $2a = 10$ unidades, y los focos están fuera de los vértices. ✓
\end{solucion}

\begin{ejemplo}[title=De ecuación general a canónica - Trayectoria de cometa]
Un cometa sigue una trayectoria descrita por la ecuación:
\[
9x^2 - 16y^2 - 54x - 64y - 127 = 0
\]
Convierta esta ecuación a la forma canónica e identifique todos los elementos de la hipérbola.
\end{ejemplo}

\begin{solucion}
\textbf{Paso 1:} Agrupar términos en $x$ e $y$

\[
9x^2 - 54x - 16y^2 - 64y = 127
\]

\textbf{Paso 2:} Factorizar coeficientes de $x^2$ e $y^2$

\[
9(x^2 - 6x) - 16(y^2 + 4y) = 127
\]

\textbf{Paso 3:} Completar cuadrados

Para $x^2 - 6x$:
\[
x^2 - 6x + 9 = (x-3)^2 \quad \text{(sumamos $9$)}
\]

Para $y^2 + 4y$:
\[
y^2 + 4y + 4 = (y+2)^2 \quad \text{(sumamos $4$)}
\]

\textbf{Paso 4:} Ajustar la ecuación

\[
9(x^2 - 6x + 9) - 16(y^2 + 4y + 4) = 127 + 9(9) - 16(4)
\]
\[
9(x-3)^2 - 16(y+2)^2 = 127 + 81 - 64
\]
\[
9(x-3)^2 - 16(y+2)^2 = 144
\]

\textbf{Paso 5:} Dividir por 144 para obtener la forma canónica

\[
\frac{9(x-3)^2}{144} - \frac{16(y+2)^2}{144} = 1
\]
\[
\frac{(x-3)^2}{16} - \frac{(y+2)^2}{9} = 1
\]

\[
\boxed{\frac{(x-3)^2}{16} - \frac{(y+2)^2}{9} = 1}
\]

\textbf{Paso 6:} Identificar elementos

\begin{itemize}
\item $h = 3$, $k = -2$ → Centro: $\boxed{C(3,-2)}$
\item $a^2 = 16$ → $a = 4$
\item $b^2 = 9$ → $b = 3$
\item $c^2 = a^2 + b^2 = 16 + 9 = 25$ → $c = 5$
\item Hipérbola horizontal (término $x^2$ es positivo)
\end{itemize}

\textbf{Vértices:}
\[
\boxed{V_1(7,-2) \text{ y } V_2(-1,-2)}
\]

\textbf{Focos:}
\[
\boxed{F_1(8,-2) \text{ y } F_2(-2,-2)}
\]

\textbf{Excentricidad:}
\[
\boxed{e = \frac{5}{4} = 1.25}
\]

\textbf{Asíntotas:}
\[
y + 2 = \pm \frac{3}{4}(x - 3)
\]
\[
\boxed{y = \frac{3}{4}x - \frac{17}{4} \quad \text{y} \quad y = -\frac{3}{4}x - \frac{1}{4}}
\]

\begin{center}
\begin{tikzpicture}
\begin{axis}[
    width=0.90\textwidth, height=0.60\textwidth,
    axis equal image,
    axis lines=middle,
    xlabel={$x$}, ylabel={$y$},
    xmin=-4, xmax=10,
    ymin=-7, ymax=3,
    xtick={-4,-2,...,10},
    ytick={-7,-6,...,3},
    grid=both,
    grid style={line width=.1pt, draw=gray!30},
    axis line style={-{Latex},thick},
    tick label style={font=\small},
    samples=100,
]

% Hipérbola
\addplot[red, very thick, domain=7:9.5] {-2 + 3*sqrt((x-3)^2/16 - 1)};
\addplot[red, very thick, domain=7:9.5] {-2 - 3*sqrt((x-3)^2/16 - 1)};
\addplot[red, very thick, domain=-3.5:-1] {-2 + 3*sqrt((x-3)^2/16 - 1)};
\addplot[red, very thick, domain=-3.5:-1] {-2 - 3*sqrt((x-3)^2/16 - 1)};

% Asíntotas
\addplot[blue, dashed, thick, domain=-4:10] {0.75*x - 4.25};
\addplot[blue, dashed, thick, domain=-4:10] {-0.75*x - 0.25};

% Centro
\node[circle, fill=black, inner sep=2.5pt] at (3,-2) {};
\node[below right] at (3,-2) {$C(3,-2)$};

% Vértices
\node[circle, fill=green!60!black, inner sep=2.5pt] at (7,-2) {};
\node[above] at (7,-2) {$V_1(7,-2)$};
\node[circle, fill=green!60!black, inner sep=2.5pt] at (-1,-2) {};
\node[above] at (-1,-2) {$V_2(-1,-2)$};

% Focos
\node[circle, fill=blue, inner sep=2.5pt] at (8,-2) {};
\node[below] at (8,-2) {$F_1$};
\node[circle, fill=blue, inner sep=2.5pt] at (-2,-2) {};
\node[below] at (-2,-2) {$F_2$};

\end{axis}
\end{tikzpicture}
\end{center}
\end{solucion}

\begin{ejemplo}[title=Hipérbola vertical - Torre de enfriamiento]
Una torre de enfriamiento de una planta nuclear tiene una sección transversal con forma de hipérbola vertical. La ecuación que describe su perfil es:
\[
\frac{y^2}{36} - \frac{x^2}{64} = 1
\]
Determine los elementos de esta hipérbola y grafíquela.
\end{ejemplo}

\begin{solucion}
\textbf{Paso 1:} Identificar tipo y parámetros

De la forma $\frac{y^2}{a^2} - \frac{x^2}{b^2} = 1$:
\begin{itemize}
\item Hipérbola vertical (término $y^2$ es positivo)
\item $a^2 = 36$ → $a = 6$
\item $b^2 = 64$ → $b = 8$
\item Centro en el origen: $C(0,0)$
\end{itemize}

\textbf{Paso 2:} Calcular $c$

\[
c^2 = a^2 + b^2 = 36 + 64 = 100
\]
\[
c = 10
\]

\textbf{Paso 3:} Vértices

Para hipérbola vertical: $V_1(0,a)$ y $V_2(0,-a)$
\[
\boxed{V_1(0,6) \text{ y } V_2(0,-6)}
\]

\textbf{Paso 4:} Focos

Para hipérbola vertical: $F_1(0,c)$ y $F_2(0,-c)$
\[
\boxed{F_1(0,10) \text{ y } F_2(0,-10)}
\]

\textbf{Paso 5:} Excentricidad

\[
e = \frac{c}{a} = \frac{10}{6} = \frac{5}{3} \approx 1.67
\]
\[
\boxed{e = \frac{5}{3} \approx 1.67}
\]

\textbf{Paso 6:} Asíntotas

Para hipérbola vertical:
\[
y = \pm \frac{a}{b}x = \pm \frac{6}{8}x = \pm \frac{3}{4}x
\]
\[
\boxed{y = \frac{3}{4}x \quad \text{y} \quad y = -\frac{3}{4}x}
\]

\textbf{Paso 7:} Gráfica

\begin{center}
\begin{tikzpicture}
\begin{axis}[
    width=0.90\textwidth, height=0.60\textwidth,
    axis equal image,
    axis lines=middle,
    xlabel={$x$}, ylabel={$y$},
    xmin=-12, xmax=12,
    ymin=-14, ymax=14,
    xtick={-12,-10,...,12},
    ytick={-14,-12,...,14},
    grid=both,
    grid style={line width=.1pt, draw=gray!30},
    axis line style={-{Latex},thick},
    tick label style={font=\small},
    samples=100,
]

% Hipérbola vertical: y²/36 - x²/64 = 1
% Despejando y: y = ±√(36(1 + x²/64)) = ±6√(1 + x²/64)
\addplot[red, very thick, domain=-11:11, samples=200] {6*sqrt(1 + x^2/64)};
\addplot[red, very thick, domain=-11:11, samples=200] {-6*sqrt(1 + x^2/64)};

% Asíntotas
\addplot[blue, dashed, thick, domain=-12:12] {0.75*x};
\addplot[blue, dashed, thick, domain=-12:12] {-0.75*x};

% Centro
\node[circle, fill=black, inner sep=2pt] at (0,0) {};
\node[below right] at (0,0) {$C(0,0)$};

% Vértices
\node[circle, fill=green!60!black, inner sep=2.5pt] at (0,6) {};
\node[right] at (0.3,6) {$V_1(0,6)$};
\node[circle, fill=green!60!black, inner sep=2.5pt] at (0,-6) {};
\node[right] at (0.3,-6) {$V_2(0,-6)$};

% Focos
\node[circle, fill=blue, inner sep=2.5pt] at (0,10) {};
\node[right] at (0.3,10) {$F_1(0,10)$};
\node[circle, fill=blue, inner sep=2.5pt] at (0,-10) {};
\node[right] at (0.3,-10) {$F_2(0,-10)$};

\end{axis}
\end{tikzpicture}
\end{center}

\textbf{Interpretación física:} El punto más estrecho de la torre está a $6$ metros del centro (vértices), y la curvatura aumenta según nos alejamos del centro, siguiendo las ramas de la hipérbola.
\end{solucion}

\begin{ejemplo}[title=Problema inverso - Construir ecuación dadas condiciones]
Encuentre la ecuación de la hipérbola que cumple las siguientes condiciones:
\begin{itemize}
\item Centro en $C(2,-1)$
\item Un vértice en $V_1(2,4)$
\item Excentricidad $e = \frac{13}{5}$
\end{itemize}
\end{ejemplo}

\begin{solucion}
\textbf{Paso 1:} Determinar orientación

Como el centro está en $(2,-1)$ y un vértice en $(2,4)$:
\begin{itemize}
\item Ambos tienen la misma coordenada $x = 2$
\item La hipérbola es vertical (eje focal paralelo al eje $y$)
\end{itemize}

\textbf{Paso 2:} Calcular $a$

Distancia del centro al vértice:
\[
a = |4 - (-1)| = 5
\]

\textbf{Paso 3:} Calcular $c$ usando la excentricidad

\[
e = \frac{c}{a}
\]
\[
\frac{13}{5} = \frac{c}{5}
\]
\[
c = 13
\]

\textbf{Paso 4:} Calcular $b$ usando la relación fundamental

\[
c^2 = a^2 + b^2
\]
\[
169 = 25 + b^2
\]
\[
b^2 = 144
\]
\[
b = 12
\]

\textbf{Paso 5:} Escribir la ecuación canónica

Para hipérbola vertical con centro en $(h,k) = (2,-1)$:
\[
\frac{(y-k)^2}{a^2} - \frac{(x-h)^2}{b^2} = 1
\]
\[
\frac{(y-(-1))^2}{25} - \frac{(x-2)^2}{144} = 1
\]
\[
\boxed{\frac{(y+1)^2}{25} - \frac{(x-2)^2}{144} = 1}
\]

\textbf{Paso 6:} Determinar otros elementos

\textbf{Segundo vértice:}
\[
V_2(2, -1-5) = \boxed{V_2(2,-6)}
\]

\textbf{Focos:}
\[
F_1(2, -1+13) = \boxed{F_1(2,12)}
\]
\[
F_2(2, -1-13) = \boxed{F_2(2,-14)}
\]

\textbf{Asíntotas:}
\[
y - (-1) = \pm \frac{a}{b}(x - 2)
\]
\[
y + 1 = \pm \frac{5}{12}(x - 2)
\]
\[
\boxed{y = \frac{5}{12}x - \frac{17}{12} \quad \text{y} \quad y = -\frac{5}{12}x - \frac{1}{12}}
\]

\begin{center}
\begin{tikzpicture}
\begin{axis}[
    width=0.90\textwidth, height=0.60\textwidth,
    axis equal image,
    axis lines=middle,
    xlabel={$x$}, ylabel={$y$},
    xmin=-12, xmax=16,
    ymin=-16, ymax=14,
    xtick={-12,-10,...,16},
    ytick={-16,-14,...,14},
    grid=both,
    grid style={line width=.1pt, draw=gray!30},
    axis line style={-{Latex},thick},
    tick label style={font=\small},
    samples=100,
]

% Hipérbola vertical trasladada
\addplot[red, very thick, domain=-10:14] {-1 + sqrt(25*(1 + (x-2)^2/144))};
\addplot[red, very thick, domain=-10:14] {-1 - sqrt(25*(1 + (x-2)^2/144))};

% Asíntotas
\addplot[blue, dashed, thick, domain=-12:16] {(5/12)*x - 17/12};
\addplot[blue, dashed, thick, domain=-12:16] {-(5/12)*x - 1/12};

% Centro
\node[circle, fill=black, inner sep=2.5pt] at (2,-1) {};
\node[below right] at (2,-1) {$C(2,-1)$};

% Vértices
\node[circle, fill=green!60!black, inner sep=2.5pt] at (2,4) {};
\node[right] at (2.3,4) {$V_1(2,4)$};
\node[circle, fill=green!60!black, inner sep=2.5pt] at (2,-6) {};
\node[right] at (2.3,-6) {$V_2(2,-6)$};

% Focos
\node[circle, fill=blue, inner sep=2.5pt] at (2,12) {};
\node[right] at (2.3,12) {$F_1(2,12)$};
\node[circle, fill=blue, inner sep=2.5pt] at (2,-14) {};
\node[right] at (2.3,-14) {$F_2(2,-14)$};

\end{axis}
\end{tikzpicture}
\end{center}

\textbf{Verificación:} La excentricidad $e = \frac{13}{5} = 2.6 > 1$ confirma que es una hipérbola con apertura considerable.
\end{solucion}

\newpage

\section{Ejercicios Propuestos}

\begin{ejercicio}[title=Hipérbola básica con centro en el origen]
Dada la ecuación de la hipérbola:
\[
\frac{x^2}{49} - \frac{y^2}{25} = 1
\]
Determine:
\begin{enumerate}[label=\alph*)]
\item Centro, vértices y focos
\item Excentricidad
\item Ecuaciones de las asíntotas
\item Grafique la hipérbola
\end{enumerate}
\end{ejercicio}

\begin{ejercicio}[title=Hipérbola vertical]
Dada la ecuación:
\[
\frac{y^2}{100} - \frac{x^2}{64} = 1
\]
Determine todos los elementos de la hipérbola y grafíquela.
\end{ejercicio}

\begin{ejercicio}[title=Hipérbola con centro trasladado]
Dada la ecuación:
\[
\frac{(x+4)^2}{36} - \frac{(y-5)^2}{20} = 1
\]
Identifique todos los elementos y trace la gráfica.
\end{ejercicio}

\begin{ejercicio}[title=Conversión de forma general a canónica]
Convierta la siguiente ecuación a la forma canónica e identifique todos los elementos:
\[
4x^2 - 9y^2 - 16x + 18y - 29 = 0
\]
\end{ejercicio}

\begin{ejercicio}[title=Construir ecuación - Dados focos y vértices]
Encuentre la ecuación de la hipérbola que tiene:
\begin{itemize}
\item Focos en $F_1(0,5)$ y $F_2(0,-5)$
\item Vértices en $V_1(0,3)$ y $V_2(0,-3)$
\end{itemize}
\end{ejercicio}

\begin{ejercicio}[title=Hipérbola con excentricidad dada]
Determine la ecuación de la hipérbola horizontal con centro en el origen, un vértice en $(6,0)$ y excentricidad $e = \frac{5}{3}$.
\end{ejercicio}

\begin{ejercicio}[title=Aplicación - Sistema LORAN de navegación]
Un sistema de navegación LORAN utiliza dos estaciones transmisoras ubicadas en los puntos $A(0,0)$ y $B(300,0)$ (distancias en kilómetros). Un barco recibe señales de ambas estaciones y determina que la diferencia de las distancias a las estaciones es de $180$ km.

\begin{enumerate}[label=\alph*)]
\item Encuentre la ecuación de la hipérbola que describe las posibles posiciones del barco
\item Si el barco está en el punto $(150, y)$ con $y > 0$, ¿cuál es su coordenada $y$?
\item Calcule la excentricidad del sistema
\end{enumerate}
\end{ejercicio}

\begin{ejercicio}[title=Problema avanzado - Asíntotas dadas]
Encuentre la ecuación de la hipérbola con centro en $(3,-2)$, asíntotas $y + 2 = \pm 2(x-3)$, y que pasa por el punto $(4,0)$.
\end{ejercicio}

\newpage

\section{Soluciones Detalladas}

\subsection*{Solución Ejercicio 1}

\begin{solucion}
\textbf{Ecuación:} $\frac{x^2}{49} - \frac{y^2}{25} = 1$

\textbf{a) Centro, vértices y focos}

De la forma $\frac{x^2}{a^2} - \frac{y^2}{b^2} = 1$:
\begin{itemize}
\item $a^2 = 49$ → $a = 7$
\item $b^2 = 25$ → $b = 5$
\item Hipérbola horizontal
\end{itemize}

\textbf{Centro:} $\boxed{C(0,0)}$

\textbf{Vértices:} $V_1(a,0)$ y $V_2(-a,0)$
\[
\boxed{V_1(7,0) \text{ y } V_2(-7,0)}
\]

\textbf{Focos:} Calculamos $c$:
\[
c^2 = a^2 + b^2 = 49 + 25 = 74
\]
\[
c = \sqrt{74} \approx 8.60
\]

Focos: $F_1(c,0)$ y $F_2(-c,0)$
\[
\boxed{F_1(\sqrt{74},0) \text{ y } F_2(-\sqrt{74},0)}
\]

\textbf{b) Excentricidad}

\[
e = \frac{c}{a} = \frac{\sqrt{74}}{7} \approx \frac{8.60}{7} \approx 1.23
\]
\[
\boxed{e = \frac{\sqrt{74}}{7} \approx 1.23}
\]

\textbf{c) Asíntotas}

\[
y = \pm \frac{b}{a}x = \pm \frac{5}{7}x
\]
\[
\boxed{y = \frac{5}{7}x \quad \text{y} \quad y = -\frac{5}{7}x}
\]

\textbf{d) Gráfica}

\begin{center}
\begin{tikzpicture}
\begin{axis}[
    width=0.90\textwidth, height=0.60\textwidth,
    axis equal image,
    axis lines=middle,
    xlabel={$x$}, ylabel={$y$},
    xmin=-12, xmax=12,
    ymin=-8, ymax=8,
    xtick={-12,-10,...,12},
    ytick={-8,-6,...,8},
    grid=both,
    grid style={line width=.1pt, draw=gray!30},
    axis line style={-{Latex},thick},
    tick label style={font=\small},
    samples=100,
]

% Hipérbola
\addplot[red, very thick, domain=7:11.5] {5*sqrt(x^2/49 - 1)};
\addplot[red, very thick, domain=7:11.5] {-5*sqrt(x^2/49 - 1)};
\addplot[red, very thick, domain=-11.5:-7] {5*sqrt(x^2/49 - 1)};
\addplot[red, very thick, domain=-11.5:-7] {-5*sqrt(x^2/49 - 1)};

% Asíntotas
\addplot[blue, dashed, thick, domain=-12:12] {(5/7)*x};
\addplot[blue, dashed, thick, domain=-12:12] {-(5/7)*x};

% Centro
\node[circle, fill=black, inner sep=2pt] at (0,0) {};
\node[below left] at (0,0) {$C$};

% Vértices
\node[circle, fill=green!60!black, inner sep=2.5pt] at (7,0) {};
\node[above] at (7,0) {$V_1(7,0)$};
\node[circle, fill=green!60!black, inner sep=2.5pt] at (-7,0) {};
\node[above] at (-7,0) {$V_2(-7,0)$};

% Focos
\node[circle, fill=blue, inner sep=2.5pt] at (8.6,0) {};
\node[below] at (8.6,0) {$F_1$};
\node[circle, fill=blue, inner sep=2.5pt] at (-8.6,0) {};
\node[below] at (-8.6,0) {$F_2$};

\end{axis}
\end{tikzpicture}
\end{center}
\end{solucion}

\subsection*{Solución Ejercicio 2}

\begin{solucion}
\textbf{Ecuación:} $\frac{y^2}{100} - \frac{x^2}{64} = 1$

De la forma $\frac{y^2}{a^2} - \frac{x^2}{b^2} = 1$:
\begin{itemize}
\item $a^2 = 100$ → $a = 10$
\item $b^2 = 64$ → $b = 8$
\item Hipérbola vertical
\end{itemize}

\textbf{Centro:} $\boxed{C(0,0)}$

\textbf{Vértices:} $V_1(0,a)$ y $V_2(0,-a)$
\[
\boxed{V_1(0,10) \text{ y } V_2(0,-10)}
\]

\textbf{Focos:} Calculamos $c$:
\[
c^2 = a^2 + b^2 = 100 + 64 = 164
\]
\[
c = \sqrt{164} = 2\sqrt{41} \approx 12.81
\]

Focos: $F_1(0,c)$ y $F_2(0,-c)$
\[
\boxed{F_1(0,2\sqrt{41}) \text{ y } F_2(0,-2\sqrt{41})}
\]

\textbf{Excentricidad:}
\[
e = \frac{c}{a} = \frac{2\sqrt{41}}{10} = \frac{\sqrt{41}}{5} \approx 1.28
\]
\[
\boxed{e = \frac{\sqrt{41}}{5} \approx 1.28}
\]

\textbf{Asíntotas:}
\[
y = \pm \frac{a}{b}x = \pm \frac{10}{8}x = \pm \frac{5}{4}x
\]
\[
\boxed{y = \frac{5}{4}x \quad \text{y} \quad y = -\frac{5}{4}x}
\]

\textbf{Gráfica:}

\begin{center}
\begin{tikzpicture}
\begin{axis}[
    width=0.90\textwidth, height=0.60\textwidth,
    axis equal image,
    axis lines=middle,
    xlabel={$x$}, ylabel={$y$},
    xmin=-12, xmax=12,
    ymin=-16, ymax=16,
    xtick={-12,-10,...,12},
    ytick={-16,-14,...,16},
    grid=both,
    grid style={line width=.1pt, draw=gray!30},
    axis line style={-{Latex},thick},
    tick label style={font=\small},
    samples=100,
]

% Hipérbola vertical
\addplot[red, very thick, domain=-11:11] {sqrt(100*(1 + (x)^2/64))};
\addplot[red, very thick, domain=-11:11] {-sqrt(100*(1 + (x)^2/64))};

% Asíntotas
\addplot[blue, dashed, thick, domain=-12:12] {1.25*x};
\addplot[blue, dashed, thick, domain=-12:12] {-1.25*x};

% Centro
\node[circle, fill=black, inner sep=2pt] at (0,0) {};
\node[below right] at (0,0) {$C$};

% Vértices
\node[circle, fill=green!60!black, inner sep=2.5pt] at (0,10) {};
\node[right] at (0.3,10) {$V_1(0,10)$};
\node[circle, fill=green!60!black, inner sep=2.5pt] at (0,-10) {};
\node[right] at (0.3,-10) {$V_2(0,-10)$};

% Focos
\node[circle, fill=blue, inner sep=2.5pt] at (0,12.81) {};
\node[right] at (0.3,12.81) {$F_1$};
\node[circle, fill=blue, inner sep=2.5pt] at (0,-12.81) {};
\node[right] at (0.3,-12.81) {$F_2$};

\end{axis}
\end{tikzpicture}
\end{center}
\end{solucion}

\subsection*{Solución Ejercicio 3}

\begin{solucion}
\textbf{Ecuación:} $\frac{(x+4)^2}{36} - \frac{(y-5)^2}{20} = 1$

De la forma $\frac{(x-h)^2}{a^2} - \frac{(y-k)^2}{b^2} = 1$:
\begin{itemize}
\item $h = -4$, $k = 5$
\item $a^2 = 36$ → $a = 6$
\item $b^2 = 20$ → $b = 2\sqrt{5} \approx 4.47$
\item Hipérbola horizontal
\end{itemize}

\textbf{Centro:} $\boxed{C(-4,5)}$

\textbf{Vértices:} $V_1(h+a,k)$ y $V_2(h-a,k)$
\[
V_1(-4+6,5) = V_1(2,5)
\]
\[
V_2(-4-6,5) = V_2(-10,5)
\]
\[
\boxed{V_1(2,5) \text{ y } V_2(-10,5)}
\]

\textbf{Focos:} Calculamos $c$:
\[
c^2 = a^2 + b^2 = 36 + 20 = 56
\]
\[
c = \sqrt{56} = 2\sqrt{14} \approx 7.48
\]

Focos: $F_1(h+c,k)$ y $F_2(h-c,k)$
\[
F_1(-4+2\sqrt{14},5) \approx F_1(3.48,5)
\]
\[
F_2(-4-2\sqrt{14},5) \approx F_2(-11.48,5)
\]
\[
\boxed{F_1(-4+2\sqrt{14},5) \text{ y } F_2(-4-2\sqrt{14},5)}
\]

\textbf{Excentricidad:}
\[
e = \frac{c}{a} = \frac{2\sqrt{14}}{6} = \frac{\sqrt{14}}{3} \approx 1.25
\]
\[
\boxed{e = \frac{\sqrt{14}}{3} \approx 1.25}
\]

\textbf{Asíntotas:}
\[
y - k = \pm \frac{b}{a}(x - h)
\]
\[
y - 5 = \pm \frac{2\sqrt{5}}{6}(x + 4) = \pm \frac{\sqrt{5}}{3}(x + 4)
\]
\[
\boxed{y = \frac{\sqrt{5}}{3}x + 5 + \frac{4\sqrt{5}}{3} \quad \text{y} \quad y = -\frac{\sqrt{5}}{3}x + 5 - \frac{4\sqrt{5}}{3}}
\]

\begin{center}
\begin{tikzpicture}
\begin{axis}[
    width=0.90\textwidth, height=0.60\textwidth,
    axis equal image,
    axis lines=middle,
    xlabel={$x$}, ylabel={$y$},
    xmin=-14, xmax=6,
    ymin=0, ymax=10,
    xtick={-14,-12,...,6},
    ytick={0,2,...,10},
    grid=both,
    grid style={line width=.1pt, draw=gray!30},
    axis line style={-{Latex},thick},
    tick label style={font=\small},
    samples=100,
]

% Hipérbola
\addplot[red, very thick, domain=2:5.5] {5 + sqrt(20*((x+4)^2/36 - 1))};
\addplot[red, very thick, domain=2:5.5] {5 - sqrt(20*((x+4)^2/36 - 1))};
\addplot[red, very thick, domain=-13.5:-10] {5 + sqrt(20*((x+4)^2/36 - 1))};
\addplot[red, very thick, domain=-13.5:-10] {5 - sqrt(20*((x+4)^2/36 - 1))};

% Asíntotas
\addplot[blue, dashed, thick, domain=-14:6] {(sqrt(5)/3)*(x+4) + 5};
\addplot[blue, dashed, thick, domain=-14:6] {-(sqrt(5)/3)*(x+4) + 5};

% Centro
\node[circle, fill=black, inner sep=2.5pt] at (-4,5) {};
\node[below] at (-4,4.7) {$C(-4,5)$};

% Vértices
\node[circle, fill=green!60!black, inner sep=2.5pt] at (2,5) {};
\node[above] at (2,5.3) {$V_1(2,5)$};
\node[circle, fill=green!60!black, inner sep=2.5pt] at (-10,5) {};
\node[above] at (-10,5.3) {$V_2(-10,5)$};

% Focos
\node[circle, fill=blue, inner sep=2.5pt] at (3.48,5) {};
\node[below] at (3.48,4.7) {$F_1$};
\node[circle, fill=blue, inner sep=2.5pt] at (-11.48,5) {};
\node[below] at (-11.48,4.7) {$F_2$};

\end{axis}
\end{tikzpicture}
\end{center}
\end{solucion}

\subsection*{Solución Ejercicio 4}

\begin{solucion}
\textbf{Ecuación:} $4x^2 - 9y^2 - 16x + 18y - 29 = 0$

\textbf{Paso 1:} Agrupar términos
\[
4x^2 - 16x - 9y^2 + 18y = 29
\]

\textbf{Paso 2:} Factorizar
\[
4(x^2 - 4x) - 9(y^2 - 2y) = 29
\]

\textbf{Paso 3:} Completar cuadrados

Para $x^2 - 4x$:
\[
x^2 - 4x + 4 = (x-2)^2
\]

Para $y^2 - 2y$:
\[
y^2 - 2y + 1 = (y-1)^2
\]

\textbf{Paso 4:} Ajustar ecuación
\[
4(x^2 - 4x + 4) - 9(y^2 - 2y + 1) = 29 + 4(4) - 9(1)
\]
\[
4(x-2)^2 - 9(y-1)^2 = 29 + 16 - 9
\]
\[
4(x-2)^2 - 9(y-1)^2 = 36
\]

\textbf{Paso 5:} Dividir por 36
\[
\frac{4(x-2)^2}{36} - \frac{9(y-1)^2}{36} = 1
\]
\[
\frac{(x-2)^2}{9} - \frac{(y-1)^2}{4} = 1
\]

\[
\boxed{\frac{(x-2)^2}{9} - \frac{(y-1)^2}{4} = 1}
\]

\textbf{Paso 6:} Identificar elementos

\begin{itemize}
\item $h = 2$, $k = 1$ → Centro: $\boxed{C(2,1)}$
\item $a^2 = 9$ → $a = 3$
\item $b^2 = 4$ → $b = 2$
\item $c^2 = 9 + 4 = 13$ → $c = \sqrt{13} \approx 3.61$
\item Hipérbola horizontal
\end{itemize}

\textbf{Vértices:}
\[
V_1(2+3,1) = V_1(5,1)
\]
\[
V_2(2-3,1) = V_2(-1,1)
\]
\[
\boxed{V_1(5,1) \text{ y } V_2(-1,1)}
\]

\textbf{Focos:}
\[
\boxed{F_1(2+\sqrt{13},1) \text{ y } F_2(2-\sqrt{13},1)}
\]

\textbf{Excentricidad:}
\[
\boxed{e = \frac{\sqrt{13}}{3} \approx 1.20}
\]

\textbf{Asíntotas:}
\[
y - 1 = \pm \frac{2}{3}(x - 2)
\]
\[
\boxed{y = \frac{2}{3}x - \frac{1}{3} \quad \text{y} \quad y = -\frac{2}{3}x + \frac{7}{3}}
\]

\begin{center}
\begin{tikzpicture}
\begin{axis}[
    width=0.90\textwidth, height=0.60\textwidth,
    axis equal image,
    axis lines=middle,
    xlabel={$x$}, ylabel={$y$},
    xmin=-4, xmax=8,
    ymin=-3, ymax=5,
    xtick={-4,-2,...,8},
    ytick={-3,-2,...,5},
    grid=both,
    grid style={line width=.1pt, draw=gray!30},
    axis line style={-{Latex},thick},
    tick label style={font=\small},
    samples=100,
]

% Hipérbola
\addplot[red, very thick, domain=5:7.5] {1 + 2*sqrt((x-2)^2/9 - 1)};
\addplot[red, very thick, domain=5:7.5] {1 - 2*sqrt((x-2)^2/9 - 1)};
\addplot[red, very thick, domain=-3.5:-1] {1 + 2*sqrt((x-2)^2/9 - 1)};
\addplot[red, very thick, domain=-3.5:-1] {1 - 2*sqrt((x-2)^2/9 - 1)};

% Asíntotas
\addplot[blue, dashed, thick, domain=-4:8] {(2/3)*x - 1/3};
\addplot[blue, dashed, thick, domain=-4:8] {-(2/3)*x + 7/3};

% Centro
\node[circle, fill=black, inner sep=2.5pt] at (2,1) {};
\node[below] at (2,0.7) {$C(2,1)$};

% Vértices
\node[circle, fill=green!60!black, inner sep=2.5pt] at (5,1) {};
\node[above] at (5,1.3) {$V_1(5,1)$};
\node[circle, fill=green!60!black, inner sep=2.5pt] at (-1,1) {};
\node[above] at (-1,1.3) {$V_2(-1,1)$};

% Focos
\node[circle, fill=blue, inner sep=2.5pt] at (5.61,1) {};
\node[below] at (5.61,0.7) {$F_1$};
\node[circle, fill=blue, inner sep=2.5pt] at (-1.61,1) {};
\node[below] at (-1.61,0.7) {$F_2$};

\end{axis}
\end{tikzpicture}
\end{center}
\end{solucion}

\subsection*{Solución Ejercicio 5}

\begin{solucion}
\textbf{Datos:}
\begin{itemize}
\item Focos: $F_1(0,5)$ y $F_2(0,-5)$ → $c = 5$
\item Vértices: $V_1(0,3)$ y $V_2(0,-3)$ → $a = 3$
\item Hipérbola vertical (focos y vértices sobre eje $y$)
\item Centro en el origen
\end{itemize}

\textbf{Paso 1:} Calcular $b$

\[
c^2 = a^2 + b^2
\]
\[
25 = 9 + b^2
\]
\[
b^2 = 16
\]
\[
b = 4
\]

\textbf{Paso 2:} Ecuación canónica

Para hipérbola vertical con centro en el origen:
\[
\frac{y^2}{a^2} - \frac{x^2}{b^2} = 1
\]
\[
\boxed{\frac{y^2}{9} - \frac{x^2}{16} = 1}
\]

\textbf{Paso 3:} Excentricidad

\[
e = \frac{c}{a} = \frac{5}{3} \approx 1.67
\]
\[
\boxed{e = \frac{5}{3}}
\]

\textbf{Paso 4:} Asíntotas

\[
y = \pm \frac{a}{b}x = \pm \frac{3}{4}x
\]
\[
\boxed{y = \frac{3}{4}x \quad \text{y} \quad y = -\frac{3}{4}x}
\]

\begin{center}
\begin{tikzpicture}
\begin{axis}[
    width=0.90\textwidth, height=0.60\textwidth,
    axis equal image,
    axis lines=middle,
    xlabel={$x$}, ylabel={$y$},
    xmin=-8, xmax=8,
    ymin=-7, ymax=7,
    xtick={-8,-6,...,8},
    ytick={-7,-6,...,7},
    grid=both,
    grid style={line width=.1pt, draw=gray!30},
    axis line style={-{Latex},thick},
    tick label style={font=\small},
    samples=100,
]

% Hipérbola vertical
\addplot[red, very thick, domain=-7:7] {sqrt(9*(1 + (x)^2/16))};
\addplot[red, very thick, domain=-7:7] {-sqrt(9*(1 + (x)^2/16))};

% Asíntotas
\addplot[blue, dashed, thick, domain=-8:8] {0.75*x};
\addplot[blue, dashed, thick, domain=-8:8] {-0.75*x};

% Centro
\node[circle, fill=black, inner sep=2pt] at (0,0) {};
\node[below right] at (0,0) {$C(0,0)$};

% Vértices
\node[circle, fill=green!60!black, inner sep=2.5pt] at (0,3) {};
\node[right] at (0.3,3) {$V_1(0,3)$};
\node[circle, fill=green!60!black, inner sep=2.5pt] at (0,-3) {};
\node[right] at (0.3,-3) {$V_2(0,-3)$};

% Focos
\node[circle, fill=blue, inner sep=2.5pt] at (0,5) {};
\node[right] at (0.3,5) {$F_1(0,5)$};
\node[circle, fill=blue, inner sep=2.5pt] at (0,-5) {};
\node[right] at (0.3,-5) {$F_2(0,-5)$};

\end{axis}
\end{tikzpicture}
\end{center}
\end{solucion}

\subsection*{Solución Ejercicio 6}

\begin{solucion}
\textbf{Datos:}
\begin{itemize}
\item Hipérbola horizontal con centro en el origen
\item Un vértice en $(6,0)$ → $a = 6$
\item Excentricidad $e = \frac{5}{3}$
\end{itemize}

\textbf{Paso 1:} Calcular $c$ usando la excentricidad

\[
e = \frac{c}{a}
\]
\[
\frac{5}{3} = \frac{c}{6}
\]
\[
c = 10
\]

\textbf{Paso 2:} Calcular $b$

\[
c^2 = a^2 + b^2
\]
\[
100 = 36 + b^2
\]
\[
b^2 = 64
\]
\[
b = 8
\]

\textbf{Paso 3:} Ecuación

Para hipérbola horizontal con centro en el origen:
\[
\boxed{\frac{x^2}{36} - \frac{y^2}{64} = 1}
\]

\textbf{Elementos adicionales:}

\textbf{Vértices:}
\[
\boxed{V_1(6,0) \text{ y } V_2(-6,0)}
\]

\textbf{Focos:}
\[
\boxed{F_1(10,0) \text{ y } F_2(-10,0)}
\]

\textbf{Asíntotas:}
\[
y = \pm \frac{b}{a}x = \pm \frac{8}{6}x = \pm \frac{4}{3}x
\]
\[
\boxed{y = \frac{4}{3}x \quad \text{y} \quad y = -\frac{4}{3}x}
\]

\begin{center}
\begin{tikzpicture}
\begin{axis}[
    width=0.90\textwidth, height=0.60\textwidth,
    axis equal image,
    axis lines=middle,
    xlabel={$x$}, ylabel={$y$},
    xmin=-14, xmax=14,
    ymin=-12, ymax=12,
    xtick={-14,-12,...,14},
    ytick={-12,-10,...,12},
    grid=both,
    grid style={line width=.1pt, draw=gray!30},
    axis line style={-{Latex},thick},
    tick label style={font=\small},
    samples=100,
]

% Hipérbola
\addplot[red, very thick, domain=6:13.5] {8*sqrt(x^2/36 - 1)};
\addplot[red, very thick, domain=6:13.5] {-8*sqrt(x^2/36 - 1)};
\addplot[red, very thick, domain=-13.5:-6] {8*sqrt(x^2/36 - 1)};
\addplot[red, very thick, domain=-13.5:-6] {-8*sqrt(x^2/36 - 1)};

% Asíntotas
\addplot[blue, dashed, thick, domain=-14:14] {(4/3)*x};
\addplot[blue, dashed, thick, domain=-14:14] {-(4/3)*x};

% Centro
\node[circle, fill=black, inner sep=2pt] at (0,0) {};
\node[below right] at (0,0) {$C$};

% Vértices
\node[circle, fill=green!60!black, inner sep=2.5pt] at (6,0) {};
\node[above] at (6,0.5) {$V_1(6,0)$};
\node[circle, fill=green!60!black, inner sep=2.5pt] at (-6,0) {};
\node[above] at (-6,0.5) {$V_2(-6,0)$};

% Focos
\node[circle, fill=blue, inner sep=2.5pt] at (10,0) {};
\node[below] at (10,-0.5) {$F_1(10,0)$};
\node[circle, fill=blue, inner sep=2.5pt] at (-10,0) {};
\node[below] at (-10,-0.5) {$F_2(-10,0)$};

\end{axis}
\end{tikzpicture}
\end{center}
\end{solucion}

\subsection*{Solución Ejercicio 7}

\begin{solucion}
\textbf{Datos del sistema LORAN:}
\begin{itemize}
\item Estación A en $(0,0)$
\item Estación B en $(300,0)$
\item Diferencia de distancias: $2a = 180$ km → $a = 90$ km
\item Centro de la hipérbola: punto medio entre A y B → $C(150,0)$
\item Distancia entre estaciones: $2c = 300$ km → $c = 150$ km
\end{itemize}

\textbf{a) Ecuación de la hipérbola}

Calculamos $b$:
\[
c^2 = a^2 + b^2
\]
\[
22500 = 8100 + b^2
\]
\[
b^2 = 14400
\]
\[
b = 120
\]

Como el centro está en $(150,0)$ y la hipérbola es horizontal:
\[
\frac{(x-150)^2}{8100} - \frac{y^2}{14400} = 1
\]
\[
\boxed{\frac{(x-150)^2}{8100} - \frac{y^2}{14400} = 1}
\]

\textbf{b) Coordenada $y$ cuando $x = 150$}

Sustituimos $x = 150$ en la ecuación:
\[
\frac{(150-150)^2}{8100} - \frac{y^2}{14400} = 1
\]
\[
0 - \frac{y^2}{14400} = 1
\]
\[
-\frac{y^2}{14400} = 1
\]
\[
y^2 = -14400
\]

Esto no tiene solución real. El punto $(150, y)$ NO está en la hipérbola porque $x = 150$ es el centro, y la hipérbola no pasa por el centro.

Los vértices están en:
\[
V_1(150+90, 0) = V_1(240, 0)
\]
\[
V_2(150-90, 0) = V_2(60, 0)
\]

Por lo tanto, \boxed{\text{el barco no puede estar en } x = 150}.

Si el barco está en $x = 200$ (por ejemplo):
\[
\frac{(200-150)^2}{8100} - \frac{y^2}{14400} = 1
\]
\[
\frac{2500}{8100} - \frac{y^2}{14400} = 1
\]
\[
\frac{y^2}{14400} = \frac{2500}{8100} - 1 = -\frac{5600}{8100}
\]

Tampoco hay solución. Probemos $x = 250$:
\[
\frac{(250-150)^2}{8100} - \frac{y^2}{14400} = 1
\]
\[
\frac{10000}{8100} - \frac{y^2}{14400} = 1
\]
\[
\frac{y^2}{14400} = \frac{10000}{8100} - 1 = \frac{1900}{8100}
\]
\[
y^2 = 14400 \cdot \frac{1900}{8100} \approx 3377.78
\]
\[
y \approx 58.12 \text{ km}
\]

\textbf{c) Excentricidad}

\[
e = \frac{c}{a} = \frac{150}{90} = \frac{5}{3} \approx 1.67
\]
\[
\boxed{e = \frac{5}{3} \approx 1.67}
\]

\begin{center}
\begin{tikzpicture}
\begin{axis}[
    width=0.90\textwidth, height=0.60\textwidth,
    axis equal image,
    axis lines=middle,
    xlabel={$x$ (km)}, ylabel={$y$ (km)},
    xmin=-50, xmax=350,
    ymin=-180, ymax=180,
    xtick={0,50,...,350},
    ytick={-180,-120,...,180},
    grid=both,
    grid style={line width=.1pt, draw=gray!30},
    axis line style={-{Latex},thick},
    tick label style={font=\small},
    samples=100,
]

% Hipérbola
\addplot[red, very thick, domain=240:330] {120*sqrt((x-150)^2/8100 - 1)};
\addplot[red, very thick, domain=240:330] {-120*sqrt((x-150)^2/8100 - 1)};
\addplot[red, very thick, domain=-30:60] {120*sqrt((x-150)^2/8100 - 1)};
\addplot[red, very thick, domain=-30:60] {-120*sqrt((x-150)^2/8100 - 1)};

% Asíntotas: y = ± (b/a)(x - h) = ± (120/90)(x - 150) = ± (4/3)(x - 150)
\addplot[blue, dashed, thick, domain=-50:350] {(4/3)*(x - 150)};
\addplot[blue, dashed, thick, domain=-50:350] {-(4/3)*(x - 150)};

% Estaciones
\node[circle, fill=orange, inner sep=3pt] at (0,0) {};
\node[below left] at (0,0) {$A(0,0)$};
\node[circle, fill=orange, inner sep=3pt] at (300,0) {};
\node[below right] at (300,0) {$B(300,0)$};

% Centro
\node[circle, fill=black, inner sep=2.5pt] at (150,0) {};
\node[below] at (150,-10) {$C(150,0)$};

% Vértices
\node[circle, fill=green!60!black, inner sep=2.5pt] at (240,0) {};
\node[above] at (240,10) {$V_1(240,0)$};
\node[circle, fill=green!60!black, inner sep=2.5pt] at (60,0) {};
\node[above] at (60,10) {$V_2(60,0)$};

\end{axis}
\end{tikzpicture}
\end{center}
\end{solucion}

\subsection*{Solución Ejercicio 8}

\begin{solucion}
\textbf{Datos:}
\begin{itemize}
\item Centro: $C(3,-2)$
\item Asíntotas: $y + 2 = \pm 2(x-3)$
\item Pasa por el punto $(4,0)$
\end{itemize}

\textbf{Paso 1:} Determinar orientación y relación $\frac{b}{a}$ o $\frac{a}{b}$

De las asíntotas $y + 2 = \pm 2(x-3)$:
\[
y + 2 = 2(x-3) \quad \text{o} \quad y + 2 = -2(x-3)
\]

La pendiente es $\pm 2$.

Si la hipérbola es horizontal: $\frac{b}{a} = 2$ → $b = 2a$

Si la hipérbola es vertical: $\frac{a}{b} = 2$ → $a = 2b$

\textbf{Paso 2:} Probar con hipérbola horizontal

Ecuación: $\frac{(x-3)^2}{a^2} - \frac{(y+2)^2}{b^2} = 1$ donde $b = 2a$

Sustituimos el punto $(4,0)$:
\[
\frac{(4-3)^2}{a^2} - \frac{(0+2)^2}{(2a)^2} = 1
\]
\[
\frac{1}{a^2} - \frac{4}{4a^2} = 1
\]
\[
\frac{1}{a^2} - \frac{1}{a^2} = 1
\]
\[
0 = 1 \quad \text{(Contradicción)}
\]

No funciona con hipérbola horizontal.

\textbf{Paso 3:} Probar con hipérbola vertical

Ecuación: $\frac{(y+2)^2}{a^2} - \frac{(x-3)^2}{b^2} = 1$ donde $a = 2b$

Asíntotas para hipérbola vertical: $y + 2 = \pm \frac{a}{b}(x-3) = \pm \frac{2b}{b}(x-3) = \pm 2(x-3)$ ✓

Sustituimos el punto $(4,0)$:
\[
\frac{(0+2)^2}{(2b)^2} - \frac{(4-3)^2}{b^2} = 1
\]
\[
\frac{4}{4b^2} - \frac{1}{b^2} = 1
\]
\[
\frac{1}{b^2} - \frac{1}{b^2} = 1
\]
\[
0 = 1 \quad \text{(Contradicción)}
\]

\textbf{Paso 4:} Revisar el enfoque

Probemos dejando la relación general. Para hipérbola con centro en $(3,-2)$:

Hipérbola horizontal: $\frac{(x-3)^2}{a^2} - \frac{(y+2)^2}{b^2} = 1$ con $\frac{b}{a} = 2$

Esto significa $b = 2a$, así $b^2 = 4a^2$.

Ecuación: $\frac{(x-3)^2}{a^2} - \frac{(y+2)^2}{4a^2} = 1$

Punto $(4,0)$:
\[
\frac{1}{a^2} - \frac{4}{4a^2} = 1
\]
\[
\frac{1}{a^2} - \frac{1}{a^2} = 1
\]
\[
0 = 1
\]

El problema tiene un error o necesitamos considerar que el punto $(4,0)$ esté muy cerca del centro $(3,-2)$, lo que haría difícil que esté en una rama de la hipérbola.

Asumiendo que hay un error tipográfico y el punto debería ser, por ejemplo, $(5,0)$:

\[
\frac{(5-3)^2}{a^2} - \frac{(0+2)^2}{4a^2} = 1
\]
\[
\frac{4}{a^2} - \frac{4}{4a^2} = 1
\]
\[
\frac{4}{a^2} - \frac{1}{a^2} = 1
\]
\[
\frac{3}{a^2} = 1
\]
\[
a^2 = 3
\]
\[
b^2 = 4a^2 = 12
\]

Ecuación:
\[
\boxed{\frac{(x-3)^2}{3} - \frac{(y+2)^2}{12} = 1}
\]

(Nota: El ejercicio original podría tener un error en el punto dado. Con $(5,0)$ funciona correctamente.)
\end{solucion}

\newpage

\section{Conclusión}

¡Felicitaciones! Has completado esta guía completa sobre la hipérbola. Ahora tienes las herramientas necesarias para identificar, analizar y resolver problemas relacionados con esta fascinante cónica.

\subsection*{Resumen de conceptos clave}

\begin{itemize}
\item La \textbf{hipérbola} es el lugar geométrico de puntos donde la \textcolor{red}{diferencia} de distancias a dos focos es constante
\item Relación fundamental: $c^2 = a^2 + b^2$ (donde $c > a$)
\item Excentricidad: $e = \frac{c}{a} > 1$
\item Dos orientaciones: horizontal (término $x^2$ positivo) y vertical (término $y^2$ positivo)
\item Las \textbf{asíntotas} son rectas que la hipérbola se aproxima pero nunca toca
\item Para convertir de forma general a canónica, usamos completación de cuadrados
\end{itemize}

\subsection*{Tabla de fórmulas esenciales}

\begin{center}
\begin{tabular}{|l|l|l|}
\hline
\textbf{Elemento} & \textbf{Hipérbola Horizontal} & \textbf{Hipérbola Vertical} \\
\hline
Ecuación canónica & $\frac{(x-h)^2}{a^2} - \frac{(y-k)^2}{b^2} = 1$ & $\frac{(y-k)^2}{a^2} - \frac{(x-h)^2}{b^2} = 1$ \\
\hline
Centro & $(h,k)$ & $(h,k)$ \\
\hline
Vértices & $(h\pm a, k)$ & $(h, k\pm a)$ \\
\hline
Focos & $(h\pm c, k)$ & $(h, k\pm c)$ \\
\hline
Asíntotas & $y - k = \pm \frac{b}{a}(x-h)$ & $y - k = \pm \frac{a}{b}(x-h)$ \\
\hline
Relación & $c^2 = a^2 + b^2$ & $c^2 = a^2 + b^2$ \\
\hline
Excentricidad & $e = \frac{c}{a} > 1$ & $e = \frac{c}{a} > 1$ \\
\hline
\end{tabular}
\end{center}

\subsection*{Diferencias clave: Hipérbola vs Elipse}

\begin{center}
\begin{tabular}{|l|l|l|}
\hline
\textbf{Característica} & \textbf{Hipérbola} & \textbf{Elipse} \\
\hline
Definición & Diferencia de distancias & Suma de distancias \\
\hline
Ecuación & Signo negativo (-) & Signo positivo (+) \\
\hline
Relación fundamental & $c^2 = a^2 + b^2$ & $a^2 = b^2 + c^2$ \\
\hline
Excentricidad & $e > 1$ & $0 < e < 1$ \\
\hline
Posición de focos & $c > a$ (fuera vértices) & $c < a$ (dentro vértices) \\
\hline
Forma & Dos ramas abiertas & Curva cerrada \\
\hline
Asíntotas & Sí tiene & No tiene \\
\hline
\end{tabular}
\end{center}

\subsection*{Consejos para el éxito}

\begin{enumerate}
\item \textbf{Identificar orientación primero:} Observa qué término es positivo ($x^2$ o $y^2$)
\item \textbf{Recordar la relación:} En hipérbola es $c^2 = a^2 + b^2$ (sumar), en elipse es $a^2 = b^2 + c^2$ (restar)
\item \textbf{Verificar $e > 1$:} Si calculas $e \leq 1$, revisa tus cálculos
\item \textbf{Graficar las asíntotas:} Te ayudan a visualizar la apertura de la hipérbola
\item \textbf{Completación de cuadrados:} Practica este método, es esencial para pasar de forma general a canónica
\item \textbf{Verificar resultados:} Sustituye puntos conocidos en la ecuación final
\end{enumerate}

\subsection*{Aplicaciones en el mundo real}

La hipérbola no es solo un objeto matemático abstracto. Recuerda que aparece en:
\begin{itemize}
\item Sistemas de navegación (GPS, LORAN)
\item Trayectorias astronómicas (cometas, objetos expulsados)
\item Diseño de estructuras (torres de enfriamiento)
\item Óptica (espejos de telescopios)
\item Física de partículas (trayectorias de repulsión)
\end{itemize}

\subsection*{Siguiente paso}

Ahora que dominas la hipérbola, estás listo para:
\begin{itemize}
\item Estudiar la ecuación general de segundo grado que engloba todas las cónicas
\item Explorar rotación de cónicas
\item Aplicar estos conceptos en cálculo (derivadas e integrales de cónicas)
\item Resolver problemas de optimización con restricciones hiperbólicas
\end{itemize}

\vspace{1cm}

\begin{center}
\textit{``Las matemáticas no mienten, las hipérbolas siempre encuentran su camino hacia el infinito.''}

\vspace{0.5cm}

¡Sigue practicando y explorando el maravilloso mundo de la geometría analítica!
\end{center}

\end{document}
