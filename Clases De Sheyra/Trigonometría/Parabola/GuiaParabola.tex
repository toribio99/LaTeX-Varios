% !TEX program = lualatex
\documentclass[12pt,a4paper,twoside]{article}
\usepackage{fontspec}
\usepackage[spanish,es-nodecimaldot]{babel}
\usepackage{amsmath,amssymb}
\usepackage[margin=2.5cm]{geometry}
\usepackage{xcolor}
\usepackage{tikz,pgfplots}
\usetikzlibrary{calc,arrows.meta,babel,intersections}
\usepackage{multicol}
\usepackage{enumitem}
\pgfplotsset{compat=1.18}
\definecolor{maincolor}{RGB}{26,35,126}
\definecolor{accentcolor}{RGB}{255,87,34}

% Configuración de títulos y formato
\usepackage{titlesec}
\titleformat{\section}{\Large\bfseries\color{maincolor}}{\thesection}{1em}{}
\titleformat{\subsection}{\large\bfseries\color{accentcolor}}{\thesubsection}{1em}{}

% Configuración de cajas para ejemplos
\usepackage{tcolorbox}
\tcbuselibrary{skins,breakable}

\usepackage{fancyhdr}

\pagestyle{fancy}
\fancyhf{}
\fancyhead[LE]{\small\textcolor{maincolor}{\thepage \quad La Parábola}}
\fancyhead[RO]{\small\textcolor{maincolor}{La Parábola \quad \thepage}}
\fancyhead[LO]{\small\textcolor{maincolor}{Grado 10 - Trigonometría}}
\fancyhead[RE]{\small\textcolor{maincolor}{Prof. Toribio De J Arrieta F}}
\fancyfoot[C]{}
\renewcommand{\headrulewidth}{0.5pt}
\renewcommand{\footrulewidth}{0pt}
\setlength{\headheight}{14pt}

\newtcolorbox{ejemplo}[1][]{
  enhanced,
  breakable,
  colback=maincolor!5,
  colframe=maincolor,
  fonttitle=\bfseries,
  title=Ejemplo Resuelto,
  #1
}

\newtcolorbox{ejercicio}[1][]{
  enhanced,
  breakable,
  colback=accentcolor!5,
  colframe=accentcolor,
  fonttitle=\bfseries,
  title=Ejercicio,
  #1
}

\newtcolorbox{solucion}[1][]{
  enhanced,
  breakable,
  colback=green!5,
  colframe=green!60!black,
  fonttitle=\bfseries,
  title=Solución,
  #1
}

\newtcolorbox{nota}[1][]{
  enhanced,
  colback=yellow!10,
  colframe=orange!80!black,
  fonttitle=\bfseries,
  title=Nota Importante,
  #1
}

\newtcolorbox{definicion}[1][]{
  enhanced,
  breakable,
  colback=blue!5,
  colframe=blue!60!black,
  fonttitle=\bfseries,
  title=Definición,
  #1
}

\newtcolorbox{teorema}[1][]{
  enhanced,
  breakable,
  colback=red!5,
  colframe=red!60!black,
  fonttitle=\bfseries,
  title=Teorema,
  #1
}

% Título
\title{\textbf{\Huge GEOMETRÍA ANALÍTICA}\\[0.5cm]
\Large La Parábola}
\author{Prof. Toribio De J Arrieta F\\
\textit{La Pruebita}\\
Grado 10}
\date{\today}

\begin{document}

\maketitle

\tableofcontents
\newpage

\section{Introducción}

¡Hola! Seguramente has visto parábolas por todas partes sin darte cuenta. ¿Has notado la forma de un chorro de agua que sale de una manguera? ¿O la curva que hace una pelota de básquet cuando la lanzas hacia el aro? ¿Has visto alguna vez una antena parabólica? Todas estas formas tienen algo en común: ¡son parábolas!

La parábola es una de las curvas más fascinantes y útiles de las matemáticas. Imagínate que tienes una linterna y la apuntas hacia una pared de forma inclinada... la forma de luz que se proyecta puede ser una parábola. O piensa en los faros de un automóvil: usan espejos parabólicos para concentrar la luz y alumbrar mejor el camino.

\subsection*{¿Qué es una parábola?}

De manera sencilla, una parábola es como una curva en forma de ``U'' (puede estar abierta hacia arriba, hacia abajo, hacia la izquierda o hacia la derecha). Lo genial es que no es cualquier curva... es una curva muy especial con propiedades únicas que la hacen perfecta para muchas aplicaciones.

Digamos que estás en un parque y lanzas una pelota hacia arriba. La trayectoria que sigue la pelota es... ¡exacto, una parábola! Esto no es casualidad, es física pura: la gravedad hace que todos los objetos lanzados (sin considerar la resistencia del aire) sigan trayectorias parabólicas.

\subsection*{Definición intuitiva}

Aquí viene lo interesante. Una parábola es el conjunto de todos los puntos que están a la misma distancia de:
\begin{itemize}
    \item Un punto fijo llamado \textbf{foco} (imagínalo como un imán especial)
    \item Una línea recta fija llamada \textbf{directriz} (como una barrera invisible)
\end{itemize}

Es como si cada punto de la parábola estuviera ``indeciso'' entre acercarse al foco o a la directriz, así que se mantiene exactamente a la misma distancia de ambos. ¡Por eso la curva tiene esa forma tan particular!

\subsection*{¿Por qué son tan importantes las parábolas?}

Las parábolas aparecen en nuestra vida diaria más de lo que piensas:

\begin{itemize}
    \item \textbf{Antenas parabólicas:} ¿Te has preguntado por qué las antenas de TV satelital tienen esa forma? La forma parabólica hace que todas las señales que llegan paralelas se concentren en un solo punto (el foco), donde está el receptor. ¡Por eso captan señales de satélites que están a miles de kilómetros!

    \item \textbf{Puentes colgantes:} Los cables principales de muchos puentes colgantes forman parábolas casi perfectas. Esta forma distribuye el peso de manera óptima, haciendo que el puente sea fuerte y estable.

    \item \textbf{Trayectorias de proyectiles:} Desde una pelota de fútbol hasta un cohete espacial (en su fase inicial), todos siguen trayectorias parabólicas. Los videojuegos usan esta matemática para hacer que los movimientos se vean realistas.

    \item \textbf{Faros de automóviles:} Los faros usan reflectores parabólicos. Cuando la bombilla está en el foco, la luz sale en rayos paralelos, iluminando eficientemente el camino. ¡Sin parábolas, manejar de noche sería mucho más difícil!

    \item \textbf{Telescopios reflectores:} Los grandes telescopios usan espejos parabólicos para concentrar la luz de las estrellas en un punto. Gracias a las parábolas podemos ver galaxias que están a millones de años luz.
\end{itemize}

\subsection*{Un dato curioso}

¿Sabías que Galileo Galilei fue uno de los primeros en descubrir que las trayectorias de los proyectiles son parabólicas? Antes de él, la gente pensaba que los objetos lanzados seguían líneas rectas y luego caían verticalmente. Galileo demostró que no, que la trayectoria es una hermosa curva parabólica. ¡Revolucionó la física con esta observación!

\subsection*{¿Qué aprenderemos en esta guía?}

En esta guía vamos a explorar:
\begin{enumerate}
    \item Cómo construir una parábola usando su definición geométrica
    \item Los elementos importantes: vértice, foco, directriz, eje de simetría
    \item Las diferentes ecuaciones de la parábola y cómo usarlas
    \item Cómo graficar parábolas y entender sus propiedades
    \item Problemas reales donde las parábolas son las protagonistas
\end{enumerate}

Así que prepárate para ver el mundo con otros ojos. Después de esta guía, cada vez que veas un chorro de agua, un puente o una antena parabólica, entenderás la matemática que hay detrás. ¡Vamos a descubrir juntos el fascinante mundo de las parábolas!

\newpage

\section{Conceptos Fundamentales}

\subsection{Construcción Geométrica de la Parábola}

Vamos a construir una parábola paso a paso para entender realmente qué es. Imagina que tienes:
\begin{itemize}
    \item Un punto fijo \textbf{F} (el foco) - piénsalo como un faro
    \item Una línea recta \textbf{L} (la directriz) - piénsala como una pared
\end{itemize}

La parábola es el conjunto de todos los puntos \textbf{P} que cumplen esta condición especial:
\[
\text{Distancia de P al foco} = \text{Distancia de P a la directriz}
\]

Es decir: $d(P,F) = d(P,L)$

\begin{center}
\begin{tikzpicture}[scale=1.5]
    % Directriz
    \draw[thick, blue] (-3,-2) -- (3,-2) node[right] {Directriz};

    % Foco
    \filldraw[red] (0,2) circle (0.08) node[above right] {Foco $F$};

    % Parábola
    \draw[very thick, maincolor, domain=-2.5:2.5, samples=100] plot (\x, {0.25*\x*\x});

    % Punto en la parábola
    \coordinate (P) at (2,1);
    \filldraw[black] (P) circle (0.05) node[above right] {$P$};

    % Distancia al foco
    \draw[dashed, red] (P) -- (0,2) node[midway, above left] {$d_1$};

    % Distancia a la directriz
    \draw[dashed, blue] (P) -- (2,-2) node[midway, right] {$d_2$};

    % Vértice
    \filldraw[green!60!black] (0,0) circle (0.05) node[below left] {Vértice};

    % Eje de simetría
    \draw[dotted, thick] (0,-2.5) -- (0,3);
    \node at (-0.3,3) {Eje};
\end{tikzpicture}
\end{center}

¿Ves cómo cada punto de la parábola mantiene el equilibrio perfecto entre el foco y la directriz? ¡Es como una competencia de tira y jalar donde siempre hay empate!

\subsection{Definición Formal}

\begin{definicion}
Una \textbf{parábola} es el lugar geométrico de todos los puntos del plano que equidistan de un punto fijo llamado \textbf{foco} y de una recta fija llamada \textbf{directriz}.

Matemáticamente, si $F$ es el foco y $L$ es la directriz, entonces un punto $P(x,y)$ pertenece a la parábola si y solo si:
\[
d(P,F) = d(P,L)
\]
\end{definicion}

\subsection{Elementos de la Parábola}

Cada parábola tiene varios elementos importantes que debemos conocer:

\subsubsection{1. Vértice (V)}
Es el punto de la parábola más cercano a la directriz (o al foco). Es como la ``punta'' de la U. El vértice está exactamente a la mitad entre el foco y la directriz.

\subsubsection{2. Foco (F)}
Es el punto fijo que define la parábola. Todas las señales o rayos que llegan paralelos al eje se reflejan hacia el foco (por eso las antenas parabólicas funcionan).

\subsubsection{3. Directriz (L)}
Es la línea recta fija. Siempre es perpendicular al eje de simetría.

\subsubsection{4. Eje de Simetría}
Es la línea recta que pasa por el foco y es perpendicular a la directriz. La parábola es simétrica respecto a este eje (si la doblas por este eje, las dos mitades coinciden perfectamente).

\subsubsection{5. Parámetro (p)}
Es la distancia del vértice al foco (también es la distancia del vértice a la directriz). Este valor determina qué tan ``abierta'' o ``cerrada'' está la parábola:
\begin{itemize}
    \item Si $p$ es grande, la parábola es más abierta
    \item Si $p$ es pequeño, la parábola es más cerrada
\end{itemize}

\subsubsection{6. Lado Recto}
Es el segmento de recta que pasa por el foco, es perpendicular al eje de simetría y tiene sus extremos en la parábola. Su longitud es siempre $|4p|$.

\begin{center}
\begin{tikzpicture}[scale=1.2]
    % Parábola vertical con vértice en origen
    \begin{axis}[
        axis lines = center,
        xlabel = {$x$},
        ylabel = {$y$},
        xmin=-4, xmax=4,
        ymin=-1, ymax=5,
        grid = major,
        width=0.9\textwidth,
        height=8cm,
        axis equal image
    ]

    % Parábola x² = 4y (p=1)
    \addplot[
        domain=-3.5:3.5,
        samples=100,
        very thick,
        maincolor
    ] {x^2/4};

    % Vértice
    \node[label={180:{Vértice $(0,0)$}},circle,fill,inner sep=2pt] at (axis cs:0,0) {};

    % Foco
    \node[label={45:{Foco $(0,1)$}},circle,fill,red,inner sep=2pt] at (axis cs:0,1) {};

    % Directriz
    \addplot[blue, thick, dashed] coordinates {(-4,-1) (4,-1)};
    \node[blue] at (axis cs:3.5,-1.3) {Directriz: $y = -1$};

    % Eje de simetría
    \addplot[dotted, thick] coordinates {(0,-1) (0,5)};

    % Lado recto
    \addplot[green!60!black, very thick] coordinates {(-2,1) (2,1)};
    \node[green!60!black] at (axis cs:0,0.5) {Lado recto: $4p = 4$};

    % Parámetro p
    \draw[<->, thick, orange] (axis cs:-0.3,0) -- (axis cs:-0.3,1);
    \node[orange] at (axis cs:-0.7,0.5) {$p=1$};

    \end{axis}
\end{tikzpicture}
\end{center}

\subsection{Ecuación Canónica con Vértice en el Origen (0,0)}

Cuando el vértice de la parábola está en el origen, las ecuaciones son más sencillas. Hay cuatro casos según hacia dónde se abre la parábola:

\subsubsection{Caso 1: Parábola que abre hacia la derecha}

\begin{center}
\fbox{\Large $y^2 = 4px$ \quad (donde $p > 0$)}
\end{center}

\begin{itemize}
    \item Vértice: $(0,0)$
    \item Foco: $(p,0)$
    \item Directriz: $x = -p$
    \item Eje de simetría: eje $x$ (horizontal)
    \item Lado recto: $4p$
\end{itemize}

\subsubsection{Caso 2: Parábola que abre hacia la izquierda}

\begin{center}
\fbox{\Large $y^2 = -4px$ \quad (donde $p > 0$)}
\end{center}

\begin{itemize}
    \item Vértice: $(0,0)$
    \item Foco: $(-p,0)$
    \item Directriz: $x = p$
    \item Eje de simetría: eje $x$ (horizontal)
    \item Lado recto: $4p$
\end{itemize}

\subsubsection{Caso 3: Parábola que abre hacia arriba}

\begin{center}
\fbox{\Large $x^2 = 4py$ \quad (donde $p > 0$)}
\end{center}

\begin{itemize}
    \item Vértice: $(0,0)$
    \item Foco: $(0,p)$
    \item Directriz: $y = -p$
    \item Eje de simetría: eje $y$ (vertical)
    \item Lado recto: $4p$
\end{itemize}

\subsubsection{Caso 4: Parábola que abre hacia abajo}

\begin{center}
\fbox{\Large $x^2 = -4py$ \quad (donde $p > 0$)}
\end{center}

\begin{itemize}
    \item Vértice: $(0,0)$
    \item Foco: $(0,-p)$
    \item Directriz: $y = p$
    \item Eje de simetría: eje $y$ (vertical)
    \item Lado recto: $4p$
\end{itemize}

\begin{nota}
Un truco para recordar:
\begin{itemize}
    \item Si la variable al cuadrado es $y^2$, la parábola es horizontal
    \item Si la variable al cuadrado es $x^2$, la parábola es vertical
    \item El signo determina hacia dónde se abre
\end{itemize}
\end{nota}

\begin{center}
\begin{tikzpicture}[scale=0.8]
    % Cuatro parábolas con vértice en el origen
    \begin{axis}[
        axis lines = center,
        xlabel = {$x$},
        ylabel = {$y$},
        xmin=-4, xmax=4,
        ymin=-4, ymax=4,
        grid = major,
        width=0.85\textwidth,
        height=0.85\textwidth,
        axis equal image,
        legend pos=outer north east
    ]

    % Parábola hacia la derecha: y² = 4x
    \addplot[
        domain=0:3.5,
        samples=100,
        very thick,
        blue,
        name path=A
    ] ({x^2/4}, x);
    \addplot[
        domain=-3.5:0,
        samples=100,
        very thick,
        blue
    ] ({x^2/4}, x);
    \addlegendentry{$y^2 = 4x$ (derecha)}

    % Parábola hacia la izquierda: y² = -4x
    \addplot[
        domain=0:3.5,
        samples=100,
        very thick,
        red
    ] ({-x^2/4}, x);
    \addplot[
        domain=-3.5:0,
        samples=100,
        very thick,
        red
    ] ({-x^2/4}, x);
    \addlegendentry{$y^2 = -4x$ (izquierda)}

    % Parábola hacia arriba: x² = 4y
    \addplot[
        domain=-3.5:3.5,
        samples=100,
        very thick,
        green!60!black
    ] {x^2/4};
    \addlegendentry{$x^2 = 4y$ (arriba)}

    % Parábola hacia abajo: x² = -4y
    \addplot[
        domain=-3.5:3.5,
        samples=100,
        very thick,
        orange
    ] {-x^2/4};
    \addlegendentry{$x^2 = -4y$ (abajo)}

    % Vértice común
    \node[label={45:{$(0,0)$}},circle,fill,inner sep=2pt] at (axis cs:0,0) {};

    \end{axis}
\end{tikzpicture}
\end{center}

\subsection{Ecuación Canónica con Vértice en (h,k)}

Cuando el vértice no está en el origen sino en un punto $(h,k)$, las ecuaciones se transforman mediante una traslación:

\subsubsection{Parábolas Horizontales (eje paralelo al eje x)}

\begin{center}
\begin{tabular}{|c|c|}
\hline
\textbf{Abre hacia la derecha} & \textbf{Abre hacia la izquierda} \\
\hline
$(y-k)^2 = 4p(x-h)$ & $(y-k)^2 = -4p(x-h)$ \\
$p > 0$ & $p > 0$ \\
\hline
Vértice: $(h,k)$ & Vértice: $(h,k)$ \\
Foco: $(h+p,k)$ & Foco: $(h-p,k)$ \\
Directriz: $x = h-p$ & Directriz: $x = h+p$ \\
\hline
\end{tabular}
\end{center}

\subsubsection{Parábolas Verticales (eje paralelo al eje y)}

\begin{center}
\begin{tabular}{|c|c|}
\hline
\textbf{Abre hacia arriba} & \textbf{Abre hacia abajo} \\
\hline
$(x-h)^2 = 4p(y-k)$ & $(x-h)^2 = -4p(y-k)$ \\
$p > 0$ & $p > 0$ \\
\hline
Vértice: $(h,k)$ & Vértice: $(h,k)$ \\
Foco: $(h,k+p)$ & Foco: $(h,k-p)$ \\
Directriz: $y = k-p$ & Directriz: $y = k+p$ \\
\hline
\end{tabular}
\end{center}

\begin{ejemplo}
Encuentra los elementos de la parábola $(x-3)^2 = 8(y-2)$

\textbf{Solución:}
Comparando con $(x-h)^2 = 4p(y-k)$:
\begin{itemize}
    \item $h = 3$, $k = 2$, $4p = 8 \Rightarrow p = 2$
    \item Vértice: $(3, 2)$
    \item Como es de la forma $(x-h)^2 = 4p(y-k)$ con $p > 0$, abre hacia arriba
    \item Foco: $(h, k+p) = (3, 2+2) = (3, 4)$
    \item Directriz: $y = k-p = 2-2 = 0$ (el eje $x$)
    \item Eje de simetría: $x = 3$ (línea vertical)
    \item Lado recto: $4p = 8$
\end{itemize}
\end{ejemplo}

\subsection{Ecuación General de la Parábola}

La ecuación general de segundo grado:
\[
Ax^2 + Bxy + Cy^2 + Dx + Ey + F = 0
\]

representa una parábola cuando:
\begin{itemize}
    \item $B = 0$ (no hay término $xy$)
    \item Exactamente uno de $A$ o $C$ es cero (no ambos)
\end{itemize}

Esto nos da dos formas posibles:

\subsubsection{Parábola con eje vertical}
\[
Ax^2 + Dx + Ey + F = 0 \quad (A \neq 0, C = 0)
\]

Se puede reescribir como:
\[
x^2 + \frac{D}{A}x + \frac{E}{A}y + \frac{F}{A} = 0
\]

\subsubsection{Parábola con eje horizontal}
\[
Cy^2 + Dx + Ey + F = 0 \quad (C \neq 0, A = 0)
\]

Se puede reescribir como:
\[
y^2 + \frac{D}{C}x + \frac{E}{C}y + \frac{F}{C} = 0
\]

\subsection{Conversión entre Formas}

\subsubsection{De General a Canónica}

Para convertir de la forma general a la canónica, completamos el cuadrado:

\begin{ejemplo}
Convierte $x^2 - 6x - 4y + 1 = 0$ a la forma canónica.

\textbf{Solución:}
\begin{align}
x^2 - 6x - 4y + 1 &= 0 \\
x^2 - 6x &= 4y - 1 \\
x^2 - 6x + 9 &= 4y - 1 + 9 \\
(x - 3)^2 &= 4y + 8 \\
(x - 3)^2 &= 4(y + 2)
\end{align}

Por lo tanto, la forma canónica es $(x-3)^2 = 4(y+2)$ con:
\begin{itemize}
    \item Vértice: $(3, -2)$
    \item $4p = 4 \Rightarrow p = 1$
    \item Foco: $(3, -1)$
    \item Directriz: $y = -3$
\end{itemize}
\end{ejemplo}

\subsubsection{De Canónica a General}

Para convertir de canónica a general, expandimos los binomios:

\begin{ejemplo}
Convierte $(y-2)^2 = 12(x+1)$ a la forma general.

\textbf{Solución:}
\begin{align}
(y-2)^2 &= 12(x+1) \\
y^2 - 4y + 4 &= 12x + 12 \\
y^2 - 4y + 4 - 12x - 12 &= 0 \\
y^2 - 12x - 4y - 8 &= 0
\end{align}

La forma general es: $y^2 - 12x - 4y - 8 = 0$
\end{ejemplo}

\subsection{Tabla Resumen de Fórmulas}

\begin{center}
\begin{tcolorbox}[enhanced,colback=maincolor!10,colframe=maincolor,title=Resumen de Ecuaciones y Elementos]
\renewcommand{\arraystretch}{1.5}
\begin{tabular}{|l|c|c|c|c|}
\hline
\textbf{Orientación} & \textbf{Ecuación} & \textbf{Vértice} & \textbf{Foco} & \textbf{Directriz} \\
\hline
\multicolumn{5}{|c|}{\textbf{Vértice en el origen $(0,0)$}} \\
\hline
Abre derecha & $y^2 = 4px$ & $(0,0)$ & $(p,0)$ & $x = -p$ \\
Abre izquierda & $y^2 = -4px$ & $(0,0)$ & $(-p,0)$ & $x = p$ \\
Abre arriba & $x^2 = 4py$ & $(0,0)$ & $(0,p)$ & $y = -p$ \\
Abre abajo & $x^2 = -4py$ & $(0,0)$ & $(0,-p)$ & $y = p$ \\
\hline
\multicolumn{5}{|c|}{\textbf{Vértice en $(h,k)$}} \\
\hline
Abre derecha & $(y-k)^2 = 4p(x-h)$ & $(h,k)$ & $(h+p,k)$ & $x = h-p$ \\
Abre izquierda & $(y-k)^2 = -4p(x-h)$ & $(h,k)$ & $(h-p,k)$ & $x = h+p$ \\
Abre arriba & $(x-h)^2 = 4p(y-k)$ & $(h,k)$ & $(h,k+p)$ & $y = k-p$ \\
Abre abajo & $(x-h)^2 = -4p(y-k)$ & $(h,k)$ & $(h,k-p)$ & $y = k+p$ \\
\hline
\end{tabular}
\end{tcolorbox}
\end{center}

\subsection{Gráficas Ilustrativas}

Veamos algunos ejemplos de parábolas con diferentes valores de $p$ para entender cómo afecta la abertura:

\begin{center}
\begin{tikzpicture}
    \begin{axis}[
        axis lines = center,
        xlabel = {$x$},
        ylabel = {$y$},
        xmin=-5, xmax=5,
        ymin=-1, ymax=8,
        grid = major,
        width=0.95\textwidth,
        height=10cm,
        axis equal image,
        legend pos=outer north east
    ]

    % Parábola con p = 0.5 (más cerrada)
    \addplot[
        domain=-3:3,
        samples=100,
        very thick,
        blue
    ] {x^2/2};
    \addlegendentry{$x^2 = 2y$ $(p = 0.5)$}

    % Parábola con p = 1
    \addplot[
        domain=-3.5:3.5,
        samples=100,
        very thick,
        red
    ] {x^2/4};
    \addlegendentry{$x^2 = 4y$ $(p = 1)$}

    % Parábola con p = 2 (más abierta)
    \addplot[
        domain=-4.5:4.5,
        samples=100,
        very thick,
        green!60!black
    ] {x^2/8};
    \addlegendentry{$x^2 = 8y$ $(p = 2)$}

    % Focos
    \node[label={0:{$F_1$}},circle,fill,blue,inner sep=1.5pt] at (axis cs:0,0.5) {};
    \node[label={0:{$F_2$}},circle,fill,red,inner sep=1.5pt] at (axis cs:0,1) {};
    \node[label={0:{$F_3$}},circle,fill,green!60!black,inner sep=1.5pt] at (axis cs:0,2) {};

    \end{axis}
\end{tikzpicture}
\end{center}

\begin{nota}
Observa que mientras más grande es $p$, más ``abierta'' es la parábola. Esto es porque el foco está más lejos del vértice, entonces los puntos de la parábola tienen que ``abrirse'' más para mantener la condición de equidistancia.
\end{nota}

Ahora veamos parábolas con vértice trasladado:

\begin{center}
\begin{tikzpicture}
    \begin{axis}[
        axis lines = center,
        xlabel = {$x$},
        ylabel = {$y$},
        xmin=-2, xmax=8,
        ymin=-2, ymax=6,
        grid = major,
        width=0.95\textwidth,
        height=10cm,
        axis equal image
    ]

    % Parábola (x-3)² = 4(y-1)
    \addplot[
        domain=0:6,
        samples=100,
        very thick,
        maincolor
    ] {(x-3)^2/4 + 1};

    % Vértice
    \node[label={225:{Vértice $(3,1)$}},circle,fill,inner sep=2pt] at (axis cs:3,1) {};

    % Foco
    \node[label={45:{Foco $(3,2)$}},circle,fill,red,inner sep=2pt] at (axis cs:3,2) {};

    % Directriz
    \addplot[blue, thick, dashed] coordinates {(-1,0) (7,0)};
    \node[blue] at (axis cs:6.5,-0.3) {Directriz: $y = 0$};

    % Eje de simetría
    \addplot[dotted, thick] coordinates {(3,-1.5) (3,5.5)};
    \node at (axis cs:3.5,5) {Eje: $x = 3$};

    % Lado recto
    \addplot[green!60!black, very thick] coordinates {(1,2) (5,2)};

    % Ecuación
    \node[maincolor] at (axis cs:5,4.5) {$(x-3)^2 = 4(y-1)$};

    \end{axis}
\end{tikzpicture}
\end{center}

\section{Ejemplos Resueltos}

\begin{ejemplo}[title={Ecuación Canónica con Vértice en el Origen - Antena Parabólica}]
Una antena parabólica para comunicaciones satelitales tiene su vértice en el origen y su eje de simetría sobre el eje $x$. Si el receptor (foco) está ubicado a 2.5 metros del vértice, encuentra la ecuación de la parábola y la posición de un punto sobre la antena que está a 4 metros de altura.

\vspace{0.3cm}
\textbf{Solución:}

\textbf{Paso 1:} Identificar los datos y el sistema de coordenadas.

Como el vértice está en el origen $(0,0)$ y el eje de simetría es horizontal (eje $x$), la parábola tiene la forma:
\[
y^2 = 4px
\]
donde $p$ es la distancia del vértice al foco.

\textbf{Paso 2:} Determinar el valor del parámetro $p$.

Dado que el foco está a 2.5 metros del vértice:
\[
p = 2.5 \text{ metros}
\]

Como la antena "abre" hacia el foco (hacia la derecha), $p > 0$.

\textbf{Paso 3:} Escribir la ecuación de la parábola.

Sustituyendo $p = 2.5$ en la ecuación canónica:
\begin{align*}
y^2 &= 4px \\
y^2 &= 4(2.5)x \\
y^2 &= 10x
\end{align*}

\textbf{Paso 4:} Encontrar la posición del punto a 4 metros de altura.

Si $y = 4$ metros, sustituimos en la ecuación:
\begin{align*}
(4)^2 &= 10x \\
16 &= 10x \\
x &= \frac{16}{10} \\
x &= 1.6 \text{ metros}
\end{align*}

\textbf{Paso 5:} Verificación del resultado.

Comprobamos que el punto $(1.6, 4)$ satisface la ecuación:
\[
y^2 = (4)^2 = 16 = 10(1.6) = 16 \quad \checkmark
\]

\textbf{Paso 6:} Análisis adicional - Directriz.

La directriz está a una distancia $p$ del vértice, en el lado opuesto al foco:
\[
x = -p = -2.5
\]

\textbf{Paso 7:} Graficar la parábola.

\begin{center}
\begin{tikzpicture}
\begin{axis}[
    width=0.9\textwidth,
    height=0.6\textwidth,
    axis equal image,
    xmin=-3, xmax=6,
    ymin=-5, ymax=5,
    xlabel={$x$ (metros)},
    ylabel={$y$ (metros)},
    grid=major,
    axis lines=center,
    xtick={-3,-2,-1,0,1,2,3,4,5,6},
    ytick={-5,-4,-3,-2,-1,0,1,2,3,4,5},
]

% Parábola
\addplot[maincolor, very thick, smooth, samples=100, domain=-5:5]
    ({y^2/10}, {y});

% Vértice
\addplot[only marks, mark=*, mark size=3pt, red] coordinates {(0,0)};
\node[below right] at (axis cs:0,0) {$V(0,0)$};

% Foco
\addplot[only marks, mark=*, mark size=3pt, blue] coordinates {(2.5,0)};
\node[below] at (axis cs:2.5,0) {$F(2.5,0)$};

% Directriz
\addplot[dashed, thick, red, domain=-5:5] coordinates {(-2.5,-5) (-2.5,5)};
\node[above] at (axis cs:-2.5,5) {Directriz};

% Punto a 4m de altura
\addplot[only marks, mark=*, mark size=3pt, green!60!black] coordinates {(1.6,4)};
\node[right] at (axis cs:1.6,4) {$(1.6, 4)$};

% Eje de simetría
\addplot[dashed, gray, domain=-3:6] coordinates {(-3,0) (6,0)};

\end{axis}
\end{tikzpicture}
\end{center}

\textbf{Paso 8:} Interpretación física.

La antena parabólica refleja todas las ondas electromagnéticas paralelas al eje hacia el foco, donde se ubica el receptor. Esta propiedad hace que las parábolas sean ideales para antenas y telescopios.

\textbf{Respuesta:}
\[
\boxed{
\begin{aligned}
&\text{Ecuación de la parábola: } y^2 = 10x \\
&\text{Punto a 4m de altura: } (1.6, 4)
\end{aligned}
}
\]
\end{ejemplo}

\begin{ejemplo}[title={Ecuación Canónica con Vértice Trasladado - Puente Colgante}]
Los cables principales de un puente colgante forman una parábola. El vértice del cable está 10 metros por encima del tablero del puente y a 50 metros del inicio del puente. Si el cable alcanza una altura de 40 metros a 100 metros del inicio del puente, encuentra la ecuación de la parábola y determina la altura del cable a 20 metros del inicio.

\vspace{0.3cm}
\textbf{Solución:}

\textbf{Paso 1:} Establecer el sistema de coordenadas.

Ubicamos el origen en el inicio del puente al nivel del tablero. Por lo tanto:
\begin{itemize}
    \item Vértice: $V(50, 10)$ (50m horizontal, 10m de altura)
    \item Punto conocido: $P(100, 40)$
    \item Eje de simetría: vertical, $x = 50$
\end{itemize}

\textbf{Paso 2:} Escribir la forma canónica con vértice trasladado.

Para una parábola vertical con vértice en $(h, k)$:
\[
(x - h)^2 = 4p(y - k)
\]

Con vértice $V(50, 10)$:
\[
(x - 50)^2 = 4p(y - 10)
\]

\textbf{Paso 3:} Determinar el parámetro $p$ usando el punto conocido.

El punto $(100, 40)$ debe satisfacer la ecuación:
\begin{align*}
(100 - 50)^2 &= 4p(40 - 10) \\
(50)^2 &= 4p(30) \\
2500 &= 120p \\
p &= \frac{2500}{120} \\
p &= \frac{125}{6} \approx 20.83 \text{ metros}
\end{align*}

\textbf{Paso 4:} Escribir la ecuación completa.

Sustituyendo $p = \frac{125}{6}$:
\begin{align*}
(x - 50)^2 &= 4 \cdot \frac{125}{6}(y - 10) \\
(x - 50)^2 &= \frac{500}{6}(y - 10) \\
(x - 50)^2 &= \frac{250}{3}(y - 10)
\end{align*}

\textbf{Paso 5:} Encontrar la altura del cable a 20 metros del inicio.

Para $x = 20$, sustituimos en la ecuación:
\begin{align*}
(20 - 50)^2 &= \frac{250}{3}(y - 10) \\
(-30)^2 &= \frac{250}{3}(y - 10) \\
900 &= \frac{250}{3}(y - 10) \\
900 \cdot \frac{3}{250} &= y - 10 \\
\frac{2700}{250} &= y - 10 \\
10.8 &= y - 10 \\
y &= 20.8 \text{ metros}
\end{align*}

\textbf{Paso 6:} Verificación con el punto conocido.

Verificamos que $(100, 40)$ satisface la ecuación:
\begin{align*}
(100 - 50)^2 &= \frac{250}{3}(40 - 10) \\
2500 &= \frac{250}{3}(30) \\
2500 &= \frac{7500}{3} \\
2500 &= 2500 \quad \checkmark
\end{align*}

\textbf{Paso 7:} Encontrar el foco de la parábola.

El foco está a una distancia $p = \frac{125}{6}$ del vértice, en dirección del eje:
\[
F = (50, 10 + \frac{125}{6}) = (50, \frac{185}{6}) \approx (50, 30.83)
\]

\textbf{Paso 8:} Graficar el cable del puente.

\begin{center}
\begin{tikzpicture}
\begin{axis}[
    width=0.95\textwidth,
    height=0.65\textwidth,
    xmin=0, xmax=120,
    ymin=0, ymax=50,
    xlabel={Distancia horizontal (m)},
    ylabel={Altura (m)},
    grid=major,
    axis lines=left,
    xtick={0,20,40,50,60,80,100,120},
    ytick={0,10,20,30,40,50},
]

% Parábola del cable
\addplot[maincolor, very thick, smooth, samples=100, domain=0:120]
    {10 + (3/250)*(x-50)^2};

% Vértice
\addplot[only marks, mark=*, mark size=3pt, red] coordinates {(50,10)};
\node[above right] at (axis cs:50,10) {$V(50,10)$};

% Punto conocido
\addplot[only marks, mark=*, mark size=3pt, blue] coordinates {(100,40)};
\node[right] at (axis cs:100,40) {$(100,40)$};

% Punto a x=20
\addplot[only marks, mark=*, mark size=3pt, green!60!black] coordinates {(20,20.8)};
\node[left] at (axis cs:20,20.8) {$(20, 20.8)$};

% Foco
\addplot[only marks, mark=*, mark size=3pt, orange] coordinates {(50,30.83)};
\node[right] at (axis cs:50,30.83) {$F$};

% Eje de simetría
\addplot[dashed, gray, domain=0:50] coordinates {(50,0) (50,50)};

% Tablero del puente
\addplot[thick, brown, domain=0:120] coordinates {(0,0) (120,0)};

\end{axis}
\end{tikzpicture}
\end{center}

\textbf{Paso 9:} Forma general de la ecuación.

Desarrollando la ecuación canónica:
\begin{align*}
(x - 50)^2 &= \frac{250}{3}(y - 10) \\
x^2 - 100x + 2500 &= \frac{250}{3}y - \frac{2500}{3} \\
3x^2 - 300x + 7500 &= 250y - 2500 \\
3x^2 - 300x - 250y + 10000 &= 0
\end{align*}

\textbf{Respuesta:}
\[
\boxed{
\begin{aligned}
&\text{Ecuación canónica: } (x - 50)^2 = \frac{250}{3}(y - 10) \\
&\text{Altura a 20m del inicio: } 20.8 \text{ metros}
\end{aligned}
}
\]
\end{ejemplo}

\begin{ejemplo}[title={De Ecuación General a Canónica - Faro de Automóvil}]
El reflector parabólico de un faro de automóvil tiene la ecuación $x^2 - 8x - 12y + 28 = 0$, donde las unidades están en centímetros. Encuentra el vértice, el foco y la directriz del reflector. Determina también la profundidad del reflector si tiene un diámetro de 20 cm.

\vspace{0.3cm}
\textbf{Solución:}

\textbf{Paso 1:} Reorganizar la ecuación general.

Partimos de:
\[
x^2 - 8x - 12y + 28 = 0
\]

Agrupamos los términos en $x$ y despejamos los términos en $y$:
\[
x^2 - 8x = 12y - 28
\]

\textbf{Paso 2:} Completar el cuadrado en $x$.

Para completar el cuadrado en $x^2 - 8x$:
\begin{align*}
x^2 - 8x &= x^2 - 8x + 16 - 16 \\
&= (x - 4)^2 - 16
\end{align*}

\textbf{Paso 3:} Sustituir y simplificar.

Reemplazando en la ecuación:
\begin{align*}
(x - 4)^2 - 16 &= 12y - 28 \\
(x - 4)^2 &= 12y - 28 + 16 \\
(x - 4)^2 &= 12y - 12 \\
(x - 4)^2 &= 12(y - 1)
\end{align*}

\textbf{Paso 4:} Identificar los elementos de la parábola.

La ecuación está en la forma $(x - h)^2 = 4p(y - k)$ donde:
\begin{itemize}
    \item Vértice: $V(h, k) = (4, 1)$
    \item $4p = 12$, entonces $p = 3$ cm
\end{itemize}

\textbf{Paso 5:} Encontrar el foco.

Como $p > 0$ y la parábola abre hacia arriba (forma $(x-h)^2 = 4p(y-k)$):
\[
F = (h, k + p) = (4, 1 + 3) = (4, 4)
\]

\textbf{Paso 6:} Encontrar la directriz.

La directriz es una línea horizontal:
\[
y = k - p = 1 - 3 = -2
\]

\textbf{Paso 7:} Calcular la profundidad del reflector.

Si el diámetro es 20 cm, los extremos del reflector están a 10 cm del eje de simetría.
En $x = 4 + 10 = 14$ (o $x = 4 - 10 = -6$):

\begin{align*}
(14 - 4)^2 &= 12(y - 1) \\
100 &= 12(y - 1) \\
\frac{100}{12} &= y - 1 \\
\frac{25}{3} &= y - 1 \\
y &= 1 + \frac{25}{3} = \frac{28}{3} \approx 9.33 \text{ cm}
\end{align*}

Profundidad del reflector = altura en el borde - altura en el vértice:
\[
\text{Profundidad} = \frac{28}{3} - 1 = \frac{25}{3} \approx 8.33 \text{ cm}
\]

\textbf{Paso 8:} Verificación de la ecuación.

Verificamos que el vértice $(4, 1)$ satisface la ecuación original:
\begin{align*}
(4)^2 - 8(4) - 12(1) + 28 &= 16 - 32 - 12 + 28 \\
&= 0 \quad \checkmark
\end{align*}

\textbf{Paso 9:} Graficar el reflector.

\begin{center}
\begin{tikzpicture}
\begin{axis}[
    width=0.85\textwidth,
    height=0.85\textwidth,
    axis equal image,
    xmin=-8, xmax=16,
    ymin=-4, ymax=12,
    xlabel={$x$ (cm)},
    ylabel={$y$ (cm)},
    grid=major,
    axis lines=center,
    xtick={-8,-6,-4,-2,0,2,4,6,8,10,12,14,16},
    ytick={-4,-2,0,2,4,6,8,10,12},
]

% Parábola completa
\addplot[maincolor!30, thin, smooth, samples=100, domain=-8:16]
    {1 + (x-4)^2/12};

% Reflector (parte usada)
\addplot[maincolor, very thick, smooth, samples=100, domain=-6:14]
    {1 + (x-4)^2/12};

% Vértice
\addplot[only marks, mark=*, mark size=3pt, red] coordinates {(4,1)};
\node[below right] at (axis cs:4,1) {$V(4,1)$};

% Foco
\addplot[only marks, mark=*, mark size=3pt, blue] coordinates {(4,4)};
\node[right] at (axis cs:4,4) {$F(4,4)$};

% Directriz
\addplot[dashed, thick, red, domain=-8:16] coordinates {(-8,-2) (16,-2)};
\node[right] at (axis cs:16,-2) {Directriz: $y=-2$};

% Extremos del reflector
\addplot[only marks, mark=*, mark size=3pt, green!60!black]
    coordinates {(-6,9.33) (14,9.33)};
\node[left] at (axis cs:-6,9.33) {$(-6, 9.33)$};
\node[right] at (axis cs:14,9.33) {$(14, 9.33)$};

% Eje de simetría
\addplot[dashed, gray, domain=-4:12] coordinates {(4,-4) (4,12)};

% Diámetro
\draw[<->, thick, orange] (axis cs:-6,9.33) -- (axis cs:14,9.33)
    node[midway,above] {20 cm};

\end{axis}
\end{tikzpicture}
\end{center}

\textbf{Paso 10:} Interpretación física.

La bombilla se coloca en el foco $(4, 4)$. Los rayos de luz que emanan del foco se reflejan en la superficie parabólica y salen paralelos al eje, creando un haz de luz dirigido.

\textbf{Respuesta:}
\[
\boxed{
\begin{aligned}
&\text{Vértice: } V(4, 1) \text{ cm} \\
&\text{Foco: } F(4, 4) \text{ cm} \\
&\text{Directriz: } y = -2 \\
&\text{Profundidad del reflector: } \frac{25}{3} \approx 8.33 \text{ cm}
\end{aligned}
}
\]
\end{ejemplo}

\begin{ejemplo}[title={Hallar Ecuación dados Vértice y Foco - Telescopio Reflector}]
Un telescopio reflector tiene su espejo principal con forma parabólica. El vértice del espejo está en el punto $(0, 0)$ y el foco donde se concentran los rayos de luz está en el punto $(0, 15)$. Si el espejo tiene un diámetro de 60 cm, encuentra su ecuación y determina la profundidad del espejo.

\vspace{0.3cm}
\textbf{Solución:}

\textbf{Paso 1:} Identificar la orientación de la parábola.

Datos:
\begin{itemize}
    \item Vértice: $V(0, 0)$
    \item Foco: $F(0, 15)$
    \item Diámetro: 60 cm (radio = 30 cm)
\end{itemize}

Como el foco está directamente arriba del vértice, el eje de simetría es vertical (eje $y$) y la parábola abre hacia arriba.

\textbf{Paso 2:} Calcular el parámetro $p$.

La distancia del vértice al foco es:
\[
p = |y_F - y_V| = |15 - 0| = 15 \text{ cm}
\]

Como la parábola abre hacia arriba, $p > 0$.

\textbf{Paso 3:} Escribir la ecuación de la parábola.

Para una parábola con vértice en el origen y eje vertical:
\[
x^2 = 4py
\]

Sustituyendo $p = 15$:
\begin{align*}
x^2 &= 4(15)y \\
x^2 &= 60y
\end{align*}

\textbf{Paso 4:} Encontrar la directriz.

La directriz está a una distancia $p$ del vértice, en el lado opuesto al foco:
\[
y = -p = -15
\]

\textbf{Paso 5:} Calcular la profundidad del espejo.

El borde del espejo está a 30 cm del eje (radio del espejo). Para $x = 30$:
\begin{align*}
(30)^2 &= 60y \\
900 &= 60y \\
y &= \frac{900}{60} \\
y &= 15 \text{ cm}
\end{align*}

La profundidad del espejo es la diferencia entre la altura en el borde y en el vértice:
\[
\text{Profundidad} = 15 - 0 = 15 \text{ cm}
\]

\textbf{Paso 6:} Verificación importante.

Notamos que el punto del borde $(30, 15)$ está a la misma altura que el foco. Esto significa que el foco está al nivel del borde del espejo.

\textbf{Paso 7:} Calcular el punto de Latera Recta.

La longitud del latus rectum es $|4p| = 60$ cm. Los extremos del latus rectum (cuerda focal perpendicular al eje) están en:
\[
(\pm 30, 15)
\]

\textbf{Paso 8:} Análisis de la propiedad focal.

Cualquier rayo de luz paralelo al eje $y$ que incida en el espejo se reflejará hacia el foco. La distancia de cualquier punto $(x, y)$ de la parábola al foco es:
\[
d = \sqrt{x^2 + (y - 15)^2}
\]

Y su distancia a la directriz es:
\[
d' = |y - (-15)| = y + 15
\]

Para verificar que son iguales en un punto, tomemos $(30, 15)$:
\begin{align*}
d &= \sqrt{30^2 + (15 - 15)^2} = \sqrt{900} = 30 \\
d' &= 15 + 15 = 30 \quad \checkmark
\end{align*}

\textbf{Paso 9:} Graficar el telescopio reflector.

\begin{center}
\begin{tikzpicture}
\begin{axis}[
    width=0.9\textwidth,
    height=0.9\textwidth,
    axis equal image,
    xmin=-40, xmax=40,
    ymin=-20, ymax=30,
    xlabel={$x$ (cm)},
    ylabel={$y$ (cm)},
    grid=major,
    axis lines=center,
    xtick={-40,-30,-20,-10,0,10,20,30,40},
    ytick={-20,-15,-10,-5,0,5,10,15,20,25,30},
]

% Parábola completa (línea delgada)
\addplot[maincolor!30, thin, smooth, samples=100, domain=-40:40]
    {x^2/60};

% Espejo (parte utilizada)
\addplot[maincolor, very thick, smooth, samples=100, domain=-30:30]
    {x^2/60};

% Vértice
\addplot[only marks, mark=*, mark size=3pt, red] coordinates {(0,0)};
\node[below right] at (axis cs:0,0) {$V(0,0)$};

% Foco
\addplot[only marks, mark=*, mark size=3pt, blue] coordinates {(0,15)};
\node[right] at (axis cs:0,15) {$F(0,15)$};

% Directriz
\addplot[dashed, thick, red, domain=-40:40] coordinates {(-40,-15) (40,-15)};
\node[right] at (axis cs:40,-15) {Directriz: $y=-15$};

% Extremos del espejo
\addplot[only marks, mark=*, mark size=3pt, green!60!black]
    coordinates {(-30,15) (30,15)};
\node[left] at (axis cs:-30,15) {$(-30, 15)$};
\node[right] at (axis cs:30,15) {$(30, 15)$};

% Eje de simetría
\addplot[dashed, gray, domain=-20:30] coordinates {(0,-20) (0,30)};

% Rayos de luz paralelos
\pgfplotsinvokeforeach{-25,-15,-5,5,15,25}{
    \draw[->, yellow!80!orange, thick] (axis cs:#1,28) -- (axis cs:#1,{#1*#1/60});
    \draw[->, yellow!80!orange, thick] (axis cs:#1,{#1*#1/60}) -- (axis cs:0,15);
}

% Diámetro del espejo
\draw[<->, thick, orange] (axis cs:-30,15) -- (axis cs:30,15)
    node[midway,above] {60 cm};

\end{axis}
\end{tikzpicture}
\end{center}

\textbf{Paso 10:} Aplicación práctica.

Los rayos de luz de estrellas distantes llegan paralelos al eje del telescopio. Al reflejarse en el espejo parabólico, todos convergen en el foco donde se coloca el sensor o espejo secundario.

\textbf{Respuesta:}
\[
\boxed{
\begin{aligned}
&\text{Ecuación del espejo: } x^2 = 60y \\
&\text{Profundidad del espejo: } 15 \text{ cm} \\
&\text{Directriz: } y = -15
\end{aligned}
}
\]
\end{ejemplo}

\begin{ejemplo}[title={Parábola por Tres Puntos - Trayectoria de Proyectil}]
Un proyectil es lanzado desde el suelo y su trayectoria parabólica pasa por los puntos $A(20, 15)$, $B(40, 20)$ y $C(60, 15)$, donde las distancias están en metros. Encuentra la ecuación de la trayectoria, la altura máxima alcanzada y el alcance horizontal del proyectil.

\vspace{0.3cm}
\textbf{Solución:}

\textbf{Paso 1:} Plantear la forma general de la parábola.

Como la trayectoria es parabólica con eje vertical, usamos:
\[
y = ax^2 + bx + c
\]

\textbf{Paso 2:} Formar el sistema de ecuaciones.

Sustituyendo los tres puntos:

Para $A(20, 15)$:
\[
15 = a(20)^2 + b(20) + c = 400a + 20b + c
\]

Para $B(40, 20)$:
\[
20 = a(40)^2 + b(40) + c = 1600a + 40b + c
\]

Para $C(60, 15)$:
\[
15 = a(60)^2 + b(60) + c = 3600a + 60b + c
\]

\textbf{Paso 3:} Resolver el sistema de ecuaciones.

Sistema:
\begin{align}
400a + 20b + c &= 15 \\
1600a + 40b + c &= 20 \\
3600a + 60b + c &= 15
\end{align}

Restando ecuación (1) de ecuación (2):
\begin{align*}
1200a + 20b &= 5 \\
60a + b &= \frac{1}{4} \quad \text{...(4)}
\end{align*}

Restando ecuación (2) de ecuación (3):
\begin{align*}
2000a + 20b &= -5 \\
100a + b &= -\frac{1}{4} \quad \text{...(5)}
\end{align*}

Restando ecuación (4) de ecuación (5):
\begin{align*}
40a &= -\frac{1}{2} \\
a &= -\frac{1}{80}
\end{align*}

\textbf{Paso 4:} Encontrar $b$ y $c$.

De ecuación (4):
\begin{align*}
60\left(-\frac{1}{80}\right) + b &= \frac{1}{4} \\
-\frac{3}{4} + b &= \frac{1}{4} \\
b &= 1
\end{align*}

De ecuación (1):
\begin{align*}
400\left(-\frac{1}{80}\right) + 20(1) + c &= 15 \\
-5 + 20 + c &= 15 \\
c &= 0
\end{align*}

\textbf{Paso 5:} Escribir la ecuación de la trayectoria.

\[
y = -\frac{1}{80}x^2 + x
\]

Factorizando:
\[
y = x\left(1 - \frac{x}{80}\right) = \frac{x(80 - x)}{80}
\]

\textbf{Paso 6:} Encontrar el vértice (altura máxima).

El vértice ocurre en $x = -\frac{b}{2a} = -\frac{1}{2(-1/80)} = 40$

Altura máxima:
\begin{align*}
y_{max} &= -\frac{1}{80}(40)^2 + 40 \\
&= -\frac{1600}{80} + 40 \\
&= -20 + 40 = 20 \text{ metros}
\end{align*}

\textbf{Paso 7:} Encontrar el alcance horizontal.

El proyectil toca el suelo cuando $y = 0$:
\begin{align*}
0 &= -\frac{1}{80}x^2 + x \\
0 &= x\left(-\frac{1}{80}x + 1\right) \\
x &= 0 \quad \text{o} \quad x = 80
\end{align*}

Alcance horizontal = 80 metros

\textbf{Paso 8:} Verificación con los puntos dados.

Para $A(20, 15)$: $y = -\frac{1}{80}(400) + 20 = -5 + 20 = 15$ ✓

Para $B(40, 20)$: $y = -\frac{1}{80}(1600) + 40 = -20 + 40 = 20$ ✓

Para $C(60, 15)$: $y = -\frac{1}{80}(3600) + 60 = -45 + 60 = 15$ ✓

\textbf{Paso 9:} Graficar la trayectoria.

\begin{center}
\begin{tikzpicture}
\begin{axis}[
    width=0.95\textwidth,
    height=0.6\textwidth,
    xmin=0, xmax=85,
    ymin=0, ymax=25,
    xlabel={Distancia horizontal (m)},
    ylabel={Altura (m)},
    grid=major,
    axis lines=left,
    xtick={0,10,20,30,40,50,60,70,80},
    ytick={0,5,10,15,20,25},
]

% Trayectoria parabólica
\addplot[maincolor, very thick, smooth, samples=100, domain=0:80]
    {x*(1 - x/80)};

% Puntos dados
\addplot[only marks, mark=*, mark size=3pt, red]
    coordinates {(20,15) (40,20) (60,15)};
\node[above] at (axis cs:20,15) {$A$};
\node[above] at (axis cs:40,20) {$B$};
\node[above] at (axis cs:60,15) {$C$};

% Vértice (altura máxima)
\addplot[only marks, mark=*, mark size=3pt, blue] coordinates {(40,20)};
\draw[dashed, blue] (axis cs:40,0) -- (axis cs:40,20);
\node[right] at (axis cs:40,10) {Altura máxima};

% Punto de lanzamiento y caída
\addplot[only marks, mark=*, mark size=3pt, green!60!black]
    coordinates {(0,0) (80,0)};
\node[below] at (axis cs:0,0) {Lanzamiento};
\node[below] at (axis cs:80,0) {Impacto};

% Alcance
\draw[<->, thick, orange] (axis cs:0,-2) -- (axis cs:80,-2)
    node[midway,below] {Alcance = 80 m};

\end{axis}
\end{tikzpicture}
\end{center}

\textbf{Paso 10:} Análisis físico.

La trayectoria simétrica indica que no hay resistencia del aire. El proyectil alcanza su altura máxima exactamente a la mitad del recorrido (40 m), característica de un tiro parabólico ideal.

\textbf{Respuesta:}
\[
\boxed{
\begin{aligned}
&\text{Ecuación de la trayectoria: } y = -\frac{1}{80}x^2 + x \\
&\text{Altura máxima: } 20 \text{ metros (en } x = 40 \text{m)} \\
&\text{Alcance horizontal: } 80 \text{ metros}
\end{aligned}
}
\]
\end{ejemplo}

\begin{ejemplo}[title={Aplicación Integral - Diseño de Espejo Parabólico Solar}]
Una empresa de energía solar necesita diseñar un concentrador parabólico para calentar un tubo colector. El espejo debe tener 8 metros de ancho y una profundidad de 2 metros. El tubo colector se ubicará en el foco. Encuentra: la ecuación del espejo, la posición del foco, el área de la superficie reflectante y el ángulo de apertura del espejo.

\vspace{0.3cm}
\textbf{Solución:}

\textbf{Paso 1:} Establecer el sistema de coordenadas.

Colocamos el vértice en el origen con el eje de simetría vertical:
\begin{itemize}
    \item Vértice: $V(0, 0)$
    \item Ancho: 8 m (de $x = -4$ a $x = 4$)
    \item Profundidad: 2 m (altura en los extremos)
\end{itemize}

\textbf{Paso 2:} Determinar la ecuación usando un punto conocido.

La parábola tiene la forma $x^2 = 4py$. En el extremo $(4, 2)$:
\begin{align*}
(4)^2 &= 4p(2) \\
16 &= 8p \\
p &= 2 \text{ metros}
\end{align*}

Por lo tanto, la ecuación es:
\[
x^2 = 8y \quad \text{o} \quad y = \frac{x^2}{8}
\]

\textbf{Paso 3:} Localizar el foco.

El foco está a una distancia $p = 2$ del vértice:
\[
F(0, 2)
\]

El tubo colector se colocará horizontalmente pasando por este punto.

\textbf{Paso 4:} Calcular el área de la superficie reflectante.

El área bajo la curva desde $x = -4$ hasta $x = 4$ es:
\begin{align*}
A_{bajo} &= \int_{-4}^{4} \frac{x^2}{8} \, dx \\
&= \frac{1}{8} \int_{-4}^{4} x^2 \, dx \\
&= \frac{1}{8} \left[ \frac{x^3}{3} \right]_{-4}^{4} \\
&= \frac{1}{8} \cdot \frac{1}{3} \left[ 64 - (-64) \right] \\
&= \frac{1}{24} \cdot 128 = \frac{16}{3} \text{ m}^2
\end{align*}

\textbf{Paso 5:} Calcular la longitud del arco parabólico.

La longitud del arco se calcula con:
\[
L = \int_{-4}^{4} \sqrt{1 + \left(\frac{dy}{dx}\right)^2} \, dx
\]

Donde $\frac{dy}{dx} = \frac{x}{4}$, entonces:
\begin{align*}
L &= \int_{-4}^{4} \sqrt{1 + \frac{x^2}{16}} \, dx \\
&= \int_{-4}^{4} \sqrt{\frac{16 + x^2}{16}} \, dx \\
&= \frac{1}{4} \int_{-4}^{4} \sqrt{16 + x^2} \, dx
\end{align*}

Usando la sustitución $x = 4\tan\theta$ y evaluando:
\[
L \approx 8.94 \text{ metros}
\]

\textbf{Paso 6:} Calcular el ángulo de apertura.

El ángulo de apertura es el ángulo que forma la tangente en el extremo con la vertical.

En el punto $(4, 2)$, la pendiente es:
\[
m = \frac{dy}{dx}\Big|_{x=4} = \frac{4}{4} = 1
\]

El ángulo con la horizontal es $\arctan(1) = 45°$.
El ángulo con la vertical (ángulo de apertura) es:
\[
\alpha = 90° - 45° = 45°
\]

El ángulo de apertura total del espejo es $2\alpha = 90°$.

\textbf{Paso 7:} Análisis de eficiencia.

La distancia focal $f = p = 2$ metros.
La razón focal-diámetro es:
\[
\frac{f}{D} = \frac{2}{8} = 0.25
\]

Esta razón indica un concentrador de alta concentración.

\textbf{Paso 8:} Verificar la propiedad focal.

Cualquier rayo paralelo al eje $y$ que incida en el punto $(x, y)$ del espejo se reflejará hacia el foco. Por ejemplo, un rayo que llega a $(3, 9/8)$:

Distancia al foco:
\[
d_F = \sqrt{3^2 + (9/8 - 2)^2} = \sqrt{9 + 49/64} = \sqrt{625/64} = 25/8
\]

Distancia a la directriz ($y = -2$):
\[
d_D = 9/8 - (-2) = 9/8 + 2 = 25/8 \quad \checkmark
\]

\textbf{Paso 9:} Graficar el diseño del concentrador solar.

\begin{center}
\begin{tikzpicture}
\begin{axis}[
    width=0.9\textwidth,
    height=0.75\textwidth,
    axis equal image,
    xmin=-5, xmax=5,
    ymin=-1, ymax=4,
    xlabel={$x$ (metros)},
    ylabel={$y$ (metros)},
    grid=major,
    axis lines=center,
    xtick={-5,-4,-3,-2,-1,0,1,2,3,4,5},
    ytick={-1,0,1,2,3,4},
]

% Parábola (espejo)
\addplot[maincolor, very thick, smooth, samples=100, domain=-4:4]
    {x^2/8};

% Vértice
\addplot[only marks, mark=*, mark size=3pt, red] coordinates {(0,0)};
\node[below right] at (axis cs:0,0) {$V(0,0)$};

% Foco (tubo colector)
\addplot[only marks, mark=*, mark size=4pt, blue] coordinates {(0,2)};
\draw[blue, very thick] (axis cs:-0.5,2) -- (axis cs:0.5,2);
\node[right] at (axis cs:0.5,2) {Tubo colector};

% Extremos del espejo
\addplot[only marks, mark=*, mark size=3pt, green!60!black]
    coordinates {(-4,2) (4,2)};

% Rayos solares paralelos
\pgfplotsinvokeforeach{-3.5,-2.5,-1.5,-0.5,0.5,1.5,2.5,3.5}{
    \draw[->, yellow!80!orange, thick]
        (axis cs:#1,3.5) -- (axis cs:#1,{#1*#1/8});
    \draw[->, red!80!orange, thick]
        (axis cs:#1,{#1*#1/8}) -- (axis cs:0,2);
}

% Directriz
\addplot[dashed, thick, red, domain=-5:5] coordinates {(-5,-2) (5,-2)};
\node[right] at (axis cs:5,-2) {Directriz};

% Dimensiones
\draw[<->, thick, orange] (axis cs:-4,2.5) -- (axis cs:4,2.5)
    node[midway,above] {8 m};
\draw[<->, thick, orange] (axis cs:4.5,0) -- (axis cs:4.5,2)
    node[midway,right] {2 m};

% Ángulo de apertura
\draw[dashed, gray] (axis cs:0,0) -- (axis cs:4,2);
\draw[blue,->,thick] (axis cs:0,0.5) arc (90:45:0.5);
\node[blue] at (axis cs:0.3,0.7) {$45°$};

\end{axis}
\end{tikzpicture}
\end{center}

\textbf{Paso 10:} Cálculo de potencia concentrada.

Si la irradiancia solar es 1000 W/m² y la eficiencia de reflexión es 90%, la potencia concentrada en el tubo es aproximadamente:
\[
P = 1000 \times \frac{16}{3} \times 0.9 \approx 4800 \text{ W}
\]

\textbf{Respuesta:}
\[
\boxed{
\begin{aligned}
&\text{Ecuación del espejo: } x^2 = 8y \text{ o } y = \frac{x^2}{8} \\
&\text{Posición del foco: } F(0, 2) \text{ metros} \\
&\text{Área reflectante: } \frac{16}{3} \approx 5.33 \text{ m}^2 \\
&\text{Ángulo de apertura total: } 90°
\end{aligned}
}
\]
\end{ejemplo}

\section{Ejercicios Inversos}

\begin{ejercicio}[title={El Ingeniero de Telecomunicaciones y el Diseño de Antena Satelital}]
Un ingeniero debe diseñar una antena parabólica para recepción satelital. La señal del satélite llega con un ángulo de elevación de $30°$ respecto a la horizontal. Si el diámetro de la antena debe ser 2.4 metros y se requiere que el receptor (LNB) esté a 90 cm del vértice del plato, determina:

\begin{enumerate}[label=\alph*)]
    \item La ecuación de la parábola que forma el plato
    \item La profundidad del plato parabólico
    \item El área de recepción efectiva
    \item La ganancia teórica de la antena si opera a 12 GHz
\end{enumerate}

Pista: La antena debe orientarse de modo que su eje apunte al satélite. Considera que la eficiencia de iluminación es del 65\%.
\end{ejercicio}

\begin{ejercicio}[title={El Arquitecto y el Puente Parabólico}]
Un arquitecto está diseñando un puente peatonal con arcos parabólicos. El puente debe salvar un río de 80 metros de ancho. Los puntos de apoyo en las orillas están al mismo nivel. El punto más alto del arco debe estar a 25 metros sobre el nivel de los apoyos. Además, por razones estructurales, se necesitan cables verticales de soporte cada 10 metros.

El arquitecto necesita determinar:
\begin{enumerate}[label=\alph*)]
    \item La ecuación del arco parabólico
    \item La longitud de cada cable de soporte
    \item La pendiente del arco en los puntos de apoyo
    \item El volumen de concreto necesario si el arco tiene un grosor uniforme de 0.5 metros
\end{enumerate}

Considera que el origen del sistema coordenado está en el punto medio entre los apoyos, al nivel del río.
\end{ejercicio}

\begin{ejercicio}[title={El Físico y la Trayectoria del Cohete}]
Un cohete experimental es lanzado con una velocidad inicial de 100 m/s formando un ángulo de $60°$ con la horizontal. Despreciando la resistencia del aire y considerando $g = 10$ m/s²:

\begin{enumerate}[label=\alph*)]
    \item Deduce la ecuación de la trayectoria parabólica en términos de las coordenadas $(x, y)$
    \item Encuentra las coordenadas del punto más alto de la trayectoria
    \item Determina el alcance horizontal del cohete
    \item Si se coloca un sensor a 300 metros del punto de lanzamiento y a 100 metros de altura, ¿detectará el paso del cohete?
\end{enumerate}

Nota: Usa las ecuaciones cinemáticas $x = v_0 \cos\theta \cdot t$ y $y = v_0 \sin\theta \cdot t - \frac{1}{2}gt^2$.
\end{ejercicio}

\section{Soluciones de Ejercicios Inversos}

\begin{solucion}[title={Solución: El Ingeniero de Telecomunicaciones y el Diseño de Antena Satelital}]

\textbf{Datos del problema:}
\begin{itemize}
    \item Diámetro: $D = 2.4$ m (radio $r = 1.2$ m)
    \item Distancia focal: $f = 0.9$ m (distancia vértice-receptor)
    \item Ángulo de elevación: $30°$
\end{itemize}

\textbf{Parte a) Ecuación de la parábola}

\textbf{Paso 1:} Establecer el sistema de coordenadas.

Colocamos el vértice en el origen con el eje de simetría en el eje $x$:
\[
y^2 = 4px
\]

\textbf{Paso 2:} Determinar el parámetro $p$.

El foco está a 0.9 m del vértice, entonces $p = 0.9$ m.

\textbf{Paso 3:} Escribir la ecuación.
\[
y^2 = 4(0.9)x = 3.6x
\]

\textbf{Parte b) Profundidad del plato}

\textbf{Paso 4:} Calcular la profundidad.

En el borde del plato, $y = 1.2$ m:
\begin{align*}
(1.2)^2 &= 3.6x \\
1.44 &= 3.6x \\
x &= \frac{1.44}{3.6} = 0.4 \text{ m}
\end{align*}

La profundidad del plato es 0.4 metros o 40 cm.

\textbf{Parte c) Área de recepción efectiva}

\textbf{Paso 5:} Calcular el área proyectada.

El área del círculo de diámetro 2.4 m es:
\[
A_{total} = \pi r^2 = \pi(1.2)^2 = 1.44\pi \text{ m}^2
\]

\textbf{Paso 6:} Aplicar la eficiencia.

Con 65\% de eficiencia de iluminación:
\[
A_{efectiva} = 0.65 \times 1.44\pi = 0.936\pi \approx 2.94 \text{ m}^2
\]

\textbf{Parte d) Ganancia teórica}

\textbf{Paso 7:} Calcular la longitud de onda.

A 12 GHz:
\[
\lambda = \frac{c}{f} = \frac{3 \times 10^8}{12 \times 10^9} = 0.025 \text{ m} = 2.5 \text{ cm}
\]

\textbf{Paso 8:} Calcular la ganancia.

La ganancia de una antena parabólica es:
\[
G = \eta \left(\frac{\pi D}{\lambda}\right)^2
\]

Donde $\eta = 0.65$ es la eficiencia:
\begin{align*}
G &= 0.65 \times \left(\frac{\pi \times 2.4}{0.025}\right)^2 \\
&= 0.65 \times (301.59)^2 \\
&= 0.65 \times 90,957 \\
&\approx 59,122
\end{align*}

En decibelios:
\[
G_{dB} = 10\log_{10}(59,122) \approx 47.7 \text{ dB}
\]

\textbf{Paso 9:} Graficar la antena.

\begin{center}
\begin{tikzpicture}
\begin{axis}[
    width=0.85\textwidth,
    height=0.7\textwidth,
    axis equal image,
    xmin=-0.5, xmax=1.5,
    ymin=-1.5, ymax=1.5,
    xlabel={$x$ (metros)},
    ylabel={$y$ (metros)},
    grid=major,
    axis lines=center,
]

% Parábola (antena)
\addplot[maincolor, very thick, smooth, samples=100, domain=-1.2:1.2]
    ({y^2/3.6}, {y});

% Vértice
\addplot[only marks, mark=*, mark size=3pt, red] coordinates {(0,0)};
\node[left] at (axis cs:0,0) {Vértice};

% Foco (LNB)
\addplot[only marks, mark=*, mark size=4pt, blue] coordinates {(0.9,0)};
\node[below] at (axis cs:0.9,0) {LNB};

% Extremos de la antena
\addplot[only marks, mark=*, mark size=3pt, green!60!black]
    coordinates {(0.4,1.2) (0.4,-1.2)};

% Rayos del satélite (paralelos)
\pgfplotsinvokeforeach{-1.0,-0.5,0,0.5,1.0}{
    \draw[->, yellow!80!orange, thick]
        (axis cs:1.3,#1) -- (axis cs:{#1*#1/3.6},#1);
    \draw[->, red!80!orange, thick]
        (axis cs:{#1*#1/3.6},#1) -- (axis cs:0.9,0);
}

% Diámetro
\draw[<->, thick, orange] (axis cs:0.4,-1.3) -- (axis cs:0.4,1.3)
    node[midway,right] {2.4 m};

\end{axis}
\end{tikzpicture}
\end{center}

\textbf{Respuesta:}
\[
\boxed{
\begin{aligned}
&\text{a) Ecuación: } y^2 = 3.6x \\
&\text{b) Profundidad: } 0.4 \text{ m} = 40 \text{ cm} \\
&\text{c) Área efectiva: } 2.94 \text{ m}^2 \\
&\text{d) Ganancia: } 47.7 \text{ dB}
\end{aligned}
}
\]
\end{solucion}

\begin{solucion}[title={Solución: El Arquitecto y el Puente Parabólico}]

\textbf{Datos:}
\begin{itemize}
    \item Ancho del río: 80 m
    \item Altura máxima del arco: 25 m
    \item Cables cada 10 m
    \item Grosor del arco: 0.5 m
\end{itemize}

\textbf{Parte a) Ecuación del arco parabólico}

\textbf{Paso 1:} Sistema de coordenadas.

Origen en el centro del río, al nivel del agua. El arco es una parábola invertida con vértice en $(0, 25)$.

\textbf{Paso 2:} Forma de la ecuación.

Para una parábola con vértice en $(0, 25)$ y que abre hacia abajo:
\[
y - 25 = -a(x - 0)^2
\]
\[
y = 25 - ax^2
\]

\textbf{Paso 3:} Determinar $a$ usando los puntos de apoyo.

Los apoyos están en $(\pm 40, 0)$:
\begin{align*}
0 &= 25 - a(40)^2 \\
a(1600) &= 25 \\
a &= \frac{25}{1600} = \frac{1}{64}
\end{align*}

\textbf{Paso 4:} Ecuación final.
\[
y = 25 - \frac{x^2}{64}
\]

\textbf{Parte b) Longitud de los cables de soporte}

\textbf{Paso 5:} Calcular alturas en cada posición.

Cables en $x = 0, \pm 10, \pm 20, \pm 30, \pm 40$:

\begin{align*}
x = 0: \quad y &= 25 - 0 = 25 \text{ m} \\
x = \pm 10: \quad y &= 25 - \frac{100}{64} = 25 - 1.5625 = 23.4375 \text{ m} \\
x = \pm 20: \quad y &= 25 - \frac{400}{64} = 25 - 6.25 = 18.75 \text{ m} \\
x = \pm 30: \quad y &= 25 - \frac{900}{64} = 25 - 14.0625 = 10.9375 \text{ m} \\
x = \pm 40: \quad y &= 25 - \frac{1600}{64} = 0 \text{ m}
\end{align*}

\textbf{Parte c) Pendiente en los puntos de apoyo}

\textbf{Paso 6:} Calcular la derivada.
\[
\frac{dy}{dx} = -\frac{2x}{64} = -\frac{x}{32}
\]

\textbf{Paso 7:} Evaluar en los apoyos.

En $x = 40$: $m = -\frac{40}{32} = -1.25$

En $x = -40$: $m = -\frac{-40}{32} = 1.25$

El ángulo con la horizontal es $\arctan(1.25) \approx 51.34°$

\textbf{Parte d) Volumen de concreto}

\textbf{Paso 8:} Calcular la longitud del arco.

\[
L = \int_{-40}^{40} \sqrt{1 + \left(\frac{dy}{dx}\right)^2} \, dx = \int_{-40}^{40} \sqrt{1 + \frac{x^2}{1024}} \, dx
\]

Por simetría:
\[
L = 2\int_{0}^{40} \sqrt{1 + \frac{x^2}{1024}} \, dx \approx 85.77 \text{ m}
\]

\textbf{Paso 9:} Calcular el volumen.

Con grosor uniforme de 0.5 m:
\[
V = L \times \text{área de sección} = 85.77 \times 0.5 \times 0.5 = 21.44 \text{ m}^3
\]

\textbf{Paso 10:} Graficar el puente.

\begin{center}
\begin{tikzpicture}
\begin{axis}[
    width=0.95\textwidth,
    height=0.6\textwidth,
    xmin=-45, xmax=45,
    ymin=-2, ymax=28,
    xlabel={Distancia desde el centro (m)},
    ylabel={Altura (m)},
    grid=major,
    axis lines=center,
    xtick={-40,-30,-20,-10,0,10,20,30,40},
    ytick={0,5,10,15,20,25},
]

% Arco parabólico
\addplot[maincolor, very thick, smooth, samples=100, domain=-40:40]
    {25 - x^2/64};

% Cables de soporte
\pgfplotsinvokeforeach{-40,-30,-20,-10,0,10,20,30,40}{
    \addplot[thick, gray, domain=0:{25-#1*#1/64}]
        coordinates {(#1,0) (#1,{25-#1*#1/64})};
}

% Puntos de apoyo
\addplot[only marks, mark=*, mark size=3pt, red]
    coordinates {(-40,0) (40,0)};

% Vértice
\addplot[only marks, mark=*, mark size=3pt, blue] coordinates {(0,25)};
\node[above] at (axis cs:0,25) {Altura máxima};

% Río
\addplot[thick, cyan, domain=-45:45] coordinates {(-45,0) (45,0)};
\node[below] at (axis cs:0,0) {Nivel del río};

% Dimensiones
\draw[<->, thick, orange] (axis cs:-40,-1) -- (axis cs:40,-1)
    node[midway,below] {80 m};

\end{axis}
\end{tikzpicture}
\end{center}

\textbf{Respuesta:}
\[
\boxed{
\begin{aligned}
&\text{a) Ecuación: } y = 25 - \frac{x^2}{64} \\
&\text{b) Cables: Centro: 25m, } \pm 10\text{m: 23.44m, } \pm 20\text{m: 18.75m,} \\
&\quad\quad\quad\quad \pm 30\text{m: 10.94m, Apoyos: 0m} \\
&\text{c) Pendiente en apoyos: } \pm 1.25 \text{ (ángulo } 51.34°) \\
&\text{d) Volumen de concreto: } 21.44 \text{ m}^3
\end{aligned}
}
\]
\end{solucion}

\begin{solucion}[title={Solución: El Físico y la Trayectoria del Cohete}]

\textbf{Datos:}
\begin{itemize}
    \item Velocidad inicial: $v_0 = 100$ m/s
    \item Ángulo de lanzamiento: $\theta = 60°$
    \item Gravedad: $g = 10$ m/s²
    \item Componentes de velocidad: $v_{0x} = 100\cos 60° = 50$ m/s, $v_{0y} = 100\sin 60° = 50\sqrt{3}$ m/s
\end{itemize}

\textbf{Parte a) Ecuación de la trayectoria}

\textbf{Paso 1:} Ecuaciones paramétricas.
\begin{align*}
x &= v_{0x} t = 50t \\
y &= v_{0y} t - \frac{1}{2}gt^2 = 50\sqrt{3}t - 5t^2
\end{align*}

\textbf{Paso 2:} Eliminar el parámetro $t$.

De $x = 50t$: $t = \frac{x}{50}$

Sustituyendo en $y$:
\begin{align*}
y &= 50\sqrt{3} \cdot \frac{x}{50} - 5\left(\frac{x}{50}\right)^2 \\
y &= \sqrt{3}x - \frac{5x^2}{2500} \\
y &= \sqrt{3}x - \frac{x^2}{500}
\end{align*}

\textbf{Parte b) Punto más alto de la trayectoria}

\textbf{Paso 3:} Encontrar el tiempo en el punto más alto.

En el punto más alto, $v_y = 0$:
\begin{align*}
v_y &= v_{0y} - gt = 0 \\
50\sqrt{3} - 10t &= 0 \\
t &= 5\sqrt{3} \text{ s}
\end{align*}

\textbf{Paso 4:} Coordenadas del punto más alto.
\begin{align*}
x_{max} &= 50(5\sqrt{3}) = 250\sqrt{3} \approx 433.01 \text{ m} \\
y_{max} &= 50\sqrt{3}(5\sqrt{3}) - 5(5\sqrt{3})^2 \\
&= 750 - 375 = 375 \text{ m}
\end{align*}

\textbf{Parte c) Alcance horizontal}

\textbf{Paso 5:} Tiempo de vuelo total.

El cohete toca el suelo cuando $y = 0$:
\begin{align*}
0 &= 50\sqrt{3}t - 5t^2 \\
0 &= t(50\sqrt{3} - 5t) \\
t &= 0 \text{ o } t = 10\sqrt{3} \text{ s}
\end{align*}

\textbf{Paso 6:} Alcance horizontal.
\[
R = v_{0x} \cdot t_{total} = 50 \cdot 10\sqrt{3} = 500\sqrt{3} \approx 866.03 \text{ m}
\]

\textbf{Parte d) Detección del sensor}

\textbf{Paso 7:} Verificar si el punto $(300, 100)$ está en la trayectoria.

Sustituyendo $x = 300$ en la ecuación:
\begin{align*}
y &= \sqrt{3}(300) - \frac{(300)^2}{500} \\
&= 300\sqrt{3} - \frac{90000}{500} \\
&= 300\sqrt{3} - 180 \\
&\approx 519.62 - 180 = 339.62 \text{ m}
\end{align*}

\textbf{Paso 8:} Comparación.

El cohete pasa a 339.62 m de altura cuando $x = 300$ m.
Como el sensor está a solo 100 m de altura, SÍ detectará el cohete pasando por encima.

\textbf{Paso 9:} Graficar la trayectoria completa.

\begin{center}
\begin{tikzpicture}
\begin{axis}[
    width=0.95\textwidth,
    height=0.65\textwidth,
    xmin=0, xmax=900,
    ymin=0, ymax=400,
    xlabel={Distancia horizontal (m)},
    ylabel={Altura (m)},
    grid=major,
    axis lines=left,
    xtick={0,100,200,300,400,500,600,700,800,900},
    ytick={0,100,200,300,400},
]

% Trayectoria del cohete
\addplot[maincolor, very thick, smooth, samples=100, domain=0:866.03]
    {1.732*x - x^2/500};

% Punto de lanzamiento
\addplot[only marks, mark=*, mark size=3pt, green!60!black] coordinates {(0,0)};
\node[above right] at (axis cs:0,0) {Lanzamiento};

% Punto más alto
\addplot[only marks, mark=*, mark size=3pt, blue] coordinates {(433.01,375)};
\node[above] at (axis cs:433.01,375) {Altura máxima};
\draw[dashed, blue] (axis cs:433.01,0) -- (axis cs:433.01,375);

% Punto de impacto
\addplot[only marks, mark=*, mark size=3pt, red] coordinates {(866.03,0)};
\node[below] at (axis cs:866.03,0) {Impacto};

% Posición del sensor
\addplot[only marks, mark=square*, mark size=4pt, orange] coordinates {(300,100)};
\node[below] at (axis cs:300,100) {Sensor};

% Altura del cohete en x=300
\addplot[only marks, mark=*, mark size=3pt, purple] coordinates {(300,339.62)};
\draw[dashed, purple] (axis cs:300,100) -- (axis cs:300,339.62);
\node[right] at (axis cs:300,220) {239.62 m};

% Vector velocidad inicial
\draw[->, very thick, red] (axis cs:0,0) -- (axis cs:50,86.6)
    node[midway,above,sloped] {$v_0$};

\end{axis}
\end{tikzpicture}
\end{center}

\textbf{Paso 10:} Verificación usando fórmulas de alcance.

Alcance teórico:
\[
R = \frac{v_0^2 \sin(2\theta)}{g} = \frac{100^2 \sin(120°)}{10} = \frac{10000 \cdot \frac{\sqrt{3}}{2}}{10} = 500\sqrt{3} \quad \checkmark
\]

\textbf{Respuesta:}
\[
\boxed{
\begin{aligned}
&\text{a) Ecuación: } y = \sqrt{3}x - \frac{x^2}{500} \\
&\text{b) Punto más alto: } (250\sqrt{3}, 375) \approx (433.01, 375) \text{ m} \\
&\text{c) Alcance: } 500\sqrt{3} \approx 866.03 \text{ m} \\
&\text{d) Sí, el sensor detectará el cohete (pasa 239.62 m por encima)}
\end{aligned}
}
\]
\end{solucion}

\end{document}
% PARTE 3: EJERCICIOS PROPUESTOS Y SOLUCIONES
% Guía sobre la Parábola - Geometría Analítica

\section{Ejercicios Propuestos}

Los siguientes ejercicios están diseñados para reforzar tu comprensión de las parábolas. Intenta resolverlos antes de consultar las soluciones detalladas.

\begin{ejercicio}[title={Ejercicio 1: Ecuación Canónica - Vértice en el Origen}]
Para cada parábola con vértice en el origen, escribe su ecuación canónica y determina el foco y la directriz:
\begin{enumerate}[label=\alph*)]
    \item Parábola que abre hacia arriba con $p = 3$
    \item Parábola que abre hacia la izquierda con distancia focal de 5 unidades
\end{enumerate}
\end{ejercicio}

\begin{ejercicio}[title={Ejercicio 2: Ecuación Canónica - Vértice Trasladado}]
Encuentra la ecuación canónica de cada parábola:
\begin{enumerate}[label=\alph*)]
    \item Vértice en $(2, -1)$, abre hacia la derecha, $p = 4$
    \item Vértice en $(-3, 5)$, abre hacia abajo, foco a 2 unidades del vértice
\end{enumerate}
\end{ejercicio}

\begin{ejercicio}[title={Ejercicio 3: Identificación de Elementos}]
Para cada ecuación, encuentra el vértice, foco, directriz y eje de simetría:
\begin{enumerate}[label=\alph*)]
    \item $(x - 4)^2 = 12(y + 2)$
    \item $(y + 1)^2 = -8(x - 3)$
\end{enumerate}
\end{ejercicio}

\begin{ejercicio}[title={Ejercicio 4: De Forma General a Canónica}]
Transforma cada ecuación a su forma canónica completando el cuadrado:
\begin{enumerate}[label=\alph*)]
    \item $x^2 - 6x - 8y + 17 = 0$
    \item $y^2 + 4y + 12x - 8 = 0$
\end{enumerate}
\end{ejercicio}

\begin{ejercicio}[title={Ejercicio 5: Parábola Dados Foco y Directriz}]
Encuentra la ecuación de la parábola con:
\begin{enumerate}[label=\alph*)]
    \item Foco en $F(0, 4)$ y directriz $y = -4$
    \item Foco en $F(5, 2)$ y directriz $x = -1$
\end{enumerate}
\end{ejercicio}

\begin{ejercicio}[title={Ejercicio 6: Parábola por Tres Puntos}]
Encuentra la ecuación de la parábola vertical que pasa por los puntos:
\begin{enumerate}[label=\alph*)]
    \item $A(0, 3)$, $B(2, -1)$ y $C(4, 3)$
    \item $P(1, 0)$, $Q(3, 4)$ y $R(-1, 4)$
\end{enumerate}
\end{ejercicio}

\begin{ejercicio}[title={Ejercicio 7: Determinación de Orientación y Elementos}]
Para cada ecuación, determina la orientación de la parábola y todos sus elementos geométricos:
\begin{enumerate}[label=\alph*)]
    \item $x^2 + 8x - 4y + 20 = 0$
    \item $2y^2 - 12y - 5x + 13 = 0$
\end{enumerate}
\end{ejercicio}

\begin{ejercicio}[title={Ejercicio 8: Aplicación Práctica - Antena Parabólica}]
Una antena parabólica tiene 8 metros de diámetro y 2 metros de profundidad.
\begin{enumerate}[label=\alph*)]
    \item Encuentra la ecuación de la parábola considerando el vértice en el origen
    \item Determina la posición del foco (receptor de señales)
    \item Si un rayo llega paralelo al eje de simetría a 3 metros del vértice, ¿a qué distancia del vértice se encuentra el punto donde el rayo toca la parábola?
\end{enumerate}
\end{ejercicio}

\newpage

\section{Soluciones Detalladas}

\begin{solucion}[title={Solución Ejercicio 1}]
\textbf{Parte a)} Parábola con vértice en el origen, abre hacia arriba, $p = 3$

\textbf{Paso 1:} Identificar la orientación y el valor de $p$.
- Abre hacia arriba: forma $x^2 = 4py$
- $p = 3$ (positivo porque abre hacia arriba)

\textbf{Paso 2:} Escribir la ecuación canónica.
\[
x^2 = 4py = 4(3)y = 12y
\]
\[
\boxed{x^2 = 12y}
\]

\textbf{Paso 3:} Determinar el foco.
- Como abre hacia arriba, el foco está en $(0, p)$
\[
F(0, 3)
\]

\textbf{Paso 4:} Determinar la directriz.
- La directriz es una línea horizontal: $y = -p$
\[
y = -3
\]

\textbf{Paso 5:} Verificación de la definición.
- Punto en la parábola: si $x = 6$, entonces $36 = 12y$, así $y = 3$
- Punto $P(6, 3)$
- Distancia al foco: $d(P, F) = \sqrt{(6-0)^2 + (3-3)^2} = 6$
- Distancia a la directriz: $d(P, y=-3) = |3-(-3)| = 6$ ✓

\textbf{Parte b)} Parábola con vértice en el origen, abre hacia la izquierda, distancia focal 5

\textbf{Paso 1:} Interpretar la información.
- Abre hacia la izquierda: forma $y^2 = 4px$ con $p < 0$
- Distancia focal = $|p| = 5$, entonces $p = -5$

\textbf{Paso 2:} Escribir la ecuación canónica.
\[
y^2 = 4px = 4(-5)x = -20x
\]
\[
\boxed{y^2 = -20x}
\]

\textbf{Paso 3:} Determinar el foco.
- Como abre hacia la izquierda, el foco está en $(p, 0)$
\[
F(-5, 0)
\]

\textbf{Paso 4:} Determinar la directriz.
- La directriz es una línea vertical: $x = -p$
\[
x = 5
\]

\textbf{Paso 5:} Graficar para visualizar.

\begin{center}
\begin{tikzpicture}[scale=0.4]
    \begin{axis}[
        axis lines=middle,
        xlabel={$x$},
        ylabel={$y$},
        xmin=-8, xmax=8,
        ymin=-10, ymax=10,
        grid=major,
        width=0.9\textwidth,
        height=0.6\textwidth,
        axis equal image
    ]
    % Parte a: x^2 = 12y
    \addplot[domain=-8:8, samples=100, thick, blue] {x^2/12};
    \addplot[only marks, mark=*, blue] coordinates {(0,3)};
    \node[blue] at (axis cs: 0,3) [above right] {$F(0,3)$};
    \addplot[dashed, blue, domain=-8:8] {-3};
    \node[blue] at (axis cs: 7,-3) [below] {$y=-3$};

    % Parte b: y^2 = -20x
    \addplot[domain=-8:0, samples=100, thick, red, parametric] ({-t^2/20}, {t});
    \addplot[only marks, mark=*, red] coordinates {(-5,0)};
    \node[red] at (axis cs: -5,0) [below left] {$F(-5,0)$};
    \addplot[dashed, red, samples=2] coordinates {(5,-10) (5,10)};
    \node[red] at (axis cs: 5,8) [right] {$x=5$};
    \end{axis}
\end{tikzpicture}
\end{center}

\textbf{Respuesta completa:}
\begin{itemize}
    \item \textbf{Parte a:} Ecuación: $x^2 = 12y$, Foco: $F(0,3)$, Directriz: $y = -3$
    \item \textbf{Parte b:} Ecuación: $y^2 = -20x$, Foco: $F(-5,0)$, Directriz: $x = 5$
\end{itemize}
\end{solucion}

\begin{solucion}[title={Solución Ejercicio 2}]
\textbf{Parte a)} Vértice en $(2, -1)$, abre hacia la derecha, $p = 4$

\textbf{Paso 1:} Identificar la forma de la ecuación.
- Vértice: $V(h, k) = (2, -1)$
- Abre hacia la derecha: eje horizontal
- Forma: $(y - k)^2 = 4p(x - h)$

\textbf{Paso 2:} Sustituir valores.
\[
(y - (-1))^2 = 4(4)(x - 2)
\]
\[
(y + 1)^2 = 16(x - 2)
\]
\[
\boxed{(y + 1)^2 = 16(x - 2)}
\]

\textbf{Paso 3:} Determinar el foco.
- El foco está a $p = 4$ unidades a la derecha del vértice
\[
F(h + p, k) = F(2 + 4, -1) = F(6, -1)
\]

\textbf{Paso 4:} Determinar la directriz.
- La directriz es vertical: $x = h - p$
\[
x = 2 - 4 = -2
\]

\textbf{Paso 5:} Verificar con un punto.
- Si $x = 6$: $(y + 1)^2 = 16(6 - 2) = 64$
- Entonces $y + 1 = \pm 8$, así $y = 7$ o $y = -9$
- Puntos: $(6, 7)$ y $(6, -9)$

\textbf{Parte b)} Vértice en $(-3, 5)$, abre hacia abajo, foco a 2 unidades

\textbf{Paso 1:} Determinar el valor de $p$.
- Abre hacia abajo: $p < 0$
- Distancia del foco al vértice: $|p| = 2$
- Por tanto: $p = -2$

\textbf{Paso 2:} Escribir la ecuación.
- Vértice: $V(-3, 5)$
- Abre hacia abajo: eje vertical
- Forma: $(x - h)^2 = 4p(y - k)$
\[
(x - (-3))^2 = 4(-2)(y - 5)
\]
\[
(x + 3)^2 = -8(y - 5)
\]
\[
\boxed{(x + 3)^2 = -8(y - 5)}
\]

\textbf{Paso 3:} Determinar el foco.
- El foco está a 2 unidades debajo del vértice
\[
F(h, k + p) = F(-3, 5 + (-2)) = F(-3, 3)
\]

\textbf{Paso 4:} Determinar la directriz.
- La directriz es horizontal: $y = k - p$
\[
y = 5 - (-2) = 7
\]

\textbf{Paso 5:} Desarrollar a forma general.
\begin{align}
(x + 3)^2 &= -8(y - 5)\\
x^2 + 6x + 9 &= -8y + 40\\
x^2 + 6x + 8y - 31 &= 0
\end{align}

\textbf{Respuestas:}
\begin{itemize}
    \item \textbf{Parte a:} $(y + 1)^2 = 16(x - 2)$, Foco: $F(6, -1)$, Directriz: $x = -2$
    \item \textbf{Parte b:} $(x + 3)^2 = -8(y - 5)$, Foco: $F(-3, 3)$, Directriz: $y = 7$
\end{itemize}
\end{solucion}

\begin{solucion}[title={Solución Ejercicio 3}]
\textbf{Parte a)} $(x - 4)^2 = 12(y + 2)$

\textbf{Paso 1:} Identificar la forma y orientación.
- Forma: $(x - h)^2 = 4p(y - k)$
- Variable al cuadrado: $x$ → eje vertical
- Coeficiente positivo → abre hacia arriba

\textbf{Paso 2:} Identificar $h$, $k$ y $4p$.
- Comparando: $(x - 4)^2 = 12(y - (-2))$
- $h = 4$, $k = -2$
- $4p = 12$, entonces $p = 3$

\textbf{Paso 3:} Determinar el vértice.
\[
V(h, k) = V(4, -2)
\]

\textbf{Paso 4:} Determinar el foco.
- Como abre hacia arriba: $F(h, k + p)$
\[
F(4, -2 + 3) = F(4, 1)
\]

\textbf{Paso 5:} Determinar la directriz.
- Línea horizontal: $y = k - p$
\[
y = -2 - 3 = -5
\]

\textbf{Paso 6:} Determinar el eje de simetría.
- Línea vertical que pasa por el vértice
\[
x = 4
\]

\textbf{Parte b)} $(y + 1)^2 = -8(x - 3)$

\textbf{Paso 1:} Identificar la forma y orientación.
- Forma: $(y - k)^2 = 4p(x - h)$
- Variable al cuadrado: $y$ → eje horizontal
- Coeficiente negativo → abre hacia la izquierda

\textbf{Paso 2:} Identificar $h$, $k$ y $4p$.
- Comparando: $(y - (-1))^2 = -8(x - 3)$
- $h = 3$, $k = -1$
- $4p = -8$, entonces $p = -2$

\textbf{Paso 3:} Determinar el vértice.
\[
V(h, k) = V(3, -1)
\]

\textbf{Paso 4:} Determinar el foco.
- Como abre hacia la izquierda: $F(h + p, k)$
\[
F(3 + (-2), -1) = F(1, -1)
\]

\textbf{Paso 5:} Determinar la directriz.
- Línea vertical: $x = h - p$
\[
x = 3 - (-2) = 5
\]

\textbf{Paso 6:} Determinar el eje de simetría.
- Línea horizontal que pasa por el vértice
\[
y = -1
\]

\textbf{Paso 7:} Graficar ambas parábolas.

\begin{center}
\begin{tikzpicture}[scale=0.5]
    \begin{axis}[
        axis lines=middle,
        xlabel={$x$},
        ylabel={$y$},
        xmin=-2, xmax=10,
        ymin=-7, ymax=5,
        grid=major,
        width=0.95\textwidth,
        height=0.65\textwidth,
        axis equal image
    ]
    % Parte a
    \addplot[domain=0:8, samples=100, thick, blue] ({x}, {(x-4)^2/12 - 2});
    \addplot[only marks, mark=*, blue] coordinates {(4,-2) (4,1)};
    \node[blue] at (axis cs: 4,-2) [below right] {$V(4,-2)$};
    \node[blue] at (axis cs: 4,1) [above right] {$F(4,1)$};
    \addplot[dashed, blue] coordinates {(-2,-5) (10,-5)};
    \node[blue] at (axis cs: 8,-5) [below] {$y=-5$};

    % Parte b
    \addplot[domain=-5:3, samples=100, thick, red, parametric] ({3-(t+1)^2/8}, {t});
    \addplot[only marks, mark=*, red] coordinates {(3,-1) (1,-1)};
    \node[red] at (axis cs: 3,-1) [above left] {$V(3,-1)$};
    \node[red] at (axis cs: 1,-1) [below left] {$F(1,-1)$};
    \addplot[dashed, red] coordinates {(5,-7) (5,5)};
    \node[red] at (axis cs: 5,3) [right] {$x=5$};
    \end{axis}
\end{tikzpicture}
\end{center}

\textbf{Respuestas completas:}
\begin{itemize}
    \item \textbf{Parte a:} $V(4,-2)$, $F(4,1)$, Directriz: $y=-5$, Eje: $x=4$
    \item \textbf{Parte b:} $V(3,-1)$, $F(1,-1)$, Directriz: $x=5$, Eje: $y=-1$
\end{itemize}
\end{solucion}

\begin{solucion}[title={Solución Ejercicio 4}]
\textbf{Parte a)} $x^2 - 6x - 8y + 17 = 0$

\textbf{Paso 1:} Agrupar términos con la misma variable.
\[
x^2 - 6x = 8y - 17
\]

\textbf{Paso 2:} Completar el cuadrado para $x$.
- Coeficiente de $x$: $-6$
- Mitad del coeficiente: $-3$
- Cuadrado: $(-3)^2 = 9$

\textbf{Paso 3:} Añadir y restar 9.
\[
x^2 - 6x + 9 = 8y - 17 + 9
\]
\[
(x - 3)^2 = 8y - 8
\]

\textbf{Paso 4:} Factorizar el lado derecho.
\[
(x - 3)^2 = 8(y - 1)
\]
\[
\boxed{(x - 3)^2 = 8(y - 1)}
\]

\textbf{Paso 5:} Identificar elementos.
- Vértice: $V(3, 1)$
- $4p = 8$, entonces $p = 2$
- Foco: $F(3, 1 + 2) = F(3, 3)$
- Directriz: $y = 1 - 2 = -1$

\textbf{Parte b)} $y^2 + 4y + 12x - 8 = 0$

\textbf{Paso 1:} Reorganizar la ecuación.
\[
y^2 + 4y = -12x + 8
\]

\textbf{Paso 2:} Completar el cuadrado para $y$.
- Coeficiente de $y$: $4$
- Mitad del coeficiente: $2$
- Cuadrado: $(2)^2 = 4$

\textbf{Paso 3:} Añadir 4 a ambos lados.
\[
y^2 + 4y + 4 = -12x + 8 + 4
\]
\[
(y + 2)^2 = -12x + 12
\]

\textbf{Paso 4:} Factorizar el lado derecho.
\[
(y + 2)^2 = -12(x - 1)
\]
\[
\boxed{(y + 2)^2 = -12(x - 1)}
\]

\textbf{Paso 5:} Identificar elementos.
- Vértice: $V(1, -2)$
- $4p = -12$, entonces $p = -3$
- Abre hacia la izquierda
- Foco: $F(1 - 3, -2) = F(-2, -2)$
- Directriz: $x = 1 - (-3) = 4$

\textbf{Paso 6:} Verificación expandiendo.
\begin{align}
(y + 2)^2 &= -12(x - 1)\\
y^2 + 4y + 4 &= -12x + 12\\
y^2 + 4y + 12x - 8 &= 0 \checkmark
\end{align}

\textbf{Respuestas:}
\begin{itemize}
    \item \textbf{Parte a:} $(x - 3)^2 = 8(y - 1)$, $V(3,1)$, $F(3,3)$, Directriz: $y = -1$
    \item \textbf{Parte b:} $(y + 2)^2 = -12(x - 1)$, $V(1,-2)$, $F(-2,-2)$, Directriz: $x = 4$
\end{itemize}
\end{solucion}

\begin{solucion}[title={Solución Ejercicio 5}]
\textbf{Parte a)} Foco en $F(0, 4)$ y directriz $y = -4$

\textbf{Paso 1:} Encontrar el vértice.
- El vértice está a mitad de camino entre el foco y la directriz
- Coordenada $x$ del vértice: $x = 0$ (mismo que el foco)
- Coordenada $y$ del vértice: $y = \frac{4 + (-4)}{2} = 0$
- Vértice: $V(0, 0)$

\textbf{Paso 2:} Calcular $p$.
- Distancia del vértice al foco: $p = 4 - 0 = 4$
- Como el foco está arriba del vértice: $p > 0$

\textbf{Paso 3:} Determinar la orientación.
- Directriz horizontal, foco arriba → abre hacia arriba
- Forma: $x^2 = 4py$

\textbf{Paso 4:} Escribir la ecuación.
\[
x^2 = 4(4)y = 16y
\]
\[
\boxed{x^2 = 16y}
\]

\textbf{Paso 5:} Verificación con la definición.
- Punto de prueba: $(4, 1)$ debe estar en la parábola
- Verificar: $16 = 16(1)$ ✓
- Distancia al foco: $d = \sqrt{(4-0)^2 + (1-4)^2} = \sqrt{16 + 9} = 5$
- Distancia a la directriz: $d = |1 - (-4)| = 5$ ✓

\textbf{Parte b)} Foco en $F(5, 2)$ y directriz $x = -1$

\textbf{Paso 1:} Encontrar el vértice.
- Directriz vertical en $x = -1$
- Foco en $x = 5$
- Coordenada $x$ del vértice: $x = \frac{5 + (-1)}{2} = 2$
- Coordenada $y$ del vértice: $y = 2$ (mismo que el foco)
- Vértice: $V(2, 2)$

\textbf{Paso 2:} Calcular $p$.
- Distancia del vértice al foco: $p = 5 - 2 = 3$
- Como el foco está a la derecha: $p > 0$

\textbf{Paso 3:} Determinar la orientación.
- Directriz vertical, foco a la derecha → abre hacia la derecha
- Forma: $(y - k)^2 = 4p(x - h)$

\textbf{Paso 4:} Escribir la ecuación.
\[
(y - 2)^2 = 4(3)(x - 2)
\]
\[
\boxed{(y - 2)^2 = 12(x - 2)}
\]

\textbf{Paso 5:} Desarrollar a forma general.
\begin{align}
(y - 2)^2 &= 12(x - 2)\\
y^2 - 4y + 4 &= 12x - 24\\
y^2 - 4y - 12x + 28 &= 0
\end{align}

\textbf{Paso 6:} Graficar ambas parábolas.

\begin{center}
\begin{tikzpicture}[scale=0.35]
    \begin{axis}[
        axis lines=middle,
        xlabel={$x$},
        ylabel={$y$},
        xmin=-6, xmax=8,
        ymin=-6, ymax=8,
        grid=major,
        width=0.9\textwidth,
        height=0.9\textwidth,
        axis equal image
    ]
    % Parte a: x^2 = 16y
    \addplot[domain=-8:8, samples=100, thick, blue] {x^2/16};
    \addplot[only marks, mark=*, blue] coordinates {(0,0) (0,4)};
    \node[blue] at (axis cs: 0,0) [below left] {$V$};
    \node[blue] at (axis cs: 0,4) [above right] {$F(0,4)$};
    \addplot[dashed, blue, domain=-8:8] {-4};
    \node[blue] at (axis cs: 6,-4) [below] {$y=-4$};

    % Parte b: (y-2)^2 = 12(x-2)
    \addplot[domain=-2:8, samples=100, thick, red, parametric] ({2 + t^2/12}, {2 + t});
    \addplot[only marks, mark=*, red] coordinates {(2,2) (5,2)};
    \node[red] at (axis cs: 2,2) [below left] {$V(2,2)$};
    \node[red] at (axis cs: 5,2) [above] {$F(5,2)$};
    \addplot[dashed, red] coordinates {(-1,-6) (-1,8)};
    \node[red] at (axis cs: -1,6) [left] {$x=-1$};
    \end{axis}
\end{tikzpicture}
\end{center}

\textbf{Respuestas:}
\begin{itemize}
    \item \textbf{Parte a:} $x^2 = 16y$ o en forma general: $x^2 - 16y = 0$
    \item \textbf{Parte b:} $(y - 2)^2 = 12(x - 2)$ o $y^2 - 4y - 12x + 28 = 0$
\end{itemize}
\end{solucion}

\begin{solucion}[title={Solución Ejercicio 6}]
\textbf{Parte a)} Parábola vertical por $A(0, 3)$, $B(2, -1)$ y $C(4, 3)$

\textbf{Paso 1:} Plantear la forma general de una parábola vertical.
\[
y = ax^2 + bx + c
\]

\textbf{Paso 2:} Sustituir cada punto para obtener un sistema de ecuaciones.

Para $A(0, 3)$:
\[
3 = a(0)^2 + b(0) + c \Rightarrow c = 3
\]

Para $B(2, -1)$:
\[
-1 = a(2)^2 + b(2) + 3 \Rightarrow 4a + 2b = -4
\]

Para $C(4, 3)$:
\[
3 = a(4)^2 + b(4) + 3 \Rightarrow 16a + 4b = 0
\]

\textbf{Paso 3:} Resolver el sistema.
De la tercera ecuación: $16a + 4b = 0 \Rightarrow b = -4a$

Sustituyendo en la segunda ecuación:
\[
4a + 2(-4a) = -4
\]
\[
4a - 8a = -4
\]
\[
-4a = -4
\]
\[
a = 1
\]

Por lo tanto: $b = -4(1) = -4$

\textbf{Paso 4:} Escribir la ecuación.
\[
y = x^2 - 4x + 3
\]

\textbf{Paso 5:} Convertir a forma canónica.
\begin{align}
y &= x^2 - 4x + 3\\
y &= (x^2 - 4x + 4) - 4 + 3\\
y &= (x - 2)^2 - 1\\
y + 1 &= (x - 2)^2
\end{align}
\[
\boxed{(x - 2)^2 = y + 1}
\]

\textbf{Paso 6:} Verificar con los puntos originales.
- $A(0, 3)$: $(0-2)^2 = 4 = 3+1$ ✓
- $B(2, -1)$: $(2-2)^2 = 0 = -1+1$ ✓
- $C(4, 3)$: $(4-2)^2 = 4 = 3+1$ ✓

\textbf{Parte b)} Parábola vertical por $P(1, 0)$, $Q(3, 4)$ y $R(-1, 4)$

\textbf{Paso 1:} Usar la forma general $y = ax^2 + bx + c$.

\textbf{Paso 2:} Sustituir los puntos.

Para $P(1, 0)$:
\[
0 = a(1)^2 + b(1) + c \Rightarrow a + b + c = 0
\]

Para $Q(3, 4)$:
\[
4 = a(3)^2 + b(3) + c \Rightarrow 9a + 3b + c = 4
\]

Para $R(-1, 4)$:
\[
4 = a(-1)^2 + b(-1) + c \Rightarrow a - b + c = 4
\]

\textbf{Paso 3:} Resolver el sistema.
De la primera ecuación: $c = -a - b$

Restando la tercera ecuación de la primera:
\[
(a + b + c) - (a - b + c) = 0 - 4
\]
\[
2b = -4 \Rightarrow b = -2
\]

Sustituyendo $b = -2$ y $c = -a - b = -a + 2$ en la segunda ecuación:
\[
9a + 3(-2) + (-a + 2) = 4
\]
\[
9a - 6 - a + 2 = 4
\]
\[
8a - 4 = 4
\]
\[
8a = 8 \Rightarrow a = 1
\]

Por lo tanto: $c = -1 - (-2) = 1$

\textbf{Paso 4:} Escribir la ecuación.
\[
y = x^2 - 2x + 1
\]

\textbf{Paso 5:} Reconocer el cuadrado perfecto.
\[
y = (x - 1)^2
\]
\[
\boxed{(x - 1)^2 = y}
\]

\textbf{Paso 6:} Identificar elementos.
- Vértice: $V(1, 0)$ (¡coincide con el punto $P$!)
- $4p = 1$, entonces $p = 1/4$
- Foco: $F(1, 1/4)$
- Directriz: $y = -1/4$

\textbf{Respuestas:}
\begin{itemize}
    \item \textbf{Parte a:} $(x - 2)^2 = y + 1$ o $y = x^2 - 4x + 3$
    \item \textbf{Parte b:} $(x - 1)^2 = y$ o $y = x^2 - 2x + 1$
\end{itemize}
\end{solucion}

\begin{solucion}[title={Solución Ejercicio 7}]
\textbf{Parte a)} $x^2 + 8x - 4y + 20 = 0$

\textbf{Paso 1:} Determinar la orientación.
- Variable al cuadrado: $x^2$ → eje vertical
- Reorganizar: $x^2 + 8x = 4y - 20$

\textbf{Paso 2:} Completar el cuadrado.
\[
x^2 + 8x + 16 = 4y - 20 + 16
\]
\[
(x + 4)^2 = 4y - 4
\]
\[
(x + 4)^2 = 4(y - 1)
\]

\textbf{Paso 3:} Identificar la orientación definitiva.
- Coeficiente positivo → abre hacia arriba

\textbf{Paso 4:} Determinar todos los elementos.
- Vértice: $V(-4, 1)$
- $4p = 4$, entonces $p = 1$
- Foco: $F(-4, 1 + 1) = F(-4, 2)$
- Directriz: $y = 1 - 1 = 0$
- Eje de simetría: $x = -4$
- Lado recto: $|4p| = 4$

\textbf{Paso 5:} Puntos adicionales para graficar.
- Extremos del lado recto: cuando $y = 2$ (altura del foco)
  $(x + 4)^2 = 4(2 - 1) = 4$
  $x + 4 = \pm 2$
  $x = -6$ o $x = -2$
- Puntos: $(-6, 2)$ y $(-2, 2)$

\textbf{Parte b)} $2y^2 - 12y - 5x + 13 = 0$

\textbf{Paso 1:} Normalizar el coeficiente principal.
\[
y^2 - 6y - \frac{5x}{2} + \frac{13}{2} = 0
\]
\[
y^2 - 6y = \frac{5x}{2} - \frac{13}{2}
\]

\textbf{Paso 2:} Completar el cuadrado.
\[
y^2 - 6y + 9 = \frac{5x}{2} - \frac{13}{2} + 9
\]
\[
(y - 3)^2 = \frac{5x}{2} + \frac{5}{2}
\]
\[
(y - 3)^2 = \frac{5}{2}(x + 1)
\]

\textbf{Paso 3:} Identificar la orientación.
- Variable al cuadrado: $y$ → eje horizontal
- Coeficiente positivo → abre hacia la derecha

\textbf{Paso 4:} Determinar elementos.
- Vértice: $V(-1, 3)$
- $4p = \frac{5}{2}$, entonces $p = \frac{5}{8}$
- Foco: $F(-1 + \frac{5}{8}, 3) = F(-\frac{3}{8}, 3)$
- Directriz: $x = -1 - \frac{5}{8} = -\frac{13}{8}$
- Eje de simetría: $y = 3$
- Lado recto: $|4p| = \frac{5}{2}$

\textbf{Paso 5:} Graficar ambas parábolas.

\begin{center}
\begin{tikzpicture}[scale=0.4]
    \begin{axis}[
        axis lines=middle,
        xlabel={$x$},
        ylabel={$y$},
        xmin=-8, xmax=2,
        ymin=-2, ymax=6,
        grid=major,
        width=0.95\textwidth,
        height=0.7\textwidth,
        axis equal image
    ]
    % Parte a
    \addplot[domain=-8:0, samples=100, thick, blue] ({x}, {(x+4)^2/4 + 1});
    \addplot[only marks, mark=*, blue] coordinates {(-4,1) (-4,2)};
    \node[blue] at (axis cs: -4,1) [below right] {$V(-4,1)$};
    \node[blue] at (axis cs: -4,2) [above right] {$F(-4,2)$};
    \addplot[dashed, blue, domain=-8:0] {0};
    \node[blue] at (axis cs: -2,0) [below] {$y=0$};

    % Parte b
    \addplot[domain=-1:2, samples=100, thick, red, parametric] ({-1 + 2*(t-3)^2/5}, {t});
    \addplot[only marks, mark=*, red] coordinates {(-1,3) (-3/8,3)};
    \node[red] at (axis cs: -1,3) [above left] {$V(-1,3)$};
    \node[red] at (axis cs: -3/8,3) [above] {$F$};
    \addplot[dashed, red] coordinates {(-13/8,-2) (-13/8,6)};
    \end{axis}
\end{tikzpicture}
\end{center}

\textbf{Respuestas completas:}
\begin{itemize}
    \item \textbf{Parte a:} Abre hacia arriba, $V(-4,1)$, $F(-4,2)$, Directriz: $y=0$, Eje: $x=-4$
    \item \textbf{Parte b:} Abre hacia la derecha, $V(-1,3)$, $F(-\frac{3}{8},3)$, Directriz: $x=-\frac{13}{8}$, Eje: $y=3$
\end{itemize}
\end{solucion}

\begin{solucion}[title={Solución Ejercicio 8}]
\textbf{Antena parabólica:} 8 metros de diámetro, 2 metros de profundidad

\textbf{Parte a)} Ecuación con vértice en el origen

\textbf{Paso 1:} Establecer el sistema de coordenadas.
- Vértice en el origen: $V(0, 0)$
- Eje de simetría: eje $x$ (antena abre hacia la derecha)
- Forma: $y^2 = 4px$

\textbf{Paso 2:} Usar la información dada.
- Diámetro = 8 m → radio = 4 m
- Profundidad = 2 m
- El punto $(2, 4)$ está en la parábola (borde superior)
- El punto $(2, -4)$ está en la parábola (borde inferior)

\textbf{Paso 3:} Encontrar $p$ usando un punto.
Sustituyendo $(2, 4)$ en $y^2 = 4px$:
\[
16 = 4p(2)
\]
\[
16 = 8p
\]
\[
p = 2
\]

\textbf{Paso 4:} Escribir la ecuación.
\[
y^2 = 4(2)x
\]
\[
\boxed{y^2 = 8x}
\]

\textbf{Paso 5:} Verificar con el otro punto.
Para $(2, -4)$: $(-4)^2 = 16 = 8(2)$ ✓

\textbf{Parte b)} Posición del foco (receptor)

\textbf{Paso 1:} Usar el valor de $p$.
- $p = 2$ metros
- El foco está a $p$ unidades del vértice en la dirección de apertura

\textbf{Paso 2:} Determinar la posición.
\[
F(p, 0) = F(2, 0)
\]

\textbf{Respuesta:} El receptor debe colocarse a 2 metros del vértice de la antena.

\textbf{Parte c)} Punto donde un rayo paralelo toca la parábola

\textbf{Paso 1:} Entender el problema.
- Rayo paralelo al eje de simetría
- A 3 metros del vértice → altura $y = 3$
- Necesitamos encontrar la coordenada $x$

\textbf{Paso 2:} Sustituir en la ecuación.
\[
(3)^2 = 8x
\]
\[
9 = 8x
\]
\[
x = \frac{9}{8} = 1.125
\]

\textbf{Paso 3:} Interpretar el resultado.
- El punto es $\left(\frac{9}{8}, 3\right)$
- Distancia del vértice: $\frac{9}{8}$ metros = 1.125 metros

\textbf{Paso 4:} Verificar la propiedad reflectora.
- Todo rayo paralelo al eje se refleja hacia el foco
- Distancia del punto al foco:
\[
d = \sqrt{\left(\frac{9}{8} - 2\right)^2 + (3 - 0)^2} = \sqrt{\left(-\frac{7}{8}\right)^2 + 9}
\]
\[
d = \sqrt{\frac{49}{64} + \frac{576}{64}} = \sqrt{\frac{625}{64}} = \frac{25}{8}
\]

\textbf{Paso 5:} Graficar la situación.

\begin{center}
\begin{tikzpicture}[scale=0.7]
    \begin{axis}[
        axis lines=middle,
        xlabel={$x$ (metros)},
        ylabel={$y$ (metros)},
        xmin=-0.5, xmax=3,
        ymin=-5, ymax=5,
        grid=major,
        width=0.9\textwidth,
        height=0.7\textwidth,
        axis equal image
    ]
    % Parábola
    \addplot[domain=-4:4, samples=100, thick, blue, parametric] ({t^2/8}, {t});

    % Vértice y foco
    \addplot[only marks, mark=*, blue] coordinates {(0,0) (2,0)};
    \node[blue] at (axis cs: 0,0) [below left] {$V(0,0)$};
    \node[blue] at (axis cs: 2,0) [below] {$F(2,0)$};

    % Puntos del borde
    \addplot[only marks, mark=o, red] coordinates {(2,4) (2,-4)};
    \node[red] at (axis cs: 2,4) [above] {$(2,4)$};
    \node[red] at (axis cs: 2,-4) [below] {$(2,-4)$};

    % Rayo y punto de incidencia
    \draw[green, thick, -{Latex}] (axis cs: -0.5,3) -- (axis cs: 9/8,3);
    \addplot[only marks, mark=*, green] coordinates {(9/8,3)};
    \node[green] at (axis cs: 9/8,3) [above right] {$(\frac{9}{8},3)$};

    % Rayo reflejado hacia el foco
    \draw[green, thick, dashed, -{Latex}] (axis cs: 9/8,3) -- (axis cs: 2,0);

    % Dimensiones
    \draw[<->, thick, gray] (axis cs: 0,-4.5) -- (axis cs: 2,-4.5);
    \node[gray] at (axis cs: 1,-4.5) [below] {2 m};
    \draw[<->, thick, gray] (axis cs: 2.5,-4) -- (axis cs: 2.5,4);
    \node[gray] at (axis cs: 2.5,0) [right] {8 m};
    \end{axis}
\end{tikzpicture}
\end{center}

\textbf{Respuestas completas:}
\begin{itemize}
    \item \textbf{Parte a:} Ecuación: $y^2 = 8x$
    \item \textbf{Parte b:} Foco (receptor) en $F(2, 0)$, a 2 metros del vértice
    \item \textbf{Parte c:} El rayo toca la parábola a $\frac{9}{8} = 1.125$ metros del vértice
\end{itemize}

\textbf{Nota práctica:} Esta propiedad reflectora es la que hace que las antenas parabólicas sean tan eficientes: todos los rayos paralelos (señales de satélite) se concentran en un único punto (el receptor en el foco).
\end{solucion}

\end{document}
\newpage

\section{Conclusión}

¡Felicidades! Has completado un viaje fascinante por el mundo de las parábolas. Ahora ya sabes que estas curvas no son solo figuras matemáticas abstractas, sino formas que aparecen constantemente en nuestro mundo y que tienen aplicaciones increíbles.

\subsection*{Resumen de Conceptos Clave}

Repasemos lo más importante que hemos aprendido:

\begin{enumerate}
    \item \textbf{Definición fundamental:} Una parábola es el conjunto de todos los puntos que están a la misma distancia de un punto fijo (foco) y de una línea recta fija (directriz).

    \item \textbf{Elementos esenciales:}
    \begin{itemize}
        \item Vértice: El punto ``clave'' de la parábola
        \item Foco: El punto especial que define la curva
        \item Directriz: La línea de referencia
        \item Eje de simetría: La línea que divide la parábola en dos partes iguales
        \item Parámetro $p$: Determina qué tan abierta está la parábola
        \item Lado recto: Siempre mide $4p$
    \end{itemize}

    \item \textbf{Ecuaciones que debes dominar:}
    \begin{itemize}
        \item Con vértice en el origen: $y^2 = 4px$, $x^2 = 4py$
        \item Con vértice en $(h,k)$: $(y-k)^2 = 4p(x-h)$, $(x-h)^2 = 4p(y-k)$
    \end{itemize}
\end{enumerate}

\subsection*{Fórmulas Esenciales - Tu Kit de Supervivencia}

\begin{tcolorbox}[enhanced,colback=accentcolor!10,colframe=accentcolor,title=Fórmulas Rápidas para Parábolas]
\textbf{Para identificar rápidamente:}
\begin{itemize}
    \item Si ves $y^2$: La parábola es horizontal (abre a derecha o izquierda)
    \item Si ves $x^2$: La parábola es vertical (abre arriba o abajo)
    \item Signo positivo: Abre hacia la derecha (si $y^2$) o hacia arriba (si $x^2$)
    \item Signo negativo: Abre hacia la izquierda (si $y^2$) o hacia abajo (si $x^2$)
\end{itemize}

\textbf{Para encontrar elementos:}
\begin{itemize}
    \item Distancia vértice-foco = $p$
    \item Distancia vértice-directriz = $p$
    \item Longitud del lado recto = $4p$
\end{itemize}

\textbf{Para convertir ecuaciones:}
\begin{itemize}
    \item General → Canónica: Completa el cuadrado
    \item Canónica → General: Expande los binomios
\end{itemize}
\end{tcolorbox}

\subsection*{Consejos para Estudiantes}

Aquí van algunos tips que te ayudarán a dominar las parábolas:

\begin{enumerate}
    \item \textbf{Dibuja siempre:} Hacer un bosquejo rápido te ayuda a visualizar el problema. No tiene que ser perfecto, solo ubica el vértice, el foco y la directriz.

    \item \textbf{Identifica el tipo primero:} Antes de resolver, determina si la parábola es horizontal o vertical. Esto te dirá qué fórmulas usar.

    \item \textbf{El parámetro $p$ es tu amigo:} Una vez que encuentras $p$, puedes hallar todos los demás elementos fácilmente.

    \item \textbf{Verifica con un punto:} Cuando termines, sustituye las coordenadas de un punto conocido (como el vértice) en tu ecuación para verificar.

    \item \textbf{Practica la conversión:} Saber cambiar entre forma general y canónica es súper útil. Es como traducir entre dos idiomas matemáticos.

    \item \textbf{Relaciona con la vida real:} Cuando veas una antena parabólica o un chorro de agua, piensa en la matemática detrás. Esto hace que el aprendizaje sea más significativo.
\end{enumerate}

\subsection*{Tabla de Referencia Rápida}

\begin{center}
\begin{tcolorbox}[enhanced,colback=blue!5,colframe=blue!60!black,title=Guía Rápida de Parábolas]
\renewcommand{\arraystretch}{1.3}
\begin{tabular}{|p{3cm}|p{5cm}|p{5cm}|}
\hline
\textbf{Si tienes...} & \textbf{Forma de la ecuación} & \textbf{Característica principal} \\
\hline
Parábola horizontal abriendo a la derecha & $(y-k)^2 = 4p(x-h)$ & Foco a la derecha del vértice \\
\hline
Parábola horizontal abriendo a la izquierda & $(y-k)^2 = -4p(x-h)$ & Foco a la izquierda del vértice \\
\hline
Parábola vertical abriendo hacia arriba & $(x-h)^2 = 4p(y-k)$ & Foco arriba del vértice \\
\hline
Parábola vertical abriendo hacia abajo & $(x-h)^2 = -4p(y-k)$ & Foco abajo del vértice \\
\hline
Ecuación con $x^2$ & $Ax^2 + Dx + Ey + F = 0$ & Parábola vertical \\
\hline
Ecuación con $y^2$ & $Cy^2 + Dx + Ey + F = 0$ & Parábola horizontal \\
\hline
\end{tabular}
\end{tcolorbox}
\end{center}

\subsection*{¿Qué sigue?}

Ahora que dominas las bases de las parábolas, estás listo para:

\begin{itemize}
    \item Resolver problemas más complejos de aplicación
    \item Estudiar otras cónicas (elipse, hipérbola)
    \item Explorar las parábolas en coordenadas polares
    \item Aplicar tus conocimientos en física (movimiento de proyectiles)
    \item Entender mejor el diseño de estructuras parabólicas
\end{itemize}

\subsection*{Reflexión Final}

Las parábolas son mucho más que curvas en un papel. Son la forma que toma el agua al salir de una fuente, el camino de una pelota de béisbol hacia el home run, la estructura de los puentes que cruzan ríos, y la forma de las antenas que nos conectan con el espacio.

Cada vez que resuelvas un problema de parábolas, recuerda que estás trabajando con una de las formas más elegantes y útiles de la naturaleza. La misma matemática que usas en clase es la que usan los ingenieros para diseñar los faros de los autos, los arquitectos para crear edificios impresionantes, y los científicos para entender el universo.

\begin{center}
\textit{``La naturaleza está escrita en lenguaje matemático, y las letras son triángulos, círculos y otras figuras geométricas.''} \\
--- Galileo Galilei
\end{center}

\vspace{1cm}

\begin{center}
\Large
\textbf{¡Sigue explorando el fascinante mundo de las cónicas!}
\end{center}

% No cerramos el documento aquí, será cerrado en el ensamblaje final
\end{document}
