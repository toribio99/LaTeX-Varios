% PARTE 3: EJERCICIOS PROPUESTOS Y SOLUCIONES
% Guía sobre la Parábola - Geometría Analítica

\section{Ejercicios Propuestos}

Los siguientes ejercicios están diseñados para reforzar tu comprensión de las parábolas. Intenta resolverlos antes de consultar las soluciones detalladas.

\begin{ejercicio}[title={Ejercicio 1: Ecuación Canónica - Vértice en el Origen}]
Para cada parábola con vértice en el origen, escribe su ecuación canónica y determina el foco y la directriz:
\begin{enumerate}[label=\alph*)]
    \item Parábola que abre hacia arriba con $p = 3$
    \item Parábola que abre hacia la izquierda con distancia focal de 5 unidades
\end{enumerate}
\end{ejercicio}

\begin{ejercicio}[title={Ejercicio 2: Ecuación Canónica - Vértice Trasladado}]
Encuentra la ecuación canónica de cada parábola:
\begin{enumerate}[label=\alph*)]
    \item Vértice en $(2, -1)$, abre hacia la derecha, $p = 4$
    \item Vértice en $(-3, 5)$, abre hacia abajo, foco a 2 unidades del vértice
\end{enumerate}
\end{ejercicio}

\begin{ejercicio}[title={Ejercicio 3: Identificación de Elementos}]
Para cada ecuación, encuentra el vértice, foco, directriz y eje de simetría:
\begin{enumerate}[label=\alph*)]
    \item $(x - 4)^2 = 12(y + 2)$
    \item $(y + 1)^2 = -8(x - 3)$
\end{enumerate}
\end{ejercicio}

\begin{ejercicio}[title={Ejercicio 4: De Forma General a Canónica}]
Transforma cada ecuación a su forma canónica completando el cuadrado:
\begin{enumerate}[label=\alph*)]
    \item $x^2 - 6x - 8y + 17 = 0$
    \item $y^2 + 4y + 12x - 8 = 0$
\end{enumerate}
\end{ejercicio}

\begin{ejercicio}[title={Ejercicio 5: Parábola Dados Foco y Directriz}]
Encuentra la ecuación de la parábola con:
\begin{enumerate}[label=\alph*)]
    \item Foco en $F(0, 4)$ y directriz $y = -4$
    \item Foco en $F(5, 2)$ y directriz $x = -1$
\end{enumerate}
\end{ejercicio}

\begin{ejercicio}[title={Ejercicio 6: Parábola por Tres Puntos}]
Encuentra la ecuación de la parábola vertical que pasa por los puntos:
\begin{enumerate}[label=\alph*)]
    \item $A(0, 3)$, $B(2, -1)$ y $C(4, 3)$
    \item $P(1, 0)$, $Q(3, 4)$ y $R(-1, 4)$
\end{enumerate}
\end{ejercicio}

\begin{ejercicio}[title={Ejercicio 7: Determinación de Orientación y Elementos}]
Para cada ecuación, determina la orientación de la parábola y todos sus elementos geométricos:
\begin{enumerate}[label=\alph*)]
    \item $x^2 + 8x - 4y + 20 = 0$
    \item $2y^2 - 12y - 5x + 13 = 0$
\end{enumerate}
\end{ejercicio}

\begin{ejercicio}[title={Ejercicio 8: Aplicación Práctica - Antena Parabólica}]
Una antena parabólica tiene 8 metros de diámetro y 2 metros de profundidad.
\begin{enumerate}[label=\alph*)]
    \item Encuentra la ecuación de la parábola considerando el vértice en el origen
    \item Determina la posición del foco (receptor de señales)
    \item Si un rayo llega paralelo al eje de simetría a 3 metros del vértice, ¿a qué distancia del vértice se encuentra el punto donde el rayo toca la parábola?
\end{enumerate}
\end{ejercicio}

\newpage

\section{Soluciones Detalladas}

\begin{solucion}[title={Solución Ejercicio 1}]
\textbf{Parte a)} Parábola con vértice en el origen, abre hacia arriba, $p = 3$

\textbf{Paso 1:} Identificar la orientación y el valor de $p$.
- Abre hacia arriba: forma $x^2 = 4py$
- $p = 3$ (positivo porque abre hacia arriba)

\textbf{Paso 2:} Escribir la ecuación canónica.
\[
x^2 = 4py = 4(3)y = 12y
\]
\[
\boxed{x^2 = 12y}
\]

\textbf{Paso 3:} Determinar el foco.
- Como abre hacia arriba, el foco está en $(0, p)$
\[
F(0, 3)
\]

\textbf{Paso 4:} Determinar la directriz.
- La directriz es una línea horizontal: $y = -p$
\[
y = -3
\]

\textbf{Paso 5:} Verificación de la definición.
- Punto en la parábola: si $x = 6$, entonces $36 = 12y$, así $y = 3$
- Punto $P(6, 3)$
- Distancia al foco: $d(P, F) = \sqrt{(6-0)^2 + (3-3)^2} = 6$
- Distancia a la directriz: $d(P, y=-3) = |3-(-3)| = 6$ ✓

\textbf{Parte b)} Parábola con vértice en el origen, abre hacia la izquierda, distancia focal 5

\textbf{Paso 1:} Interpretar la información.
- Abre hacia la izquierda: forma $y^2 = 4px$ con $p < 0$
- Distancia focal = $|p| = 5$, entonces $p = -5$

\textbf{Paso 2:} Escribir la ecuación canónica.
\[
y^2 = 4px = 4(-5)x = -20x
\]
\[
\boxed{y^2 = -20x}
\]

\textbf{Paso 3:} Determinar el foco.
- Como abre hacia la izquierda, el foco está en $(p, 0)$
\[
F(-5, 0)
\]

\textbf{Paso 4:} Determinar la directriz.
- La directriz es una línea vertical: $x = -p$
\[
x = 5
\]

\textbf{Paso 5:} Graficar para visualizar.

\begin{center}
\begin{tikzpicture}[scale=0.4]
    \begin{axis}[
        axis lines=middle,
        xlabel={$x$},
        ylabel={$y$},
        xmin=-8, xmax=8,
        ymin=-10, ymax=10,
        grid=major,
        width=0.9\textwidth,
        height=0.6\textwidth,
        axis equal image
    ]
    % Parte a: x^2 = 12y
    \addplot[domain=-8:8, samples=100, thick, blue] {x^2/12};
    \addplot[only marks, mark=*, blue] coordinates {(0,3)};
    \node[blue] at (axis cs: 0,3) [above right] {$F(0,3)$};
    \addplot[dashed, blue, domain=-8:8] {-3};
    \node[blue] at (axis cs: 7,-3) [below] {$y=-3$};

    % Parte b: y^2 = -20x
    \addplot[domain=-8:0, samples=100, thick, red, parametric] ({-t^2/20}, {t});
    \addplot[only marks, mark=*, red] coordinates {(-5,0)};
    \node[red] at (axis cs: -5,0) [below left] {$F(-5,0)$};
    \addplot[dashed, red, samples=2] coordinates {(5,-10) (5,10)};
    \node[red] at (axis cs: 5,8) [right] {$x=5$};
    \end{axis}
\end{tikzpicture}
\end{center}

\textbf{Respuesta completa:}
\begin{itemize}
    \item \textbf{Parte a:} Ecuación: $x^2 = 12y$, Foco: $F(0,3)$, Directriz: $y = -3$
    \item \textbf{Parte b:} Ecuación: $y^2 = -20x$, Foco: $F(-5,0)$, Directriz: $x = 5$
\end{itemize}
\end{solucion}

\begin{solucion}[title={Solución Ejercicio 2}]
\textbf{Parte a)} Vértice en $(2, -1)$, abre hacia la derecha, $p = 4$

\textbf{Paso 1:} Identificar la forma de la ecuación.
- Vértice: $V(h, k) = (2, -1)$
- Abre hacia la derecha: eje horizontal
- Forma: $(y - k)^2 = 4p(x - h)$

\textbf{Paso 2:} Sustituir valores.
\[
(y - (-1))^2 = 4(4)(x - 2)
\]
\[
(y + 1)^2 = 16(x - 2)
\]
\[
\boxed{(y + 1)^2 = 16(x - 2)}
\]

\textbf{Paso 3:} Determinar el foco.
- El foco está a $p = 4$ unidades a la derecha del vértice
\[
F(h + p, k) = F(2 + 4, -1) = F(6, -1)
\]

\textbf{Paso 4:} Determinar la directriz.
- La directriz es vertical: $x = h - p$
\[
x = 2 - 4 = -2
\]

\textbf{Paso 5:} Verificar con un punto.
- Si $x = 6$: $(y + 1)^2 = 16(6 - 2) = 64$
- Entonces $y + 1 = \pm 8$, así $y = 7$ o $y = -9$
- Puntos: $(6, 7)$ y $(6, -9)$

\textbf{Parte b)} Vértice en $(-3, 5)$, abre hacia abajo, foco a 2 unidades

\textbf{Paso 1:} Determinar el valor de $p$.
- Abre hacia abajo: $p < 0$
- Distancia del foco al vértice: $|p| = 2$
- Por tanto: $p = -2$

\textbf{Paso 2:} Escribir la ecuación.
- Vértice: $V(-3, 5)$
- Abre hacia abajo: eje vertical
- Forma: $(x - h)^2 = 4p(y - k)$
\[
(x - (-3))^2 = 4(-2)(y - 5)
\]
\[
(x + 3)^2 = -8(y - 5)
\]
\[
\boxed{(x + 3)^2 = -8(y - 5)}
\]

\textbf{Paso 3:} Determinar el foco.
- El foco está a 2 unidades debajo del vértice
\[
F(h, k + p) = F(-3, 5 + (-2)) = F(-3, 3)
\]

\textbf{Paso 4:} Determinar la directriz.
- La directriz es horizontal: $y = k - p$
\[
y = 5 - (-2) = 7
\]

\textbf{Paso 5:} Desarrollar a forma general.
\begin{align}
(x + 3)^2 &= -8(y - 5)\\
x^2 + 6x + 9 &= -8y + 40\\
x^2 + 6x + 8y - 31 &= 0
\end{align}

\textbf{Respuestas:}
\begin{itemize}
    \item \textbf{Parte a:} $(y + 1)^2 = 16(x - 2)$, Foco: $F(6, -1)$, Directriz: $x = -2$
    \item \textbf{Parte b:} $(x + 3)^2 = -8(y - 5)$, Foco: $F(-3, 3)$, Directriz: $y = 7$
\end{itemize}
\end{solucion}

\begin{solucion}[title={Solución Ejercicio 3}]
\textbf{Parte a)} $(x - 4)^2 = 12(y + 2)$

\textbf{Paso 1:} Identificar la forma y orientación.
- Forma: $(x - h)^2 = 4p(y - k)$
- Variable al cuadrado: $x$ → eje vertical
- Coeficiente positivo → abre hacia arriba

\textbf{Paso 2:} Identificar $h$, $k$ y $4p$.
- Comparando: $(x - 4)^2 = 12(y - (-2))$
- $h = 4$, $k = -2$
- $4p = 12$, entonces $p = 3$

\textbf{Paso 3:} Determinar el vértice.
\[
V(h, k) = V(4, -2)
\]

\textbf{Paso 4:} Determinar el foco.
- Como abre hacia arriba: $F(h, k + p)$
\[
F(4, -2 + 3) = F(4, 1)
\]

\textbf{Paso 5:} Determinar la directriz.
- Línea horizontal: $y = k - p$
\[
y = -2 - 3 = -5
\]

\textbf{Paso 6:} Determinar el eje de simetría.
- Línea vertical que pasa por el vértice
\[
x = 4
\]

\textbf{Parte b)} $(y + 1)^2 = -8(x - 3)$

\textbf{Paso 1:} Identificar la forma y orientación.
- Forma: $(y - k)^2 = 4p(x - h)$
- Variable al cuadrado: $y$ → eje horizontal
- Coeficiente negativo → abre hacia la izquierda

\textbf{Paso 2:} Identificar $h$, $k$ y $4p$.
- Comparando: $(y - (-1))^2 = -8(x - 3)$
- $h = 3$, $k = -1$
- $4p = -8$, entonces $p = -2$

\textbf{Paso 3:} Determinar el vértice.
\[
V(h, k) = V(3, -1)
\]

\textbf{Paso 4:} Determinar el foco.
- Como abre hacia la izquierda: $F(h + p, k)$
\[
F(3 + (-2), -1) = F(1, -1)
\]

\textbf{Paso 5:} Determinar la directriz.
- Línea vertical: $x = h - p$
\[
x = 3 - (-2) = 5
\]

\textbf{Paso 6:} Determinar el eje de simetría.
- Línea horizontal que pasa por el vértice
\[
y = -1
\]

\textbf{Paso 7:} Graficar ambas parábolas.

\begin{center}
\begin{tikzpicture}[scale=0.5]
    \begin{axis}[
        axis lines=middle,
        xlabel={$x$},
        ylabel={$y$},
        xmin=-2, xmax=10,
        ymin=-7, ymax=5,
        grid=major,
        width=0.95\textwidth,
        height=0.65\textwidth,
        axis equal image
    ]
    % Parte a
    \addplot[domain=0:8, samples=100, thick, blue] ({x}, {(x-4)^2/12 - 2});
    \addplot[only marks, mark=*, blue] coordinates {(4,-2) (4,1)};
    \node[blue] at (axis cs: 4,-2) [below right] {$V(4,-2)$};
    \node[blue] at (axis cs: 4,1) [above right] {$F(4,1)$};
    \addplot[dashed, blue] coordinates {(-2,-5) (10,-5)};
    \node[blue] at (axis cs: 8,-5) [below] {$y=-5$};

    % Parte b
    \addplot[domain=-5:3, samples=100, thick, red, parametric] ({3-(t+1)^2/8}, {t});
    \addplot[only marks, mark=*, red] coordinates {(3,-1) (1,-1)};
    \node[red] at (axis cs: 3,-1) [above left] {$V(3,-1)$};
    \node[red] at (axis cs: 1,-1) [below left] {$F(1,-1)$};
    \addplot[dashed, red] coordinates {(5,-7) (5,5)};
    \node[red] at (axis cs: 5,3) [right] {$x=5$};
    \end{axis}
\end{tikzpicture}
\end{center}

\textbf{Respuestas completas:}
\begin{itemize}
    \item \textbf{Parte a:} $V(4,-2)$, $F(4,1)$, Directriz: $y=-5$, Eje: $x=4$
    \item \textbf{Parte b:} $V(3,-1)$, $F(1,-1)$, Directriz: $x=5$, Eje: $y=-1$
\end{itemize}
\end{solucion}

\begin{solucion}[title={Solución Ejercicio 4}]
\textbf{Parte a)} $x^2 - 6x - 8y + 17 = 0$

\textbf{Paso 1:} Agrupar términos con la misma variable.
\[
x^2 - 6x = 8y - 17
\]

\textbf{Paso 2:} Completar el cuadrado para $x$.
- Coeficiente de $x$: $-6$
- Mitad del coeficiente: $-3$
- Cuadrado: $(-3)^2 = 9$

\textbf{Paso 3:} Añadir y restar 9.
\[
x^2 - 6x + 9 = 8y - 17 + 9
\]
\[
(x - 3)^2 = 8y - 8
\]

\textbf{Paso 4:} Factorizar el lado derecho.
\[
(x - 3)^2 = 8(y - 1)
\]
\[
\boxed{(x - 3)^2 = 8(y - 1)}
\]

\textbf{Paso 5:} Identificar elementos.
- Vértice: $V(3, 1)$
- $4p = 8$, entonces $p = 2$
- Foco: $F(3, 1 + 2) = F(3, 3)$
- Directriz: $y = 1 - 2 = -1$

\textbf{Parte b)} $y^2 + 4y + 12x - 8 = 0$

\textbf{Paso 1:} Reorganizar la ecuación.
\[
y^2 + 4y = -12x + 8
\]

\textbf{Paso 2:} Completar el cuadrado para $y$.
- Coeficiente de $y$: $4$
- Mitad del coeficiente: $2$
- Cuadrado: $(2)^2 = 4$

\textbf{Paso 3:} Añadir 4 a ambos lados.
\[
y^2 + 4y + 4 = -12x + 8 + 4
\]
\[
(y + 2)^2 = -12x + 12
\]

\textbf{Paso 4:} Factorizar el lado derecho.
\[
(y + 2)^2 = -12(x - 1)
\]
\[
\boxed{(y + 2)^2 = -12(x - 1)}
\]

\textbf{Paso 5:} Identificar elementos.
- Vértice: $V(1, -2)$
- $4p = -12$, entonces $p = -3$
- Abre hacia la izquierda
- Foco: $F(1 - 3, -2) = F(-2, -2)$
- Directriz: $x = 1 - (-3) = 4$

\textbf{Paso 6:} Verificación expandiendo.
\begin{align}
(y + 2)^2 &= -12(x - 1)\\
y^2 + 4y + 4 &= -12x + 12\\
y^2 + 4y + 12x - 8 &= 0 \checkmark
\end{align}

\textbf{Respuestas:}
\begin{itemize}
    \item \textbf{Parte a:} $(x - 3)^2 = 8(y - 1)$, $V(3,1)$, $F(3,3)$, Directriz: $y = -1$
    \item \textbf{Parte b:} $(y + 2)^2 = -12(x - 1)$, $V(1,-2)$, $F(-2,-2)$, Directriz: $x = 4$
\end{itemize}
\end{solucion}

\begin{solucion}[title={Solución Ejercicio 5}]
\textbf{Parte a)} Foco en $F(0, 4)$ y directriz $y = -4$

\textbf{Paso 1:} Encontrar el vértice.
- El vértice está a mitad de camino entre el foco y la directriz
- Coordenada $x$ del vértice: $x = 0$ (mismo que el foco)
- Coordenada $y$ del vértice: $y = \frac{4 + (-4)}{2} = 0$
- Vértice: $V(0, 0)$

\textbf{Paso 2:} Calcular $p$.
- Distancia del vértice al foco: $p = 4 - 0 = 4$
- Como el foco está arriba del vértice: $p > 0$

\textbf{Paso 3:} Determinar la orientación.
- Directriz horizontal, foco arriba → abre hacia arriba
- Forma: $x^2 = 4py$

\textbf{Paso 4:} Escribir la ecuación.
\[
x^2 = 4(4)y = 16y
\]
\[
\boxed{x^2 = 16y}
\]

\textbf{Paso 5:} Verificación con la definición.
- Punto de prueba: $(4, 1)$ debe estar en la parábola
- Verificar: $16 = 16(1)$ ✓
- Distancia al foco: $d = \sqrt{(4-0)^2 + (1-4)^2} = \sqrt{16 + 9} = 5$
- Distancia a la directriz: $d = |1 - (-4)| = 5$ ✓

\textbf{Parte b)} Foco en $F(5, 2)$ y directriz $x = -1$

\textbf{Paso 1:} Encontrar el vértice.
- Directriz vertical en $x = -1$
- Foco en $x = 5$
- Coordenada $x$ del vértice: $x = \frac{5 + (-1)}{2} = 2$
- Coordenada $y$ del vértice: $y = 2$ (mismo que el foco)
- Vértice: $V(2, 2)$

\textbf{Paso 2:} Calcular $p$.
- Distancia del vértice al foco: $p = 5 - 2 = 3$
- Como el foco está a la derecha: $p > 0$

\textbf{Paso 3:} Determinar la orientación.
- Directriz vertical, foco a la derecha → abre hacia la derecha
- Forma: $(y - k)^2 = 4p(x - h)$

\textbf{Paso 4:} Escribir la ecuación.
\[
(y - 2)^2 = 4(3)(x - 2)
\]
\[
\boxed{(y - 2)^2 = 12(x - 2)}
\]

\textbf{Paso 5:} Desarrollar a forma general.
\begin{align}
(y - 2)^2 &= 12(x - 2)\\
y^2 - 4y + 4 &= 12x - 24\\
y^2 - 4y - 12x + 28 &= 0
\end{align}

\textbf{Paso 6:} Graficar ambas parábolas.

\begin{center}
\begin{tikzpicture}[scale=0.35]
    \begin{axis}[
        axis lines=middle,
        xlabel={$x$},
        ylabel={$y$},
        xmin=-6, xmax=8,
        ymin=-6, ymax=8,
        grid=major,
        width=0.9\textwidth,
        height=0.9\textwidth,
        axis equal image
    ]
    % Parte a: x^2 = 16y
    \addplot[domain=-8:8, samples=100, thick, blue] {x^2/16};
    \addplot[only marks, mark=*, blue] coordinates {(0,0) (0,4)};
    \node[blue] at (axis cs: 0,0) [below left] {$V$};
    \node[blue] at (axis cs: 0,4) [above right] {$F(0,4)$};
    \addplot[dashed, blue, domain=-8:8] {-4};
    \node[blue] at (axis cs: 6,-4) [below] {$y=-4$};

    % Parte b: (y-2)^2 = 12(x-2)
    \addplot[domain=-2:8, samples=100, thick, red, parametric] ({2 + t^2/12}, {2 + t});
    \addplot[only marks, mark=*, red] coordinates {(2,2) (5,2)};
    \node[red] at (axis cs: 2,2) [below left] {$V(2,2)$};
    \node[red] at (axis cs: 5,2) [above] {$F(5,2)$};
    \addplot[dashed, red] coordinates {(-1,-6) (-1,8)};
    \node[red] at (axis cs: -1,6) [left] {$x=-1$};
    \end{axis}
\end{tikzpicture}
\end{center}

\textbf{Respuestas:}
\begin{itemize}
    \item \textbf{Parte a:} $x^2 = 16y$ o en forma general: $x^2 - 16y = 0$
    \item \textbf{Parte b:} $(y - 2)^2 = 12(x - 2)$ o $y^2 - 4y - 12x + 28 = 0$
\end{itemize}
\end{solucion}

\begin{solucion}[title={Solución Ejercicio 6}]
\textbf{Parte a)} Parábola vertical por $A(0, 3)$, $B(2, -1)$ y $C(4, 3)$

\textbf{Paso 1:} Plantear la forma general de una parábola vertical.
\[
y = ax^2 + bx + c
\]

\textbf{Paso 2:} Sustituir cada punto para obtener un sistema de ecuaciones.

Para $A(0, 3)$:
\[
3 = a(0)^2 + b(0) + c \Rightarrow c = 3
\]

Para $B(2, -1)$:
\[
-1 = a(2)^2 + b(2) + 3 \Rightarrow 4a + 2b = -4
\]

Para $C(4, 3)$:
\[
3 = a(4)^2 + b(4) + 3 \Rightarrow 16a + 4b = 0
\]

\textbf{Paso 3:} Resolver el sistema.
De la tercera ecuación: $16a + 4b = 0 \Rightarrow b = -4a$

Sustituyendo en la segunda ecuación:
\[
4a + 2(-4a) = -4
\]
\[
4a - 8a = -4
\]
\[
-4a = -4
\]
\[
a = 1
\]

Por lo tanto: $b = -4(1) = -4$

\textbf{Paso 4:} Escribir la ecuación.
\[
y = x^2 - 4x + 3
\]

\textbf{Paso 5:} Convertir a forma canónica.
\begin{align}
y &= x^2 - 4x + 3\\
y &= (x^2 - 4x + 4) - 4 + 3\\
y &= (x - 2)^2 - 1\\
y + 1 &= (x - 2)^2
\end{align}
\[
\boxed{(x - 2)^2 = y + 1}
\]

\textbf{Paso 6:} Verificar con los puntos originales.
- $A(0, 3)$: $(0-2)^2 = 4 = 3+1$ ✓
- $B(2, -1)$: $(2-2)^2 = 0 = -1+1$ ✓
- $C(4, 3)$: $(4-2)^2 = 4 = 3+1$ ✓

\textbf{Parte b)} Parábola vertical por $P(1, 0)$, $Q(3, 4)$ y $R(-1, 4)$

\textbf{Paso 1:} Usar la forma general $y = ax^2 + bx + c$.

\textbf{Paso 2:} Sustituir los puntos.

Para $P(1, 0)$:
\[
0 = a(1)^2 + b(1) + c \Rightarrow a + b + c = 0
\]

Para $Q(3, 4)$:
\[
4 = a(3)^2 + b(3) + c \Rightarrow 9a + 3b + c = 4
\]

Para $R(-1, 4)$:
\[
4 = a(-1)^2 + b(-1) + c \Rightarrow a - b + c = 4
\]

\textbf{Paso 3:} Resolver el sistema.
De la primera ecuación: $c = -a - b$

Restando la tercera ecuación de la primera:
\[
(a + b + c) - (a - b + c) = 0 - 4
\]
\[
2b = -4 \Rightarrow b = -2
\]

Sustituyendo $b = -2$ y $c = -a - b = -a + 2$ en la segunda ecuación:
\[
9a + 3(-2) + (-a + 2) = 4
\]
\[
9a - 6 - a + 2 = 4
\]
\[
8a - 4 = 4
\]
\[
8a = 8 \Rightarrow a = 1
\]

Por lo tanto: $c = -1 - (-2) = 1$

\textbf{Paso 4:} Escribir la ecuación.
\[
y = x^2 - 2x + 1
\]

\textbf{Paso 5:} Reconocer el cuadrado perfecto.
\[
y = (x - 1)^2
\]
\[
\boxed{(x - 1)^2 = y}
\]

\textbf{Paso 6:} Identificar elementos.
- Vértice: $V(1, 0)$ (¡coincide con el punto $P$!)
- $4p = 1$, entonces $p = 1/4$
- Foco: $F(1, 1/4)$
- Directriz: $y = -1/4$

\textbf{Respuestas:}
\begin{itemize}
    \item \textbf{Parte a:} $(x - 2)^2 = y + 1$ o $y = x^2 - 4x + 3$
    \item \textbf{Parte b:} $(x - 1)^2 = y$ o $y = x^2 - 2x + 1$
\end{itemize}
\end{solucion}

\begin{solucion}[title={Solución Ejercicio 7}]
\textbf{Parte a)} $x^2 + 8x - 4y + 20 = 0$

\textbf{Paso 1:} Determinar la orientación.
- Variable al cuadrado: $x^2$ → eje vertical
- Reorganizar: $x^2 + 8x = 4y - 20$

\textbf{Paso 2:} Completar el cuadrado.
\[
x^2 + 8x + 16 = 4y - 20 + 16
\]
\[
(x + 4)^2 = 4y - 4
\]
\[
(x + 4)^2 = 4(y - 1)
\]

\textbf{Paso 3:} Identificar la orientación definitiva.
- Coeficiente positivo → abre hacia arriba

\textbf{Paso 4:} Determinar todos los elementos.
- Vértice: $V(-4, 1)$
- $4p = 4$, entonces $p = 1$
- Foco: $F(-4, 1 + 1) = F(-4, 2)$
- Directriz: $y = 1 - 1 = 0$
- Eje de simetría: $x = -4$
- Lado recto: $|4p| = 4$

\textbf{Paso 5:} Puntos adicionales para graficar.
- Extremos del lado recto: cuando $y = 2$ (altura del foco)
  $(x + 4)^2 = 4(2 - 1) = 4$
  $x + 4 = \pm 2$
  $x = -6$ o $x = -2$
- Puntos: $(-6, 2)$ y $(-2, 2)$

\textbf{Parte b)} $2y^2 - 12y - 5x + 13 = 0$

\textbf{Paso 1:} Normalizar el coeficiente principal.
\[
y^2 - 6y - \frac{5x}{2} + \frac{13}{2} = 0
\]
\[
y^2 - 6y = \frac{5x}{2} - \frac{13}{2}
\]

\textbf{Paso 2:} Completar el cuadrado.
\[
y^2 - 6y + 9 = \frac{5x}{2} - \frac{13}{2} + 9
\]
\[
(y - 3)^2 = \frac{5x}{2} + \frac{5}{2}
\]
\[
(y - 3)^2 = \frac{5}{2}(x + 1)
\]

\textbf{Paso 3:} Identificar la orientación.
- Variable al cuadrado: $y$ → eje horizontal
- Coeficiente positivo → abre hacia la derecha

\textbf{Paso 4:} Determinar elementos.
- Vértice: $V(-1, 3)$
- $4p = \frac{5}{2}$, entonces $p = \frac{5}{8}$
- Foco: $F(-1 + \frac{5}{8}, 3) = F(-\frac{3}{8}, 3)$
- Directriz: $x = -1 - \frac{5}{8} = -\frac{13}{8}$
- Eje de simetría: $y = 3$
- Lado recto: $|4p| = \frac{5}{2}$

\textbf{Paso 5:} Graficar ambas parábolas.

\begin{center}
\begin{tikzpicture}[scale=0.4]
    \begin{axis}[
        axis lines=middle,
        xlabel={$x$},
        ylabel={$y$},
        xmin=-8, xmax=2,
        ymin=-2, ymax=6,
        grid=major,
        width=0.95\textwidth,
        height=0.7\textwidth,
        axis equal image
    ]
    % Parte a
    \addplot[domain=-8:0, samples=100, thick, blue] ({x}, {(x+4)^2/4 + 1});
    \addplot[only marks, mark=*, blue] coordinates {(-4,1) (-4,2)};
    \node[blue] at (axis cs: -4,1) [below right] {$V(-4,1)$};
    \node[blue] at (axis cs: -4,2) [above right] {$F(-4,2)$};
    \addplot[dashed, blue, domain=-8:0] {0};
    \node[blue] at (axis cs: -2,0) [below] {$y=0$};

    % Parte b
    \addplot[domain=-1:2, samples=100, thick, red, parametric] ({-1 + 2*(t-3)^2/5}, {t});
    \addplot[only marks, mark=*, red] coordinates {(-1,3) (-3/8,3)};
    \node[red] at (axis cs: -1,3) [above left] {$V(-1,3)$};
    \node[red] at (axis cs: -3/8,3) [above] {$F$};
    \addplot[dashed, red] coordinates {(-13/8,-2) (-13/8,6)};
    \end{axis}
\end{tikzpicture}
\end{center}

\textbf{Respuestas completas:}
\begin{itemize}
    \item \textbf{Parte a:} Abre hacia arriba, $V(-4,1)$, $F(-4,2)$, Directriz: $y=0$, Eje: $x=-4$
    \item \textbf{Parte b:} Abre hacia la derecha, $V(-1,3)$, $F(-\frac{3}{8},3)$, Directriz: $x=-\frac{13}{8}$, Eje: $y=3$
\end{itemize}
\end{solucion}

\begin{solucion}[title={Solución Ejercicio 8}]
\textbf{Antena parabólica:} 8 metros de diámetro, 2 metros de profundidad

\textbf{Parte a)} Ecuación con vértice en el origen

\textbf{Paso 1:} Establecer el sistema de coordenadas.
- Vértice en el origen: $V(0, 0)$
- Eje de simetría: eje $x$ (antena abre hacia la derecha)
- Forma: $y^2 = 4px$

\textbf{Paso 2:} Usar la información dada.
- Diámetro = 8 m → radio = 4 m
- Profundidad = 2 m
- El punto $(2, 4)$ está en la parábola (borde superior)
- El punto $(2, -4)$ está en la parábola (borde inferior)

\textbf{Paso 3:} Encontrar $p$ usando un punto.
Sustituyendo $(2, 4)$ en $y^2 = 4px$:
\[
16 = 4p(2)
\]
\[
16 = 8p
\]
\[
p = 2
\]

\textbf{Paso 4:} Escribir la ecuación.
\[
y^2 = 4(2)x
\]
\[
\boxed{y^2 = 8x}
\]

\textbf{Paso 5:} Verificar con el otro punto.
Para $(2, -4)$: $(-4)^2 = 16 = 8(2)$ ✓

\textbf{Parte b)} Posición del foco (receptor)

\textbf{Paso 1:} Usar el valor de $p$.
- $p = 2$ metros
- El foco está a $p$ unidades del vértice en la dirección de apertura

\textbf{Paso 2:} Determinar la posición.
\[
F(p, 0) = F(2, 0)
\]

\textbf{Respuesta:} El receptor debe colocarse a 2 metros del vértice de la antena.

\textbf{Parte c)} Punto donde un rayo paralelo toca la parábola

\textbf{Paso 1:} Entender el problema.
- Rayo paralelo al eje de simetría
- A 3 metros del vértice → altura $y = 3$
- Necesitamos encontrar la coordenada $x$

\textbf{Paso 2:} Sustituir en la ecuación.
\[
(3)^2 = 8x
\]
\[
9 = 8x
\]
\[
x = \frac{9}{8} = 1.125
\]

\textbf{Paso 3:} Interpretar el resultado.
- El punto es $\left(\frac{9}{8}, 3\right)$
- Distancia del vértice: $\frac{9}{8}$ metros = 1.125 metros

\textbf{Paso 4:} Verificar la propiedad reflectora.
- Todo rayo paralelo al eje se refleja hacia el foco
- Distancia del punto al foco:
\[
d = \sqrt{\left(\frac{9}{8} - 2\right)^2 + (3 - 0)^2} = \sqrt{\left(-\frac{7}{8}\right)^2 + 9}
\]
\[
d = \sqrt{\frac{49}{64} + \frac{576}{64}} = \sqrt{\frac{625}{64}} = \frac{25}{8}
\]

\textbf{Paso 5:} Graficar la situación.

\begin{center}
\begin{tikzpicture}[scale=0.7]
    \begin{axis}[
        axis lines=middle,
        xlabel={$x$ (metros)},
        ylabel={$y$ (metros)},
        xmin=-0.5, xmax=3,
        ymin=-5, ymax=5,
        grid=major,
        width=0.9\textwidth,
        height=0.7\textwidth,
        axis equal image
    ]
    % Parábola
    \addplot[domain=-4:4, samples=100, thick, blue, parametric] ({t^2/8}, {t});

    % Vértice y foco
    \addplot[only marks, mark=*, blue] coordinates {(0,0) (2,0)};
    \node[blue] at (axis cs: 0,0) [below left] {$V(0,0)$};
    \node[blue] at (axis cs: 2,0) [below] {$F(2,0)$};

    % Puntos del borde
    \addplot[only marks, mark=o, red] coordinates {(2,4) (2,-4)};
    \node[red] at (axis cs: 2,4) [above] {$(2,4)$};
    \node[red] at (axis cs: 2,-4) [below] {$(2,-4)$};

    % Rayo y punto de incidencia
    \draw[green, thick, -{Latex}] (axis cs: -0.5,3) -- (axis cs: 9/8,3);
    \addplot[only marks, mark=*, green] coordinates {(9/8,3)};
    \node[green] at (axis cs: 9/8,3) [above right] {$(\frac{9}{8},3)$};

    % Rayo reflejado hacia el foco
    \draw[green, thick, dashed, -{Latex}] (axis cs: 9/8,3) -- (axis cs: 2,0);

    % Dimensiones
    \draw[<->, thick, gray] (axis cs: 0,-4.5) -- (axis cs: 2,-4.5);
    \node[gray] at (axis cs: 1,-4.5) [below] {2 m};
    \draw[<->, thick, gray] (axis cs: 2.5,-4) -- (axis cs: 2.5,4);
    \node[gray] at (axis cs: 2.5,0) [right] {8 m};
    \end{axis}
\end{tikzpicture}
\end{center}

\textbf{Respuestas completas:}
\begin{itemize}
    \item \textbf{Parte a:} Ecuación: $y^2 = 8x$
    \item \textbf{Parte b:} Foco (receptor) en $F(2, 0)$, a 2 metros del vértice
    \item \textbf{Parte c:} El rayo toca la parábola a $\frac{9}{8} = 1.125$ metros del vértice
\end{itemize}

\textbf{Nota práctica:} Esta propiedad reflectora es la que hace que las antenas parabólicas sean tan eficientes: todos los rayos paralelos (señales de satélite) se concentran en un único punto (el receptor en el foco).
\end{solucion}

\end{document}