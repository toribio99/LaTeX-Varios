% PARTE 3: EJERCICIOS PROPUESTOS Y SOLUCIONES DETALLADAS
% Guía sobre la Elipse - Geometría Analítica

\section{Ejercicios Propuestos}

Los siguientes ejercicios están diseñados para reforzar tu comprensión de las elipses. Intenta resolverlos antes de consultar las soluciones detalladas que se presentan más adelante.

\begin{ejercicio}[title={Ejercicio 1: Ecuación Canónica - Centro en el Origen (BÁSICO)}]
Para cada elipse con centro en el origen, escribe su ecuación canónica y determina los focos, vértices y excentricidad:
\begin{enumerate}[label=\alph*)]
    \item Eje mayor horizontal de longitud 16, eje menor de longitud 12
    \item Eje mayor vertical con $a = 7$, eje menor con $b = 5$
\end{enumerate}
\end{ejercicio}

\begin{ejercicio}[title={Ejercicio 2: Ecuación Canónica - Centro Trasladado (BÁSICO)}]
Encuentra la ecuación canónica de cada elipse:
\begin{enumerate}[label=\alph*)]
    \item Centro en $(4, -3)$, eje mayor horizontal de longitud 10, eje menor de longitud 6
    \item Centro en $(-2, 5)$, eje mayor vertical con $a = 8$, $b = 6$
\end{enumerate}
\end{ejercicio}

\begin{ejercicio}[title={Ejercicio 3: Identificación de Elementos (BÁSICO-INTERMEDIO)}]
Para cada ecuación canónica, identifica el centro, vértices, focos, longitud de los ejes y excentricidad:
\begin{enumerate}[label=\alph*)]
    \item $\frac{x^2}{64} + \frac{y^2}{36} = 1$
    \item $\frac{(x + 3)^2}{25} + \frac{(y - 2)^2}{49} = 1$
\end{enumerate}
\end{ejercicio}

\begin{ejercicio}[title={Ejercicio 4: De Forma General a Canónica (INTERMEDIO)}]
Transforma cada ecuación general a su forma canónica mediante completación de cuadrados e identifica todos los elementos:
\begin{enumerate}[label=\alph*)]
    \item $4x^2 + 9y^2 - 16x + 18y - 11 = 0$
    \item $16x^2 + 25y^2 + 32x - 150y - 159 = 0$
\end{enumerate}
\end{ejercicio}

\begin{ejercicio}[title={Ejercicio 5: Elipse Dados Focos y Vértice (INTERMEDIO)}]
Encuentra la ecuación de la elipse que cumple las siguientes condiciones:
\begin{enumerate}[label=\alph*)]
    \item Focos en $F_1(-4, 0)$ y $F_2(4, 0)$; vértice mayor en $(6, 0)$
    \item Focos en $F_1(2, 1)$ y $F_2(2, 7)$; vértice mayor en $(2, 9)$
\end{enumerate}
\end{ejercicio}

\begin{ejercicio}[title={Ejercicio 6: Elipse con Excentricidad Dada (INTERMEDIO-AVANZADO)}]
Determina la ecuación de la elipse que satisface:
\begin{enumerate}[label=\alph*)]
    \item Centro en $(0, 0)$, vértice mayor en $(0, 10)$, excentricidad $e = 0.6$
    \item Centro en $(3, -2)$, eje mayor horizontal de longitud 20, excentricidad $e = 0.8$
\end{enumerate}
\end{ejercicio}

\begin{ejercicio}[title={Ejercicio 7: Aplicaciones Prácticas (AVANZADO)}]
Resuelve los siguientes problemas de aplicación:
\begin{enumerate}[label=\alph*)]
    \item Un planeta orbita alrededor de una estrella. La órbita es elíptica con la estrella en uno de los focos. Si el semieje mayor mide 200 millones de km y la excentricidad es 0.05, encuentra la ecuación de la órbita y las distancias mínima y máxima del planeta a la estrella.
    \item Se diseña un estadio elíptico de 120 metros de largo y 80 metros de ancho. Si se coloca el centro en el origen con el eje mayor horizontal, encuentra la ecuación y determina a qué distancia del centro deben ubicarse dos torres de iluminación para que estén en los focos.
    \item En el diseño de un puente elíptico, el arco principal tiene 60 metros de ancho en su base y una altura máxima de 20 metros en el centro. Si se coloca el centro del arco en el origen con el eje mayor horizontal, encuentra la ecuación y calcula la altura del arco a 15 metros del centro.
\end{enumerate}
\end{ejercicio}

\begin{ejercicio}[title={Ejercicio 8: Problema Integral - Sistema Complejo (AVANZADO)}]
Resuelve los siguientes problemas desafiantes:
\begin{enumerate}[label=\alph*)]
    \item Una elipse tiene centro en $(1, 2)$, pasa por los puntos $(1, 8)$ y $(5, 2)$, y tiene eje mayor vertical. Encuentra su ecuación completa y todos sus elementos.
    \item Demuestra que la suma de distancias de cualquier punto $P(x, y)$ en la elipse $\frac{x^2}{25} + \frac{y^2}{16} = 1$ a los focos es constante e igual a $2a = 10$. Verifica con el punto $P(3, \frac{16}{5})$.
    \item Determina el área encerrada por la elipse $\frac{x^2}{36} + \frac{y^2}{16} = 1$ sabiendo que el área de una elipse es $A = \pi ab$. Compara con el área del círculo de radio $a = 6$.
\end{enumerate}
\end{ejercicio}

\newpage

% ============================================================
% SOLUCIONES DETALLADAS
% ============================================================
\section{Soluciones Detalladas}

\begin{solucion}[title={Solución Ejercicio 1}]
\textbf{Parte a)} Eje mayor horizontal: longitud 16, eje menor: longitud 12

\textbf{Paso 1:} Determinar $a$ y $b$.

\begin{align*}
2a &= 16 \Rightarrow a = 8 \\
2b &= 12 \Rightarrow b = 6
\end{align*}

\textbf{Paso 2:} Verificar $a > b$: $8 > 6$ \checkmark

\textbf{Paso 3:} Escribir la ecuación canónica (eje mayor horizontal).

\[
\boxed{\frac{x^2}{64} + \frac{y^2}{36} = 1}
\]

\textbf{Paso 4:} Calcular la distancia focal.

\[
c = \sqrt{a^2 - b^2} = \sqrt{64 - 36} = \sqrt{28} = 2\sqrt{7} \approx 5.29
\]

\textbf{Paso 5:} Determinar los focos (horizontal).

\[
F_1(-2\sqrt{7}, 0) \approx (-5.29, 0), \quad F_2(2\sqrt{7}, 0) \approx (5.29, 0)
\]

\textbf{Paso 6:} Determinar los vértices.

Vértices mayores: $V_1(-8, 0)$, $V_2(8, 0)$

Vértices menores: $B_1(0, -6)$, $B_2(0, 6)$

\textbf{Paso 7:} Calcular la excentricidad.

\[
e = \frac{c}{a} = \frac{2\sqrt{7}}{8} = \frac{\sqrt{7}}{4} \approx 0.661
\]

\vspace{0.5cm}

\textbf{Parte b)} Eje mayor vertical: $a = 7$, $b = 5$

\textbf{Paso 1:} Verificar $a > b$: $7 > 5$ \checkmark

\textbf{Paso 2:} Escribir la ecuación canónica (eje mayor vertical).

\[
\boxed{\frac{x^2}{25} + \frac{y^2}{49} = 1}
\]

\textbf{Paso 3:} Calcular la distancia focal.

\[
c = \sqrt{a^2 - b^2} = \sqrt{49 - 25} = \sqrt{24} = 2\sqrt{6} \approx 4.90
\]

\textbf{Paso 4:} Determinar los focos (vertical).

\[
F_1(0, -2\sqrt{6}) \approx (0, -4.90), \quad F_2(0, 2\sqrt{6}) \approx (0, 4.90)
\]

\textbf{Paso 5:} Determinar los vértices.

Vértices mayores: $V_1(0, -7)$, $V_2(0, 7)$

Vértices menores: $B_1(-5, 0)$, $B_2(5, 0)$

\textbf{Paso 6:} Calcular la excentricidad.

\[
e = \frac{c}{a} = \frac{2\sqrt{6}}{7} \approx 0.70
\]

\textbf{Respuesta completa:}
\begin{itemize}
    \item \textbf{Parte a:} $\frac{x^2}{64} + \frac{y^2}{36} = 1$; Focos: $(\pm 2\sqrt{7}, 0)$; $e \approx 0.661$
    \item \textbf{Parte b:} $\frac{x^2}{25} + \frac{y^2}{49} = 1$; Focos: $(0, \pm 2\sqrt{6})$; $e \approx 0.70$
\end{itemize}
\end{solucion}

\newpage

\begin{solucion}[title={Solución Ejercicio 2}]
\textbf{Parte a)} Centro $(4, -3)$, eje mayor horizontal: longitud 10, eje menor: longitud 6

\textbf{Paso 1:} Determinar $h, k, a, b$.

\begin{align*}
(h, k) &= (4, -3) \\
2a &= 10 \Rightarrow a = 5 \\
2b &= 6 \Rightarrow b = 3
\end{align*}

\textbf{Paso 2:} Verificar $a > b$: $5 > 3$ \checkmark

\textbf{Paso 3:} Escribir la ecuación canónica (eje horizontal).

\[
\frac{(x - h)^2}{a^2} + \frac{(y - k)^2}{b^2} = 1
\]
\[
\boxed{\frac{(x - 4)^2}{25} + \frac{(y + 3)^2}{9} = 1}
\]

\textbf{Paso 4:} Calcular $c$ y determinar focos.

\[
c = \sqrt{25 - 9} = \sqrt{16} = 4
\]
\[
F_1(4 - 4, -3) = (0, -3), \quad F_2(4 + 4, -3) = (8, -3)
\]

\vspace{0.5cm}

\textbf{Parte b)} Centro $(-2, 5)$, eje mayor vertical: $a = 8$, $b = 6$

\textbf{Paso 1:} Identificar parámetros.

\begin{align*}
(h, k) &= (-2, 5) \\
a &= 8, \quad b = 6
\end{align*}

\textbf{Paso 2:} Escribir la ecuación canónica (eje vertical).

\[
\frac{(x - h)^2}{b^2} + \frac{(y - k)^2}{a^2} = 1
\]
\[
\boxed{\frac{(x + 2)^2}{36} + \frac{(y - 5)^2}{64} = 1}
\]

\textbf{Paso 3:} Calcular $c$ y focos.

\[
c = \sqrt{64 - 36} = \sqrt{28} = 2\sqrt{7} \approx 5.29
\]
\[
F_1(-2, 5 - 2\sqrt{7}) \approx (-2, -0.29), \quad F_2(-2, 5 + 2\sqrt{7}) \approx (-2, 10.29)
\]

\textbf{Respuesta completa:}
\begin{itemize}
    \item \textbf{Parte a:} $\frac{(x - 4)^2}{25} + \frac{(y + 3)^2}{9} = 1$
    \item \textbf{Parte b:} $\frac{(x + 2)^2}{36} + \frac{(y - 5)^2}{64} = 1$
\end{itemize}
\end{solucion}

\newpage

\begin{solucion}[title={Solución Ejercicio 3}]
\textbf{Parte a)} $\frac{x^2}{64} + \frac{y^2}{36} = 1$

\textbf{Paso 1:} Identificar parámetros.

\begin{align*}
\text{Centro: } &(0, 0) \\
a^2 = 64 &\Rightarrow a = 8 \quad \text{(mayor, con } x^2\text{)} \\
b^2 = 36 &\Rightarrow b = 6 \\
\text{Eje mayor: } &\text{horizontal}
\end{align*}

\textbf{Paso 2:} Calcular $c$.

\[
c = \sqrt{64 - 36} = \sqrt{28} = 2\sqrt{7} \approx 5.29
\]

\textbf{Paso 3:} Determinar focos y vértices.

\begin{itemize}
    \item Focos: $F_1(-2\sqrt{7}, 0)$, $F_2(2\sqrt{7}, 0)$
    \item Vértices mayores: $V_1(-8, 0)$, $V_2(8, 0)$
    \item Vértices menores: $B_1(0, -6)$, $B_2(0, 6)$
    \item Eje mayor: $2a = 16$
    \item Eje menor: $2b = 12$
\end{itemize}

\textbf{Paso 4:} Calcular excentricidad.

\[
e = \frac{2\sqrt{7}}{8} = \frac{\sqrt{7}}{4} \approx 0.661
\]

\vspace{0.5cm}

\textbf{Parte b)} $\frac{(x + 3)^2}{25} + \frac{(y - 2)^2}{49} = 1$

\textbf{Paso 1:} Identificar parámetros.

\begin{align*}
\text{Centro: } &(-3, 2) \\
a^2 = 49 &\Rightarrow a = 7 \quad \text{(mayor, con } y^2\text{)} \\
b^2 = 25 &\Rightarrow b = 5 \\
\text{Eje mayor: } &\text{vertical}
\end{align*}

\textbf{Paso 2:} Calcular $c$.

\[
c = \sqrt{49 - 25} = \sqrt{24} = 2\sqrt{6} \approx 4.90
\]

\textbf{Paso 3:} Determinar focos y vértices.

\begin{itemize}
    \item Focos: $F_1(-3, 2 - 2\sqrt{6})$, $F_2(-3, 2 + 2\sqrt{6})$
    \item Vértices mayores: $V_1(-3, -5)$, $V_2(-3, 9)$
    \item Vértices menores: $B_1(-8, 2)$, $B_2(2, 2)$
    \item Eje mayor: $2a = 14$ (vertical)
    \item Eje menor: $2b = 10$ (horizontal)
\end{itemize}

\textbf{Paso 4:} Calcular excentricidad.

\[
e = \frac{2\sqrt{6}}{7} \approx 0.70
\]

\textbf{Respuesta completa:}
\begin{itemize}
    \item \textbf{Parte a:} Centro: $(0,0)$; Focos: $(\pm 2\sqrt{7}, 0)$; Ejes: $16 \times 12$; $e \approx 0.661$
    \item \textbf{Parte b:} Centro: $(-3,2)$; Focos: $(-3, 2 \pm 2\sqrt{6})$; Ejes: $10 \times 14$; $e \approx 0.70$
\end{itemize}
\end{solucion}

\newpage

\begin{solucion}[title={Solución Ejercicio 4}]
\textbf{Parte a)} $4x^2 + 9y^2 - 16x + 18y - 11 = 0$

\textbf{Paso 1:} Agrupar términos y factorizar.

\begin{align*}
(4x^2 - 16x) + (9y^2 + 18y) &= 11 \\
4(x^2 - 4x) + 9(y^2 + 2y) &= 11
\end{align*}

\textbf{Paso 2:} Completar cuadrados.

Para $x^2 - 4x$: $(x - 2)^2 - 4$

Para $y^2 + 2y$: $(y + 1)^2 - 1$

\textbf{Paso 3:} Sustituir.

\begin{align*}
4[(x - 2)^2 - 4] + 9[(y + 1)^2 - 1] &= 11 \\
4(x - 2)^2 - 16 + 9(y + 1)^2 - 9 &= 11 \\
4(x - 2)^2 + 9(y + 1)^2 &= 36
\end{align*}

\textbf{Paso 4:} Dividir por 36.

\[
\frac{(x - 2)^2}{9} + \frac{(y + 1)^2}{4} = 1
\]
\[
\boxed{\frac{(x - 2)^2}{9} + \frac{(y + 1)^2}{4} = 1}
\]

\textbf{Paso 5:} Identificar elementos.

\begin{itemize}
    \item Centro: $(2, -1)$
    \item $a^2 = 9 \Rightarrow a = 3$ (horizontal)
    \item $b^2 = 4 \Rightarrow b = 2$
    \item $c = \sqrt{9 - 4} = \sqrt{5} \approx 2.24$
    \item Focos: $(2 \pm \sqrt{5}, -1)$
    \item Vértices: $(2 \pm 3, -1)$ y $(2, -1 \pm 2)$
    \item Excentricidad: $e = \frac{\sqrt{5}}{3} \approx 0.745$
\end{itemize}

\vspace{0.5cm}

\textbf{Parte b)} $16x^2 + 25y^2 + 32x - 150y - 159 = 0$

\textbf{Paso 1:} Agrupar y factorizar.

\begin{align*}
16(x^2 + 2x) + 25(y^2 - 6y) &= 159
\end{align*}

\textbf{Paso 2:} Completar cuadrados.

Para $x^2 + 2x$: $(x + 1)^2 - 1$

Para $y^2 - 6y$: $(y - 3)^2 - 9$

\textbf{Paso 3:} Sustituir.

\begin{align*}
16[(x + 1)^2 - 1] + 25[(y - 3)^2 - 9] &= 159 \\
16(x + 1)^2 - 16 + 25(y - 3)^2 - 225 &= 159 \\
16(x + 1)^2 + 25(y - 3)^2 &= 400
\end{align*}

\textbf{Paso 4:} Dividir por 400.

\[
\frac{(x + 1)^2}{25} + \frac{(y - 3)^2}{16} = 1
\]
\[
\boxed{\frac{(x + 1)^2}{25} + \frac{(y - 3)^2}{16} = 1}
\]

\textbf{Paso 5:} Identificar elementos.

\begin{itemize}
    \item Centro: $(-1, 3)$
    \item $a^2 = 25 \Rightarrow a = 5$ (horizontal)
    \item $b^2 = 16 \Rightarrow b = 4$
    \item $c = \sqrt{25 - 16} = 3$
    \item Focos: $(-1 \pm 3, 3) = (-4, 3)$ y $(2, 3)$
    \item Vértices: $(-6, 3)$, $(4, 3)$, $(-1, -1)$, $(-1, 7)$
    \item Excentricidad: $e = \frac{3}{5} = 0.6$
\end{itemize}

\textbf{Respuesta completa:}
\begin{itemize}
    \item \textbf{Parte a:} $\frac{(x - 2)^2}{9} + \frac{(y + 1)^2}{4} = 1$; Centro: $(2, -1)$; $e \approx 0.745$
    \item \textbf{Parte b:} $\frac{(x + 1)^2}{25} + \frac{(y - 3)^2}{16} = 1$; Centro: $(-1, 3)$; $e = 0.6$
\end{itemize}
\end{solucion}

\newpage

\begin{solucion}[title={Solución Ejercicio 5}]
\textbf{Parte a)} Focos: $F_1(-4, 0)$, $F_2(4, 0)$; Vértice mayor: $(6, 0)$

\textbf{Paso 1:} Determinar el centro.

El centro es el punto medio entre los focos:
\[
C = \left( \frac{-4 + 4}{2}, 0 \right) = (0, 0)
\]

\textbf{Paso 2:} Determinar la orientación y $c$.

Focos en eje $x \Rightarrow$ eje mayor horizontal

\[
c = 4
\]

\textbf{Paso 3:} Determinar $a$.

El vértice mayor $(6, 0)$ está a distancia $a$ del centro:
\[
a = 6
\]

\textbf{Paso 4:} Calcular $b$.

\begin{align*}
a^2 &= b^2 + c^2 \\
36 &= b^2 + 16 \\
b^2 &= 20 \\
b &= 2\sqrt{5}
\end{align*}

\textbf{Paso 5:} Escribir la ecuación.

\[
\boxed{\frac{x^2}{36} + \frac{y^2}{20} = 1}
\]

\vspace{0.5cm}

\textbf{Parte b)} Focos: $F_1(2, 1)$, $F_2(2, 7)$; Vértice mayor: $(2, 9)$

\textbf{Paso 1:} Determinar el centro.

\[
C = \left( 2, \frac{1 + 7}{2} \right) = (2, 4)
\]

\textbf{Paso 2:} Determinar la orientación y $c$.

Focos en línea vertical $\Rightarrow$ eje mayor vertical

\[
c = \frac{|7 - 1|}{2} = 3
\]

\textbf{Paso 3:} Determinar $a$.

Distancia del centro $(2, 4)$ al vértice $(2, 9)$:
\[
a = |9 - 4| = 5
\]

\textbf{Paso 4:} Calcular $b$.

\begin{align*}
25 &= b^2 + 9 \\
b^2 &= 16 \\
b &= 4
\end{align*}

\textbf{Paso 5:} Escribir la ecuación.

\[
\boxed{\frac{(x - 2)^2}{16} + \frac{(y - 4)^2}{25} = 1}
\]

\textbf{Respuesta completa:}
\begin{itemize}
    \item \textbf{Parte a:} $\frac{x^2}{36} + \frac{y^2}{20} = 1$
    \item \textbf{Parte b:} $\frac{(x - 2)^2}{16} + \frac{(y - 4)^2}{25} = 1$
\end{itemize}
\end{solucion}

\newpage

\begin{solucion}[title={Solución Ejercicio 6}]
\textbf{Parte a)} Centro $(0, 0)$, vértice mayor $(0, 10)$, $e = 0.6$

\textbf{Paso 1:} Determinar la orientación y $a$.

Vértice en $(0, 10) \Rightarrow$ eje mayor vertical

\[
a = 10
\]

\textbf{Paso 2:} Usar la excentricidad para encontrar $c$.

\begin{align*}
e &= \frac{c}{a} \\
0.6 &= \frac{c}{10} \\
c &= 6
\end{align*}

\textbf{Paso 3:} Calcular $b$.

\begin{align*}
a^2 &= b^2 + c^2 \\
100 &= b^2 + 36 \\
b^2 &= 64 \\
b &= 8
\end{align*}

\textbf{Paso 4:} Escribir la ecuación (eje vertical).

\[
\boxed{\frac{x^2}{64} + \frac{y^2}{100} = 1}
\]

\vspace{0.5cm}

\textbf{Parte b)} Centro $(3, -2)$, eje mayor horizontal de longitud 20, $e = 0.8$

\textbf{Paso 1:} Determinar $a$.

\[
2a = 20 \Rightarrow a = 10
\]

\textbf{Paso 2:} Usar la excentricidad para encontrar $c$.

\begin{align*}
0.8 &= \frac{c}{10} \\
c &= 8
\end{align*}

\textbf{Paso 3:} Calcular $b$.

\begin{align*}
100 &= b^2 + 64 \\
b^2 &= 36 \\
b &= 6
\end{align*}

\textbf{Paso 4:} Escribir la ecuación (eje horizontal, centro trasladado).

\[
\boxed{\frac{(x - 3)^2}{100} + \frac{(y + 2)^2}{36} = 1}
\]

\textbf{Respuesta completa:}
\begin{itemize}
    \item \textbf{Parte a:} $\frac{x^2}{64} + \frac{y^2}{100} = 1$
    \item \textbf{Parte b:} $\frac{(x - 3)^2}{100} + \frac{(y + 2)^2}{36} = 1$
\end{itemize}
\end{solucion}

\newpage

\begin{solucion}[title={Solución Ejercicio 7 - Aplicaciones Prácticas}]
\textbf{Parte a)} Órbita planetaria: $a = 200$ millones de km, $e = 0.05$

\textbf{Paso 1:} Calcular $c$ usando la excentricidad.

\begin{align*}
e &= \frac{c}{a} \\
0.05 &= \frac{c}{200} \\
c &= 10 \text{ millones de km}
\end{align*}

\textbf{Paso 2:} Calcular $b$.

\begin{align*}
a^2 &= b^2 + c^2 \\
40000 &= b^2 + 100 \\
b^2 &= 39900 \\
b &= \sqrt{39900} \approx 199.75 \text{ millones de km}
\end{align*}

\textbf{Paso 3:} Escribir la ecuación (centro en origen, eje horizontal).

\[
\boxed{\frac{x^2}{40000} + \frac{y^2}{39900} = 1}
\]

(Unidades: millones de km)

\textbf{Paso 4:} Calcular distancias mínima y máxima a la estrella.

La estrella está en un foco, por ejemplo $F_2(10, 0)$.

\begin{itemize}
    \item Distancia mínima (perihelio): $a - c = 200 - 10 = 190$ millones de km
    \item Distancia máxima (afelio): $a + c = 200 + 10 = 210$ millones de km
\end{itemize}

\vspace{0.5cm}

\textbf{Parte b)} Estadio: 120 m de largo, 80 m de ancho

\textbf{Paso 1:} Determinar $a$ y $b$.

\begin{align*}
2a &= 120 \Rightarrow a = 60 \text{ m} \\
2b &= 80 \Rightarrow b = 40 \text{ m}
\end{align*}

\textbf{Paso 2:} Escribir la ecuación.

\[
\boxed{\frac{x^2}{3600} + \frac{y^2}{1600} = 1}
\]

\textbf{Paso 3:} Calcular la posición de los focos.

\begin{align*}
c &= \sqrt{3600 - 1600} = \sqrt{2000} = 20\sqrt{5} \approx 44.72 \text{ m}
\end{align*}

Las torres deben ubicarse a $\pm 44.72$ metros del centro sobre el eje mayor (horizontal).

\vspace{0.5cm}

\textbf{Parte c)} Puente: 60 m de ancho (base), 20 m de altura máxima

\textbf{Paso 1:} Determinar parámetros.

\begin{align*}
2a &= 60 \Rightarrow a = 30 \text{ m (horizontal)} \\
2b &= 20 \Rightarrow b = 10 \text{ m (vertical)}
\end{align*}

\textbf{Paso 2:} Escribir la ecuación (eje mayor horizontal).

\[
\boxed{\frac{x^2}{900} + \frac{y^2}{100} = 1}
\]

\textbf{Paso 3:} Calcular la altura a 15 m del centro.

Cuando $x = 15$:
\begin{align*}
\frac{225}{900} + \frac{y^2}{100} &= 1 \\
\frac{1}{4} + \frac{y^2}{100} &= 1 \\
\frac{y^2}{100} &= \frac{3}{4} \\
y^2 &= 75 \\
y &= \sqrt{75} = 5\sqrt{3} \approx 8.66 \text{ m}
\end{align*}

La altura del arco a 15 m del centro es aproximadamente $8.66$ metros.

\textbf{Respuesta completa:}
\begin{itemize}
    \item \textbf{Parte a:} Ecuación: $\frac{x^2}{40000} + \frac{y^2}{39900} = 1$; Distancias: 190-210 millones de km
    \item \textbf{Parte b:} Ecuación: $\frac{x^2}{3600} + \frac{y^2}{1600} = 1$; Torres en $x = \pm 44.72$ m
    \item \textbf{Parte c:} Ecuación: $\frac{x^2}{900} + \frac{y^2}{100} = 1$; Altura a 15m: $8.66$ m
\end{itemize}
\end{solucion}

\newpage

\begin{solucion}[title={Solución Ejercicio 8 - Problemas Avanzados}]
\textbf{Parte a)} Centro $(1, 2)$, pasa por $(1, 8)$ y $(5, 2)$, eje mayor vertical

\textbf{Paso 1:} Usar el punto $(1, 8)$.

Como $x = 1$ (igual al centro), este punto está en el eje vertical. Es un vértice mayor:
\[
a = |8 - 2| = 6
\]

\textbf{Paso 2:} Usar el punto $(5, 2)$.

Como $y = 2$ (igual al centro), este punto está en el eje horizontal. Es un vértice menor:
\[
b = |5 - 1| = 4
\]

\textbf{Paso 3:} Escribir la ecuación (eje vertical).

\[
\boxed{\frac{(x - 1)^2}{16} + \frac{(y - 2)^2}{36} = 1}
\]

\textbf{Paso 4:} Calcular $c$ y focos.

\begin{align*}
c &= \sqrt{36 - 16} = \sqrt{20} = 2\sqrt{5} \approx 4.47
\end{align*}

Focos: $(1, 2 \pm 2\sqrt{5})$

\textbf{Paso 5:} Vértices.

\begin{itemize}
    \item Mayores: $(1, -4)$, $(1, 8)$
    \item Menores: $(-3, 2)$, $(5, 2)$
\end{itemize}

\textbf{Paso 6:} Excentricidad.

\[
e = \frac{2\sqrt{5}}{6} = \frac{\sqrt{5}}{3} \approx 0.745
\]

\vspace{0.5cm}

\textbf{Parte b)} Demostrar que para $\frac{x^2}{25} + \frac{y^2}{16} = 1$, la suma de distancias a los focos es $2a = 10$

\textbf{Paso 1:} Identificar parámetros.

\begin{align*}
a^2 &= 25 \Rightarrow a = 5 \\
b^2 &= 16 \Rightarrow b = 4 \\
c &= \sqrt{25 - 16} = 3
\end{align*}

Focos: $F_1(-3, 0)$, $F_2(3, 0)$

\textbf{Paso 2:} Verificar con $P(3, \frac{16}{5})$.

Primero verificamos que $P$ está en la elipse:
\[
\frac{9}{25} + \frac{(\frac{16}{5})^2}{16} = \frac{9}{25} + \frac{256/25}{16} = \frac{9}{25} + \frac{16}{25} = 1 \quad \checkmark
\]

\textbf{Paso 3:} Calcular distancias.

\begin{align*}
d(P, F_1) &= \sqrt{(3 - (-3))^2 + (\frac{16}{5} - 0)^2} \\
&= \sqrt{36 + \frac{256}{25}} \\
&= \sqrt{\frac{900 + 256}{25}} \\
&= \sqrt{\frac{1156}{25}} \\
&= \frac{34}{5}
\end{align*}

\begin{align*}
d(P, F_2) &= \sqrt{(3 - 3)^2 + (\frac{16}{5})^2} \\
&= \frac{16}{5}
\end{align*}

\textbf{Paso 4:} Verificar la suma.

\[
d(P, F_1) + d(P, F_2) = \frac{34}{5} + \frac{16}{5} = \frac{50}{5} = 10 = 2a \quad \checkmark
\]

\vspace{0.5cm}

\textbf{Parte c)} Área de la elipse $\frac{x^2}{36} + \frac{y^2}{16} = 1$

\textbf{Paso 1:} Identificar $a$ y $b$.

\[
a = 6, \quad b = 4
\]

\textbf{Paso 2:} Calcular el área usando $A = \pi ab$.

\[
A = \pi \cdot 6 \cdot 4 = 24\pi \approx 75.4 \text{ unidades}^2
\]

\textbf{Paso 3:} Comparar con el círculo de radio $a = 6$.

\[
A_{\text{círculo}} = \pi r^2 = \pi \cdot 36 = 36\pi \approx 113.1 \text{ unidades}^2
\]

\textbf{Paso 4:} Análisis.

\[
\frac{A_{\text{elipse}}}{A_{\text{círculo}}} = \frac{24\pi}{36\pi} = \frac{2}{3} \approx 0.667
\]

La elipse tiene aproximadamente el $67\%$ del área del círculo de radio $a$.

\textbf{Respuesta completa:}
\begin{itemize}
    \item \textbf{Parte a:} $\frac{(x - 1)^2}{16} + \frac{(y - 2)^2}{36} = 1$; Focos: $(1, 2 \pm 2\sqrt{5})$; $e \approx 0.745$
    \item \textbf{Parte b:} Demostrado: $d(P, F_1) + d(P, F_2) = 10 = 2a$ ✓
    \item \textbf{Parte c:} Área elipse: $24\pi \approx 75.4$; Área círculo: $36\pi \approx 113.1$ (67\%)
\end{itemize}
\end{solucion}

