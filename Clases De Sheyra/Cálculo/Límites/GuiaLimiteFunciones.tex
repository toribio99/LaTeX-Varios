\documentclass[12pt,a4paper]{article}
\usepackage{fontspec}
\usepackage[spanish,es-nodecimaldot]{babel}
\usepackage{amsmath,amssymb}
\usepackage[margin=2.5cm]{geometry}
\usepackage{xcolor}
\usepackage{tikz,pgfplots}
\usetikzlibrary{calc,arrows.meta,babel}
\usepackage{multicol}
\usepackage{enumitem}

\pgfplotsset{compat=1.18}

\definecolor{maincolor}{RGB}{26,35,126}
\definecolor{accentcolor}{RGB}{255,87,34}

\title{\textbf{\Large Límites de Funciones Reales:\\ Fundamentos y Aplicaciones}}
\author{\textbf{Prof. Toribio de J Arrieta F}}
\date{
    \textbf{Institución:} La Pruebita\\
    \textbf{Asignatura:} Cálculo Diferencial\\
    \textbf{Grado:} 11\\
    Noviembre 2025
}

\setlength{\parindent}{0pt}
\setlength{\parskip}{8pt}

\begin{document}

\maketitle
\thispagestyle{empty}

\newpage

\tableofcontents

\newpage

\section{Introducción}

El concepto de límite es uno de los pilares fundamentales del cálculo diferencial e integral. A través del estudio de límites, podemos analizar el comportamiento de funciones en puntos específicos o cuando las variables tienden a infinito, lo que nos permite comprender fenómenos dinámicos en física, ingeniería, economía y otras ciencias aplicadas.

Esta guía presenta de manera rigurosa y sistemática los fundamentos teóricos de los límites de funciones reales, junto con técnicas de cálculo, aplicaciones prácticas y ejercicios que desarrollarán su capacidad analítica y resolución de problemas.

\section{Fundamentos Teóricos}

\subsection{Definición Intuitiva de Límite}

Sea $f(x)$ una función definida en un intervalo abierto que contiene a $a$, excepto posiblemente en $a$ mismo. Decimos que el límite de $f(x)$ cuando $x$ tiende a $a$ es $L$, y escribimos:

\[
\lim_{x \to a} f(x) = L
\]

si podemos hacer que $f(x)$ esté arbitrariamente cerca de $L$ tomando valores de $x$ suficientemente cercanos a $a$, pero diferentes de $a$.

\textbf{Interpretación intuitiva:} El límite describe el valor al cual se aproxima la función cuando la variable independiente se acerca a un punto determinado, sin importar si la función está o no definida en ese punto.

\subsection{Notación Matemática}

La notación estándar para límites es:

\[
\lim_{x \to a} f(x) = L
\]

Se lee: El límite de $f(x)$ cuando $x$ tiende a $a$ es igual a $L$.

Esta notación fue introducida por el matemático alemán Karl Weierstrass en el siglo XIX y se ha convertido en el estándar universal para el cálculo.

\subsection{Interpretación Gráfica}

Gráficamente, el límite representa el valor de altura al cual se aproxima la curva de la función cuando nos acercamos al punto $x = a$ desde ambos lados. Es importante notar que:

\begin{itemize}[leftmargin=*]
    \item El límite puede existir incluso si la función no está definida en $x = a$
    \item El valor del límite puede ser diferente del valor de la función en $x = a$
    \item La función puede tener una discontinuidad en $x = a$ y aún así tener límite
\end{itemize}

\begin{center}
\begin{tikzpicture}
    \begin{axis}[
        width=12cm, height=8cm,
        axis lines=middle,
        xlabel={$x$}, ylabel={$y$},
        xmin=-1, xmax=5,
        ymin=-1, ymax=5,
        grid=both,
        samples=100,
        axis background/.style={fill=yellow!2},
        legend pos=north west,
    ]
    % Función con discontinuidad removible
    \addplot[red, very thick, domain=-1:1.95] {x + 2};
    \addplot[red, very thick, domain=2.05:5] {x + 2};
    \addplot[blue, only marks, mark size=3pt] coordinates {(2,3)};
    \draw[blue, fill=white] (axis cs:2,4) circle (3pt);
    \node at (axis cs:3.5,2) {\Large $\displaystyle \lim_{x \to 2} f(x) = 4$};
    \node at (axis cs:3.5,1.3) {pero \Large $\displaystyle f(2) = 3$};
    \end{axis}
\end{tikzpicture}
\end{center}

\subsection{Límites Laterales}

Los límites laterales describen el comportamiento de una función cuando nos aproximamos a un punto desde una dirección específica.

\textbf{Límite por la derecha (límite lateral derecho):}
\[
\lim_{x \to a^+} f(x) = L^+
\]
significa que $f(x)$ se aproxima a $L^+$ cuando $x$ se acerca a $a$ por valores mayores que $a$.

\textbf{Límite por la izquierda (límite lateral izquierdo):}
\[
\lim_{x \to a^-} f(x) = L^-
\]
significa que $f(x)$ se aproxima a $L^-$ cuando $x$ se acerca a $a$ por valores menores que $a$.

\subsection{Existencia del Límite}

\textbf{Teorema (Existencia del Límite):} El límite $\displaystyle\lim_{x \to a} f(x)$ existe y es igual a $L$ si y solo si:

\[
\lim_{x \to a^-} f(x) = \lim_{x \to a^+} f(x) = L
\]

Es decir, el límite existe cuando los límites laterales existen y son iguales.

\begin{center}
\begin{tikzpicture}
    \begin{axis}[
        width=12cm, height=8cm,
        axis lines=middle,
        xlabel={$x$}, ylabel={$y$},
        xmin=-1, xmax=5,
        ymin=-1, ymax=5,
        grid=both,
        samples=100,
        legend pos=north west,
    ]
    % Función con límites laterales diferentes
    \addplot[red, very thick, domain=-1:2] {0.5*x + 1};
    \addplot[red, very thick, domain=2:5] {-0.5*x + 4};
    \draw[blue, fill=white] (axis cs:2,2) circle (3pt);
    \draw[blue, fill=white] (axis cs:2,3) circle (3pt);
    \node at (axis cs:1.2,2.3) {$\lim_{x \to 2^{-}} f(x) = 2$};
    \node at (axis cs:2.1,3.4) {$\lim_{x \to 2^{+}} f(x) = 3$};
    \node at (axis cs:2.2,1) {$\therefore \lim_{x \to 2} f(x)$ no existe};
    \end{axis}
\end{tikzpicture}
\end{center}

\subsection{Propiedades de los Límites}

Sean $\displaystyle\lim_{x \to a} f(x) = L$ y $\displaystyle\lim_{x \to a} g(x) = M$, donde $L$ y $M$ son números reales. Entonces:

\begin{enumerate}[leftmargin=*]
    \item \textbf{Límite de una constante:} $\displaystyle\lim_{x \to a} c = c$

    \item \textbf{Límite de la función identidad:} $\displaystyle\lim_{x \to a} x = a$

    \item \textbf{Suma:} $\displaystyle\lim_{x \to a} [f(x) + g(x)] = L + M$

    \item \textbf{Diferencia:} $\displaystyle\lim_{x \to a} [f(x) - g(x)] = L - M$

    \item \textbf{Producto por constante:} $\displaystyle\lim_{x \to a} [c \cdot f(x)] = c \cdot L$

    \item \textbf{Producto:} $\displaystyle\lim_{x \to a} [f(x) \cdot g(x)] = L \cdot M$

    \item \textbf{Cociente:} $\displaystyle\lim_{x \to a} \frac{f(x)}{g(x)} = \frac{L}{M}$, siempre que $M \neq 0$

    \item \textbf{Potencia:} $\displaystyle\lim_{x \to a} [f(x)]^n = L^n$, donde $n$ es un entero positivo

    \item \textbf{Raíz:} $\displaystyle\lim_{x \to a} \sqrt[n]{f(x)} = \sqrt[n]{L}$, donde $n$ es un entero positivo (si $n$ es par, se requiere $L \geq 0$)
\end{enumerate}

\subsection{Indeterminaciones}

Cuando aplicamos las propiedades de límites, pueden surgir expresiones indeterminadas que requieren técnicas especiales:

\textbf{Formas indeterminadas comunes:}
\[
\frac{0}{0}, \quad \frac{\infty}{\infty}, \quad 0 \cdot \infty, \quad \infty - \infty, \quad 0^0, \quad 1^\infty, \quad \infty^0
\]

Las más frecuentes en cálculo elemental son $\dfrac{0}{0}$ y $\dfrac{\infty}{\infty}$.

\subsection{Técnicas para Resolver Indeterminaciones}

\subsubsection{Factorización}

Útil cuando aparece la indeterminación $\dfrac{0}{0}$ en expresiones polinómicas. Se factoriza el numerador y denominador para cancelar factores comunes.

\subsubsection{Racionalización}

Se emplea cuando hay radicales en el numerador o denominador. Consiste en multiplicar por el conjugado para eliminar la raíz.

\subsubsection{Simplificación Algebraica}

Incluye operaciones como combinar fracciones, expandir productos notables, o realizar sustituciones apropiadas.

\subsection{Límites al Infinito}

Cuando $x$ crece o decrece sin límite, estudiamos el comportamiento de $f(x)$ mediante:

\[
\lim_{x \to \infty} f(x) \quad \text{o} \quad \lim_{x \to -\infty} f(x)
\]

\textbf{Para funciones racionales:} Si $f(x) = \dfrac{P(x)}{Q(x)}$ donde $P$ y $Q$ son polinomios:

\begin{itemize}[leftmargin=*]
    \item Si grado$(P) <$ grado$(Q)$: $\displaystyle\lim_{x \to \infty} f(x) = 0$
    \item Si grado$(P) =$ grado$(Q)$: $\displaystyle\lim_{x \to \infty} f(x) = \frac{a_n}{b_m}$ (razón de coeficientes principales)
    \item Si grado$(P) >$ grado$(Q)$: $\displaystyle\lim_{x \to \infty} f(x) = \pm\infty$
\end{itemize}

\subsection{Límites Infinitos (Asíntotas Verticales)}

Un límite es infinito cuando $f(x)$ crece o decrece sin límite cuando $x$ se aproxima a $a$:

\[
\lim_{x \to a} f(x) = \infty \quad \text{o} \quad \lim_{x \to a} f(x) = -\infty
\]

Esto indica la presencia de una \textbf{asíntota vertical} en $x = a$.

\begin{center}
\begin{tikzpicture}
    \begin{axis}[
        width=12cm, height=8cm,
        axis lines=middle,
        xlabel={$x$}, ylabel={$y$},
        xmin=-1, xmax=5,
        ymin=-8, ymax=8,
        grid=both,
        samples=200,
        restrict y to domain=-10:10,
    ]
    \addplot[red, very thick, domain=-1:1.8] {1/(x-2)};
    \addplot[red, very thick, domain=2.2:5] {1/(x-2)};
    \draw[blue, dashed, very thick] (axis cs:2,-8) -- (axis cs:2,8);
    \node[blue, rotate=90] at (axis cs:1.8,2) {Asíntota vertical: $x = 2$};
    \end{axis}
\end{tikzpicture}
\end{center}

\subsection{Asíntotas Horizontales}

Si $\displaystyle\lim_{x \to \infty} f(x) = L$ o $\displaystyle\lim_{x \to -\infty} f(x) = L$, entonces la recta $y = L$ es una \textbf{asíntota horizontal}.

\begin{center}
\begin{tikzpicture}
    \begin{axis}[
        width=12cm, height=8cm,
        axis lines=middle,
        xlabel={$x$}, ylabel={$y$},
        xmin=-5, xmax=5,
        ymin=-1, ymax=4,
        grid=both,
        samples=200,
        axis background/.style={fill=yellow!2},
    ]
    \addplot[red, very thick, domain=-5:5] {2 + 3/(x^2+1)};
    \draw[blue, dashed, very thick] (axis cs:-5,2) -- (axis cs:5,2);
    \node[blue] at (axis cs:0,2.25) {Asíntota horizontal: $y = 2$};
    \end{axis}
\end{tikzpicture}
\end{center}

\subsection{Continuidad de Funciones}

Una función $f$ es \textbf{continua} en $x = a$ si se cumplen tres condiciones:

\begin{enumerate}[leftmargin=*]
    \item $f(a)$ está definida
    \item $\displaystyle\lim_{x \to a} f(x)$ existe
    \item $\displaystyle\lim_{x \to a} f(x) = f(a)$
\end{enumerate}

Si alguna de estas condiciones no se cumple, la función es \textbf{discontinua} en $x = a$.

\textbf{Tipos de discontinuidad:}

\begin{itemize}[leftmargin=*]
    \item \textbf{Evitable (removible):} El límite existe pero $f(a) \neq \lim_{x \to a} f(x)$ o $f(a)$ no está definida
    \item \textbf{De salto:} Los límites laterales existen pero son diferentes
    \item \textbf{Infinita:} Al menos uno de los límites laterales es infinito
    \item \textbf{Esencial:} Ninguno de los límites laterales existe
\end{itemize}

\subsection{Límites Trigonométricos Fundamentales}

Dos límites trigonométricos son fundamentales en el cálculo:

\[
\lim_{x \to 0} \frac{\sin x}{x} = 1
\]

\[
\lim_{x \to 0} \frac{1 - \cos x}{x} = 0
\]

Estos límites son esenciales para derivar funciones trigonométricas y resolver problemas aplicados.

\begin{center}
\begin{tikzpicture}
    \begin{axis}[
        width=12cm, height=8cm,
        axis lines=middle,
        xlabel={$x$}, ylabel={$y$},
        xmin=-6.5, xmax=6.5,
        ymin=-0.5, ymax=1.5,
        grid=both,
        grid style={line width=.2pt, draw=gray!40},
        samples=300,
		legend style={
		text=black,       % color del texto
		font=\small,
		axis background/.style={fill=yellow!2},
		},
    ]
    \addplot[red, very thick, domain=-6.28:6.28] {sin(deg(x))/x};
    \addplot[blue, dashed] coordinates {(-6.5,1) (6.5,1)};
    \addplot[blue, only marks, mark size=3pt] coordinates {(0,1)};
    \legend{$y = \frac{\sin x}{x}$, $y = 1$}
    \end{axis}
\end{tikzpicture}
\end{center}

\section{Aplicaciones de los Límites}

\subsection{Aplicación 1: Velocidad Instantánea en Física}

En física, la velocidad instantánea de un objeto en movimiento se define mediante un límite. Si $s(t)$ representa la posición de un objeto en el tiempo $t$, la velocidad instantánea en $t = t_0$ es:

\[
v(t_0) = \lim_{h \to 0} \frac{s(t_0 + h) - s(t_0)}{h}
\]

Este límite representa la derivada de la posición respecto al tiempo y es fundamental en cinemática.

\textbf{Ejemplo:} Un objeto cae libremente desde el reposo. Su posición está dada por $s(t) = 4.9t^2$ metros. La velocidad instantánea en $t = 2$ segundos es:

\[
v(2) = \lim_{h \to 0} \frac{4.9(2+h)^2 - 4.9(2)^2}{h} = \lim_{h \to 0} \frac{4.9(4 + 4h + h^2) - 19.6}{h} = \lim_{h \to 0} \frac{19.6h + 4.9h^2}{h} = 19.6 \text{ m/s}
\]

\subsection{Aplicación 2: Costo Marginal en Economía}

El costo marginal es el costo de producir una unidad adicional y se calcula mediante:

\[
C'(x) = \lim_{h \to 0} \frac{C(x + h) - C(x)}{h}
\]

donde $C(x)$ es la función de costo total para producir $x$ unidades.

\textbf{Ejemplo:} Si $C(x) = 1000 + 50x + 0.5x^2$ es el costo total (en dólares) de producir $x$ unidades, el costo marginal cuando $x = 100$ es:

\[
C'(100) = \lim_{h \to 0} \frac{[1000 + 50(100+h) + 0.5(100+h)^2] - [1000 + 5000 + 5000]}{h} = 150 \text{ dólares/unidad}
\]

\subsection{Aplicación 3: Circuitos Eléctricos en Estado Estacionario}

En circuitos RC (resistencia-capacitor), la corriente $i(t)$ tiende a cero cuando $t \to \infty$:

\[
\lim_{t \to \infty} i(t) = \lim_{t \to \infty} \frac{V}{R}e^{-t/(RC)} = 0
\]

donde $V$ es el voltaje, $R$ la resistencia, y $C$ la capacitancia. Esto describe el comportamiento del circuito en estado estacionario.

\subsection{Aplicación 4: Modelos de Crecimiento Poblacional}

El modelo logístico de crecimiento poblacional está dado por:

\[
P(t) = \frac{K}{1 + Ae^{-rt}}
\]

donde $K$ es la capacidad de carga del ambiente. El límite cuando $t \to \infty$ es:

\[
\lim_{t \to \infty} P(t) = \lim_{t \to \infty} \frac{K}{1 + Ae^{-rt}} = K
\]

Esto indica que la población se estabiliza en la capacidad de carga del ecosistema.

\subsection{Aplicación 5: Concentración de Medicamentos (Vida Media)}

La concentración de un medicamento en la sangre disminuye exponencialmente:

\[
C(t) = C_0 e^{-kt}
\]

donde $C_0$ es la concentración inicial y $k$ es la constante de eliminación. El límite cuando $t \to \infty$ es:

\[
\lim_{t \to \infty} C(t) = \lim_{t \to \infty} C_0 e^{-kt} = 0
\]

Esto indica que eventualmente el medicamento es completamente eliminado del organismo.

\subsection{Aplicación 6: Análisis de Estructuras}

En ingeniería estructural, el esfuerzo en una viga bajo carga distribuida $w(x)$ requiere calcular:

\[
\sigma(x_0) = \lim_{\Delta x \to 0} \frac{M(x_0 + \Delta x) - M(x_0)}{\Delta x}
\]

donde $M(x)$ es el momento flector. Este límite determina la distribución de esfuerzos y es crucial para el diseño seguro de estructuras.

\section{Ejemplos Resueltos}

\subsection{Ejemplo 1: Límite por Sustitución Directa}

\textbf{Problema:} Calcular $\displaystyle\lim_{x \to 2} (3x^2 - 5x + 1)$

\textbf{Solución:}

Paso 1: Verificamos si la función es continua en $x = 2$. Como se trata de una función polinómica, es continua en todos los números reales.

Paso 2: Aplicamos sustitución directa:
\begin{align*}
\lim_{x \to 2} (3x^2 - 5x + 1) &= 3(2)^2 - 5(2) + 1\\
&= 3(4) - 10 + 1\\
&= 12 - 10 + 1\\
&= 3
\end{align*}

\textbf{Respuesta:} $\boxed{\displaystyle\lim_{x \to 2} (3x^2 - 5x + 1) = 3}$

\textbf{Representación gráfica:}

\begin{center}
\begin{tikzpicture}
    \begin{axis}[
        width=12cm, height=8cm,
        axis lines=middle,
        xlabel={$x$}, ylabel={$y$},
        xmin=-1, xmax=4,
        ymin=-2, ymax=8,
        grid=both,
        samples=100,
    ]
    \addplot[red, very thick, domain=-1:4] {3*x^2 - 5*x + 1};
    \addplot[blue, only marks, mark size=4pt] coordinates {(2,3)};
    \node[blue] at (axis cs:2.8,3.5) {$(2, 3)$};
    \draw[green!60!black, dashed] (axis cs:2,-2) -- (axis cs:2,3);
    \draw[green!60!black, dashed] (axis cs:-1,3) -- (axis cs:2,3);
    \end{axis}
\end{tikzpicture}
\end{center}

\subsection{Ejemplo 2: Indeterminación $\dfrac{0}{0}$ con Factorización}

\textbf{Problema:} Calcular $\displaystyle\lim_{x \to 3} \frac{x^2 - 9}{x - 3}$

\textbf{Solución:}

Paso 1: Intentamos sustitución directa:
\[
\frac{3^2 - 9}{3 - 3} = \frac{0}{0} \quad \text{(Indeterminación)}
\]

Paso 2: Factorizamos el numerador usando diferencia de cuadrados:
\[
x^2 - 9 = (x - 3)(x + 3)
\]

Paso 3: Simplificamos la expresión:
\begin{align*}
\lim_{x \to 3} \frac{x^2 - 9}{x - 3} &= \lim_{x \to 3} \frac{(x - 3)(x + 3)}{x - 3}\\
&= \lim_{x \to 3} (x + 3) \quad \text{(para } x \neq 3\text{)}
\end{align*}

Paso 4: Ahora aplicamos sustitución directa:
\[
\lim_{x \to 3} (x + 3) = 3 + 3 = 6
\]

\textbf{Respuesta:} $\boxed{\displaystyle\lim_{x \to 3} \frac{x^2 - 9}{x - 3} = 6}$

\textbf{Interpretación gráfica:}

La función tiene una discontinuidad removible en $x = 3$. El límite existe y es igual a 6, aunque la función no está definida en ese punto.

\begin{center}
\begin{tikzpicture}
    \begin{axis}[
        width=12cm, height=8cm,
        axis lines=middle,
        xlabel={$x$}, ylabel={$y$},
        xmin=-1, xmax=6,
        ymin=-2, ymax=10,
        grid=both,
        samples=100,
    ]
    \addplot[red, very thick, domain=-1:2.9] {x + 3};
    \addplot[red, very thick, domain=3.1:6] {x + 3};
    \draw[blue, fill=white] (axis cs:3,6) circle (3pt);
    \node[blue, rotate=18.5] at (axis cs:3,6.8) {Discontinuidad removible en $(3, 6)$};
    \draw[green!60!black, dashed] (axis cs:3,-2) -- (axis cs:3,6);
    \end{axis}
\end{tikzpicture}
\end{center}

\subsection{Ejemplo 3: Racionalización}

\textbf{Problema:} Calcular $\displaystyle\lim_{x \to 0} \frac{\sqrt{x + 4} - 2}{x}$

\textbf{Solución:}

Paso 1: Verificamos si hay indeterminación:
\[
\frac{\sqrt{0 + 4} - 2}{0} = \frac{2 - 2}{0} = \frac{0}{0} \quad \text{(Indeterminación)}
\]

Paso 2: Multiplicamos numerador y denominador por el conjugado del numerador:
\begin{align*}
\lim_{x \to 0} \frac{\sqrt{x + 4} - 2}{x} &= \lim_{x \to 0} \frac{\sqrt{x + 4} - 2}{x} \cdot \frac{\sqrt{x + 4} + 2}{\sqrt{x + 4} + 2}
\end{align*}

Paso 3: Aplicamos diferencia de cuadrados en el numerador:
\begin{align*}
&= \lim_{x \to 0} \frac{(\sqrt{x + 4})^2 - 2^2}{x(\sqrt{x + 4} + 2)}\\
&= \lim_{x \to 0} \frac{x + 4 - 4}{x(\sqrt{x + 4} + 2)}\\
&= \lim_{x \to 0} \frac{x}{x(\sqrt{x + 4} + 2)}
\end{align*}

Paso 4: Simplificamos cancelando $x$ (para $x \neq 0$):
\begin{align*}
&= \lim_{x \to 0} \frac{1}{\sqrt{x + 4} + 2}
\end{align*}

Paso 5: Aplicamos sustitución directa:
\[
= \frac{1}{\sqrt{0 + 4} + 2} = \frac{1}{2 + 2} = \frac{1}{4}
\]

\textbf{Respuesta:} $\boxed{\displaystyle\lim_{x \to 0} \frac{\sqrt{x + 4} - 2}{x} = \frac{1}{4}}$

\textbf{Representación gráfica:}

\begin{center}
\begin{tikzpicture}
    \begin{axis}[
        width=12cm, height=8cm,
        axis lines=middle,
        xlabel={$x$}, ylabel={$y$},
        xmin=-3, xmax=3,
        ymin=-0.2, ymax=0.8,
        grid=both,
        samples=200,
    ]
    \addplot[red, very thick, domain=-3.9:-0.05] {(sqrt(x + 4) - 2)/x};
    \addplot[red, very thick, domain=0.05:3] {(sqrt(x + 4) - 2)/x};
    \draw[blue, fill=white] (axis cs:0,0.25) circle (3pt);
    \addplot[green!60!black, dashed] coordinates {(-3,0.25) (3,0.25)};
    \node[blue] at (axis cs:0.9,0.3) {$\lim_{x \to 0} f(x) = \frac{1}{4}$};
    \end{axis}
\end{tikzpicture}
\end{center}

\subsection{Ejemplo 4: Límites Laterales con Función por Partes}

\textbf{Problema:} Dada la función por partes
\[
f(x) = \begin{cases}
x^2 + 1 & \text{si } x < 2\\
3x - 1 & \text{si } x \geq 2
\end{cases}
\]
Calcular $\displaystyle\lim_{x \to 2^-} f(x)$, $\displaystyle\lim_{x \to 2^+} f(x)$ y $\displaystyle\lim_{x \to 2} f(x)$.

\textbf{Solución:}

Paso 1: Calculamos el límite lateral izquierdo ($x \to 2^-$):

Cuando $x$ se aproxima a 2 por la izquierda, usamos la primera expresión:
\begin{align*}
\lim_{x \to 2^-} f(x) &= \lim_{x \to 2^-} (x^2 + 1)\\
&= 2^2 + 1\\
&= 5
\end{align*}

Paso 2: Calculamos el límite lateral derecho ($x \to 2^+$):

Cuando $x$ se aproxima a 2 por la derecha, usamos la segunda expresión:
\begin{align*}
\lim_{x \to 2^+} f(x) &= \lim_{x \to 2^+} (3x - 1)\\
&= 3(2) - 1\\
&= 5
\end{align*}

Paso 3: Determinamos si existe el límite:

Como $\displaystyle\lim_{x \to 2^-} f(x) = \lim_{x \to 2^+} f(x) = 5$, concluimos que:
\[
\lim_{x \to 2} f(x) = 5
\]

Paso 4: Verificamos continuidad:

El valor de la función en $x = 2$ es:
\[
f(2) = 3(2) - 1 = 5
\]

Como $\displaystyle\lim_{x \to 2} f(x) = f(2) = 5$, la función es continua en $x = 2$.

\textbf{Respuesta:} $\boxed{\displaystyle\lim_{x \to 2^-} f(x) = 5, \quad \lim_{x \to 2^+} f(x) = 5, \quad \lim_{x \to 2} f(x) = 5}$

\textbf{Representación gráfica:}

\begin{center}
\begin{tikzpicture}
    \begin{axis}[
        width=12cm, height=8cm,
        axis lines=middle,
        xlabel={$x$}, ylabel={$y$},
        xmin=-1, xmax=5,
        ymin=-1, ymax=12,
        grid=both,
        samples=100,
    ]
    \addplot[red, very thick, domain=-1:2] {x^2 + 1};
    \addplot[red, very thick, domain=2:5] {3*x - 1};
    \addplot[blue, only marks, mark size=4pt] coordinates {(2,5)};
    \draw[green!60!black, dashed] (axis cs:2,-1) -- (axis cs:2,5);
    \draw[green!60!black, dashed] (axis cs:-1,5) -- (axis cs:2,5);
    \node[blue] at (axis cs:3.5,4.9) {Función continua en $x = 2$};
    \end{axis}
\end{tikzpicture}
\end{center}

\subsection{Ejemplo 5: Límite al Infinito}

\textbf{Problema:} Calcular $\displaystyle\lim_{x \to \infty} \frac{3x^2 + 5x - 1}{2x^2 - x + 7}$

\textbf{Solución:}

Paso 1: Identificamos los grados del numerador y denominador:
\begin{itemize}
    \item Grado del numerador: 2
    \item Grado del denominador: 2
\end{itemize}

Como los grados son iguales, el límite es la razón de los coeficientes principales.

Paso 2: Método alternativo (dividir por la mayor potencia):

Dividimos numerador y denominador por $x^2$:
\begin{align*}
\lim_{x \to \infty} \frac{3x^2 + 5x - 1}{2x^2 - x + 7} &= \lim_{x \to \infty} \frac{\dfrac{3x^2}{x^2} + \dfrac{5x}{x^2} - \dfrac{1}{x^2}}{\dfrac{2x^2}{x^2} - \dfrac{x}{x^2} + \dfrac{7}{x^2}}\\
&= \lim_{x \to \infty} \frac{3 + \dfrac{5}{x} - \dfrac{1}{x^2}}{2 - \dfrac{1}{x} + \dfrac{7}{x^2}}
\end{align*}

Paso 3: Aplicamos los límites:

Sabemos que $\displaystyle\lim_{x \to \infty} \frac{1}{x} = 0$ y $\displaystyle\lim_{x \to \infty} \frac{1}{x^2} = 0$

Por lo tanto:
\begin{align*}
\lim_{x \to \infty} \frac{3 + \dfrac{5}{x} - \dfrac{1}{x^2}}{2 - \dfrac{1}{x} + \dfrac{7}{x^2}} &= \frac{3 + 0 - 0}{2 - 0 + 0}\\
&= \frac{3}{2}
\end{align*}

\textbf{Respuesta:} $\boxed{\displaystyle\lim_{x \to \infty} \frac{3x^2 + 5x - 1}{2x^2 - x + 7} = \frac{3}{2}}$

\textbf{Interpretación:} La función tiene una asíntota horizontal en $y = \dfrac{3}{2}$.

\textbf{Representación gráfica:}

\begin{center}
\begin{tikzpicture}
    \begin{axis}[
        width=12cm, height=8cm,
        axis lines=middle,
        xlabel={$x$}, ylabel={$y$},
        xmin=-5, xmax=20,
        ymin=-0.5, ymax=3,
        grid=both,
        samples=200,
    ]
    \addplot[red, very thick, domain=-5:20] {(3*x^2 + 5*x - 1)/(2*x^2 - x + 7)};
    \addplot[blue, dashed, very thick] coordinates {(-5,1.5) (20,1.5)};
    \node[blue] at (axis cs:13,1.3) {Asíntota horizontal: $y = \frac{3}{2}$};
    \end{axis}
\end{tikzpicture}
\end{center}

\subsection{Ejemplo 6: Límite Infinito (Asíntota Vertical)}

\textbf{Problema:} Calcular $\displaystyle\lim_{x \to 2^+} \frac{1}{x - 2}$

\textbf{Solución:}

Paso 1: Analizamos el comportamiento cerca de $x = 2$ por la derecha.

Cuando $x \to 2^+$, tenemos $x > 2$, por lo tanto:
\begin{itemize}
    \item $x - 2 > 0$ (el denominador es positivo)
    \item $x - 2 \to 0^+$ (el denominador se aproxima a cero por valores positivos)
\end{itemize}

Paso 2: Construimos una tabla de valores para $x$ cercano a 2 por la derecha:

\begin{center}
\begin{tabular}{|c|c|c|}
\hline
$x$ & $x - 2$ & $f(x) = \dfrac{1}{x-2}$\\
\hline
2.1 & 0.1 & 10\\
2.01 & 0.01 & 100\\
2.001 & 0.001 & 1000\\
2.0001 & 0.0001 & 10000\\
\hline
\end{tabular}
\end{center}

Paso 3: Conclusión:

A medida que $x$ se aproxima a 2 por la derecha, el denominador se hace cada vez más pequeño (pero positivo), causando que la fracción crezca sin límite hacia valores positivos.

\[
\lim_{x \to 2^+} \frac{1}{x - 2} = +\infty
\]

\textbf{Respuesta:} $\boxed{\displaystyle\lim_{x \to 2^+} \frac{1}{x - 2} = +\infty}$

\textbf{Interpretación:} La función tiene una asíntota vertical en $x = 2$.

\textbf{Nota adicional:} Si calculáramos $\displaystyle\lim_{x \to 2^-} \frac{1}{x - 2}$, obtendríamos $-\infty$ porque el denominador sería negativo.

\textbf{Representación gráfica:}

\begin{center}
\begin{tikzpicture}
    \begin{axis}[
        width=12cm, height=10cm,
        axis lines=middle,
        xlabel={$x$}, ylabel={$y$},
        xmin=0, xmax=4,
        ymin=-10, ymax=10,
        grid=both,
        samples=500,
        restrict y to domain=-15:15,
    ]
    \addplot[red, very thick, domain=0:1.9] {1/(x-2)};
    \addplot[red, very thick, domain=2.1:4] {1/(x-2)};
    \draw[blue, dashed, very thick] (axis cs:2,-10) -- (axis cs:2,10);
    \node[blue, rotate=90] at (axis cs:1.85,3.5) {Asíntota vertical};
    \node[blue, rotate=90] at (axis cs:1.85,8.5) {$x = 2$};
    \draw[green!60!black, -{Latex}, very thick] (axis cs:2.3,6) -- (axis cs:2.05,8);
    \node[green!60!black] at (axis cs:3,6) {$\lim_{x \to 2^{+}} f(x) = +\infty$};
    \end{axis}
\end{tikzpicture}
\end{center}

\subsection{Ejemplo 7: Límite Trigonométrico}

\textbf{Problema:} Calcular $\displaystyle\lim_{x \to 0} \frac{\sin(3x)}{x}$

\textbf{Solución:}

Paso 1: Intentamos sustitución directa:
\[
\frac{\sin(3 \cdot 0)}{0} = \frac{0}{0} \quad \text{(Indeterminación)}
\]

Paso 2: Utilizamos el límite fundamental $\displaystyle\lim_{u \to 0} \frac{\sin u}{u} = 1$.

Multiplicamos y dividimos por 3 para crear la forma del límite fundamental:
\begin{align*}
\lim_{x \to 0} \frac{\sin(3x)}{x} &= \lim_{x \to 0} \frac{\sin(3x)}{x} \cdot \frac{3}{3}\\
&= \lim_{x \to 0} 3 \cdot \frac{\sin(3x)}{3x}\\
&= 3 \cdot \lim_{x \to 0} \frac{\sin(3x)}{3x}
\end{align*}

Paso 3: Realizamos la sustitución $u = 3x$:

Cuando $x \to 0$, entonces $u = 3x \to 0$.

\begin{align*}
3 \cdot \lim_{x \to 0} \frac{\sin(3x)}{3x} &= 3 \cdot \lim_{u \to 0} \frac{\sin u}{u}\\
&= 3 \cdot 1\\
&= 3
\end{align*}

\textbf{Respuesta:} $\boxed{\displaystyle\lim_{x \to 0} \frac{\sin(3x)}{x} = 3}$

\textbf{Verificación numérica:}

\begin{center}
\begin{tabular}{|c|c|}
\hline
$x$ & $\dfrac{\sin(3x)}{x}$\\
\hline
0.1 & 2.9552\\
0.01 & 2.9996\\
0.001 & 3.0000\\
-0.1 & 2.9552\\
-0.01 & 2.9996\\
\hline
\end{tabular}
\end{center}

\textbf{Representación gráfica:}

\begin{center}
\begin{tikzpicture}
    \begin{axis}[
        width=12cm, height=8cm,
        axis lines=middle,
        xlabel={$x$}, ylabel={$y$},
        xmin=-2, xmax=2,
        ymin=-1, ymax=5,
        grid=both,
        samples=500,
        restrict y to domain=-2:6,
    ]
    \addplot[red, very thick, domain=-2:-0.05] {sin(3*deg(x))/x};
    \addplot[red, very thick, domain=0.05:2] {sin(3*deg(x))/x};
    \draw[blue, fill=white] (axis cs:0,3) circle (3pt);
    \addplot[green!60!black, dashed] coordinates {(-2,3) (2,3)};
    \node[blue] at (axis cs:0.65,3.3) {$\lim_{x \to 0} \frac{\sin(3x)}{x} = 3$};
    \end{axis}
\end{tikzpicture}
\end{center}

\section{Ejercicios Propuestos}

Resuelva los siguientes ejercicios aplicando las técnicas estudiadas. Justifique cada paso de su solución.

\subsection{Límites Algebraicos}

\begin{enumerate}[leftmargin=*]
    \item Calcular $\displaystyle\lim_{x \to 4} \frac{x^2 - 16}{x - 4}$

    \item Calcular $\displaystyle\lim_{x \to 1} \frac{x^3 - 1}{x^2 - 1}$

    \item Calcular $\displaystyle\lim_{x \to 0} \frac{\sqrt{1 + x} - 1}{x}$
\end{enumerate}

\subsection{Indeterminaciones}

\begin{enumerate}[leftmargin=*, resume]
    \item Calcular $\displaystyle\lim_{x \to 2} \frac{x^3 - 8}{x^2 - 4}$

    \item Calcular $\displaystyle\lim_{x \to 0} \frac{(1 + x)^3 - 1}{x}$
\end{enumerate}

\subsection{Límites al Infinito}

\begin{enumerate}[leftmargin=*, resume]
    \item Calcular $\displaystyle\lim_{x \to \infty} \frac{5x^3 - 2x + 1}{3x^3 + x^2 - 7}$

    \item Calcular $\displaystyle\lim_{x \to -\infty} \frac{4x^2 + 3x - 1}{2x^3 - x + 5}$
\end{enumerate}

\subsection{Límites Trigonométricos}

\begin{enumerate}[leftmargin=*, resume]
    \item Calcular $\displaystyle\lim_{x \to 0} \frac{\sin(5x)}{\sin(2x)}$
\end{enumerate}

\section{Ejercicios Inversos}

Estos ejercicios requieren un enfoque de análisis inverso, donde se proporciona información sobre el límite y se debe determinar características de la función.

\subsection{Ejercicio Inverso 1: Determinar Constantes}

Dada la función
\[
f(x) = \begin{cases}
ax^2 + b & \text{si } x < 3\\
2x + 1 & \text{si } x \geq 3
\end{cases}
\]

Determine los valores de $a$ y $b$ para que $\displaystyle\lim_{x \to 3} f(x)$ exista y sea igual a 7.

\textbf{Pista:} Para que el límite exista, los límites laterales deben ser iguales. Además, ambos deben ser iguales a 7.

\subsection{Ejercicio Inverso 2: Continuidad Condicionada}

Determine el valor de la constante $k$ para que la función
\[
g(x) = \begin{cases}
\dfrac{x^2 - 4}{x - 2} & \text{si } x \neq 2\\
k & \text{si } x = 2
\end{cases}
\]
sea continua en $x = 2$.

\textbf{Pista:} Para que sea continua, se debe cumplir que $g(2) = \displaystyle\lim_{x \to 2} g(x)$.

\subsection{Ejercicio Inverso 3: Construir Función con Comportamiento Dado}

Construya una función racional $h(x) = \dfrac{P(x)}{Q(x)}$ (donde $P$ y $Q$ son polinomios) que satisfaga las siguientes condiciones:

\begin{itemize}[leftmargin=*]
    \item Tiene una asíntota vertical en $x = 1$
    \item Tiene una asíntota horizontal en $y = 2$
    \item $h(0) = 3$
\end{itemize}

\textbf{Pista:} Una asíntota vertical en $x = 1$ sugiere que $(x - 1)$ es factor del denominador. Una asíntota horizontal en $y = 2$ cuando los grados del numerador y denominador son iguales con razón de coeficientes 2.

\subsection{Ejercicio Inverso 4: Hallar Asíntotas}

Dada la función $f(x) = \dfrac{2x^3 + 3x^2 - 5x + 1}{x^2 - 4}$:

\begin{enumerate}[label=\alph*), leftmargin=*]
    \item Determine todas las asíntotas verticales
    \item Investigue si existe asíntota horizontal
    \item Determine el comportamiento asintótico cuando $x \to \infty$ y cuando $x \to -\infty$
\end{enumerate}

\textbf{Pista:} Para asíntotas verticales, busque donde el denominador es cero. Para asíntotas horizontales, compare los grados del numerador y denominador.

\subsection{Ejercicio Inverso 5: Intervalos de Continuidad}

Dada la función
\[
f(x) = \frac{x^2 - 9}{|x - 3|}
\]

\begin{enumerate}[label=\alph*), leftmargin=*]
    \item Determine el dominio de $f(x)$
    \item Calcule $\displaystyle\lim_{x \to 3^-} f(x)$ y $\displaystyle\lim_{x \to 3^+} f(x)$
    \item Determine los intervalos donde la función es continua
    \item Clasifique el tipo de discontinuidad en $x = 3$
\end{enumerate}

\textbf{Pista:} Recuerde que $|x - 3| = \begin{cases} x - 3 & \text{si } x \geq 3\\ -(x - 3) & \text{si } x < 3 \end{cases}$

\section{Teoremas Importantes sobre Límites}

\subsection{Teorema del Sandwich (o del Emparedado)}

Si $g(x) \leq f(x) \leq h(x)$ para todo $x$ en un intervalo abierto que contiene a $a$ (excepto posiblemente en $a$), y si
\[
\lim_{x \to a} g(x) = \lim_{x \to a} h(x) = L
\]
entonces
\[
\lim_{x \to a} f(x) = L
\]

Este teorema es especialmente útil para demostrar límites trigonométricos.

\subsection{Teorema de Existencia de Límites}

Si $f$ es una función creciente (o decreciente) en un intervalo $(a, b)$ y está acotada superiormente (o inferiormente), entonces los límites $\displaystyle\lim_{x \to a^+} f(x)$ y $\displaystyle\lim_{x \to b^-} f(x)$ existen.

\subsection{Teorema del Valor Intermedio}

Si $f$ es continua en el intervalo cerrado $[a, b]$ y $N$ es cualquier número entre $f(a)$ y $f(b)$, entonces existe al menos un número $c$ en $(a, b)$ tal que $f(c) = N$.

Este teorema garantiza que una función continua toma todos los valores intermedios entre dos puntos.

\section{Estrategias Generales para Calcular Límites}

\subsection{Procedimiento Sistemático}

Al enfrentar un problema de límites, siga este proceso:

\begin{enumerate}[leftmargin=*]
    \item \textbf{Sustitución directa:} Intente evaluar la función en el punto. Si obtiene un valor definido, ese es el límite.

    \item \textbf{Identificar indeterminaciones:} Si obtiene $\dfrac{0}{0}$, $\dfrac{\infty}{\infty}$, u otra forma indeterminada, proceda al siguiente paso.

    \item \textbf{Simplificación algebraica:}
    \begin{itemize}
        \item Factorización (para expresiones polinómicas)
        \item Racionalización (para expresiones con radicales)
        \item Operaciones con fracciones complejas
    \end{itemize}

    \item \textbf{Límites al infinito:} Divida por la mayor potencia de $x$ en el denominador.

    \item \textbf{Límites laterales:} Para funciones por partes, evalúe cada expresión según corresponda.

    \item \textbf{Límites trigonométricos:} Use identidades y los límites fundamentales.
\end{enumerate}

\subsection{Errores Comunes a Evitar}

\begin{itemize}[leftmargin=*]
    \item No confunda $\dfrac{0}{0}$ (indeterminado) con $\dfrac{k}{0}$ donde $k \neq 0$ (infinito)

    \item No cancele términos antes de verificar que son factores

    \item No asuma que si hay una discontinuidad, el límite no existe

    \item En límites al infinito, no ignore términos de menor grado antes de dividir

    \item Para límites laterales, use la expresión correcta de la función por partes
\end{itemize}

\section{Conexión con la Derivada}

El concepto de límite es fundamental para definir la derivada. La derivada de una función $f$ en un punto $x = a$ se define como:

\[
f'(a) = \lim_{h \to 0} \frac{f(a + h) - f(a)}{h}
\]

Este límite representa la pendiente de la recta tangente a la curva en el punto $(a, f(a))$ y es la base del cálculo diferencial.

\textbf{Interpretación geométrica:}

\begin{center}
\begin{tikzpicture}
    \begin{axis}[
        width=12cm, height=8cm,
        axis lines=middle,
        xlabel={$x$}, ylabel={$y$},
        xmin=0, xmax=5,
        ymin=0, ymax=10,
        grid=both,
        samples=100,
    ]
    \addplot[red, very thick, domain=0:5] {0.5*x^2 + 1};
    \addplot[blue, thick, domain=1:4] {2*x - 1};
    \addplot[green!60!black, only marks, mark size=3pt] coordinates {(2,3)};
    \draw[green!60!black, dashed] (axis cs:2,0) -- (axis cs:2,3);
    \node[red, rotate=50] at (axis cs:4.2,8.8) {$y = f(x)$};
    \node[blue, rotate=35] at (axis cs:4,6.5) {Recta tangente};
    \node[green!60!black] at (axis cs:1.5,3.5) {$(a, f(a))$};
    \end{axis}
\end{tikzpicture}
\end{center}

\section{Ejercicios Adicionales de Desafío}

\subsection{Desafío 1}

Calcular $\displaystyle\lim_{x \to 0} \frac{\sqrt{1 + x} - \sqrt{1 - x}}{x}$

\subsection{Desafío 2}

Si $\displaystyle\lim_{x \to 2} \frac{f(x) - 5}{x - 2} = 3$, determine $\displaystyle\lim_{x \to 2} f(x)$.

\subsection{Desafío 3}

Demuestre que $\displaystyle\lim_{x \to 0} x^2 \sin\left(\frac{1}{x}\right) = 0$ usando el teorema del sandwich.

\textbf{Pista:} Use que $-1 \leq \sin\left(\dfrac{1}{x}\right) \leq 1$ para todo $x \neq 0$.

\subsection{Desafío 4}

Calcular $\displaystyle\lim_{x \to 1} \frac{\sqrt{x} - 1}{\sqrt[3]{x} - 1}$

\subsection{Desafío 5}

Determine para qué valores de $a$ y $b$ la función
\[
f(x) = \begin{cases}
\dfrac{x^2 - 1}{x - 1} & \text{si } x < 1\\
ax + b & \text{si } 1 \leq x < 2\\
3x - 2 & \text{si } x \geq 2
\end{cases}
\]
es continua en todo $\mathbb{R}$.

\section{Conclusiones}

El estudio de límites es esencial para:

\begin{itemize}[leftmargin=*]
    \item Comprender el comportamiento de funciones cerca de puntos específicos
    \item Analizar asíntotas y comportamiento en el infinito
    \item Determinar la continuidad de funciones
    \item Preparar el terreno para el cálculo de derivadas e integrales
    \item Modelar fenómenos dinámicos en ciencias e ingeniería
\end{itemize}

El dominio de las técnicas de cálculo de límites proporciona las herramientas necesarias para avanzar en temas más complejos del cálculo diferencial e integral, incluyendo derivadas, aplicaciones de la derivada, integración y series infinitas.

\section{Referencias y Lecturas Recomendadas}

\begin{enumerate}[leftmargin=*]
    \item Stewart, J. (2016). \textit{Cálculo de una Variable: Trascendentes Tempranas}. Cengage Learning.

    \item Larson, R. \& Edwards, B. (2018). \textit{Cálculo I}. Cengage Learning.

    \item Thomas, G. B. (2015). \textit{Cálculo: Una Variable}. Pearson Educación.

    \item Leithold, L. (1998). \textit{El Cálculo}. Oxford University Press.

    \item Apostol, T. M. (1967). \textit{Calculus, Volume I}. John Wiley \& Sons.
\end{enumerate}

\vfill

\begin{center}
\line(1,0){400}

\textbf{Fin de la Guía}

Prof. Toribio de J Arrieta F\\
La Pruebita\\
Noviembre 2025
\end{center}

\end{document}
