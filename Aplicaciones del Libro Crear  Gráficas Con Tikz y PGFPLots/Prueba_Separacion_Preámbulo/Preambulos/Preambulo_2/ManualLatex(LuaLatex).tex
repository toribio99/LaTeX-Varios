% ========================================================
% PREÁMBULO ORDENADO (LuaLaTeX)
% ========================================================

% 0) LA CLASE VA EN main.tex (NO AQUÍ)
% \documentclass[...]{book}

% --------------------------------------------------------
% 1) Idioma y tipografías (LuaLaTeX)
%    - fontspec: selecciona fuentes del sistema (Unicode; reemplaza inputenc/fontenc)
%    - babel[spanish,es-tabla]: localización en español para LuaLaTeX
%    - unicode-math: tipografías matemáticas Unicode
%      * Elegimos "TeX Gyre Pagella" (Palatino-like) para texto
%      * y "TeX Gyre Pagella Math" para matemáticas (pareja coherente)
% --------------------------------------------------------
\usepackage{fontspec}
\usepackage[spanish,es-tabla]{babel}

% Fuente de texto (Palatino-like). Cambia si prefieres otra.
\setmainfont{TeX Gyre Pagella}[
Ligatures=TeX,
Numbers=OldStyle,
Contextuals=Alternate
]
\setsansfont{TeX Gyre Heros}    % opcional: sans (Helvetica-like)
\setmonofont{Inconsolata}       % opcional: mono

% Matemáticas Unicode (evita cargar amssymb/newtx/mathpazo)
\usepackage{unicode-math}
\setmathfont{TeX Gyre Pagella Math}
% \setmathfont{Latin Modern Math} % ← alternativa, por si no tienes TGP Math

% --------------------------------------------------------
% 2) Microtipografía
%    - microtype funciona con LuaLaTeX (protrusión/expansión)
% --------------------------------------------------------
\usepackage{microtype}

% --------------------------------------------------------
% 3) Página y márgenes
% --------------------------------------------------------
\usepackage[margin=2.54cm]{geometry} % 1 inch = 2.54 cm

% --------------------------------------------------------
% 4) Gráficos y color
%    - graphicx: inclusión de imágenes (PDF/PNG/JPG)
%    - \graphicspath: ruta por defecto de imágenes
%    - xcolor[table,xcdraw]: colores + ayudas en tablas
% --------------------------------------------------------
\usepackage{graphicx}
\graphicspath{{figuras/}}
\usepackage[table,xcdraw]{xcolor}

% --------------------------------------------------------
% 5) Bloques de código (verbatim mejorado)
%    - fvextra extiende fancyvrb (lo carga internamente)
%    - Define entornos preformateados listos para usar
% --------------------------------------------------------
\IfFileExists{fvextra.sty}{%
	\usepackage{fvextra}
	\DefineVerbatimEnvironment{codeblock}{Verbatim}{
		breaklines, breakanywhere,
		breaksymbol=\tiny\ensuremath{\hookrightarrow}\,,
		breaksymbolindentleft=0pt,
		fontsize=\small,
		frame=single, framerule=0.4pt, rulecolor=\color{black!40},
		tabsize=2
	}
	\DefineVerbatimEnvironment{codeplain}{Verbatim}{
		breaklines, breakanywhere, fontsize=\small, frame=none, tabsize=2
	}
	\DefineVerbatimEnvironment{codefine}{Verbatim}{
		breaklines, breakanywhere,
		fontsize=\small, frame=single, framerule=0.2pt, rulecolor=\color{black!20},
		tabsize=2
	}
	\DefineVerbatimEnvironment{codebg}{Verbatim}{
		breaklines, breakanywhere,
		fontsize=\small, frame=single, framerule=0.4pt, rulecolor=\color{black!30},
		bgcolor=black!3, tabsize=2
	}
	\DefineVerbatimEnvironment{codelines}{Verbatim}{
		breaklines, breakanywhere,
		fontsize=\small, numbers=left, numberstyle=\tiny, numbersep=6pt,
		frame=single, framerule=0.4pt, rulecolor=\color{black!40}, tabsize=2
	}
	\DefineVerbatimEnvironment{codenosymbol}{Verbatim}{
		breaklines, breakanywhere, breaksymbol={}, fontsize=\small,
		frame=single, framerule=0.4pt, rulecolor=\color{black!40}, tabsize=2
	}
	\DefineVerbatimEnvironment{codegobbletwo}{Verbatim}{
		breaklines, breakanywhere, gobble=2,
		fontsize=\small, frame=single, framerule=0.4pt, rulecolor=\color{black!40},
		tabsize=2
	}
}{}

% --------------------------------------------------------
% 6) Tablas
%    - booktabs: reglas tipográficas de calidad
%    - longtable: tablas que atraviesan páginas
%    - multirow: celdas que abarcan varias filas
%      (para varias columnas se usa \multicolumn, macro de tabular)
% --------------------------------------------------------
\usepackage{booktabs}
\usepackage{longtable}
\usepackage{multirow}

% --------------------------------------------------------
% 7) Flotantes y subtítulos
%    - float: opción [H] (fijar flotantes) — úsala con moderación
%    - caption: formato de leyendas; aquí etiqueta en negrita
% --------------------------------------------------------
\usepackage{float}
\usepackage{caption}
\captionsetup{labelfont=bf}

% --------------------------------------------------------
% 8) Matemáticas (complementos a unicode-math)
%    - mathtools: extensiones de amsmath compatibles con unicode-math
%    - NO cargar amssymb (ya cubierto por unicode-math)
%    - Operadores en español + numeración por capítulo
% --------------------------------------------------------
\usepackage{mathtools} % compatible con unicode-math
\DeclareMathOperator{\sen}{sen}
\DeclareMathOperator*{\argmin}{arg\,min}
% \DeclareMathOperator*{\argmax}{arg\,max}

\numberwithin{figure}{chapter}
\numberwithin{table}{chapter}
% \numberwithin{equation}{chapter} % ← activa si quieres ecuaciones por capítulo

% --------------------------------------------------------
% 9) Paginación y maquetación multi-columna
% --------------------------------------------------------
\usepackage{lastpage}   % "Página X de Y"
\usepackage{multicol}   % columnas múltiples en el texto

% --------------------------------------------------------
% 10) Listas y espaciado de párrafos
% --------------------------------------------------------
\usepackage{enumitem}
\setlist{nosep}         % listas compactas
\setlength{\parindent}{1em}
\setlength{\parskip}{6pt}

% --------------------------------------------------------
% 11) Encabezados y pies (fancyhdr) — tras geometry/graphicx
%     - Ajusta \headheight si insertas logos para evitar warnings
%     - Se redefine también el estilo 'plain' (inicio de capítulo)
% --------------------------------------------------------
\usepackage{fancyhdr}
\pagestyle{fancy}
\setlength{\headheight}{18pt}
\fancyhf{} % limpia

% Cabecera con logos (ajusta rutas de archivos)
\fancyhead[LE]{\raisebox{-0.2\height}{%
		\includegraphics[height=\dimexpr\headheight-2pt\relax,keepaspectratio]{Latex3}}}
\fancyhead[RO]{\raisebox{-0.2\height}{%
		\includegraphics[height=\dimexpr\headheight-2pt\relax,keepaspectratio]{Latex1}}}

% Pie con "Página X de Y" y datos laterales
\fancyfoot[C]{\thepage\ de \pageref{LastPage}}
\fancyfoot[LE]{Toribio de J Arrieta F}
\fancyfoot[RO]{Universidad de La Vida}

\renewcommand{\headrulewidth}{1.4pt}
\renewcommand{\footrulewidth}{0.1pt}

% 'plain' (páginas de inicio de capítulo) con el mismo diseño
\fancypagestyle{plain}{%
	\fancyhf{}
	\fancyhead[LE]{\raisebox{-0.2\height}{%
			\includegraphics[height=\dimexpr\headheight-2pt\relax,keepaspectratio]{Latex3}}}
	\fancyhead[RO]{\raisebox{-0.2\height}{%
			\includegraphics[height=\dimexpr\headheight-2pt\relax,keepaspectratio]{Latex1}}}
	\fancyfoot[C]{\thepage\ de \pageref{LastPage}}
	\fancyfoot[LE]{Toribio de J Arrieta F}
	\fancyfoot[RO]{Universidad de La Vida}
	\renewcommand{\headrulewidth}{0.1pt}
	\renewcommand{\footrulewidth}{0.1pt}
}

% --------------------------------------------------------
% 12) Hipervínculos y marcadores — al final del preámbulo
%     - hyperref primero; bookmark después (optimiza marcadores PDF)
%     - \pdfstringdefDisableCommands: adapta macros para metadatos/TOC PDF
% --------------------------------------------------------
\usepackage[hidelinks]{hyperref}
\usepackage{bookmark}

\pdfstringdefDisableCommands{%
	\def\texttt#1{#1}%
	\def\backmatter{backmatter}%
	\def\frontmatter{frontmatter}%
	\def\mainmatter{mainmatter}%
	\def\include#1{#1}%
	\def\input#1{#1}%
}

% --------------------------------------------------------
% 13) cleveref (opcional) con nombres en español
%     - Debe ir DESPUÉS de hyperref para 'nameinlink'
% --------------------------------------------------------
\newif\ifhascleveref
\IfFileExists{cleveref.sty}{\hasclevereftrue}{\hasclevereffalse}

\ifhascleveref
\usepackage[capitalise,noabbrev,nameinlink]{cleveref}
\crefname{chapter}{capítulo}{capítulos}
\Crefname{chapter}{Capítulo}{Capítulos}
\crefname{section}{sección}{secciones}
\Crefname{section}{Sección}{Secciones}
\crefname{figure}{figura}{figuras}
\Crefname{figure}{Figura}{Figuras}
\crefname{table}{tabla}{tablas}
\Crefname{table}{Tabla}{Tablas}
\crefname{equation}{ecuación}{ecuaciones}
\Crefname{equation}{Ecuación}{Ecuaciones}
\else
% Fallback con autoref en español
\renewcommand{\chapterautorefname}{Capítulo}
\renewcommand{\sectionautorefname}{Sección}
\renewcommand{\subsectionautorefname}{Subsección}
\renewcommand{\subsubsectionautorefname}{Subsubsección}
\renewcommand{\figureautorefname}{Figura}
\renewcommand{\tableautorefname}{Tabla}
\renewcommand{\equationautorefname}{Ecuación}
\providecommand{\cref}[1]{\autoref{#1}}
\providecommand{\Cref}[1]{\Autoref{#1}}
\providecommand{\crefrange}[2]{\autoref{#1}--\autoref{#2}}
\fi

% --------------------------------------------------------
% 14) Índice alfabético (imakeidx preferente; si no, makeidx)
% --------------------------------------------------------
\newif\ifhasimakeidx
\IfFileExists{imakeidx.sty}{\hasimakeidxtrue}{\hasimakeidxfalse}

\ifhasimakeidx
\usepackage{imakeidx}
\makeindex[intoc,name=general,title=Índice alfabético]
\let\printdefaultindex\printindex
\renewcommand{\printindex}{\printdefaultindex[general]}
\else
\usepackage{makeidx}
\makeindex
\newcommand{\manualindexname}{Índice alfabético}
\AtBeginDocument{\renewcommand{\indexname}{\manualindexname}}
\let\printdefaultindex\printindex
\renewcommand{\printindex}{%
	\cleardoublepage
	\phantomsection
	\addcontentsline{toc}{chapter}{\manualindexname}%
	\printdefaultindex
}
\fi

% --------------------------------------------------------
% 15) Utilidades y metadatos
% --------------------------------------------------------
\newcommand{\HRule}[1]{\rule{\linewidth}{#1}} % línea horizontal para portada
\setcounter{tocdepth}{2}
\setcounter{secnumdepth}{3}

\title{Manual de \LaTeX}
\author{Toribio de J Arrieta}
\date{\today}
