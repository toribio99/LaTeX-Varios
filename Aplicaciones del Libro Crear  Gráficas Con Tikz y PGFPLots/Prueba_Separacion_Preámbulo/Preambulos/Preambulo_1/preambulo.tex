\usepackage[T1]{fontenc}
\usepackage[left=1cm, right=1cm, top=2cm, bottom=2cm]{geometry}
\usepackage{graphicx}
\usepackage{mathtools}
\usepackage{amssymb}
\usepackage{amsthm}
\usepackage{thmtools}
\usepackage{xcolor}
\usepackage{nameref}
\usepackage[spanish,es-tabla,es-noquoting]{babel}
\usepackage{hyperref}
\usepackage{lipsum}
\usepackage{comment}
\usepackage{verbatim}
% =====================================================
% PAQUETES  LIBRERIAS PARA GRÁFICAS
% =====================================================
\usepackage{graphicx}
\usepackage[dvipsnames]{xcolor} % u otra opción
\usepackage{float}
\usepackage{subcaption}
\usepackage{tikz}
\usetikzlibrary{3d}
\usetikzlibrary{babel, patterns, patterns.meta, arrows.meta, calc, positioning, shapes.geometric, shapes.misc, shapes.symbols, fadings, shadings, shadows, backgrounds}
\usetikzlibrary{backgrounds}
\tikzset{help lines/.style={very thin, draw=gray!30}}

\usepackage{pgfplots}
\usepackage{pgfmath} % LATEX
\pgfplotsset{compat=1.18}
\usepgfplotslibrary{statistics,groupplots,fillbetween,dateplot}
%\input pgfmath.tex % plain TEX
%\usemodule[pgfmath] % ConTEXt
\usepackage{dirtree}

% =====================================================
% COLORES PERSONALIZADOS (sin cambios)
% =====================================================
\definecolor{azulTitulo}{RGB}{0,51,102}
\definecolor{verdeEjemplo}{RGB}{0,128,0}
\definecolor{rojoError}{RGB}{178,34,34}
\definecolor{fondoCodigo}{RGB}{248,248,248}
\definecolor{comentario}{RGB}{0,128,0}
\definecolor{keyword}{RGB}{0,0,255}
\definecolor{string}{RGB}{163,21,21}

% =====================================================
% PLANO CARTESIANO #1 PARA GRÁFICAS
% =====================================================

\newcommand{\planoCartesiano}[4]{%
	\begin{tikzpicture}
		\begin{scope}[on background layer]
			\fill[#1] (-#3,-#4) rectangle (#3,#4);
		\end{scope}
		
		\begin{scope}
			\draw[step=.25, draw = #2!25, line width =.3pt] (-#3,-#4) grid (#3,#4);  
			\draw[<->, >=Stealth, ultra thick, #2] (0,-#4) -- (0,#4) ; 
			\draw[<->, >=Stealth, ultra thick, #2] (-#3,0) -- (#3,0) ;
			
			% Marcas y números para eje X
			\foreach \x in {1,2,...,#3}
			{
				\draw[black, thick] (\x,-0.1) -- (\x,0.1);
				\node[black, below] at (\x,-0.3) {\x};
			}
			
			\foreach \x in {-1,-2,...,-#3}
			{
				\draw[black, thick] (\x,-0.1) -- (\x,0.1);
				\node[black, below] at (\x,-0.3) {\x};
			}
			
			% Marcas y números para eje Y
			\foreach \y in {1,2,...,#4}
			{
				\draw[black, thick] (-0.1,\y) -- (0.1,\y);
				\node[black, left] at (-0.3,\y) {\y};
			}
			
			\foreach \y in {-1,-2,...,-#4}
			{
				\draw[black, thick] (-0.1,\y) -- (0.1,\y);
				\node[black, left] at (-0.3,\y) {\y};
			}
			
			% Grid mayor (cada 1 unidad) - líneas más gruesas
			\draw[step=1, draw = #2!50, line width =.5pt] (-#3,-#4) grid (#3,#4);
			
		\end{scope}
	\end{tikzpicture}
}

% =====================================================
% TÍTULO AUTOR
% =====================================================
\title{Aplicación Crear Gráficas Con Tikz-PGFPlots}
\author{Toribio}


% =====================================================
% PARA LA VERIFICACIÓN RÁPIDA DEL CÓDIGO
% =====================================================
\usepackage{syntonly}
%\syntaxonly  % Activa el modo solo-sintaxis