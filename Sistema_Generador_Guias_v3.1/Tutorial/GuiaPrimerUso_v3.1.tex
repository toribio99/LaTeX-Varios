\documentclass[12pt,a4paper]{article}
\usepackage{fontspec}
\usepackage[spanish,es-nodecimaldot]{babel}
\usepackage{amsmath,amssymb}
\usepackage[margin=2.5cm]{geometry}
\usepackage{xcolor}
\usepackage{enumitem}
\usepackage{multicol}
\usepackage{fancyhdr}
\usepackage{tcolorbox}
\usepackage{hyperref}
\usepackage{listings}

% Configuración de colores
\definecolor{azuloscuro}{RGB}{0,51,102}
\definecolor{azulclaro}{RGB}{51,153,255}
\definecolor{verde}{RGB}{0,128,0}
\definecolor{rojo}{RGB}{220,20,60}
\definecolor{gris}{RGB}{100,100,100}
\definecolor{fondogris}{RGB}{245,245,245}

% Configuración de hyperref
\hypersetup{
    colorlinks=true,
    linkcolor=azuloscuro,
    urlcolor=azulclaro,
    pdftitle={Guía de Primer Uso - Sistema Generador de Guías v3.1},
    pdfauthor={Sistema Generador de Guías}
}

% Configuración de tcolorbox
\tcbuselibrary{skins,breakable}

% Cajas personalizadas
\newtcolorbox{notaimportante}[1][]{
    colback=rojo!5!white,
    colframe=rojo!75!black,
    fonttitle=\bfseries,
    title=⚠️ IMPORTANTE,
    #1
}

\newtcolorbox{consejo}[1][]{
    colback=verde!5!white,
    colframe=verde!75!black,
    fonttitle=\bfseries,
    title=💡 CONSEJO,
    #1
}

\newtcolorbox{ejemplo}[1][]{
    colback=azulclaro!5!white,
    colframe=azulclaro!75!black,
    fonttitle=\bfseries,
    title=📝 EJEMPLO,
    #1
}

\newtcolorbox{comando}[1][]{
    colback=fondogris,
    colframe=gris,
    fonttitle=\bfseries\ttfamily,
    title=Terminal/Comando,
    #1
}

% Configuración de listings para código
\lstset{
    basicstyle=\ttfamily\small,
    backgroundcolor=\color{fondogris},
    frame=single,
    breaklines=true,
    columns=fullflexible
}

% Encabezados y pies de página
\pagestyle{fancy}
\fancyhf{}
\fancyhead[L]{\small Sistema Generador de Guías v3.1}
\fancyhead[R]{\small Guía de Primer Uso}
\fancyfoot[C]{\thepage}
\renewcommand{\headrulewidth}{0.4pt}
\renewcommand{\footrulewidth}{0.4pt}

\title{
    \vspace{-2cm}
    \Huge\textbf{\color{azuloscuro}Sistema Generador de Guías}\\
    \Large\textbf{\color{azulclaro}Versión 3.1}\\
    \vspace{1cm}
    \large Guía Completa de Primer Uso\\
    \normalsize Para Usuarios Nuevos
}
\author{Tutorial Paso a Paso con Ejemplos Detallados}
\date{\today}

\begin{document}

\maketitle
\thispagestyle{empty}
\vspace{2cm}

\begin{center}
\large\textit{«Crear guías profesionales de matemáticas\\nunca fue tan fácil»}
\end{center}

\newpage

\tableofcontents

\newpage

\section{Bienvenida}

¡Bienvenido al Sistema Generador de Guías v3.1! Este tutorial está diseñado especialmente para usuarios que utilizan el sistema por primera vez.

\subsection{¿Qué es este sistema?}

Es un framework interactivo que te permite crear guías educativas profesionales de matemáticas en formato LaTeX, con:

\begin{itemize}[leftmargin=2cm]
    \item ✅ Contenido pedagógico estructurado
    \item ✅ Gráficas matemáticas de alta calidad (TikZ)
    \item ✅ Ejemplos resueltos paso a paso
    \item ✅ Ejercicios propuestos con soluciones completas
    \item ✅ Diseño profesional y consistente
    \item ✅ Proceso guiado e interactivo
\end{itemize}

\subsection{¿Para quién es este sistema?}

\begin{itemize}[leftmargin=2cm]
    \item \textbf{Profesores} que crean material educativo
    \item \textbf{Estudiantes de licenciatura} preparando recursos didácticos
    \item \textbf{Tutores particulares} desarrollando guías personalizadas
    \item \textbf{Cualquier persona} que necesite documentos matemáticos profesionales
\end{itemize}

\begin{notaimportante}
\textbf{No necesitas ser experto en LaTeX.} El sistema te guía paso a paso y genera automáticamente el código LaTeX profesional.
\end{notaimportante}

\subsection{¿Qué temas puedo crear?}

El sistema soporta múltiples áreas de matemáticas:

\begin{multicols}{2}
\textbf{Geometría Analítica:}
\begin{itemize}
    \item Parábola
    \item Elipse
    \item Hipérbola
    \item Circunferencia
    \item Recta
\end{itemize}

\textbf{Álgebra:}
\begin{itemize}
    \item Funciones cuadráticas
    \item Funciones exponenciales
    \item Funciones logarítmicas
    \item Sistemas de ecuaciones
\end{itemize}

\textbf{Cálculo:}
\begin{itemize}
    \item Límites
    \item Derivadas
    \item Integrales
    \item Aplicaciones
\end{itemize}

\textbf{Trigonometría:}
\begin{itemize}
    \item Razones trigonométricas
    \item Identidades
    \item Ley de senos y cosenos
\end{itemize}
\end{multicols}

\newpage

\section{Requisitos Previos}

Antes de comenzar, asegúrate de tener instalado lo siguiente:

\subsection{Software Necesario}

\subsubsection{1. LaTeX (TeX Live)}

\textbf{¿Lo tengo instalado?} Abre una terminal y ejecuta:

\begin{comando}
lualatex --version
\end{comando}

\textbf{Si ves una versión:} ✅ Ya lo tienes instalado

\textbf{Si ves un error:} ❌ Debes instalarlo

\textbf{Cómo instalarlo:}
\begin{itemize}
    \item \textbf{macOS:} Descarga MacTeX desde \texttt{https://www.tug.org/mactex/}
    \item \textbf{Windows:} Descarga MiKTeX desde \texttt{https://miktex.org/}
    \item \textbf{Linux:} \texttt{sudo apt-get install texlive-full}
\end{itemize}

\subsubsection{2. Git}

\textbf{¿Lo tengo instalado?}

\begin{comando}
git --version
\end{comando}

\textbf{Si ves una versión:} ✅ Ya lo tienes instalado

\textbf{Si ves un error:} ❌ Debes instalarlo desde \texttt{https://git-scm.com/}

\subsubsection{3. Asistente de IA (Claude, ChatGPT, etc.)}

Necesitas acceso a un asistente de IA que pueda:
\begin{itemize}
    \item Leer archivos locales
    \item Crear y editar archivos
    \item Ejecutar comandos en terminal
    \item Generar código LaTeX
\end{itemize}

\begin{consejo}
Este tutorial está optimizado para Claude Code, pero funciona con cualquier asistente compatible.
\end{consejo}

\subsection{Conocimientos Necesarios}

\textbf{Nivel mínimo:}
\begin{itemize}
    \item Saber abrir una terminal
    \item Entender rutas de archivos básicas
    \item Conocer conceptos matemáticos del tema a desarrollar
\end{itemize}

\textbf{NO necesitas:}
\begin{itemize}
    \item Ser experto en LaTeX
    \item Saber programar
    \item Conocer TikZ o gráficas matemáticas
\end{itemize}

\newpage

\section{Ubicación del Sistema}

El Sistema Generador de Guías está en:

\begin{comando}
/Users/toribioarrieta/Documents/LaTeX-GitHub/LaTeX-Varios/Sistema\_Generador\_Guias/
\end{comando}

\subsection{Estructura de archivos}

\begin{lstlisting}
Sistema_Generador_Guias/
├── README.md                   (Documentación general)
├── PROMPT_v3.0.md              (Prompt principal)
├── CHANGELOG.md                (Historial de versiones)
├── Tutorial/
│   └── GuiaPrimerUso.tex      (Este documento)
└── Referencias/
    └── GuiaAplicacionesDerivada.tex  (Ejemplo)
\end{lstlisting}

\begin{notaimportante}
\textbf{Guarda esta ruta.} La necesitarás para iniciar el sistema con el asistente de IA.
\end{notaimportante}

\newpage

\section{Tu Primer Uso: Paso a Paso}

Vamos a crear tu primera guía. Usaremos como ejemplo una guía sobre \textbf{La Elipse}.

\subsection{Paso 1: Iniciar el Asistente de IA}

Abre tu asistente de IA (Claude Code, ChatGPT, etc.) y escribe:

\begin{ejemplo}
\textbf{Mensaje al asistente:}

\begin{quote}
\textit{``Usa el Sistema Generador de Guías v3.0 ubicado en:}

\textit{/Users/toribioarrieta/Documents/LaTeX-GitHub/LaTeX-Varios/Sistema\_Generador\_Guias/}

\textit{Para crear una nueva guía educativa.''}
\end{quote}
\end{ejemplo}

\textbf{Alternativamente}, puedes ser más específico:

\begin{ejemplo}
\textbf{Mensaje alternativo:}

\begin{quote}
\textit{``Lee el archivo PROMPT\_v3.0.md del Sistema Generador de Guías y úsalo para crear una guía sobre Elipse para grado 11.''}
\end{quote}
\end{ejemplo}

\begin{consejo}
Si es tu primera vez, usa el primer mensaje (genérico). El asistente te guiará con preguntas.
\end{consejo}

\subsection{Paso 2: Responder las Preguntas del Asistente}

El asistente te hará 14 preguntas. Aquí está el ejemplo completo:

\subsubsection{Pregunta 1: Título de la guía}

\textbf{Asistente pregunta:} \textit{``¿Cuál es el TÍTULO de la guía?''}

\textbf{Tú respondes:} \textit{``La Elipse: Propiedades y Aplicaciones''}

\begin{consejo}
Usa títulos descriptivos pero no muy largos (máximo 8-10 palabras).
\end{consejo}

\subsubsection{Pregunta 2: Autor}

\textbf{Asistente pregunta:} \textit{``¿Quién es el AUTOR?''}

\textbf{Tú respondes:} \textit{``Prof. Juan Pérez''} (o tu nombre)

\subsubsection{Pregunta 3: Institución}

\textbf{Asistente pregunta:} \textit{``¿Cuál es la INSTITUCIÓN?''}

\textbf{Tú respondes:} \textit{``Colegio San José''} (o deja en blanco si no aplica)

\subsubsection{Pregunta 4: Fecha}

\textbf{Asistente pregunta:} \textit{``¿Fecha de creación?''}

\textbf{Tú respondes:} \textit{``Noviembre 2025''} (o deja que use la fecha actual)

\subsubsection{Pregunta 5: Tema principal}

\textbf{Asistente pregunta:} \textit{``¿Cuál es el TEMA principal?''}

\textbf{Tú respondes:} \textit{``Elipse''}

\begin{notaimportante}
Sé específico. No digas solo ``cónicas'', di ``Elipse''.
\end{notaimportante}

\subsubsection{Pregunta 6: Grado}

\textbf{Asistente pregunta:} \textit{``¿Para qué GRADO es la guía?''}

\textbf{Tú respondes:} \textit{``11''}

\begin{consejo}
El grado determina el tono:
\begin{itemize}
    \item Grados 9-10: Lenguaje más coloquial
    \item Grados 11+: Lenguaje más formal
\end{itemize}
\end{consejo}

\subsubsection{Pregunta 7: Asignatura}

\textbf{Asistente pregunta:} \textit{``¿Qué ASIGNATURA/ÁREA?''}

\textbf{Tú respondes:} \textit{``Geometría Analítica''}

Opciones comunes:
\begin{itemize}
    \item Geometría Analítica
    \item Álgebra
    \item Cálculo
    \item Trigonometría
    \item Aplicaciones (Medicina/Economía/Física)
\end{itemize}

\subsubsection{Pregunta 8: Elementos clave}

\textbf{Asistente pregunta:} \textit{``¿Qué ELEMENTOS CLAVE debe incluir el concepto?''}

\textbf{Tú respondes:}

\begin{quote}
\textit{``centro, focos, vértices, covertices, eje mayor, eje menor, distancia focal, excentricidad''}
\end{quote}

\begin{consejo}
Lista TODOS los elementos importantes del concepto, separados por comas.
\end{consejo}

\subsubsection{Pregunta 9: Aplicaciones de la vida real}

\textbf{Asistente pregunta:} \textit{``¿Qué APLICACIONES de la vida real mencionar? (mínimo 3)''}

\textbf{Tú respondes:}

\begin{quote}
\textit{``Órbitas planetarias, arquitectura de cúpulas, diseño de puentes colgantes, lentes ópticas''}
\end{quote}

\subsubsection{Pregunta 10: Número de ejemplos resueltos}

\textbf{Asistente pregunta:} \textit{``¿Cuántos EJEMPLOS RESUELTOS deseas?''}

\textbf{Tú respondes:} \textit{``4''}

\begin{consejo}
Rango recomendado: 3-5 ejemplos. Más de 5 hace la guía muy larga.
\end{consejo}

\subsubsection{Pregunta 11: Número de ejercicios propuestos}

\textbf{Asistente pregunta:} \textit{``¿Cuántos EJERCICIOS PROPUESTOS deseas?''}

\textbf{Tú respondes:} \textit{``5''}

\begin{consejo}
Rango recomendado: 5-8 ejercicios.
\end{consejo}

\subsubsection{Pregunta 12: Ejercicios inversos}

\textbf{Asistente pregunta:} \textit{``¿Necesitas dos tipos de ejercicios (directos e inversos)?''}

\textbf{Tú respondes:} \textit{``Sí''}

\textbf{Explicación:}
\begin{itemize}
    \item \textbf{Ejercicios directos:} Dada la ecuación → hallar elementos
    \item \textbf{Ejercicios inversos:} Dados los elementos → hallar ecuación
\end{itemize}

Para geometría analítica, responde ``Sí''. Para otros temas, ``No''.

\subsubsection{Pregunta 13: Carpeta destino}

\textbf{Asistente pregunta:} \textit{``¿En qué CARPETA se guardará la guía?''}

\textbf{Tú respondes:}

\begin{quote}
\textit{``Clases De Sheyra/Geometría analítica/Elipse''}
\end{quote}

\begin{notaimportante}
Si la carpeta no existe, el asistente la creará automáticamente.
\end{notaimportante}

\subsubsection{Pregunta 14: Nombre del archivo}

\textbf{Asistente pregunta:} \textit{``¿Nombre del archivo .tex?''}

\textbf{Tú respondes:} \textit{``GuiaElipse.tex''}

\begin{consejo}
Formato recomendado: \texttt{Guia[Tema].tex}

Ejemplos: \texttt{GuiaElipse.tex}, \texttt{GuiaDerivadas.tex}
\end{consejo}

\subsection{Paso 3: Confirmar el Resumen}

El asistente mostrará un resumen completo:

\begin{ejemplo}
\textbf{Ejemplo de resumen:}

\begin{quote}
\texttt{=== RESUMEN DE CONFIGURACIÓN ===}

\texttt{DOCUMENTO:}\\
\texttt{  Título: La Elipse: Propiedades y Aplicaciones}\\
\texttt{  Autor: Prof. Juan Pérez}\\
\texttt{  Institución: Colegio San José}\\
\texttt{  Fecha: Noviembre 2025}

\texttt{ACADÉMICO:}\\
\texttt{  Tema: Elipse}\\
\texttt{  Grado: 11}\\
\texttt{  Asignatura: Geometría Analítica}\\
\texttt{  Tono: Formal}

\texttt{CONTENIDO:}\\
\texttt{  Elementos clave: centro, focos, vértices...}\\
\texttt{  Aplicaciones: Órbitas planetarias...}\\
\texttt{  Ejemplos resueltos: 4}\\
\texttt{  Ejercicios propuestos: 5}\\
\texttt{  Ejercicios inversos: Sí}

\texttt{UBICACIÓN:}\\
\texttt{  Carpeta: Clases De Sheyra/Geometría analítica/Elipse}\\
\texttt{  Archivo: GuiaElipse.tex}

\texttt{¿Es correcta esta información? (Sí/No)}
\end{quote}
\end{ejemplo}

\textbf{Revisa cuidadosamente}. Si todo está bien, responde \textit{``Sí''}.

Si necesitas cambiar algo, responde \textit{``No''} e indica qué modificar.

\subsection{Paso 4: Esperar la Generación}

El asistente ahora:

\begin{enumerate}
    \item Leerá el archivo de referencia
    \item Generará el documento LaTeX completo
    \item Creará todas las gráficas TikZ
    \item Compilará el documento con \texttt{lualatex}
    \item Verificará la calidad
\end{enumerate}

\begin{notaimportante}
Este proceso puede tomar 2-5 minutos. \textbf{No interrumpas al asistente}.
\end{notaimportante}

\subsection{Paso 5: Revisar el Resultado}

El asistente te mostrará:

\begin{itemize}
    \item ✅ Resultado de compilación (exitoso o errores)
    \item ✅ Ruta del archivo \texttt{.tex} creado
    \item ✅ Ruta del archivo \texttt{.pdf} generado
    \item ✅ Número de páginas del documento
\end{itemize}

\begin{consejo}
Abre el PDF generado y revísalo visualmente. Verifica:
\begin{itemize}
    \item Las gráficas se ven bien
    \item No hay texto cortado
    \item Los ejemplos están completos
    \item Las soluciones tienen sentido
\end{itemize}
\end{consejo}

\subsection{Paso 6: Versionamiento en Git}

El asistente preguntará: \textit{``¿Deseas guardar en git?''}

\textbf{Responde:} \textit{``Sí''}

El asistente:
\begin{enumerate}
    \item Ejecutará \texttt{git add [archivo].tex}
    \item Creará un commit descriptivo
    \item Hará push al repositorio remoto
\end{enumerate}

\begin{notaimportante}
Solo se versionará el archivo \texttt{.tex}. Los archivos auxiliares (\texttt{.aux}, \texttt{.log}, etc.) se ignoran automáticamente.
\end{notaimportante}

\newpage

\section{Ejemplo Completo: Diálogo Real}

Aquí tienes un ejemplo completo de cómo se ve una sesión real:

\begin{ejemplo}[breakable]
\small
\textbf{Tú:} Usa el Sistema v3.0 para crear una guía sobre Funciones Cuadráticas

\textbf{Asistente:} ¡Perfecto! Voy a usar el Sistema Generador de Guías v3.0. Necesito algunos datos. ¿Cuál es el título de la guía?

\textbf{Tú:} Funciones Cuadráticas: De la Parábola a las Aplicaciones

\textbf{Asistente:} Excelente. ¿Quién es el autor?

\textbf{Tú:} Prof. María González

\textbf{Asistente:} ¿Institución? (puedes dejarlo en blanco)

\textbf{Tú:} Instituto Pedagógico Nacional

\textbf{Asistente:} ¿Fecha de creación? (puedo usar la fecha actual)

\textbf{Tú:} Usa la fecha actual

\textbf{Asistente:} ¿Cuál es el TEMA principal?

\textbf{Tú:} Funciones Cuadráticas

\textbf{Asistente:} ¿Para qué GRADO es la guía?

\textbf{Tú:} 9

\textbf{Asistente:} ¿Qué ASIGNATURA/ÁREA?

\textbf{Tú:} Álgebra

\textbf{Asistente:} ¿Qué ELEMENTOS CLAVE debe incluir el concepto?

\textbf{Tú:} vértice, eje de simetría, concavidad, raíces, discriminante, forma estándar, forma canónica

\textbf{Asistente:} ¿Qué APLICACIONES de la vida real mencionar? (mínimo 3)

\textbf{Tú:} Trayectoria de proyectiles, diseño de puentes, optimización de ganancias en negocios, trayectoria de pelotas en deportes

\textbf{Asistente:} ¿Cuántos EJEMPLOS RESUELTOS deseas? (3-5 recomendado)

\textbf{Tú:} 5

\textbf{Asistente:} ¿Cuántos EJERCICIOS PROPUESTOS deseas? (5-8 recomendado)

\textbf{Tú:} 6

\textbf{Asistente:} ¿Necesitas dos tipos de ejercicios (directos e inversos)?

\textbf{Tú:} No, solo ejercicios de práctica mixtos

\textbf{Asistente:} ¿En qué CARPETA se guardará la guía?

\textbf{Tú:} Grado 9/Algebra/Funciones\_Cuadraticas

\textbf{Asistente:} ¿Nombre del archivo .tex?

\textbf{Tú:} GuiaFuncionesCuadraticas.tex

\textbf{Asistente:} [Muestra resumen completo...]

¿Es correcta esta información? (Sí/No)

\textbf{Tú:} Sí

\textbf{Asistente:} [Genera la guía, compila, muestra resultados...]

La guía se generó exitosamente. El PDF tiene 18 páginas. ¿Deseas guardar en git?

\textbf{Tú:} Sí

\textbf{Asistente:} [Ejecuta git add, commit, push...]

✅ Guía versionada exitosamente. ¡Listo!
\end{ejemplo}

\newpage

\section{Solución de Problemas Comunes}

\subsection{Problema 1: El asistente no encuentra el archivo}

\textbf{Síntoma:} El asistente dice ``No encuentro el archivo PROMPT\_v3.0.md''

\textbf{Solución:}
\begin{enumerate}
    \item Verifica que la ruta sea correcta
    \item Proporciona la ruta completa y absoluta
    \item Usa comillas si la ruta tiene espacios
\end{enumerate}

\begin{comando}
Ruta correcta:
/Users/toribioarrieta/Documents/LaTeX-GitHub/LaTeX-Varios/Sistema\_Generador\_Guias/PROMPT\_v3.0.md
\end{comando}

\subsection{Problema 2: Error de compilación LaTeX}

\textbf{Síntoma:} El asistente dice ``Error al compilar con lualatex''

\textbf{Solución:}
\begin{enumerate}
    \item Verifica que lualatex esté instalado: \texttt{lualatex --version}
    \item Revisa el mensaje de error específico
    \item El asistente intentará corregir automáticamente
\end{enumerate}

\begin{consejo}
Si el error persiste, pide al asistente: ``Compila con modo no interactivo''
\end{consejo}

\subsection{Problema 3: Gráficas fuera del margen}

\textbf{Síntoma:} Las gráficas TikZ se salen de la página

\textbf{Solución:}

El asistente puede ajustar el parámetro \texttt{scale} en las gráficas.

\textbf{Tú dices:} ``Reduce el tamaño de las gráficas TikZ al 80\%''

\subsection{Problema 4: El asistente se saltó preguntas}

\textbf{Síntoma:} El asistente no preguntó todos los datos

\textbf{Solución:}

\textbf{Tú dices:} ``Espera, necesito que me preguntes TODOS los datos uno por uno según el PROMPT\_v3.0''

\subsection{Problema 5: Git no está configurado}

\textbf{Síntoma:} Error al hacer \texttt{git commit} o \texttt{git push}

\textbf{Solución:}

Configura git con tus datos:

\begin{comando}
git config --global user.name "Tu Nombre"
git config --global user.email "tu@email.com"
\end{comando}

\newpage

\section{Preguntas Frecuentes (FAQ)}

\subsection{¿Puedo modificar la guía después de generarla?}

\textbf{Sí.} El archivo \texttt{.tex} es editable. Puedes:
\begin{itemize}
    \item Cambiar texto
    \item Agregar más ejemplos
    \item Ajustar gráficas
    \item Modificar colores
\end{itemize}

Después de editar, recompila con: \texttt{lualatex [archivo].tex}

\subsection{¿Puedo crear guías de otros temas no listados?}

\textbf{Sí.} El sistema es flexible. Puedes crear guías de cualquier tema matemático. Solo especifica bien los elementos clave y aplicaciones.

\subsection{¿Cuánto tiempo tarda en generar una guía?}

\textbf{Entre 2-5 minutos} dependiendo de:
\begin{itemize}
    \item Número de ejemplos
    \item Complejidad de las gráficas
    \item Velocidad del asistente de IA
\end{itemize}

\subsection{¿Puedo usar el sistema sin conexión a internet?}

\textbf{No.} Necesitas conexión porque el asistente de IA funciona en línea. Sin embargo, una vez generada la guía, puedes compilarla offline.

\subsection{¿Puedo compartir las guías generadas?}

\textbf{Sí.} Las guías son tuyas. Puedes:
\begin{itemize}
    \item Distribuirlas a tus estudiantes
    \item Publicarlas en tu sitio web
    \item Incluirlas en libros
    \item Compartirlas con colegas
\end{itemize}

\subsection{¿Qué hago si encuentro un error en la guía generada?}

\textbf{Opciones:}
\begin{enumerate}
    \item Edita manualmente el archivo \texttt{.tex}
    \item Pide al asistente: ``Corrige el error en [ubicación específica]''
    \item Regenera esa sección específica
\end{enumerate}

\subsection{¿Puedo cambiar los colores de las gráficas?}

\textbf{Sí.} El archivo usa colores definidos. Para cambiarlos, edita las líneas:

\begin{lstlisting}
\definecolor{medicina}{RGB}{220,20,60}
\definecolor{economia}{RGB}{0,128,0}
...
\end{lstlisting}

\subsection{¿Funciona en Windows?}

\textbf{Sí.} El sistema funciona en:
\begin{itemize}
    \item macOS ✅
    \item Windows ✅
    \item Linux ✅
\end{itemize}

Solo asegúrate de tener LaTeX y git instalados.

\newpage

\section{Recursos Adicionales}

\subsection{Archivos de Referencia}

Dentro del sistema hay un archivo ejemplo que puedes estudiar:

\begin{comando}
Sistema\_Generador\_Guias/Referencias/GuiaAplicacionesDerivada.tex
\end{comando}

Ábrelo para ver:
\begin{itemize}
    \item Estructura completa de una guía profesional
    \item Uso avanzado de TikZ y pgfplots
    \item Estilo pedagógico con ejemplos detallados
    \item Formato de ejercicios y soluciones
\end{itemize}

\subsection{Documentación Completa}

Lee los siguientes archivos para más información:

\begin{itemize}
    \item \texttt{README.md} - Documentación general del sistema
    \item \texttt{PROMPT\_v3.0.md} - Especificaciones técnicas completas
    \item \texttt{CHANGELOG.md} - Historial de versiones y cambios
\end{itemize}

\subsection{Comandos Útiles de LaTeX}

\subsubsection{Compilar el documento}

\begin{comando}
cd [directorio]
lualatex [archivo].tex
\end{comando}

\subsubsection{Compilar dos veces (para índices)}

\begin{comando}
lualatex [archivo].tex
lualatex [archivo].tex
\end{comando}

\subsubsection{Ver el PDF generado}

\begin{comando}
open [archivo].pdf     (macOS)
start [archivo].pdf    (Windows)
xdg-open [archivo].pdf (Linux)
\end{comando}

\subsection{Comandos Útiles de Git}

\subsubsection{Ver estado}

\begin{comando}
git status
\end{comando}

\subsubsection{Agregar archivo específico}

\begin{comando}
git add "[ruta]/[archivo].tex"
\end{comando}

\subsubsection{Crear commit}

\begin{comando}
git commit -m "Descripción del cambio"
\end{comando}

\subsubsection{Subir al remoto}

\begin{comando}
git push origin main
\end{comando}

\newpage

\section{Consejos para Guías de Calidad}

\subsection{Al definir elementos clave}

\begin{itemize}
    \item ✅ Sé exhaustivo: lista TODOS los elementos importantes
    \item ✅ Usa términos técnicos correctos
    \item ✅ Separa con comas claramente
\end{itemize}

\textbf{Ejemplo bueno:}

``centro, focos, vértices, covertices, eje mayor, eje menor, distancia focal, excentricidad, lado recto''

\textbf{Ejemplo malo:}

``las partes de la elipse''

\subsection{Al elegir aplicaciones}

\begin{itemize}
    \item ✅ Sé específico y concreto
    \item ✅ Menciona contextos variados
    \item ✅ Relaciona con la vida cotidiana
\end{itemize}

\textbf{Ejemplo bueno:}

``Órbitas de planetas alrededor del Sol, diseño de cúpulas arquitectónicas, lentes para telescopios, trayectoria de cometas''

\textbf{Ejemplo malo:}

``en astronomía, en ingeniería''

\subsection{Al elegir cantidad de ejemplos}

\begin{notaimportante}
\textbf{Recomendaciones por grado:}

\begin{itemize}
    \item Grado 9: 3-4 ejemplos (conceptos más simples)
    \item Grado 10-11: 4-5 ejemplos (mayor complejidad)
\end{itemize}
\end{notaimportante}

\subsection{Al elegir cantidad de ejercicios}

\begin{consejo}
\textbf{Fórmula útil:}

Ejercicios propuestos = Ejemplos resueltos + 1 o +2

Ejemplo: Si tienes 4 ejemplos, incluye 5-6 ejercicios.
\end{consejo}

\newpage

\section{Próximos Pasos}

\subsection{¡Ya estás listo!}

Ahora que has leído esta guía, estás preparado para:

\begin{enumerate}
    \item Crear tu primera guía educativa
    \item Usar el sistema con confianza
    \item Resolver problemas básicos
    \item Aprovechar todas las características
\end{enumerate}

\subsection{Práctica Recomendada}

\textbf{Para tu primera guía:}

\begin{enumerate}
    \item Elige un tema que conozcas bien
    \item Empieza con 3 ejemplos y 5 ejercicios (simple)
    \item Revisa cuidadosamente el PDF generado
    \item Toma notas de lo que quieras mejorar
    \item Crea una segunda guía aplicando lo aprendido
\end{enumerate}

\subsection{Mejora Continua}

Después de crear 2-3 guías:

\begin{itemize}
    \item Revisa el archivo de referencia para ideas
    \item Experimenta con diferentes estructuras
    \item Personaliza colores y estilos
    \item Comparte con colegas y recibe feedback
\end{itemize}

\newpage

\section{Contacto y Soporte}

\subsection{Si necesitas ayuda}

\textbf{Opciones:}

\begin{enumerate}
    \item \textbf{Relee esta guía} - Muchas respuestas están aquí
    \item \textbf{Consulta README.md} - Documentación técnica adicional
    \item \textbf{Revisa CHANGELOG.md} - Cambios y mejoras recientes
    \item \textbf{Pregunta al asistente de IA} - Él conoce el sistema completo
\end{enumerate}

\subsection{Mejoras al Sistema}

Si tienes ideas para mejorar el sistema:

\begin{enumerate}
    \item Documenta tu sugerencia claramente
    \item Incluye ejemplos específicos
    \item Explica el beneficio de la mejora
    \item Comparte con el mantenedor del sistema
\end{enumerate}

\subsection{Información del Sistema}

\begin{itemize}
    \item \textbf{Versión actual:} 3.1
    \item \textbf{Fecha de lanzamiento:} 2025-11-05
    \item \textbf{Ubicación:} \texttt{Sistema\_Generador\_Guias\_v3.1/}
    \item \textbf{Repositorio:} LaTeX-Varios
    \item \textbf{Mantenedor:} Toribio Arrieta
    \item \textbf{Correcciones v3.1:} Gráficas con pgfplots (corregido error "Dimension too large")
\end{itemize}

\newpage

\section{Apéndice Técnico: Gráficas con pgfplots}

Esta sección contiene información técnica crítica para evitar errores comunes en la generación de gráficas.

\subsection{Problema: Error ``Dimension too large''}

\begin{notaimportante}
Si al compilar tu guía recibes el error \texttt{Dimension too large}, probablemente estés usando TikZ básico con \texttt{\textbackslash draw ... plot} en lugar de pgfplots.
\end{notaimportante}

\textbf{Causa del error:}

TikZ básico tiene limitaciones al manejar valores grandes en coordenadas. Por ejemplo, el siguiente código \textbf{fallará}:

\begin{lstlisting}[language=TeX, backgroundcolor=\color{fondogris}, basicstyle=\ttfamily\small, breaklines=true]
\draw[red,very thick,domain=0:180,samples=100]
    plot (\x/10,{(100*\x - 0.5*\x*\x)/500});
% Error: "Dimension too large"
\end{lstlisting}

\subsection{Solución: Usar pgfplots con axis}

La solución es usar el entorno \texttt{axis} de pgfplots, que maneja correctamente cualquier escala de valores.

\textbf{Código correcto:}

\begin{lstlisting}[language=TeX, backgroundcolor=\color{verde!10}, basicstyle=\ttfamily\small, breaklines=true]
\begin{tikzpicture}
    \begin{axis}[
        width=12cm, height=8cm,
        axis lines=middle,
        xlabel={$x$}, ylabel={$y$},
        xmin=-5, xmax=5,
        ymin=-5, ymax=5,
        grid=both,
        samples=100,
    ]

    % Curva principal
    \addplot[red, very thick, domain=-5:5] {x^2};

    % Puntos importantes
    \node[circle, fill=blue, inner sep=2pt] at (0,0) {};
    \node[blue, above right] at (0,0) {Vértice};

    \end{axis}
\end{tikzpicture}
\end{lstlisting}

\subsection{Plantilla Recomendada}

\begin{consejo}
Usa esta plantilla como base para todas tus gráficas:
\end{consejo}

\begin{lstlisting}[language=TeX, backgroundcolor=\color{azulclaro!10}, basicstyle=\ttfamily\small, breaklines=true]
\begin{tikzpicture}
    \begin{axis}[
        width=12cm, height=8cm,
        axis lines=middle,
        xlabel={$x$}, ylabel={$y$},
        xmin=-10, xmax=10,
        ymin=-10, ymax=10,
        xtick={-10,-8,...,10},
        ytick={-10,-8,...,10},
        grid=both,
        grid style={line width=.1pt, draw=gray!30},
        axis line style={-{Latex},thick},
        samples=100,
        legend pos=north west,
    ]

    % Gráfica principal
    \addplot[red, very thick, domain=-10:10]
        {funcion(x)};
    \addlegendentry{$f(x)$}

    % Puntos especiales
    \node[circle, fill=blue, inner sep=2pt]
        at (x,y) {};
    \node[blue, above] at (x,y) {Etiqueta};

    % Líneas auxiliares
    \draw[green!60!black, thick, dashed]
        (x1,y1)--(x2,y2);

    \end{axis}
\end{tikzpicture}
\end{lstlisting}

\subsection{Paleta de Colores Estándar}

El sistema usa una paleta de colores consistente:

\begin{multicols}{2}
\begin{itemize}
    \item \textcolor{red}{\textbf{red}}: Curvas principales
    \item \textcolor{blue}{\textbf{blue}}: Puntos fijos
    \item \textcolor{green!60!black}{\textbf{green!60!black}}: Ejes de simetría
    \item \textcolor{orange}{\textbf{orange}}: Parámetros
    \item \textcolor{purple}{\textbf{purple}}: Distancias
    \item \textcolor{gray}{\textbf{gray!30}}: Grid
\end{itemize}
\end{multicols}

\subsection{Diferencias Clave}

\begin{center}
\begin{tabular}{|p{5.5cm}|p{5.5cm}|}
\hline
\textbf{❌ TikZ básico (incorrecto)} & \textbf{✅ pgfplots (correcto)} \\
\hline
\texttt{\textbackslash draw[domain=...] plot} & \texttt{\textbackslash addplot[domain=...]} \\
\hline
Variables con \texttt{\textbackslash x} & Variables con \texttt{x} (sin backslash) \\
\hline
Escalas manuales necesarias & Escalas automáticas \\
\hline
Limitado a valores pequeños & Cualquier escala de valores \\
\hline
Propenso a errores de dimensión & Sin errores de dimensión \\
\hline
\end{tabular}
\end{center}

\subsection{Checklist de Gráficas}

Antes de compilar, verifica que cada gráfica tenga:

\begin{itemize}
    \item[$\square$] Entorno \texttt{axis} (NO solo \texttt{tikzpicture})
    \item[$\square$] Límites definidos: \texttt{xmin, xmax, ymin, ymax}
    \item[$\square$] Etiquetas en los ejes: \texttt{xlabel, ylabel}
    \item[$\square$] Grid activado: \texttt{grid=both}
    \item[$\square$] Curvas con \texttt{\textbackslash addplot} (NO \texttt{\textbackslash draw ... plot})
    \item[$\square$] Colores según la paleta estándar
    \item[$\square$] Puntos importantes marcados y etiquetados
\end{itemize}

\newpage

\section{Checklist de Inicio Rápido}

Usa esta lista para verificar que tienes todo listo:

\subsection{Antes de empezar}

\begin{itemize}
    \item[$\square$] Tengo LaTeX instalado (\texttt{lualatex --version} funciona)
    \item[$\square$] Tengo Git instalado (\texttt{git --version} funciona)
    \item[$\square$] Tengo acceso a un asistente de IA compatible
    \item[$\square$] Conozco la ruta del Sistema Generador de Guías
    \item[$\square$] He leído esta guía completa
\end{itemize}

\subsection{Al crear mi primera guía}

\begin{itemize}
    \item[$\square$] Elegí un tema que conozco bien
    \item[$\square$] Tengo clara la lista de elementos clave
    \item[$\square$] Tengo al menos 3 aplicaciones de la vida real
    \item[$\square$] Decidí cuántos ejemplos y ejercicios quiero
    \item[$\square$] Sé dónde guardar el archivo
\end{itemize}

\subsection{Después de generar la guía}

\begin{itemize}
    \item[$\square$] Revisé el PDF generado visualmente
    \item[$\square$] Las gráficas se ven correctamente
    \item[$\square$] Los ejemplos están completos
    \item[$\square$] Las soluciones tienen sentido
    \item[$\square$] Versioné el archivo en git
\end{itemize}

\vspace{2cm}

\begin{center}
\Large\textbf{¡Éxito en la creación de tus guías educativas!}

\vspace{1cm}

\large\textit{«La educación es el arma más poderosa\\que puedes usar para cambiar el mundo.»}

\normalsize— Nelson Mandela
\end{center}

\end{document}
