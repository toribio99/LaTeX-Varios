% !TEX program = lualatex
\documentclass[12pt,a4paper]{article}

% Paquetes esenciales
\usepackage{fontspec}
\usepackage[spanish,es-nodecimaldot]{babel}
\usepackage{amsmath,amssymb}
\usepackage{geometry}
\geometry{margin=2.5cm}
\usepackage{tikz}
\usetikzlibrary{calc,arrows.meta,patterns}
\usepackage{pgfplots}
\pgfplotsset{compat=1.18}
\usepackage{xcolor}
\usepackage{multicol}
\usepackage{graphicx}
\usepackage{enumitem}

% Configuración de colores
\definecolor{medicina}{RGB}{220,20,60}
\definecolor{economia}{RGB}{0,128,0}
\definecolor{fisica}{RGB}{0,0,255}
\definecolor{social}{RGB}{255,140,0}

\title{\Huge\bfseries Aplicaciones de la Derivada\\[0.3cm]
\Large En Medicina, Economía, Ciencias Sociales y Física}
\author{\Large Guía para Bachillerato}
\date{}

\begin{document}

\maketitle
\thispagestyle{empty}
%\newpage

\tableofcontents
%\newpage

\section{Introducción}

La derivada es una de las herramientas matemáticas más poderosas y versátiles que existen. Más allá de ser solo un concepto abstracto del cálculo, la derivada tiene aplicaciones directas en casi todas las áreas del conocimiento humano.

\subsection*{¿Qué nos dice la derivada?}

La derivada nos indica \textbf{cómo cambia} algo con respecto a otra cosa. O sea, si tenemos una función $y = f(x)$, su derivada $f'(x)$ nos dice qué tan rápido está creciendo o decreciendo $y$ cuando $x$ cambia.

\textbf{Conceptos clave:}
\begin{itemize}[leftmargin=2cm]
    \item Si $f'(x) > 0$ en un punto, la función está \textbf{creciendo} en ese punto
    \item Si $f'(x) < 0$ en un punto, la función está \textbf{decreciendo} en ese punto
    \item Si $f'(x) = 0$ en un punto, ese punto podría ser un \textbf{máximo o mínimo}
    \item La derivada también representa la \textbf{pendiente de la recta tangente} en cada punto
\end{itemize}

En esta guía veremos cómo la derivada nos ayuda a resolver problemas reales en diferentes disciplinas:

\begin{itemize}[leftmargin=2cm]
    \item \textcolor{medicina}{\textbf{Medicina:}} Propagación de enfermedades, farmacocinética, crecimiento tumoral
    \item \textcolor{economia}{\textbf{Economía:}} Maximización de beneficios, costos marginales, producción óptima
    \item \textcolor{social}{\textbf{Ciencias Sociales:}} Crecimiento poblacional, difusión de información
    \item \textcolor{fisica}{\textbf{Física:}} Velocidad, aceleración, movimiento de proyectiles
\end{itemize}

%\newpage

\section{Aplicaciones en Medicina}

\subsection{Propagación de Enfermedades - Modelo SIR}

En epidemiología, el modelo SIR describe cómo se propaga una enfermedad en una población. Las letras significan:
\begin{itemize}
    \item \textbf{S} (Susceptibles): personas que pueden contraer la enfermedad
    \item \textbf{I} (Infectados): personas que tienen la enfermedad y pueden contagiar
    \item \textbf{R} (Recuperados): personas que ya se recuperaron o están inmunizadas
\end{itemize}

La tasa de cambio de infectados está dada por:
\[
\frac{dI}{dt} = \beta SI - \gamma I
\]

donde:
\begin{itemize}
    \item $\beta$ = tasa de contacto (probabilidad de contagio)
    \item $\gamma$ = tasa de recuperación
    \item $\frac{dI}{dt}$ = derivada que indica cómo cambia el número de infectados con el tiempo
\end{itemize}

\subsection*{\color{medicina}Ejemplo 1: Pico de una Epidemia}

\textbf{Enunciado:} En una ciudad de 10,000 habitantes, se modela una gripe con la función:
\[
I(t) = \frac{1000t}{1 + 0.5t^2}
\]
donde $I(t)$ es el número de infectados en el día $t$. ¿En qué día se alcanza el máximo número de infectados?

\textbf{Solución:}

Para encontrar el máximo, necesitamos encontrar dónde la derivada es cero (o sea, donde la función deja de crecer).

\textbf{Paso 1:} Calcular la derivada usando la regla del cociente:
\[
I'(t) = \frac{(1 + 0.5t^2)(1000) - (1000t)(t)}{(1 + 0.5t^2)^2}
\]

\textbf{Paso 2:} Simplificar el numerador:
\begin{align*}
I'(t) &= \frac{1000 + 500t^2 - 1000t^2}{(1 + 0.5t^2)^2}\\
&= \frac{1000 - 500t^2}{(1 + 0.5t^2)^2}\\
&= \frac{500(2 - t^2)}{(1 + 0.5t^2)^2}
\end{align*}

\textbf{Paso 3:} Igualar a cero:
\[
500(2 - t^2) = 0 \quad \Rightarrow \quad 2 - t^2 = 0 \quad \Rightarrow \quad t = \sqrt{2} \approx 1.41 \text{ días}
\]

\textbf{Paso 4:} Calcular el número máximo de infectados:
\[
I(\sqrt{2}) = \frac{1000\sqrt{2}}{1 + 0.5(2)} = \frac{1414.2}{2} = \boxed{707 \text{ personas}}
\]

\textbf{Interpretación:} El pico de la epidemia ocurre aproximadamente al día y medio, con 707 personas infectadas simultáneamente.

\begin{center}
\begin{tikzpicture}[scale=0.75]
    \def\xmin{0}\def\xmax{6}
    \def\ymin{0}\def\ymax{800}

    % Grid
    \draw[very thin,gray!30] (\xmin,0) grid[ystep=100] (\xmax,8);

    % Ejes
    \draw[-{Latex},thick] (\xmin,0)--(\xmax+0.3,0) node[right]{Días ($t$)};
    \draw[-{Latex},thick] (0,\ymin)--(0,8.5) node[above right=-3mm]{Infectados $I(t)$};

    % Etiquetas del eje y
    \foreach \y in {0,100,200,300,400,500,600,700,800}{
        \draw (0,\y/100) -- (-0.1,\y/100) node[left,scale=0.7]{\y};
    }

    % Etiquetas del eje x
    \foreach \x in {0,1,2,3,4,5,6}{
        \draw (\x,0) -- (\x,-0.1) node[below,scale=0.7]{\x};
    }

    % Curva de infectados
    \draw[medicina,very thick,domain=0.1:6,samples=200,smooth]
        plot (\x,{1000*\x/(1 + 0.5*\x*\x)/100});

    % Punto máximo
    \fill[blue] (1.414,7.07) circle (0.08) node[above right,scale=0.85]{Pico: $(1.41, 707)$};

    % Línea vertical en el máximo
    \draw[blue,dashed,thick] (1.414,0) -- (1.414,7.07);

    % Título
    \node[medicina,scale=1] at (3,9.5) {\textbf{Curva de Infectados en el Tiempo}};
\end{tikzpicture}
\end{center}

\newpage

\subsection{Concentración de Medicamentos - Farmacocinética}

\subsection*{\color{medicina}Ejemplo 2: Nivel Óptimo de un Fármaco}

\textbf{Enunciado:} La concentración de un antibiótico en la sangre (en mg/L) después de $t$ horas está dada por:
\[
C(t) = \frac{20t}{t^2 + 4}
\]
¿En qué momento se alcanza la concentración máxima? ¿Cuál es esa concentración?

\textbf{Solución:}

\textbf{Paso 1:} Derivar usando la regla del cociente:
\[
C'(t) = \frac{(t^2 + 4)(20) - (20t)(2t)}{(t^2 + 4)^2}
\]

\textbf{Paso 2:} Simplificar:
\begin{align*}
C'(t) &= \frac{20t^2 + 80 - 40t^2}{(t^2 + 4)^2}\\
&= \frac{80 - 20t^2}{(t^2 + 4)^2}\\
&= \frac{20(4 - t^2)}{(t^2 + 4)^2}
\end{align*}

\textbf{Paso 3:} Igualar a cero:
\[
20(4 - t^2) = 0 \quad \Rightarrow \quad t^2 = 4 \quad \Rightarrow \quad t = 2 \text{ horas}
\]

(tomamos $t = 2$ porque $t$ debe ser positivo)

\textbf{Paso 4:} Calcular la concentración máxima:
\[
C(2) = \frac{20(2)}{2^2 + 4} = \frac{40}{8} = \boxed{5 \text{ mg/L}}
\]

\textbf{Interpretación:} El medicamento alcanza su concentración máxima de 5 mg/L a las 2 horas de su administración.

\begin{center}
\begin{tikzpicture}[scale=0.85]
    \def\xmin{0}\def\xmax{10}

    % Grid
    \draw[very thin,gray!30] (\xmin,0) grid[ystep=0.5] (\xmax,6);

    % Ejes
    \draw[-{Latex},thick] (\xmin,0)--(\xmax+0.3,0) node[right]{Tiempo (horas)};
    \draw[-{Latex},thick] (0,0)--(0,6.5) node[above right=-2mm]{$C(t)$ (mg/L)};

    % Etiquetas eje y
    \foreach \y in {0,1,2,3,4,5,6}{
        \draw (0,\y) -- (-0.1,\y) node[left,scale=0.7]{\y};
    }

    % Etiquetas eje x
    \foreach \x in {0,2,4,6,8,10}{
        \draw (\x,0) -- (\x,-0.1) node[below,scale=0.7]{\x};
    }

    % Curva de concentración
    \draw[medicina,very thick,domain=0.1:10,samples=200,smooth]
        plot (\x,{20*\x/(\x*\x + 4)});

    % Punto máximo
    \fill[blue] (2,5) circle (0.08) node[above right,scale=0.85]{Máximo: $(2, 5)$};

    % Líneas de referencia
    \draw[blue,dashed] (2,0) -- (2,5);
    \draw[blue,dashed] (0,5) -- (2,5);

    % Nivel terapéutico
    \draw[green!60!black,thick,dashed] (0,3) -- (10,3) node[above right,scale=0.8] at(6.5,3.1) {Nivel terapéutico};

    % Título
    \node[medicina,scale=1] at (5,7.8) {\textbf{Concentración del Fármaco vs Tiempo}};

\end{tikzpicture}
\end{center}

\newpage

\section{Aplicaciones en Economía}

\subsection{Maximización de Beneficios}

En economía, las empresas buscan \textbf{maximizar sus beneficios}. El beneficio se define como:
\[
B(x) = I(x) - C(x)
\]
donde:
\begin{itemize}
    \item $B(x)$ = Beneficio al producir $x$ unidades
    \item $I(x)$ = Ingreso por ventas
    \item $C(x)$ = Costo de producción
\end{itemize}

Para maximizar el beneficio, necesitamos encontrar donde $B'(x) = 0$.

\subsection*{\color{economia}Ejemplo 3: Producción Óptima de una Fábrica}

\textbf{Enunciado:} Una fábrica produce sillas. El ingreso por vender $x$ sillas está dado por:
\[
I(x) = 100x - 0.5x^2
\]
y el costo de producirlas es:
\[
C(x) = 20x + 500
\]
¿Cuántas sillas debe producir para maximizar el beneficio? ¿Cuál es el beneficio máximo?

\textbf{Solución:}

\textbf{Paso 1:} Definir la función de beneficio:
\begin{align*}
B(x) &= I(x) - C(x)\\
&= (100x - 0.5x^2) - (20x + 500)\\
&= 100x - 0.5x^2 - 20x - 500\\
&= -0.5x^2 + 80x - 500
\end{align*}

\textbf{Paso 2:} Derivar:
\[
B'(x) = -x + 80
\]

\textbf{Paso 3:} Igualar a cero y resolver:
\[
-x + 80 = 0 \quad \Rightarrow \quad x = 80 \text{ sillas}
\]

\textbf{Paso 4:} Verificar que es un máximo (la segunda derivada):
\[
B''(x) = -1 < 0 \quad \text{(es un máximo)}
\]

\textbf{Paso 5:} Calcular el beneficio máximo:
\begin{align*}
B(80) &= -0.5(80)^2 + 80(80) - 500\\
&= -3200 + 6400 - 500\\
&= \boxed{2700 \text{ pesos}}
\end{align*}

\textbf{Interpretación:} La fábrica debe producir 80 sillas para obtener el beneficio máximo de 2700 pesos.

\begin{center}
\begin{tikzpicture}
    \begin{axis}[
        width=12cm, height=8cm,
        axis lines=middle,
        xlabel={Sillas ($x$)}, ylabel={Cientos de pesos},
        xmin=0, xmax=180,
        ymin=-10, ymax=30,
        xtick={0,20,40,60,80,100,120,140,160},
        ytick={-10,-5,0,5,10,15,20,25,30},
        grid=both,
        grid style={line width=.1pt, draw=gray!30},
        axis line style={-{Latex},thick},
        tick label style={font=\small},
        samples=100,
        legend pos=north east,
        legend style={font=\small},
    ]

    % Función de ingresos
    \addplot[blue,thick,domain=0:180] {(100*x - 0.5*x*x)/500};
    \addlegendentry{$I(x)$}

    % Función de costos
    \addplot[red,thick,domain=0:180] {(20*x + 500)/500};
    \addlegendentry{$C(x)$}

    % Función de beneficio
    \addplot[economia,very thick,domain=0:170] {(-0.5*x*x + 80*x - 500)/500};
    \addlegendentry{$B(x)$}

    % Punto máximo
    \node[circle, fill=black, inner sep=2pt] at (80,{(-0.5*80*80 + 80*80 - 500)/500}) {};
    \node[black,above,scale=0.8] at (80,5.6) {Máx: $(80, 2700)$};
    \draw[black,dashed,thick] (80,-10)--(80,{(-0.5*80*80 + 80*80 - 500)/500});

    \end{axis}
\end{tikzpicture}
\end{center}

%\newpage

\subsection{Costo Marginal}

El \textbf{costo marginal} es la derivada de la función de costo: $C'(x)$. Representa cuánto cuesta producir una unidad adicional.

\subsection*{\color{economia}Ejemplo 4: Costo Marginal y Producción Eficiente}

\textbf{Enunciado:} El costo total de producir $x$ unidades de un producto es:
\[
C(x) = 0.01x^3 - 0.6x^2 + 13x + 100
\]
a) ¿Cuál es el costo marginal cuando se producen 20 unidades?\\
b) ¿Para qué nivel de producción el costo marginal es mínimo?

\textbf{Solución:}

\textbf{Parte a:}

\textbf{Paso 1:} Calcular el costo marginal (derivada):
\begin{align*}
C'(x) &= \frac{d}{dx}(0.01x^3 - 0.6x^2 + 13x + 100)\\
&= 0.03x^2 - 1.2x + 13
\end{align*}

\textbf{Paso 2:} Evaluar en $x = 20$:
\begin{align*}
C'(20) &= 0.03(20)^2 - 1.2(20) + 13\\
&= 0.03(400) - 24 + 13\\
&= 12 - 24 + 13\\
&= \boxed{1 \text{ peso por unidad}}
\end{align*}

\textbf{Interpretación:} Cuando se están produciendo 20 unidades, producir una unidad adicional cuesta aproximadamente 1 peso.

\textbf{Parte b:}

Para encontrar el mínimo del costo marginal, derivamos nuevamente:

\textbf{Paso 3:} Calcular la segunda derivada:
\[
C''(x) = 0.06x - 1.2
\]

\textbf{Paso 4:} Igualar a cero:
\[
0.06x - 1.2 = 0 \quad \Rightarrow \quad x = \frac{1.2}{0.06} = \boxed{20 \text{ unidades}}
\]

\textbf{Interpretación:} El costo marginal es mínimo cuando se producen 20 unidades, lo que significa que esta es la producción más eficiente.

\begin{center}
\begin{tikzpicture}[scale=0.65]
    \def\xmin{0}\def\xmax{50}

    % Grid
    \draw[very thin,gray!30] (\xmin,0) grid[xstep=5,ystep=2] (\xmax/5,20);

    % Ejes
    \draw[-{Latex},thick] (\xmin,0)--(\xmax/5+0.5,0) node[right]{Unidades ($x$)};
    \draw[-{Latex},thick] (0,0)--(0,21) node[above]{Costo Marginal};

    % Etiquetas eje x
    \foreach \x in {0,5,10,15,20,25,30,35,40,45,50}{
        \draw (\x/5,0) -- (\x/5,-0.2) node[below,scale=0.7]{\x};
    }

    % Etiquetas eje y
    \foreach \y in {0,2,4,6,8,10,12,14,16,18,20}{
        \draw (0,\y) -- (-0.1,\y) node[left,scale=0.7]{\y};
    }

    % Costo marginal
    \draw[economia,very thick,domain=0:45,samples=150]
        plot (\x/5,{0.03*\x*\x - 1.2*\x + 13});

    % Punto mínimo
    \fill[blue] (20/5,1) circle (0.08) node[below right,scale=0.85]{Mínimo: $(20, 1)$};
    \draw[blue,dashed] (20/5,0) -- (20/5,1);

    % Título
    \node[economia,scale=1.1] at (5,22) {\textbf{Costo Marginal $C'(x)$}};

\end{tikzpicture}
\end{center}

%\newpage

\section{Aplicaciones en Ciencias Sociales}

\subsection{Crecimiento Poblacional}

El crecimiento de una población puede modelarse con funciones exponenciales o logísticas. La derivada nos indica la \textbf{tasa de crecimiento}.

\subsection*{\color{social}Ejemplo 5: Crecimiento de una Ciudad}

\textbf{Enunciado:} La población de una ciudad (en miles) está dada por:
\[
P(t) = \frac{100}{1 + 9e^{-0.3t}}
\]
donde $t$ es el tiempo en años desde el año 2000.\\
a) ¿Cuál es la tasa de crecimiento en el año 2010?\\
b) ¿Cuándo la población crece más rápidamente?

\textbf{Solución:}

\textbf{Parte a:}

\textbf{Paso 1:} Derivar usando regla de la cadena y del cociente:
\[
P'(t) = \frac{100 \cdot 9e^{-0.3t} \cdot 0.3}{(1 + 9e^{-0.3t})^2} = \frac{270e^{-0.3t}}{(1 + 9e^{-0.3t})^2}
\]

\textbf{Paso 2:} Evaluar en $t = 10$ (año 2010):
\begin{align*}
P'(10) &= \frac{270e^{-3}}{(1 + 9e^{-3})^2}\\
&= \frac{270(0.0498)}{(1 + 0.448)^2}\\
&= \frac{13.44}{2.097}\\
&= \boxed{6.41 \text{ mil personas/año}}
\end{align*}

\textbf{Interpretación:} En el año 2010, la ciudad estaba creciendo a una tasa de aproximadamente 6,410 personas por año.

\textbf{Parte b:}

La población crece más rápidamente cuando la derivada es máxima. Para funciones logísticas, esto ocurre cuando la población es la mitad de la capacidad de carga.

Capacidad de carga: $\lim_{t \to \infty} P(t) = 100$ mil personas

Máximo crecimiento cuando: $P(t) = 50$

\[
\frac{100}{1 + 9e^{-0.3t}} = 50 \quad \Rightarrow \quad 1 + 9e^{-0.3t} = 2 \quad \Rightarrow \quad e^{-0.3t} = \frac{1}{9}
\]

\[
-0.3t = \ln\left(\frac{1}{9}\right) = -\ln(9) \quad \Rightarrow \quad t = \frac{\ln(9)}{0.3} \approx \boxed{7.32 \text{ años}}
\]

\textbf{Interpretación:} La población crece más rápidamente alrededor del año 2007.

\begin{center}
\begin{tikzpicture}[scale=0.65]
    \def\xmin{0}\def\xmax{40}

    % Grid
    \draw[very thin,gray!30] (\xmin,0) grid[xstep=5,ystep=1] (\xmax/2,12);

    % Ejes
    \draw[-{Latex},thick] (\xmin,0)--(\xmax/2+0.5,0) node[above left]{Años desde 2000};
    \draw[-{Latex},thick] (0,0)--(0,12.5) node[above right=-3mm]{Población (miles)};

    % Etiquetas eje x
    \foreach \x in {0,5,10,15,20,25,30,35,40}{
        \draw (\x/2,0) -- (\x/2,-0.2) node[below,scale=0.7]{\x};
    }

    % Etiquetas eje y
    \foreach \y in {0,20,40,60,80,100}{
        \draw (0,\y/10) -- (-0.1,\y/10) node[left,scale=0.7]{\y};
    }

    % Curva de población
    \draw[social,very thick,domain=0:40,samples=200]
        plot (\x/2,{100/(1 + 9*exp(-0.3*\x))/10});

    % Punto de inflexión (máximo crecimiento)
    \fill[blue] (7.32/2,5) circle (0.08) node[right,scale=0.85]{$(7.32, 50)$};
    \draw[blue,dashed] (7.32/2,0) -- (7.32/2,5);
    \draw[blue,dashed] (0,5) -- (7.32/2,5);

    % Asíntota
    \draw[red,dashed,thick] (0,10) -- (20,10) node[above left,scale=0.8]{Capacidad: 100 mil};

    % Título
    \node[social,scale=1.1] at (10,13.8) {\textbf{Crecimiento Poblacional (Modelo Logístico)}};

\end{tikzpicture}
\end{center}

%\newpage

\subsection{Difusión de Información en Redes Sociales}

\subsection*{\color{social}Ejemplo 6: Viralización de un Video}

\textbf{Enunciado:} El número de visualizaciones (en millones) de un video viral está dado por:
\[
V(t) = 5(1 - e^{-0.5t})
\]
donde $t$ es el tiempo en días desde su publicación.\\
a) ¿Cuál es la tasa de visualizaciones en el día 3?\\
b) ¿Hacia qué valor tiende el número total de visualizaciones?

\textbf{Solución:}

\textbf{Parte a:}

\textbf{Paso 1:} Derivar:
\begin{align*}
V'(t) &= 5 \cdot \frac{d}{dt}(1 - e^{-0.5t})\\
&= 5 \cdot (0.5e^{-0.5t})\\
&= 2.5e^{-0.5t}
\end{align*}

\textbf{Paso 2:} Evaluar en $t = 3$:
\begin{align*}
V'(3) &= 2.5e^{-1.5}\\
&= 2.5(0.2231)\\
&= \boxed{0.558 \text{ millones/día}}
\end{align*}

\textbf{Interpretación:} En el día 3, el video está recibiendo aproximadamente 558,000 visualizaciones por día.

\textbf{Parte b:}

El límite cuando $t \to \infty$:
\[
\lim_{t \to \infty} V(t) = \lim_{t \to \infty} 5(1 - e^{-0.5t}) = 5(1 - 0) = \boxed{5 \text{ millones}}
\]

\textbf{Interpretación:} El video alcanzará aproximadamente 5 millones de visualizaciones en total.

\begin{center}
\begin{tikzpicture}[scale=0.8]
    \def\xmin{0}\def\xmax{12}

    % Grid
    \draw[very thin,gray!30] (\xmin,0) grid[xstep=1,ystep=0.5] (\xmax,6);

    % Ejes
    \draw[-{Latex},thick] (\xmin,0)--(\xmax+0.5,0) node[above left]{Días ($t$)};
    \draw[-{Latex},thick] (0,0)--(0,6.5) node[below right=.2mm]{Millones de vistas};

    % Etiquetas
    \foreach \x in {0,2,4,6,8,10,12}{
        \draw (\x,0) -- (\x,-0.1) node[below,scale=0.7]{\x};
    }
    \foreach \y in {0,1,2,3,4,5}{
        \draw (0,\y) -- (-0.1,\y) node[left,scale=0.7]{\y};
    }

    % Curva de visualizaciones
    \draw[social,very thick,domain=0:12,samples=150]
        plot (\x,{5*(1 - exp(-0.5*\x))});

    % Punto en t=3
    \fill[blue] (3,{5*(1-exp(-1.5))}) circle (0.08) node[below right,scale=0.8]{Día 3};
    \draw[blue,dashed] (3,0) -- (3,{5*(1-exp(-1.5))});

    % Asíntota
    \draw[red,dashed,thick] (0,5) -- (12,5) node[below left,scale=0.8]{Límite: 5M};

    % Título
    \node[social,scale=1.1] at (6,7.5) {\textbf{Visualizaciones del Video}};

\end{tikzpicture}
\end{center}

%\newpage

\section{Aplicaciones en Física}

\subsection{Movimiento: Velocidad y Aceleración}

En física, si $s(t)$ representa la posición de un objeto en el tiempo $t$, entonces:
\begin{itemize}
    \item \textbf{Velocidad:} $v(t) = s'(t)$ (primera derivada de la posición)
    \item \textbf{Aceleración:} $a(t) = v'(t) = s''(t)$ (segunda derivada de la posición)
\end{itemize}

\subsection*{\color{fisica}Ejemplo 7: Movimiento de un Proyectil}

\textbf{Enunciado:} Un proyectil es lanzado verticalmente hacia arriba. Su altura (en metros) en el tiempo $t$ (en segundos) está dada por:
\[
h(t) = -5t^2 + 30t + 2
\]
a) ¿Cuál es la velocidad inicial?\\
b) ¿Cuándo alcanza la altura máxima?\\
c) ¿Cuál es la altura máxima?\\
d) ¿Cuál es la aceleración del proyectil?

\textbf{Solución:}

\textbf{Parte a:} La velocidad es la derivada de la posición:
\[
v(t) = h'(t) = -10t + 30
\]

La velocidad inicial es cuando $t = 0$:
\[
v(0) = -10(0) + 30 = \boxed{30 \text{ m/s}}
\]

\textbf{Parte b:} La altura máxima ocurre cuando $v(t) = 0$:
\[
-10t + 30 = 0 \quad \Rightarrow \quad t = \boxed{3 \text{ segundos}}
\]

\textbf{Parte c:} Sustituir $t = 3$ en $h(t)$:
\begin{align*}
h(3) &= -5(3)^2 + 30(3) + 2\\
&= -45 + 90 + 2\\
&= \boxed{47 \text{ metros}}
\end{align*}

\textbf{Parte d:} La aceleración es la derivada de la velocidad:
\[
a(t) = v'(t) = h''(t) = -10 \text{ m/s}^2
\]

\textbf{Interpretación:} La aceleración es constante y negativa (es la gravedad), lo que significa que el proyectil desacelera al subir.

\begin{center}
\begin{tikzpicture}[scale=0.85]
    \def\xmin{0}\def\xmax{7}

    % Grid
    \draw[very thin,gray!30] (\xmin,0) grid[xstep=1,ystep=0.5] (\xmax,5.5);

    % Ejes
    \draw[-{Latex},thick] (\xmin,0)--(\xmax+0.3,0) node[above left]{Tiempo (s)};
    \draw[-{Latex},thick] (0,0)--(0,6) node[below right=-1mm]{Altura (m)};

    % Etiquetas
    \foreach \x in {0,1,2,3,4,5,6}{
        \draw (\x,0) -- (\x,-0.15) node[below,scale=0.7]{\x};
    }
    \foreach \y in {0,10,20,30,40,50}{
        \draw (0,\y/10) -- (-0.1,\y/10) node[left,scale=0.7]{\y};
    }

    % Trayectoria
    \draw[fisica,very thick,domain=0:6.13,samples=100]
        plot (\x,{(-5*\x*\x + 30*\x + 2)/10});

    % Punto máximo
    \fill[red] (3,4.7) circle (0.08) node[above right,scale=0.8]{Máx: $(3, 47)$};
    \draw[red,dashed] (3,0) -- (3,4.7);
    \draw[red,dashed] (0,4.7) -- (3,4.7);

    % Punto inicial
    \fill[green!60!black] (0,0.2) circle (0.06) node[right,scale=0.7]{$(0, 2)$};

    % Título
    \node[fisica,scale=1] at (3.5,7.5) {\textbf{Trayectoria del Proyectil}};

\end{tikzpicture}
\end{center}

%\newpage

\subsection{Velocidad y Aceleración en Gráficas}

\subsection*{\color{fisica}Ejemplo 8: Análisis Completo de Movimiento}

\textbf{Enunciado:} Un auto se mueve según la función de posición:
\[
s(t) = t^3 - 9t^2 + 24t
\]
donde $s$ está en metros y $t$ en segundos.\\
a) ¿Cuándo el auto está en reposo?\\
b) ¿Cuándo la velocidad es máxima?\\
c) Graficar posición, velocidad y aceleración.

\textbf{Solución:}

\textbf{Parte a:} El auto está en reposo cuando $v(t) = 0$.

\textbf{Paso 1:} Calcular la velocidad:
\[
v(t) = s'(t) = 3t^2 - 18t + 24
\]

\textbf{Paso 2:} Factorizar:
\begin{align*}
3t^2 - 18t + 24 &= 0\\
3(t^2 - 6t + 8) &= 0\\
3(t - 2)(t - 4) &= 0
\end{align*}

Por lo tanto: $t = 2$ s o $t = 4$ s

\textbf{Respuesta:} El auto está en reposo en $\boxed{t = 2 \text{ s y } t = 4 \text{ s}}$

\textbf{Parte b:} La velocidad es máxima o mínima cuando $a(t) = 0$.

\textbf{Paso 3:} Calcular la aceleración:
\[
a(t) = v'(t) = s''(t) = 6t - 18
\]

\textbf{Paso 4:} Igualar a cero:
\[
6t - 18 = 0 \quad \Rightarrow \quad t = 3 \text{ s}
\]

\textbf{Paso 5:} Verificar si es máximo o mínimo evaluando $v(3)$:
\[
v(3) = 3(3)^2 - 18(3) + 24 = 27 - 54 + 24 = -3 \text{ m/s}
\]

Como $v(2) = 0$ y $v(3) = -3$, en $t = 3$ la velocidad es \textbf{mínima} (más negativa).

\textbf{Interpretación:} El auto alcanza su velocidad hacia atrás más grande en $t = 3$ segundos.

\begin{center}
\begin{tikzpicture}[scale=0.48]
    % Gráfica de posición
    \begin{scope}
        \def\xmin{0}\def\xmax{6}

        % Grid
        \draw[very thin,gray!30] (\xmin,-0.5) grid[xstep=1,ystep=0.5] (\xmax,5);

        % Ejes
        \draw[-{Latex},thick] (\xmin,0)--(\xmax+0.3,0) node[right,scale=0.8]{$t$ (s)};
        \draw[-{Latex},thick] (0,-0.5)--(0,5.3) node[below right=-.5mm,scale=0.8]{$s(t)$ (m)};

        % Curva
        \draw[fisica,very thick,domain=0:6,samples=100]
            plot (\x,{(\x*\x*\x - 9*\x*\x + 24*\x)/20});

        % Puntos de reposo
        \fill[red] (2,{(8-36+48)/20}) circle (0.08);
        \fill[red] (4,{(64-144+96)/20}) circle (0.08);
		\fill[red] (2,{(8-36+48)/20})  circle (0.08) node[scale=0.65, rotate=90] at(2,1.8){$t=2$};
		\fill[red] (4,{(64-144+96)/20}) circle (0.08) node[scale=0.65, rotate=90] at(4,1.6){$t=4$};

        % Título
        \node[fisica,scale=0.9] at (3,6.2) {\textbf{Posición}};

        % Etiquetas
        \foreach \x in {0,1,2,3,4,5,6}{
            \node[below,scale=0.6] at (\x,-0.15) {\x};
        }
        \foreach \y in {0,20,40,60,80}{
            \node[left,scale=0.6] at (-0.15,\y/20) {\y};
        }
    \end{scope}

    % Gráfica de velocidad
    \begin{scope}[xshift=9.5cm]
        \def\xmin{0}\def\xmax{6}

        % Grid
        \draw[very thin,gray!30] (\xmin,-1.5) grid[xstep=1,ystep=0.5] (\xmax,4);

        % Ejes
        \draw[-{Latex},thick] (\xmin,0)--(\xmax+0.3,0) node[right,scale=0.8]{$t$ (s)};
        \draw[-{Latex},thick] (0,-1.5)--(0,4.3) node[below right=-.5mm,scale=0.8]{$v(t)$ (m/s)};

        % Curva
        \draw[green!60!black,very thick,domain=0:6,samples=100]
            plot (\x,{(3*\x*\x - 18*\x + 24)/10});

        % Puntos donde v=0
        \fill[red] (2,0) circle (0.08) node[scale=0.65, rotate=90] at(2,1){$t=2$};
        \fill[red] (4,0) circle (0.08) node[scale=0.65, rotate=90] at(4,1){$t=4$};

        % Punto de velocidad mínima
        \fill[blue] (3,-0.3) circle (0.08) node[below,scale=0.65]{Mín};

        % Título
        \node[green!60!black,scale=0.9] at (3,6.2) {\textbf{Velocidad}};

        % Etiquetas
        \foreach \x in {0,1,2,3,4,5,6}{
            \node[below,scale=0.6] at (\x,-0.15) {\x};
        }
    \end{scope}

    % Gráfica de aceleración
    \begin{scope}[xshift=19cm]
        \def\xmin{0}\def\xmax{6}

        % Grid
        \draw[very thin,gray!30] (\xmin,-2) grid[xstep=1,ystep=0.5] (\xmax,2);

        % Ejes
        \draw[-{Latex},thick] (\xmin,0)--(\xmax+0.3,0) node[right,scale=0.8]{$t$ (s)};
        \draw[-{Latex},thick] (0,-2)--(0,2.3) node[below right=-.5,scale=0.8]{$a(t)$ (m/s²)};

        % Curva (línea recta)
        \draw[red,very thick,domain=0:6,samples=50]
            plot (\x,{(6*\x - 18)/10});

        % Punto donde a=0
        \fill[blue] (3,0) circle (0.08) node[above right,scale=0.65, rotate=30]{$t=3;a=0$};

        % Título
        \node[red,scale=0.9] at (3,6.2) {\textbf{Aceleración}};

        % Etiquetas
        \foreach \x in {0,1,2,3,4,5,6}{
            \node[below,scale=0.6] at (\x,-0.15) {\x};
        }
        \foreach \y in {-20,-10,0,10,20}{
            \node[left,scale=0.6] at (-0.15,\y/10) {\y};
        }
    \end{scope}
\end{tikzpicture}
\end{center}

%\newpage

\section{Ejercicios Propuestos}

\subsection{Medicina}

\textbf{Ejercicio 1:} La cantidad de glucosa en la sangre después de una comida está modelada por:
\[
G(t) = 85 + \frac{40t}{1 + t^2}
\]
donde $G$ está en mg/dL y $t$ en horas.
\begin{enumerate}[label=\alph*)]
    \item ¿Cuándo alcanza el nivel máximo de glucosa?
    \item ¿Cuál es ese nivel máximo?
    \item ¿Cuál es la tasa de cambio de glucosa a las 2 horas?
\end{enumerate}

\textbf{Ejercicio 2:} El tamaño de un tumor (en cm³) está dado por:
\[
T(t) = 0.5e^{0.1t}
\]
donde $t$ está en semanas.
\begin{enumerate}[label=\alph*)]
    \item ¿Cuál es la tasa de crecimiento del tumor en la semana 10?
    \item Si el tumor debe operarse cuando alcance 5 cm³, ¿en qué semana debe programarse la cirugía?
\end{enumerate}

\subsection{Economía}

\textbf{Ejercicio 3:} Una compañía tiene ingresos y costos dados por:
\[
I(x) = 200x - 2x^2 \quad \text{y} \quad C(x) = 50x + 1000
\]
donde $x$ es el número de productos vendidos.
\begin{enumerate}[label=\alph*)]
    \item Encuentra la función de beneficio $B(x)$.
    \item ¿Cuántos productos maximizan el beneficio?
    \item ¿Cuál es el beneficio máximo?
\end{enumerate}

\textbf{Ejercicio 4:} El costo de producción está dado por:
\[
C(x) = 0.002x^3 - 0.3x^2 + 20x + 500
\]
\begin{enumerate}[label=\alph*)]
    \item Encuentra el costo marginal $C'(x)$.
    \item ¿Para qué nivel de producción el costo marginal es mínimo?
\end{enumerate}

\subsection{Ciencias Sociales}

\textbf{Ejercicio 5:} Una noticia se difunde según:
\[
N(t) = \frac{5000}{1 + 99e^{-0.6t}}
\]
donde $N(t)$ es el número de personas que conocen la noticia en el día $t$.
\begin{enumerate}[label=\alph*)]
    \item ¿Cuándo la noticia se difunde más rápidamente?
    \item ¿Cuál es la tasa máxima de difusión?
\end{enumerate}

\subsection{Física}

\textbf{Ejercicio 6:} Una partícula se mueve según:
\[
s(t) = 2t^3 - 15t^2 + 24t + 5
\]
donde $s$ está en metros y $t$ en segundos.
\begin{enumerate}[label=\alph*)]
    \item Encuentra las funciones de velocidad y aceleración.
    \item ¿Cuándo la partícula está en reposo?
    \item ¿Cuándo la aceleración es cero?
\end{enumerate}

\textbf{Ejercicio 7:} Un cohete es lanzado con altura:
\[
h(t) = -4.9t^2 + 150t
\]
\begin{enumerate}[label=\alph*)]
    \item ¿Cuál es la altura máxima alcanzada?
    \item ¿Cuándo toca el suelo?
    \item ¿Cuál es la velocidad al impactar?
\end{enumerate}

%\newpage

\section{Soluciones de los Ejercicios}

\subsection*{Solución Ejercicio 1 (Medicina - Glucosa)}

\textbf{Parte a:} Encontrar el máximo nivel de glucosa.

Derivamos:
\[
G'(t) = \frac{(1+t^2)(40) - (40t)(2t)}{(1+t^2)^2} = \frac{40 + 40t^2 - 80t^2}{(1+t^2)^2} = \frac{40(1-t^2)}{(1+t^2)^2}
\]

Igualamos a cero:
\[
40(1-t^2) = 0 \Rightarrow t^2 = 1 \Rightarrow t = 1 \text{ hora}
\]

\textbf{Parte b:} Nivel máximo:
\[
G(1) = 85 + \frac{40(1)}{1+1} = 85 + 20 = \boxed{105 \text{ mg/dL}}
\]

\textbf{Parte c:} Tasa de cambio a las 2 horas:
\[
G'(2) = \frac{40(1-4)}{(1+4)^2} = \frac{-120}{25} = \boxed{-4.8 \text{ mg/dL por hora}}
\]

\begin{center}
\begin{tikzpicture}[scale=1]
    \draw[very thin,gray!30] (0,8) grid[xstep=0.5,ystep=0.5] (6,12);
    \draw[-{Latex},thick] (0,8)--(6.3,8) node[above left]{Horas};
    \draw[-{Latex},thick] (0,8)--(0,12.5) node[below right]{Glucosa (mg/dL)};

    \draw[medicina,very thick,domain=0:6,samples=150]
        plot (\x,{8.5 + 4*\x/(1+\x*\x)});

    \fill[blue] (1,10.5) circle (0.08) node[above right]{Máx: $(1, 105)$};
    \draw[blue,dashed] (1,8) -- (1,10.5);

    \foreach \x in {0,1,2,3,4,5,6}{
        \node[below,scale=0.7] at (\x,7.9) {\x};
    }
    \foreach \y in {85,90,95,100,105,110,115,120}{
        \node[left,scale=0.6] at (-0.1,{(\y-5)/10}) {\y};
    }
\end{tikzpicture}
\end{center}

\subsection*{Solución Ejercicio 3 (Economía - Beneficio)}

\textbf{Parte a:} Función de beneficio:
\[
B(x) = I(x) - C(x) = (200x - 2x^2) - (50x + 1000) = -2x^2 + 150x - 1000
\]

\textbf{Parte b:} Maximizar beneficio:
\[
B'(x) = -4x + 150 = 0 \Rightarrow x = 37.5 \approx \boxed{38 \text{ productos}}
\]

\textbf{Parte c:} Beneficio máximo:
\[
B(37.5) = -2(37.5)^2 + 150(37.5) - 1000 = -2812.5 + 5625 - 1000 = \boxed{1812.5 \text{ pesos}}
\]

\begin{center}
\begin{tikzpicture}[scale=0.9]
    \draw[very thin,gray!30] (0,-2) grid[xstep=1,ystep=1] (8,3.5);
    \draw[-{Latex},thick] (0,0)--(8.3,0) node[above]{Productos ($x$)};
    \draw[-{Latex},thick] (0,-2)--(0,3.5) node[below right]{Beneficio (cientos)};

    \draw[economia,very thick,domain=0:75,samples=100]
        plot (\x/10,{(-2*\x*\x + 150*\x - 1000)/1000});

    \fill[blue] (3.75,1.8125) circle (0.08) node[above right,scale=0.8]{$(38, 1812)$};
    \draw[blue,dashed] (3.75,0) -- (3.75,1.8125);

    \foreach \x in {0,10,20,30,40,50,60,70}{
        \node[below,scale=0.7] at (\x/10,-0.2) {\x};
    }
	\foreach \y in {0,20,40,60}{
	\pgfmathparse{\y*(0.05)}
	\node[left, scale=0.7] at (-0.2,{ \pgfmathresult }) {\y};
	}
\end{tikzpicture}
\end{center}

\subsection*{Solución Ejercicio 6 (Física - Movimiento)}

\textbf{Parte a:} Velocidad y aceleración:
\[
v(t) = s'(t) = 6t^2 - 30t + 24
\]
\[
a(t) = v'(t) = 12t - 30
\]

\textbf{Parte b:} Partícula en reposo cuando $v(t) = 0$:
\[
6t^2 - 30t + 24 = 0 \Rightarrow t^2 - 5t + 4 = 0 \Rightarrow (t-1)(t-4) = 0
\]
\[
\boxed{t = 1 \text{ s o } t = 4 \text{ s}}
\]

\textbf{Parte c:} Aceleración cero:
\[
12t - 30 = 0 \Rightarrow \boxed{t = 2.5 \text{ s}}
\]

\begin{center}
\begin{tikzpicture}[scale=1.1]
    \draw[very thin,gray!30] (0,-1.5) grid[xstep=1,ystep=.5] (6,2.5);
    \draw[-{Latex},thick] (0,0)--(6.3,0) node[right]{$t$ (s)};
    \draw[-{Latex},thick] (0,-1.5)--(0,2.5) node[right]{$s(t)$ (m)};

    \draw[fisica,very thick,domain=0:5.5,samples=100]
        plot (\x,{(2*\x*\x*\x - 15*\x*\x + 24*\x + 5)/10});

    \fill[red] (1,{(2-15+24+5)/10}) circle (0.08) node[above,scale=1]{$v=0$};
    \fill[red] (4,{(128-240+96+5)/10}) circle (0.08) node[below,scale=1]{$v=0$};

    \foreach \x in {0,1, 2, 3, 4, 5}{
        \node[below,scale=0.7] at (\x,-0.2) {\x};
    }
    
        \foreach \y in {-15, -10, -5, 0, 5, 10, 15, 20}{
        \pgfmathparse{\y/10}
    	\node[left,scale=0.7] at (-0.1,{ \pgfmathresult }) {\y};
    }
\end{tikzpicture}
\end{center}

%\newpage

\section{Conclusiones}

Las derivadas son una herramienta fundamental en el análisis de fenómenos reales. A través de esta guía hemos visto cómo:

\begin{itemize}
    \item En \textcolor{medicina}{\textbf{medicina}}, las derivadas nos ayudan a entender:
    \begin{itemize}
        \item Cómo se propagan las enfermedades
        \item Cuándo un medicamento alcanza su efecto máximo
        \item La velocidad de crecimiento de tumores
    \end{itemize}

    \item En \textcolor{economia}{\textbf{economía}}, permiten:
    \begin{itemize}
        \item Maximizar beneficios y minimizar costos
        \item Determinar la producción óptima
        \item Calcular costos marginales
    \end{itemize}

    \item En \textcolor{social}{\textbf{ciencias sociales}}, ayudan a modelar:
    \begin{itemize}
        \item Crecimiento poblacional
        \item Difusión de información
        \item Dinámica de redes sociales
    \end{itemize}

    \item En \textcolor{fisica}{\textbf{física}}, son esenciales para:
    \begin{itemize}
        \item Describir movimiento (velocidad y aceleración)
        \item Analizar trayectorias
        \item Optimizar procesos físicos
    \end{itemize}
\end{itemize}

\subsection*{Conceptos clave para recordar:}

\begin{enumerate}
    \item La derivada mide la \textbf{tasa de cambio} de una cantidad
    \item $f'(x) = 0$ indica posibles puntos \textbf{máximos o mínimos}
    \item $f'(x) > 0$ significa que la función está \textbf{creciendo}
    \item $f'(x) < 0$ significa que la función está \textbf{decreciendo}
    \item La segunda derivada $f''(x)$ nos dice sobre la \textbf{concavidad}
    \item Para verificar máximos: $f''(x) < 0$
    \item Para verificar mínimos: $f''(x) > 0$
\end{enumerate}

\vspace{1cm}

\begin{center}
\Large\textbf{¡La derivada es el lenguaje del cambio!}
\end{center}

\end{document}
